\documentclass[12pt]{article}
%\documentclass[12pt,review,authoryear]{elsarticle}
%\usepackage{ecrc}       % needed for elsarticle
\usepackage{chicago}    % bibliography package
\usepackage{graphicx}   % insert PostScript figures
\usepackage{setspace}   % controls line spacing
\usepackage{amsmath,amsthm,amssymb,amstext} % controls equation entry and symbols
\usepackage{rotating}   % rotates graphics
\usepackage{soul}       % controls hyphenation
\usepackage{epsfig}     % helps with including graphics
\usepackage{pdflscape}  % helps with displaying rotated graphics in PDFs
\usepackage{lscape}     % helps with rotating pages
\usepackage{caption}    % controls  captions
\usepackage{adjustbox}  % shrink tables
\usepackage[margin=1in]{geometry} % margins
\usepackage{hyperref}   % displays URLs
%packages for kable tables
 \usepackage{booktabs}
 \usepackage{longtable}
 \usepackage{array}
 \usepackage{multirow}
 \usepackage{wrapfig}
 \usepackage{float}
 \usepackage{colortbl}
 \usepackage{pdflscape}
 \usepackage{tabu}
 \usepackage{threeparttable}
 \usepackage{threeparttablex}
 \usepackage[normalem]{ulem}
 \usepackage{makecell}
 \usepackage{xcolor}

\graphicspath{{../output/}}

\interfootnotelinepenalty=10000

\parskip     2.0mm       % space between paragraphs

\alph{footnote}         % make title footnotes alpha-numeric

\captionsetup[figure]{labelsep=space,labelfont=bf} % remove colon from figure name


\begin{document}

%%%%%%%%%%%%%%%%%%%%%%%%%%%%%%%%%%%%%%%%%%%%%%%%%%%%%%%%%%%%%%%%%%%%%%%%%%%%%%%%%
\section{Numerical Simulations \label{sec:sims}}
In this section, we use a similar simulation setup to that in \citeN{ST2021}, but allow for either the merging parties or their rivals to be vertically integrated before they merge. This modification allows us to explore the welfare implications of complex mergers, rather than only purely horizontal or purely vertical combinations.

%%%%%%%%%%%%%%%%%%%%%%%%%%%%%%%%%%%%%%%%%%
\subsection{Data Generating Process}
Broadly speaking, we consider four categories of mergers: downstream, upstream, vertical, and integrated.  We define a downstream merger as a merger between two unintegrated retailers or between an unintegrated retailer and a vertically integrated wholesaler/retailer combination.  Similarly, upstream mergers are those between two unintegrated wholesalers or between an unintegrated wholesaler and a vertically integrated wholesaler/retailer.  We define vertical mergers as those between an unintegrated wholesaler and unintegrated retailer.  Finally, an integrated merger is between two firms that are both already vertically integrated pre-merger.

For downstream, upstream, and vertical mergers, we simulate markets by randomly sampling shares from a Dirichlet distribution for 2 to 5 retailers or wholesalers, respectively.\footnote{We parametrize the Dirichlet distribution so it is equivalent to a uniform distribution.  }  Because integrated mergers must have two vertically integrated incumbents in the pre-merger state, for those simulations we increase the maximum allowable number of wholesalers and retailers to 7. We also assume that in the pre-merger state, there are anywhere from 0 to 4 vertically integrated incumbents (2 to 6 for integrated mergers). Vertically integrated firms are not siloed: integrated wholesalers supply inputs to retailers other than their integrated partner, and integrated retailers purchase inputs from wholesalers other than their integrated partner.

Our simulations focus on mergers that are more likely to have anti-competitive effects and to therefore come under agency scrutiny. For horizontal merger simulations, we assign the products sold by the two largest firms in the market to a single firm post-merger. Similarly, when simulating a vertical merger, we assign the products sold by the largest wholesaler and the largest retailer to a single firm post-merger.

The bargaining parameter ranges from from 0.1 (where wholesalers have the advantage) to 0.9 (where retailers have the advantage). We report our results in terms of relative bargaining power, $(1-\lambda)/\lambda$.  The bargaining parameter is identical for all of the retailers in each simulation.  We calibrate the price coefficient $\alpha$ assuming that in the pre-merger world, there is a vertically integrated outside option that sets its Bertrand equilibrium price to \$5, has 15\% market share, and has zero marginal costs.  All other goods are differenced relative to this option, resulting in the mean zero outside good normalization.  We assume that pre-merger, upstream marginal costs are 25\% of upstream firm margins and 10\% of downstream firm margins. We set the market size to 1.

For each combination of number of retailers, number of wholesalers, number of incumbent integrated firms, and bargaining parameter, we draw 1,000 different sets of market primitives.  After eliminating situations where the merger is unprofitable to the merging firms, as well as downstream markets that do not pass the Hypothetical Monopolist Test, we have 1.6 million markets remaining.\footnote{The Hypothetical Monopolist Test is a market definition exercise that checks whether a monopolist that owns all products in a candidate market would raise the price of at least one of the merging producers' products by at least a ``small but significant non-transitory increase in price'' (SSNIP), which we take to be 5\%. } Each market treats as primitives the number of retailers, the number of wholesalers, the bargaining parameter, and the wholesaler and retailer marginal costs.

%%%%%%%%%%%%%%%%%%%%%%%%%%%%%%%%%%%%%%%%%%
\subsection{Overview of Simulated Output}

Table \ref{tab:simsum} provides summary statistics for our simulations. The 2010 DOJ/FTC Horizontal Merger Guidelines categorize ``Highly Concentrated Markets'' as those with Herfindahl Hirschman Index (HHI) values over 2,500 points.  The Guidelines state that mergers with HHI changes greater than 200 points that result in Highly Concentrated markets are ``presumed likely to enhance market power.''\footnote{See \S 5.3 of the Guidelines.  }  The vast majority of our mergers meet these conditions.\footnote{We compute the post-merger HHI for vertical mergers by treating the merged firms' market share as the sum of all the shares of downstream products that either use the upstream partner's input or are sold by the downstream partner.  }  The median average wholesale pre-merger price is \$5.80, and the median average retail pre-merger price is \$15.  Recall that the market size is set to 1, meaning that average price is equal to total pre-merger expenditures.

Figure \ref{fig:surplussum} summarizes our results for welfare.  Each of the four panels shows how mergers impact surplus for a particular set of agents (consumers, retailers, wholesalers, or the entire market combined, respectively).  Surplus is reported as a percentage change of total pre-merger expenditures in the downstream market.  Each panel contains four box and whisker plots, with each plot corresponding to a different type of merger. The whiskers show the $5^{th}$ and $95^{th}$ percentiles of the outcome distribution, the boxes denote the $25^{th}$ and $75^{th}$ percentiles, and the solid horizontal line marks the median. Negative outcome values imply harm, and positive values imply benefits.

We focus first on the results for consumers in the left-most panel of Figure \ref{fig:surplussum}.  Less than half of all simulated vertical mergers are harmful, while the majority of upstream, downstream, and integrated mergers show harm.  In particular, there is a partial rank-ordering of consumer harm across types of mergers: consumer harm from downstream and integrated mergers first-order stochastically dominates consumer harm from upstream mergers which in turn dominates consumer harm from vertical mergers. However, while consumer harm is typically greater under integrated mergers than downstream mergers, it is not first-order stochastically dominant. Median consumer harm from integrated mergers is about 14\% of pre-merger expenditures, 1.3 times the magnitude of downstream mergers, and 3 times the magnitude from upstream mergers.

Moving on to retailers in the second panel of Figure \ref{fig:surplussum}, we find that whereas vertical mergers, downstream mergers, and integrated mergers almost always benefit retailers, upstream mergers harm retailers in about 60\% of all simulations. Moreover, there is a partial rank-ordering across mergers that is distinct from the consumer rank-ordering: the retailer surplus distribution from integrated mergers first-order stochastically dominates that from downstream mergers, while the retailer surplus distribution from vertical mergers stochastically dominates the distribution from upstream mergers. The median gain to retailers from integrated mergers is about 13\% of pre-merger expenditures,  1.5 times the magnitude of downstream mergers, and 1.1 times the magnitude of that from vertical mergers.

Turning to wholesaler surplus in the third panel of Figure \ref{fig:surplussum}, the effects seen there are largely reversed from those for retailers: wholesaler surplus increases in about 64\% of all upstream mergers, 11\% of vertical mergers, and about 16\% of downstream  and integrated mergers. Here, wholesaler surplus under integrated, downstream and upstream mergers each stochastically dominate the surplus from vertical mergers, but not one another.

As for total welfare, approximately 29\% of vertical mergers and 17\% of integrated mergers are beneficial, whereas only 9\% of upstream mergers and 2\% of downstream mergers are beneficial. Moreover, there is a partial rank ordering of mergers, with total harm from downstream mergers first-order stochastically dominating total harm from upstream  mergers, which dominates total harm from vertical mergers. Median consumer harm from integrated mergers is about 9\% of pre-merger total expenditures, 1.4 times the magnitude of that from downstream mergers, 2.6 times the magnitude of that from upstream mergers, and more than 4 times the magnitude of vertical mergers.

Throughout these simulations, we maintain the assumption that marginal costs are constant.  However, given that many of the mergers we study have vertical aspects where input cost considerations may drive the realization of EDM and RRC, it is reasonable to wonder how our results would change in an alternative marginal cost environment.  In Appendix \ref{app:mc}, we provide results that allow marginal costs to increase linearly.\footnote{We still maintain that these cost functions are not affected by the merger, which rules out merger cost efficiencies.  }  We find that this alternative assumption tends to result in mergers that are more harmful for consumers and total welfare.

%%%%%%%%%%%%%%%%%%%%%%%%%%%%%%%%%%%%%%%%%%
\subsection{Mergers and Vertically Integrated Incumbent Firms}

We examine merger outcomes both with and without preexisting vertical integration among the merging firms and as the number of rival incumbent integrated firms varies.  Figure \ref{fig:CVvertincumbBW_consumer} depicts box and whisker plots summarizing consumer welfare as the number of incumbent integrated firms increases. For the results on total welfare, see Appendix \ref{app:total}.  The plots when the number of incumbent integrated firms equals 0 correspond to the results depicted in Figure 1 of \citeN{ST2021} and are included as reference.\footnote{One difference between the simulations depicted in Figure \ref{fig:CVvertincumbBW_consumer} and those in Figure 1 of \citeN{ST2021} is that here we do not include downstream markets where prices are set according to a second score auction.}  For vertical mergers, the plot when the number of incumbent integrated firms equals 1 summarizes market outcomes where an unintegrated wholesaler merges with an unintegrated retailer, a single third party is integrated, and any remaining rivals are unintegrated. By contrast, for upstream and downstream  mergers, the plots when the number of incumbent integrated firms equals 1 depict the outcome of a merger between an integrated firm and an unintegrated wholesaler or retailer, respectively, with no integrated third parties. For integrated mergers, the plot when the number of incumbent integrated firms equals 0 depicts a merger between two integrated firms, again with no integrated rivals.

We begin with vertical mergers, as they are our simplest case, because there is never any pre-existing integration at either merging firm.  Absent incumbent integration by rivals (denoted by 0 integrated firms in Figure \ref{fig:CVvertincumbBW_consumer}), about half of vertical mergers benefit consumers.  The addition of a single rival integrated firm (denoted by 1 integrated firm in the figure) moves median consumer welfare from about 0\% of pre-merger revenues to a positive 2\% and widens the distribution of outcomes slightly. Adding more integrated rivals does not have much additional effect.

Moving to the second panel in Figure \ref{fig:CVvertincumbBW_consumer}, upstream mergers absent any incumbent vertical integration (0 integrated firms) are never beneficial in the range we study, with median consumer harm at about 6\% of pre-merger revenues. This result is to be expected, as upstream mergers without additional integration do not have the countervailing effects for welfare possible in the context of downstream mergers (where combined retailers may bargain for lower input prices) or vertical mergers (where EDM can create benefits).  Once one of the merging firms is allowed to be integrated (1 integrated firm in the figure) beneficial mergers appear.  Consumer welfare increases in almost 30\% of mergers, though median consumer harm is roughly steady at approximately 4\% of pre-merger revenues.  The additional vertical integration creates some opportunities for EDM to enhance welfare, but this must be balanced against the potential for RRC.  Here we find that harm dominates in most simulations.  Adding in integration by rival firms (2-4 integrated firms in the figure) narrows the inter-quartile range and somewhat trims the whiskers while having little impact on median harm.

Next we examine the welfare impacts of downstream mergers in the third panel of Figure \ref{fig:CVvertincumbBW_consumer}. As with upstream mergers, downstream mergers absent incumbent vertical integration (0 integrated firms in the figure) are never beneficial in the $5^{th}$ to $95^{th}$ percentile range, with median consumer harm from downstream mergers equal to about 9\% of pre-merger revenues. \citeN{ST2021} found the same result for mergers in downstream logit Bertrand markets.  Allowing one of the merging firms to be vertically integrated when all other market participants are unintegrated (1 integrated firm in the figure) increases the range of outcomes while also leading to more median harm: consumer welfare increases in about 20\% of mergers, but median consumer harm grows to about 12\% of pre-merger revenues. Once one of the merging firms is already integrated, the merger now has potential EDM and RRC effects, which raises the possibility of both benefits and harms to welfare.  Similar to what we saw for upstream mergers, we find that the harms dominate in most instances.  Adding integrated third party rivals (2-4 integrated firms in the figure) somewhat moves the distribution towards less harm.  With 3 integrated third parties (denoted by 4 integrated firms in the figure), the merger benefits consumers in about 34\% of markets and shrinks median consumer harm to about 5\% of pre-merger revenues.

We finish by examining integrated mergers in the right-most panel of Figure \ref{fig:CVvertincumbBW_consumer}.  Absent the presence of rival incumbent integrated firms (0 integrated firms in the figure), mergers between two integrated firms benefit consumers in about 6\% of simulated markets. Thus, we find that mergers between two firms that are already integrated are harmful to consumers in the vast majority of cases.  Adding one additional rival incumbent integrated firm (1 integrated firm in the figure) does not have a significant impact on the median or on the range of outcomes. Adding more integrated rivals (2-4 integrated firms in the figure) moves median consumer welfare to be slightly more negative.

To summarize, our analysis of Figure \ref{fig:CVvertincumbBW_consumer} yields some overall conclusions.  First, we find that mergers where one or both of the merging firms are already integrated (and all rivals are unintegrated) are harmful to consumers in most instances.  Although these mergers have the possibility of creating benefits through EDM, these gains appear to be outweighed in the majority of the mergers we study.  Second, the presence of integrated rivals has some impact on the distribution of consumer harm, but the observed patterns are noisy and the resulting box plots are largely similar to those where such rivals are absent.  Therefore, having such rival firms in and of itself does not generate significantly better outcomes for consumers.

%%%%%%%%%%%%%%%%%%%%%%%%%%%%%%%%%%%%%%%%%%
\subsection{Mergers and Bargaining Power}
\citeN{ST2021} show using numerical simulations that downstream and vertical mergers when wholesalers have relatively more bargaining power are less harmful compared to when retailers have relatively more bargaining power. Here, we find that this relationship also holds for downstream and upstream mergers when one of the merging parties is already vertically integrated and for integrated mergers.  We also show that the result persists when there are rival non-merging integrated firms.

Figure \ref{fig:CVbargincumbent_consumer} depicts box and whisker plots summarizing the consumer welfare effects for vertical, upstream, downstream, and integrated mergers as the relative bargaining power parameter goes from 9 (wholesalers have the advantage) to 1/9 (retailers have the advantage). Also depicted for each bargaining power parameter are three sets of box and whisker plots that correspond to the number of incumbent vertically integrated firms included in the simulated markets: 0 (light blue), 1 (darker blue), and 4 (darkest blue) firms.  Results for total welfare are in Appendix \ref{app:total}.  For vertical, upstream, and downstream mergers, plots when the number of integrated firms is 0 assume that pre-merger, no firms in the market are vertically integrated; these are comparable to results in \citeN{ST2021}. For vertical mergers, the plots when the number of incumbent integrated firms equals 1 depict the outcome from an unintegrated wholesaler merging with an unintegrated retailer when one third party rival is integrated. For upstream and downstream mergers, the plots when the number of incumbent integrated firms equals 1 depict the outcome of a merger between an integrated firm and an unintegrated firm when all third parties are unintegrated. In turn, plots with 4 integrated firms increase the number of rival incumbent integrated firms in vertical mergers to 4 and in upstream and downstream mergers to 3.  Finally, in integrated mergers, when the number of incumbent integrated firms equals 0, this depicts a merger between two integrated firms with all rivals are unintegrated. Plots with either 1 or 4 integrated firms depict mergers between integrated incumbents when there are either 1 or 4 rival integrated firms in the market, respectively.

We begin by examining vertical mergers in the leftmost panel of Figure \ref{fig:CVbargincumbent_consumer}.  We focus first on the case when there are no incumbent integrated firms (the light blue plots).  Consistent with the patterns seen in \citeN{ST2021}, mergers tend to benefit consumers when wholesalers have relatively more power, and harm consumers when retailers have relatively more power.  Larger wholesaler bargaining power offers more possibilities for EDM, as pre-merger input prices are likely to be high.  This channel becomes less relevant as retailers gain more power.  The impact on consumers is zero in the neighborhood of retailer to wholesaler bargaining power ratios of 1 to 2/3.  As we add 1 and then 4 integrated rival firms (shown by the darker and darkest blue box plots, respectively), the range of welfare outcomes narrows and shifts upwards, though we still see the negative relationship between consumer welfare and relative retailer bargaining power.  Across all three integrated incumbent scenarios, median consumer harm moves from about -24\% of pre-merger expenditures when wholesalers have the advantage to 7\% when retailers have the advantage.

We turn next to upstream mergers in the second panel of Figure \ref{fig:CVbargincumbent_consumer}.  Without incumbent integrated firms in the market (the light blue plots), the relationship between harm and bargaining power is unlike that for downstream and vertical mergers, as it is not monotonic.  By contrast, for mergers between an integrated firm and an unintegrated upstream supplier (the darker blue plots), the relationship between harm and bargaining power is more in line with that for downstream and vertical mergers.  This is intuitive, as now all three types of mergers involve bargaining interactions that can lower input prices, the savings from which are then passed through to consumers to varying degrees. Median consumer harm goes from roughly -21\% of pre-merger expenditures when wholesalers have relatively more bargaining power to almost 8\% of pre-merger expenditures when retailers have relatively more bargaining power. Adding 3 incumbent integrated rivals (the darkest blue plots) preserves the relationship between bargaining power and harm, though the spread of outcomes narrows and the medians shift up slightly.

For downstream mergers, like with vertical mergers, we find that higher retailer relative bargaining power leads to more negative impacts on welfare.  Absent any incumbent integration (the light blue plots), all the outcomes shown in Figure \ref{fig:CVbargincumbent_consumer} harm consumers, which mirrors our findings in Figure \ref{fig:CVvertincumbBW_consumer}.  Median consumer harm goes from about 1\% of pre-merger expenditures when wholesalers have relatively more bargaining power to 19\% of pre-merger expenditures when retailers have relatively more bargaining power. In situations where retailers already have more bargaining power, retailers are likely to have extracted significant surplus from wholesalers prior to the merger, which limits any potential benefits from increased bargaining leverage.  Thus, there is little to counteract harms from decreased downstream competition.

Once we allow one of the merging firms to be integrated (the darker blue plots), some beneficial mergers appear, particularly when wholesalers have relatively more bargaining power.  The box and whisker plots in Figure \ref{fig:CVbargincumbent_consumer} now show a stronger relationship between bargaining power and extent of consumer harm.  Median consumer harm goes from approximately -28\% of pre-merger revenues when wholesalers have the advantage to 24\% of pre-merger revenues when retailers have the advantage.  The existence of pre-merger integration at one of the merging firms creates an opportunity for EDM, the benefits from which are likely to be largest when wholesalers have high bargaining power and therefore charge high pre-merger input prices.  Compared to vertical mergers, where impacts to consumers were roughly neutral at 2/3 bargaining power, here the neutrality crossing shifts left, when wholesalers have a bit more than 3/2 times the power of retailers.  It appears that wholesalers must be relatively more powerful (and thus the likely gains from EDM larger) in order to generate net consumer benefits, compared to in a vertical merger.  This is intuitive, as here the downstream merger also causes an additional lessening in horizontal competition.  Once we add 3 rival integrated firms (the darkest blue plots) the box and whisker plots shift up. The range of outcomes also shrinks.  This is consistent with what we see for vertical and upstream mergers.  Starting from when wholesalers have relatively more bargaining power and moving right, median consumer harm goes from approximately -35\% of pre-merger revenues to 13\% of pre-merger revenues.

The last panel in Figure \ref{fig:CVbargincumbent_consumer} displays results for integrated mergers.  In terms of bargaining power and resulting harm, integrated mergers are perhaps most similar to vertical mergers. Like vertical mergers, there is a strong monotonic relationship between bargaining power and harm. Median consumer harm moves from about 0\% of pre-merger expenditures when wholesalers have relatively more bargaining power to about 26\% of pre-merger expenditures when retailers have relatively more bargaining power. A second similarity between vertical and integrated mergers is that the presence of rival integrated incumbents (see the darker and darkest blue plots) tends to narrow the range of outcomes and somewhat shift the distribution upwards. However, a key difference between vertical and integrated mergers is that while vertical mergers are often beneficial when wholesalers have relatively more bargaining power, integrated mergers are often harmful unless wholesaler relative bargaining power is roughly 4 times or more. The crossing at zero harm is shifted to the left.

In summary, our review of Figure \ref{fig:CVbargincumbent_consumer} has revealed some systematic patterns.  First, we see that vertical mergers, upstream and downstream mergers where one of the merging firms is integrated, and integrated mergers all exhibit a negative relationship between consumer harm and increasing relative retailer bargaining power.  Consumers seem most likely to benefit from these mergers when wholesalers have significant bargaining power, which are situations when we would expect the gains from EDM to be largest.  Second, the point at which this monotonic relationship between consumer welfare and bargaining power crosses zero varies across merger types, and appears to be farthest to the left in integrated mergers.  That is, wholesalers need to have a substantial bargaining power advantage over retailers in order for consumers to benefit from an integrated merger.  Third, the presence of integrated rivals tends to shrink the range of welfare outcomes and to shift this distribution upwards.  This pattern for integrated rivals is much less noisy than in Figure \ref{fig:CVvertincumbBW_consumer}, where we did not condition on relative bargaining power.

%%%%%%%%%%%%%%%%%%%%%%%%%%%%%%%%%%%%%%%%%%%%%%%%%%%%%%%%%%%%%%%%%%%%%%%%%%%%%%%%%
% Bibliography
%%%%%%%%%%%%%%%%%%%%%%%%%%%%%%%%%%%%%%%%%%%%%%%%%%%%%%%%%%%%%%%%%%%%%%%%%%%%%%%%%
\newpage
\bibliographystyle{chicago}
%\bibliographystyle{elsarticle-harv}
\bibliography{master-bib}
\clearpage

%%%%%%%%%%%%%%%%%%%%%%%%%%%%%%%%%%%%%%%%%%%%%%%%%%%%%%%%%%%%%%%%%%%%%%%%%%%%%%%%%
% Tables and Figures
%%%%%%%%%%%%%%%%%%%%%%%%%%%%%%%%%%%%%%%%%%%%%%%%%%%%%%%%%%%%%%%%%%%%%%%%%%%%%%%%%
\begin{table}
\caption{Summary Statistics for Numerical Simulations}
\label{tab:simsum}
\small{
\begin{tabular}{l|l|l|r}
\hline
variable & merger & quant & val\\
\hline
down &  & Min & 2\\

up &  & Min & 2\\

vert &  & Min & 0\\

barg &  & Min & 0\\

nestParm &  & Min & 0\\

avgpricepre.up &  & Min & 1\\

avgpricepre.down &  & Min & 6\\

mktElast & \multirow{-8}{*}{\raggedright\arraybackslash all} & Min & -60\\
\cline{1-4}
hhipre &  & Min & 2008\\

hhipost &  & Min & 2915\\

hhidelta & \multirow{-3}{*}{\raggedright\arraybackslash up} & Min & 0\\
\cline{1-4}
hhipre &  & Min & 2011\\

hhipost &  & Min & 2931\\

hhidelta & \multirow{-3}{*}{\raggedright\arraybackslash down} & Min & 0\\
\cline{1-4}
hhipre &  & Min & 2100\\

hhipost &  & Min & 3120\\

hhidelta & \multirow{-3}{*}{\raggedright\arraybackslash vertical} & Min & 32\\
\cline{1-4}
hhipre &  & Min & 2205\\

hhipost &  & Min & 3633\\

hhidelta & \multirow{-3}{*}{\raggedright\arraybackslash both} & Min & 2\\
\cline{1-4}
down &  & p25 & 3\\

up &  & p25 & 3\\

vert &  & p25 & 0\\

barg &  & p25 & 0\\

nestParm &  & p25 & 0\\

avgpricepre.up &  & p25 & 2\\

avgpricepre.down &  & p25 & 10\\

mktElast & \multirow{-8}{*}{\raggedright\arraybackslash all} & p25 & -1\\
\cline{1-4}
hhipre &  & p25 & 2393\\

hhipost &  & p25 & 4011\\

hhidelta & \multirow{-3}{*}{\raggedright\arraybackslash up} & p25 & 1546\\
\cline{1-4}
hhipre &  & p25 & 2572\\

hhipost &  & p25 & 4135\\

hhidelta & \multirow{-3}{*}{\raggedright\arraybackslash down} & p25 & 1431\\
\cline{1-4}
hhipre &  & p25 & 3623\\

hhipost &  & p25 & 5069\\

hhidelta & \multirow{-3}{*}{\raggedright\arraybackslash vertical} & p25 & 1051\\
\cline{1-4}
hhipre &  & p25 & 3591\\

hhipost &  & p25 & 6242\\

hhidelta & \multirow{-3}{*}{\raggedright\arraybackslash both} & p25 & 2463\\
\cline{1-4}
down &  & p50 & 4\\

up &  & p50 & 4\\

vert &  & p50 & 1\\

barg &  & p50 & 1\\

nestParm &  & p50 & 0\\

avgpricepre.up &  & p50 & 5\\

avgpricepre.down &  & p50 & 13\\

mktElast & \multirow{-8}{*}{\raggedright\arraybackslash all} & p50 & -1\\
\cline{1-4}
hhipre &  & p50 & 2876\\

hhipost &  & p50 & 4963\\

hhidelta & \multirow{-3}{*}{\raggedright\arraybackslash up} & p50 & 2027\\
\cline{1-4}
hhipre &  & p50 & 3166\\

hhipost &  & p50 & 5207\\

hhidelta & \multirow{-3}{*}{\raggedright\arraybackslash down} & p50 & 2040\\
\cline{1-4}
hhipre &  & p50 & 4391\\

hhipost &  & p50 & 6016\\

hhidelta & \multirow{-3}{*}{\raggedright\arraybackslash vertical} & p50 & 1292\\
\cline{1-4}
hhipre &  & p50 & 4275\\

hhipost &  & p50 & 7272\\

hhidelta & \multirow{-3}{*}{\raggedright\arraybackslash both} & p50 & 2800\\
\cline{1-4}
down &  & p75 & 5\\

up &  & p75 & 5\\

vert &  & p75 & 2\\

barg &  & p75 & 1\\

nestParm &  & p75 & 0\\

avgpricepre.up &  & p75 & 10\\

avgpricepre.down &  & p75 & 19\\

mktElast & \multirow{-8}{*}{\raggedright\arraybackslash all} & p75 & 0\\
\cline{1-4}
hhipre &  & p75 & 3794\\

hhipost &  & p75 & 6780\\

hhidelta & \multirow{-3}{*}{\raggedright\arraybackslash up} & p75 & 2891\\
\cline{1-4}
hhipre &  & p75 & 4257\\

hhipost &  & p75 & 7391\\

hhidelta & \multirow{-3}{*}{\raggedright\arraybackslash down} & p75 & 2934\\
\cline{1-4}
hhipre &  & p75 & 5679\\

hhipost &  & p75 & 7293\\

hhidelta & \multirow{-3}{*}{\raggedright\arraybackslash vertical} & p75 & 1755\\
\cline{1-4}
hhipre &  & p75 & 5358\\

hhipost &  & p75 & 8648\\

hhidelta & \multirow{-3}{*}{\raggedright\arraybackslash both} & p75 & 3187\\
\cline{1-4}
down &  & Max & 5\\

up &  & Max & 5\\

vert &  & Max & 4\\

barg &  & Max & 1\\

nestParm &  & Max & 0\\

avgpricepre.up &  & Max & 270\\

avgpricepre.down &  & Max & 297\\

mktElast & \multirow{-8}{*}{\raggedright\arraybackslash all} & Max & 0\\
\cline{1-4}
hhipre &  & Max & 10000\\

hhipost &  & Max & 10000\\

hhidelta & \multirow{-3}{*}{\raggedright\arraybackslash up} & Max & 5000\\
\cline{1-4}
hhipre &  & Max & 10000\\

hhipost &  & Max & 10000\\

hhidelta & \multirow{-3}{*}{\raggedright\arraybackslash down} & Max & 5000\\
\cline{1-4}
hhipre &  & Max & 9962\\

hhipost &  & Max & 10000\\

hhidelta & \multirow{-3}{*}{\raggedright\arraybackslash vertical} & Max & 4314\\
\cline{1-4}
hhipre &  & Max & 9998\\

hhipost &  & Max & 10000\\

hhidelta & \multirow{-3}{*}{\raggedright\arraybackslash both} & Max & 5000\\
\cline{1-4}
down &  & Markets & 2202997\\

up &  & Markets & 2202997\\

vert &  & Markets & 2202997\\

barg &  & Markets & 2202997\\

nestParm &  & Markets & 2202997\\

avgpricepre.up &  & Markets & 2202997\\

avgpricepre.down &  & Markets & 2202997\\

mktElast & \multirow{-8}{*}{\raggedright\arraybackslash all} & Markets & 2202997\\
\cline{1-4}
hhipre &  & Markets & 654731\\

hhipost &  & Markets & 654731\\

hhidelta & \multirow{-3}{*}{\raggedright\arraybackslash up} & Markets & 654731\\
\cline{1-4}
hhipre &  & Markets & 701936\\

hhipost &  & Markets & 701936\\

hhidelta & \multirow{-3}{*}{\raggedright\arraybackslash down} & Markets & 701936\\
\cline{1-4}
hhipre &  & Markets & 462005\\

hhipost &  & Markets & 462005\\

hhidelta & \multirow{-3}{*}{\raggedright\arraybackslash vertical} & Markets & 462005\\
\cline{1-4}
hhipre &  & Markets & 384325\\

hhipost &  & Markets & 384325\\

hhidelta & \multirow{-3}{*}{\raggedright\arraybackslash both} & Markets & 384325\\
\hline
\end{tabular}
}
\end{table}


\begin{sidewaysfigure}
\centering
\includegraphics[scale=0.9]{output/surplussum.png}
\caption{The figure displays box and whisker plots summarizing the extent to which mergers affect consumer, retailer, wholesaler, and total surplus. Whiskers depict the $5^{th}$ and $95^{th}$ percentiles of a particular outcome, boxes depict the $25^{th}$ and $75^{th}$ percentiles, and the solid horizontal line depicts the median. }
\label{fig:surplussum}
\end{sidewaysfigure}

\begin{sidewaysfigure}
\centering
\includegraphics[scale=0.9]{output/CVvertincumbBW_consumer.png}
\caption{The figure displays box and whisker plots summarizing the extent to which mergers affect consumer surplus as the number of vertically integrated firms present in a market changes.  Whiskers depict the $5^{th}$ and $95^{th}$ percentiles of a particular outcome, boxes depict the $25^{th}$ and $75^{th}$ percentiles, and the solid horizontal line depicts the median. }
\label{fig:CVvertincumbBW_consumer}
\end{sidewaysfigure}

\begin{sidewaysfigure}
\centering
\includegraphics[scale=0.9]{output/CVbargincumbent_consumer.png}
\caption{The figure displays box and whisker plots summarizing the extent to which vertical, upstream, downstream and integrated mergers affect consumer surplus as the bargaining power of wholesalers relative to retailers changes. The different colored boxes display how outcomes change as the number of vertically  integrated firms changes. Whiskers depict the $5^{th}$ and $95^{th}$ percentiles of a particular outcome, boxes depict the $25^{th}$ and $75^{th}$ percentiles, and the solid horizontal line depicts the median.}
\label{fig:CVbargincumbent_consumer}
\end{sidewaysfigure}



%%%%%%%%%%%%%%%%%%%%%%%%%%%%%%%%%%%%%%%%%%%%%%%%%%%%%%%%%%%%%%%%%%%%%%%%%%%%%%%%%
% Appendix
%%%%%%%%%%%%%%%%%%%%%%%%%%%%%%%%%%%%%%%%%%%%%%%%%%%%%%%%%%%%%%%%%%%%%%%%%%%%%%%%%
\appendix
\newpage
\clearpage

\numberwithin{equation}{section}
\numberwithin{figure}{section}
\numberwithin{table}{section}

%%%%%%%%%%%%%%%%%%%%%%%%%%%%%%%%%%%%%%%%%%%%%%%%%%%%%%%%%%%%%%%%%%%%%%%%%%%%%%%%%
\newpage
\section{Appendix: Additional Figures\label{app:total}}

Figures \ref{fig:CVvertincumbBW_total} and \ref{fig:CVbargincumbent_total} are the analogous results as in Figures \ref{fig:CVvertincumbBW_consumer} and \ref{fig:CVbargincumbent_consumer}, but for total welfare. The patterns for total welfare are largely similar to those for consumer welfare.  We see in Figure \ref{fig:CVvertincumbBW_total} that most upstream, downstream, and integrated mergers are harmful.  Figure \ref{fig:CVbargincumbent_total} shows that net beneficial
mergers typically only occur when wholesalers have relatively more bargaining power.  The presence of integrated rivals can narrow the range of outcomes.

\begin{sidewaysfigure}
\centering
\includegraphics[scale=0.9]{output/CVvertincumbBW_total.png}
\caption{The figure displays box and whisker plots summarizing the extent to which mergers affect total surplus as the number of vertically integrated firms present in a market changes.  Whiskers depict the $5^{th}$ and $95^{th}$ percentiles of a particular outcome, boxes depict the $25^{th}$ and $75^{th}$ percentiles, and the solid horizontal line depicts the median. }
\label{fig:CVvertincumbBW_total}
\end{sidewaysfigure}

\begin{sidewaysfigure}
\centering
\includegraphics[scale=0.9]{output/CVbargincumbent_total.png}
\caption{The figure displays box and whisker plots summarizing the extent to which vertical, upstream, downstream and integrated mergers affect total surplus as the bargaining power of wholesalers relative to retailers changes. The different colored boxes display how outcomes change as the number of vertically  integrated firms changes. Whiskers depict the $5^{th}$ and $95^{th}$ percentiles of a particular outcome, boxes depict the $25^{th}$ and $75^{th}$ percentiles, and the solid horizontal line depicts the median.}
\label{fig:CVbargincumbent_total}
\end{sidewaysfigure}


\end{document}
%Concerns about vertical competitive effects were raised in both the Republic-Santek and the Waste Management-ADS transactions. Both the Solid Waste Agency of Lake County, IL and the Solid Waste Agency of Northern Cook County submitted comments in opposition to the proposed asset divestiture from Waste Management-ADS, stating that a vertically integrated competitor was needed to maintain competition in their local market post-merger.\footnote{\tiny{See  Comments by SWALCO and SWANCC : U.S. and Plaintiff States v. Waste Management, Inc., and Advanced Disposal Services, Inc., \url{https://www.justice.gov/atr/case-document/file/1377646/download}.}}
%For example, in the DOJ complaint filed in the Republic-Santek case, horizontal anti-competitive effects were alleged for four SCCW collection markets and two MSW disposal markets. In addition, vertical anti-competitive effects were alleged to arise from the combination of their integrated assets in the Chattanooga area.\footnote{See U.S. and State of Alabama v. Republic Services, Inc. and Santek Waste Services, LLC, \url{https://www.justice.gov/atr/case-document/file/1382031/download}}
%Our application in progress further identifies local markets in which these acquisitions result in: 1) only horizontal combinations of assets, 2) only vertical combinations of assets, and 3) combinations of vertically integrated assets in the presence of existing integrated competitors to analyze the welfare effects from mergers with complex vertical arrangements. The next sections present preliminary results on the merger of vertically integrated assets in the presence of other integrated and unintegrated rivals in the context of Republic and Santek's merger in the Chattanooga area.
%The next sections present results of applying the model of Section \ref{sec:theory} to the merger of vertically integrated assets in the presence of other integrated and unintegrated rivals in the context of Republic and Santek's merger in the Chattanooga area.

% \begin{sidewaysfigure}
% \centering
% \includegraphics[scale=0.9]{../CVvertincumb_updownBW.png}
% \caption{The figure displays box and whisker plots summarizing the extent to which mergers affect consumer (blue,left) and total (orange,right) surplus as the number of vertically integrated firms present in a market change.  Whiskers depict the $5^{th}$ and $95^{th}$ percentiles of a particular outcome, boxes depict the $25^{th}$ and $75^{th}$ percentiles, and the solid horizontal line depicts the median. }
% \label{fig:CVvertincumbupdownBW}
% \end{sidewaysfigure}
%
% \begin{sidewaysfigure}
% \centering
% \includegraphics[scale=0.9]{../CVvertincumb_vertincBW.png}
% \caption{The figure displays box and whisker plots summarizing the extent to which mergers affect consumer (blue,left) and total (orange,right) surplus as the number of vertically integrated firms present in a market change.  Whiskers depict the $5^{th}$ and $95^{th}$ percentiles of a particular outcome, boxes depict the $25^{th}$ and $75^{th}$ percentiles, and the solid horizontal line depicts the median. }
% \label{fig:CVvertincumbvertincBW}
% \end{sidewaysfigure}

% \begin{sidewaysfigure}
% \centering
% \includegraphics[scale=0.9]{../CVbargupBW.png}
% \caption{The figure displays box and whisker plots summarizing the extent to which mergers among an integrated and unintegrated wholesaler affect consumer, retailer, wholesaler, and total surplus as the bargaining power of wholesalers relative to retailers changes. The different colored boxes display how outcomes change as the number of vertically  integrated firms increases. Whiskers depict the $5^{th}$ and $95^{th}$ percentiles of a particular outcome, boxes depict the $25^{th}$ and $75^{th}$ percentiles, and the solid horizontal line depicts the median.}
% \label{fig:CVbargupBW}
% \end{sidewaysfigure}
%
% \begin{sidewaysfigure}
% \centering
% \includegraphics[scale=0.9]{../CVbargdownBW.png}
% \caption{The figure displays box and whisker plots summarizing the extent to which mergers among an integrated and unintegrated retailer affect consumer, retailer, wholesaler, and total surplus as the bargaining power of wholesalers relative to retailers changes. The different colored boxes display how outcomes change as the number of vertically  integrated firms increases. Whiskers depict the $5^{th}$ and $95^{th}$ percentiles of a particular outcome, boxes depict the $25^{th}$ and $75^{th}$ percentiles, and the solid horizontal line depicts the median.}
% \label{fig:CVbargdownBW}
% \end{sidewaysfigure}
%
% \begin{sidewaysfigure}
% \centering
% \includegraphics[scale=0.9]{../CVbargvertBW.png}
% \caption{The figure displays box and whisker plots summarizing the extent to which mergers among an unintegrated wholesaler and unintegrated retailer affect consumer, retailer, wholesaler, and total surplus as the bargaining power of wholesalers relative to retailers changes. The different colored boxes display how outcomes change as the number of vertically  integrated firms increases. Whiskers depict the $5^{th}$ and $95^{th}$ percentiles of a particular outcome, boxes depict the $25^{th}$ and $75^{th}$ percentiles, and the solid horizontal line depicts the median.}
% \label{fig:CVbargvertBW}
% \end{sidewaysfigure}

% \begin{sidewaysfigure}
% \centering
% \includegraphics[scale=0.85]{CVbargbothBW.png}
% \caption{The figure displays box and whisker plots summarizing the extent to which mergers among two integrated wholesalers and retailers affect consumer, retailer, wholesaler, and total surplus as the bargaining power of wholesalers relative to retailers changes. The different colored boxes display how outcomes change as the number of vertically  integrated firms increases. Whiskers depict the $5^{th}$ and $95^{th}$ percentiles of a particular outcome, boxes depict the $25^{th}$ and $75^{th}$ percentiles, and the solid horizontal line depicts the median.}
% \label{fig:CVbargbothBW}
% \end{sidewaysfigure}
