\documentclass[12pt]{article}
\usepackage{chicago}    % bibliography package
\usepackage{graphicx}   % insert PostScript figures
\usepackage{setspace}   % controls line spacing
\usepackage{amsmath,amsthm,amssymb,amstext} % controls equation entry and symbols
\usepackage{rotating}   % rotates graphics
\usepackage{soul}       % controls hyphenation
\usepackage{epsfig}     % helps with including graphics
\usepackage{pdflscape}  % helps with displaying rotated graphics in PDFs
\usepackage{lscape}     % helps with rotating pages
\usepackage{setspace}   % controls line spacing
\usepackage{caption}    % controls  captions
\usepackage{adjustbox}  % shrink tables
\usepackage[margin=1in]{geometry} % margins
\usepackage{hyperref}   % displays URLs
%packages for kable tables
 \usepackage{booktabs}
 \usepackage{longtable}
 \usepackage{array}
 \usepackage{multirow}
 \usepackage{wrapfig}
 \usepackage{float}
 \usepackage{colortbl}
 \usepackage{pdflscape}
 \usepackage{tabu}
 \usepackage{threeparttable}
 \usepackage{threeparttablex}
 \usepackage[normalem]{ulem}
 \usepackage{makecell}
 \usepackage{xcolor}

\parskip     2.0mm       % space between paragraphs

\alph{footnote}         % make title footnotes alpha-numeric

\captionsetup[figure]{labelsep=space,labelfont=bf} % remove colon from figure name

%%%%%%%%%%%%%%%%%%%%%%%%%%%%%%%%%%%%%%%%%%%%%%%%%%%%%%%%%%%%%%%%%%%%%%%%%%%
% Title Page
%%%%%%%%%%%%%%%%%%%%%%%%%%%%%%%%%%%%%%%%%%%%%%%%%%%%%%%%%%%%%%%%%%%%%%%%%%%
\title{Beyond ``Horizontal'' and ``Vertical'':\\ The Welfare Effects of Complex Integration\footnote{The analysis and conclusions set forth are those of the authors and do not indicate concurrence by other members of the Board research staff, by the Federal Reserve Board of Governors, by the Federal Trade Commission, or by its Commissioners.  Furthermore, the views expressed here should not be purported to reflect those of the U.S. Department of Justice.}}

\newcommand*\samethanks[1][\value{footnote}]{\footnotemark[#1]}
\author{Gloria Sheu\footnote{Board of Governors of the Federal Reserve System, gloria.sheu@frb.gov. } \\ Federal Reserve Board \and  Charles Taragin\footnote{Federal Trade Commission, ctaragin@ftc.gov}  \\  Federal Trade Commission \and Margaret Loudermilk\footnote{U.S. Department of Justice, margaret.loudermilk@usdoj.gov}\\  U.S Department of Justice}

\date{\today}

\begin{document}

\pagenumbering{roman}       % Roman numerals from abstract to text
\maketitle                  % print title information
\thispagestyle{empty}       % no page number on THIS page


\begin{abstract}
\noindent We use a standard vertical supply-chain model to study the welfare impacts of mergers that have both horizontal and vertical aspects.  The model features logit Bertrand competition downstream and Nash Bargaining upstream.  We numerically simulate four types of mergers: (1) vertical mergers between an unintegrated retailer and an unintegrated wholesaler, (2) mixed downstream horizontal/vertical mergers between an unintegrated retailer and a previously integrated retailer/wholesaler pair, (3) mixed upstream horizontal/vertical mergers between an unintegrated wholesaler and a previously integrated retailer/wholesaler pair, and (4) integrated mergers between two previously integrated retailer/wholesaler pairs. The breadth of options we include better captures the variety of merger structures observed in practice, compared to the typical ``horizontal vs. vertical'' dichotomy.   We further extend our analysis to accommodate preexisting vertical integration by third-party firms and linear marginal costs.
\end{abstract}

\bigbreak Keywords: bargaining models; merger simulation; vertical markets; vertical mergers

JEL classification: L13; L40; L41; L42

\newpage                    % start a new page
\pagenumbering{arabic}      % Arabic page numbers from now on
\doublespacing

%%%%%%%%%%%%%%%%%%%%%%%%%%%%%%%%%%%%%%%%%%%%%%%%%%%%%%%%%%%%%%%%%%%%%%%%%%%%%%%%%
% Body of Paper
%%%%%%%%%%%%%%%%%%%%%%%%%%%%%%%%%%%%%%%%%%%%%%%%%%%%%%%%%%%%%%%%%%%%%%%%%%%%%%%%%
\section{Introduction}
Given the many different ways a firm may be organized, it is rare that any given merger can be neatly categorized as purely ``horizontal'' or ``vertical.''  Indeed, as the 2020 U.S. Department of Justice (DOJ) and Federal Trade Commission (FTC) Vertical Merger Guidelines (henceforth, the ``VMG'') state, mergers often have both horizontal and vertical aspects.  However, the typical theories and models used to analyze mergers rarely address these complexities.  Horizontal mergers are cast in terms of whether they create significant upward pricing pressure (UPP), while the focus for vertical mergers is on assessing the net effects of the elimination of double marginalization (EDM) versus raising rivals' costs (RRC).  Yet, a given merger may combine all of these effects simultaneously, which raises the necessity of balancing their impacts in a unified framework.

In this paper, we use a standard merger simulation model to assess the welfare implications of these complex mergers.  Our model is drawn directly from \citeN{ST2020}, and uses a Bertrand logit downstream retail setup alongside a Nash Bargaining wholesale negotiation upstream.  We use this model to numerically simulate four types of mergers: (1) vertical mergers between an unintegrated retailer and an unintegrated wholesaler, (2) mixed downstream horizontal/vertical mergers between an unintegrated retailer and a previously integrated retailer/wholesaler pair, (3) mixed upstream horizontal/vertical mergers between an unintegrated wholesaler and a previously integrated retailer/wholesaler pair, and (4) integrated mergers between two previously integrated retailer/wholesaler pairs.  In each of these simulations, we calculate the impacts on consumer welfare and firm profits.  Furthermore, we examine how the presence of additional integrated rivals in the market change our results. Our baseline simulations assume that units are produced with constant marginal costs, but we later extend our analysis to linear upward sloping marginal costs.

We then provide a case study (in progress) from an industry, solid waste management, that has seen a large number of mergers with both horizontal and vertical elements.  Broadly speaking, the supply chain of this sector involves haulers at the downstream level that bring trash to disposal facilities at the upstream level.  A variety of firms participate in these services, with some owning their own hauling and disposal assets, while others are active in only a segment of the market.


%%%%%%%%%%%%%%%%%%%%%%%%%%%%%%%%%%%%%%%%%%%%%%%%%%%%%%%%%%%%%%%%%%%%%%%%%%%%%%%%%
\section{Theory}
We begin by describing our basic framework, which uses a downstream Bertrand logit model embedded in an upstream Nash Bargaining setup, taken from \citeN{ST2020}.  This framework allows us to study a variety of merger configurations.

%%%%%%%%%%%%%%%%%%%%%%%%%%%%%%%%%%%%%%%%%%
\subsection*{Downstream and Upstream Competition}
Assume there is a set of consumers indexed by $i$ who each choose to buy a product sold by a single retailer.  Retailers are indexed by $r$, while the wholesalers that supply these retailers are indexed by $w$.  Prior to any mergers taking place, each wholesaler offers only one product, although each retailer can purchase from multiple wholesalers.  We denote the set of all retailers by $\mathbb{R}=\{1, \dots, \left\vert{\mathbb{R}}\right\vert\}$, and the set of all wholesalers by $\mathbb{W}=\{1, \dots, \left\vert{\mathbb{W}}\right\vert\}$.  We divide the set $\mathbb{W}$ into overlapping subsets, each labeled $\mathbb{W}^r$, to indicate which wholesalers' products are carried by which retailers.  Similarly, we divide the set of retailers $\mathbb{R}$ into overlapping subsets, each labeled $\mathbb{R}^w$, to indicate the retailers that carry the product sold by each wholesaler.

The share of consumers that choose product $w$ sold by retailer $r$ has the logit form,
\begin{equation}
s_{rw} = \frac{\exp(\delta_{rw} - \alpha p_{rw})}{1 + \sum_{t \in \mathbb{R}} \sum_{x \in \mathbb{W}^t} \exp(\delta_{tx} - \alpha p_{tx})},
\label{eq: nb share}
\end{equation}
where $\delta_{rw}$ is a quality parameter and $\alpha$ captures sensitivity to price, denoted by $p_{rw}$.  There is an outside good whose quality parameter and price have been normalized to zero.  The retailer's profit is given by $\pi^r = \sum_{w \in \mathbb{W}^r} [p_{rw} - p^W_{rw} - c^R_{rw}] s_{rw} M$, where $p^W_{rw}$ is the unit fee charged by wholesaler $w$ to retailer $r$, $c^R_{rw}$ captures any additional marginal costs borne by the retailer, and $M$ is the market size.  Downstream prices are set in Bertrand equilibrium, according to
\begin{equation}
\sum_{x \in \mathbb{W}^r} [p_{rx} - p^W_{rx} - c^R_{rx}] \frac{\partial s_{rx}}
{\partial p_{rw}} + s_{rw} = 0,
\label{eq: nb downstream foc}
\end{equation}
which is the first order condition for product $w$ sold by retailer $r$.

Wholesale prices are set via Nash Bargaining between retailers and wholesalers.  We assume that the negotiation for a given input price treats other input prices and all downstream prices as given.\footnote{As explained in \citeN{ST2020}, this assumption is equivalent to a situation where all negotiations and choices for downstream prices happen simultaneously.  }  Profits for wholesaler $w$ are given by $\pi^w = \sum_{r \in \mathbb{R}^w} [p^W_{rw} - c^W_{rw}] s_{rw} M$, where $c^W_{rw}$ is the wholesale marginal cost when dealing with retailer $r$.  The first order condition for the negotiation between retailer $r$ and wholesaler $w$ is
\begin{equation}
\begin{split}
&\overbrace{[p^W_{rw} - c^W_{rw}]s_{rw} - \sum_{t \in \mathbb{R}^w \setminus \{r\}} [p^W_{tw} - c^W_{tw}] \Delta s_{tw}(\mathbb{W}^r \setminus \{w\})}^{\text{wholesaler GFT}} = \\
&\frac{1-\lambda}{\lambda} \left(\underbrace{[p_{rw} - p^W_{rw} - c^R_{rw}]s_{rw} - \sum_{x \in \mathbb{W}^r \setminus \{w\}} [p_{rx} - p^W_{rx} - c^R_{rx}] \Delta s_{rx}(\mathbb{W}^r \setminus \{w\})}_{\text{retailer GFT}}\right),
\end{split}
\label{eq: upstream foc}
\end{equation}
where $\lambda \in [0,1]$ captures the bargaining power of the retailer relative to the wholesaler.  The $\Delta s_{tx}(\mathbb{W}^r \setminus \{w\}) \equiv s_{tx}(\mathbb{W}^r \setminus \{w\}) - s_{tx}$ is the difference in the share of good $x$ sold by retailer $t$ when good $w$ is not offered by retailer $r$ versus when good $w$ is offered by retailer $r$.  Thus, the wholesale price $p_{rw}$ is set such that the payoff to wholesaler $w$ when it sells to retailer $r$ less the payoff when it does not (that is, the gains from trade or ``GFT''), divided by the payoff to retailer $r$ when it buys from wholesaler $w$ less the payoff when it does not, equals the ratio of wholesaler to retailer bargaining power.

Together the series of downstream and upstream first order conditions determine market equilibrium.  This model can be solved and calibrated as described in \citeN{ST2020}.  In our baseline configuration, we assume that wholesale and retail marginal costs are constant.  We loosen this restriction to allow for linear increasing marginal cost as a robustness check.

%%%%%%%%%%%%%%%%%%%%%%%%%%%%%%%%%%%%%%%%%%
\subsection*{Mergers}
Here we describe the manner in which mergers are modeled in this framework.  We begin by describing vertical mergers, as all other cases we examine have some vertical aspects.  Suppose that retailer $r$ and wholesaler $w$ were to merge.  Then the first order condition for profit maximization for the joint firm when setting the downstream price for product $w$ is given by
\begin{equation}
\begin{split}
&\sum_{x \in \mathbb{W}^r \setminus \{w\}} [p_{rx} - p^W_{rx} - c^R_{rx}]\frac{\partial s_{rx}}{\partial p_{rw}} + s_{rw} + \overbrace{[p_{rw} - c^W_{rw} - c^R_{rw}]\frac{\partial s_{rw}}{\partial p_{rw}}}^{\text{EDM effect}} \\
&+ \underbrace{\sum_{t \in \mathbb{R}^w \setminus \{r\}} [p^W_{tw} - c^W_{tw}] \frac{\partial s_{tw}}{\partial p_{rw}}}_{\text{upstream UPP effect}} = 0.
\end{split}
\label{eq: vmerger own ret downstream foc}
\end{equation}
The pricing problem balances two effects.  On the one hand, the term labeled ``EDM effect'' captures the impact of retailer $r$ being able to access product $w$ at marginal cost.  This force would tend to lower the resulting price.  On the other hand, the term labeled ``upstream UPP effect'' captures the incentive to raise prices for retailer $r$ in order to divert sales to the merging partner wholesaler $w$.  This force would tend to raise the resulting consumer price.

Turning to input prices, when wholesaler $w$ bargains with unaffiliated retailer $s$, the first order condition becomes
\begin{equation}
\begin{split}
&[p^W_{sw} - c^W_{sw}]s_{sw} - \sum_{t \in \mathbb{R}^w \setminus \{r,s\}} [p^W_{tw} - c^W_{tw}] \Delta s_{tw}(\mathbb{W}^s \setminus \{w\})\\
& - \overbrace{\overbrace{[p_{rw}-c^W_{rw}-c^R_{rw}] \Delta s_{rw}(\mathbb{W}^s \setminus \{w\})}^{\text{indirect EDM effect}} - \sum_{x \in \mathbb{W}^r \setminus \{w\}} [p_{rx} - p^W_{rx} - c^R_{rx}] \Delta s_{rx}(\mathbb{W}^s \setminus \{w\})}^{\text{RRC effect}}=\\
&\frac{1-\lambda}{\lambda} \left([p_{sw} - p^W_{sw} - c^R_{sw}]s_{sw} - \sum_{x \in \mathbb{W}^r \setminus \{w\}} [p_{sx} - p^W_{sx} - c^R_{sx}] \Delta s_{sx}(\mathbb{W}^s \setminus \{w\})\right).
\end{split}
\label{eq: vmerger wh upstream foc}
\end{equation}
which reflects the change in the disagreement payoff coming from the merger with retailer $r$.  Now when the wholesaler considers the possible loss of sales upon ceasing to trade with retailer $s$, these losses are softened due to a potential for diversion to retailer $r$, which we label the ``RRC effect.''  Furthermore, the margin on product $w$ sold by retailer $r$ is potentially higher due to EDM, as shown through the expression labeled ``indirect EDM effect,'' which can further compensate the firm.  These impacts tend to raise the resulting input price.

When the merged firm is bargaining with the unaffiliated wholesaler $v$ over what input price to pay as a retailer, the bargaining first order condition becomes
\begin{equation}
\begin{split}
&[p^W_{rv} - c^W_{rv}]s_{rv} - \sum_{t \in \mathbb{R}^v \setminus \{r\}} [p^W_{tv} - c^W_{tv}] \Delta s_{tv}(\mathbb{W}^r \setminus \{v\}) = \\
&\frac{1-\lambda}{\lambda} \left([p_{rv} - p^W_{rv} - c^R_{rv}]s_{rv} - \sum_{x \in \mathbb{W}^r \setminus \{w,v\}} [p_{rx} - p^W_{rx} - c^R_{rx}] \Delta s_{rx}(\mathbb{W}^r \setminus \{v\})\right.\\
&\left.- \underbrace{[p_{rw} - c^W_{rw} - c^R_{rw}] \Delta s_{rw}(\mathbb{W}^r \setminus \{v\})}_{\text{EDM recapture effect}}-\underbrace{\sum_{t \in \mathbb{R}^w \setminus \{r\}} [p^W_{tw} - c^W_{tw}] \Delta s_{tw}(\mathbb{W}^r \setminus \{v\})}_{\text{wholesale recapture leverage effect}}\right).
\end{split}
\label{eq: vmerger ret upstream foc}
\end{equation}
In this case, the merged firm has two channels for potential additional profits should it cease to trade with wholesaler $v$.  First, if retail sales are diverted to product $w$ sold by retailer $r$, those sales will earn a higher margin due to lower marginal costs stemming from what we call the ``EDM recapture effect.''  Second, the loss of product $v$ carried by retailer $r$ could increase sales by wholesaler $w$ through other retailers, which we call the ``wholesale recapture leverage effect.''  Both of these effects would tend to lower the resulting input price.

Next consider a merger between a preexisting integrated retailer/wholesaler $rw$ and a standalone retailer $s$.  Such a combination has a vertical component and a horizontal component.  The first order condition for setting the downstream price $p_{rw}$ becomes
\begin{equation}
\begin{split}
&\sum_{x \in \mathbb{W}^r \setminus \{w\}} [p_{rx} - p^W_{rx} - c^R_{rx}]\frac{\partial s_{rx}}{\partial p_{rw}} + s_{rw} + \overbrace{[p_{rw} - c^W_{rw} - c^R_{rw}]\frac{\partial s_{rw}}{\partial p_{rw}}}^{\text{direct EDM effect}}+ \underbrace{\sum_{t \in \mathbb{R}^w \setminus \{r\}} [p^W_{tw} - c^W_{tw}] \frac{\partial s_{tw}}{\partial p_{rw}}}_{\text{upstream UPP effect}} \\ &+\underbrace{\underbrace{[p_{sw} - c^W_{sw} - c^R_{sw}]\frac{\partial s_{sw}}{\partial p_{rw}}}_{\text{indirect EDM Effect}} + \sum_{x \in \mathbb{W}^s \setminus \{w\}} [p_{sx} - p^W_{sx} - c^R_{sx}] \frac{\partial s_{sx}} {\partial p_{rw}}}_{\text{downstream UPP effect}}= 0.
\end{split}
\label{eq: intdmerger own ret downstream foc}
\end{equation}
Now there is the possibility for what we call the ``downstream UPP effect'' in the retail market, as the merged firm can recapture sales that are diverted to retailer $s$ when $r$ raises its prices.  EDM between $w$ and $s$ can actually increase this UPP impact, because when the sales are diverted to product $w$ sold by retailer $s$, those units earn a larger margin.  This impact is what we label the ''indirect EDM effect,'' which comes through the interaction of EDM interaction with UPP.  On the other hand, the price of retailer $s$ for product $w$ could fall due to the direct impact of EDM, since that product has a lower realized marginal cost.  The net impact on consumers is ambiguous.

If instead a preexisting integrated retailer/wholesaler $rw$ were to merge with a standalone wholesaler $v$, then the resulting first order condition for product $rw$ would look similar to equation \eqref{eq: intdmerger own ret downstream foc}.  However,  the downstream UPP effect would be replaced with an additional upstream UPP effect capturing the value of sales diverted to customers of wholesaler $v$.  The condition would also include an indirect EDM component, reflecting the ability of retailer $r$ to obtain product $v$ at marginal cost, which in turn would raise the value of diverted sales to that product.  Whether these additional incentives to raise prices will dominate the direct EDM impact of the lower input cost of $v$ for retailer $r$ is an empirical question.  When considering a merger between two integrated retailer/wholesaler pairs, both upstream and downstream UPP effects would enter.

Turning to wholesale prices, again consider a merger between a preexisting integrated retailer/wholesaler $rw$ and a standalone retailer $s$.  When wholesaler $w$ bargains with an unaffiliated retailer $u$ we have the first order condition given by,
\begin{equation}
\begin{split}
&[p^W_{uw} - c^W_{uw}]s_{uw} - \sum_{t \in \mathbb{R}^w \setminus \{r,s,u\}} [p^W_{tw} - c^W_{tw}] \Delta s_{tw}(\mathbb{W}^u \setminus \{w\})\\
& - \overbrace{\sum_{t \in \{r,s\}}\left(\overbrace{[p_{tw}-c^W_{tw}-c^R_{tw}] \Delta s_{tw}(\mathbb{W}^u \setminus \{w\})}^{\text{indirect EDM effect}} - \sum_{x \in \mathbb{W}^t \setminus \{w\}} [p_{tx} - p^W_{tx} - c^R_{tx}] \Delta s_{tx}(\mathbb{W}^u \setminus \{w\})\right)}^{\text{RRC effect}}=\\
&\frac{1-\lambda}{\lambda} \left([p_{uw} - p^W_{uw} - c^R_{uw}]s_{uw} - \sum_{x \in \mathbb{W}^u \setminus \{w\}} [p_{ux} - p^W_{ux} - c^R_{ux}] \Delta s_{ux}(\mathbb{W}^u \setminus \{w\})\right).
\end{split}
\label{eq: intdmerger wh upstream foc}
\end{equation}
The RRC effect is augmented with the profits emanating from the sales of retailer $s$, in addition to the sales of retailer $r$, both of which may potentially recapture sales should retailer $u$ lose access to product $w$.  The merged firm has higher bargaining leverage as a result.  If instead the integrated retailer/wholesaler $rw$ were to merge with a standalone wholesaler $v$, then the profits earned by firm $rw$ should the negotiation with retailer $u$ fail are augmented with the earnings of wholesaler $v$ rather than retailer $s$.  This adds a term similar to the wholesale recapture leverage effect seen in equation \eqref{eq: vmerger ret upstream foc} to the left-hand side of the bargaining first order condition.\footnote{For simplicity, we assume that when a retailer fails to reach an agreement with wholesaler $w$, that retailer's contract with wholesaler $v$ remains in place.  This assumption can be loosened.  }    In the case of a merger between two integrated retailer/wholesaler firms, all of these effects would appear.

Returning to a merger between a preexisting integrated retailer/wholesaler $rw$ and a standalone retailer $s$, when the merged firm bargains with an unaffiliated wholesaler $v$ to supply retailer $r$, the first order condition becomes
\begin{equation}
\begin{split}
&[p^W_{rv} - c^W_{rv}]s_{rv} - \sum_{t \in \mathbb{R}^v \setminus \{r\}} [p^W_{tv} - c^W_{tv}] \Delta s_{tv}(\mathbb{W}^r \setminus \{v\}) = \\
&\frac{1-\lambda}{\lambda} \left([p_{rv} - p^W_{rv} - c^R_{rv}]s_{rv} - \sum_{x \in \mathbb{W}^r \setminus \{w,v\}} [p_{rx} - p^W_{rx} - c^R_{rx}] \Delta s_{rx}(\mathbb{W}^r \setminus \{v\})\right.\\
&- \underbrace{\sum_{t \in \{r,s\}}[p_{tw} - c^W_{tw} - c^R_{tw}] \Delta s_{tw}(\mathbb{W}^r \setminus \{v\})}_{\text{EDM recapture effect}}
- \underbrace{\sum_{x \in \mathbb{W}^s \setminus \{w\}} [p_{sx} - p^W_{sx} - c^R_{sx}] \Delta s_{sx}(\mathbb{W}^r \setminus \{v\})}_{\text{retail recapture leverage effect}}\\
&-\underbrace{\left.\sum_{t \in \mathbb{R}^w \setminus \{r,s\}} [p^W_{tw} - c^W_{tw}] \Delta s_{tw}(\mathbb{W}^r \setminus \{v\})\right)}_{\text{wholesale recapture leverage effect}}.
\end{split}
\label{eq: intdmerger ret upstream foc}
\end{equation}
Compared to equation \eqref{eq: vmerger ret upstream foc}, now the EDM recapture effect applies to sales of product $w$ at both affiliated retailers $r$ and $s$.  Furthermore, there is also the possibility that sales will be diverted to retailer $s$ should retailer $r$ lose access to product $v$, which creates an additional ``retail recapture leverage effect'' alongside the wholesale recapture effect.  Both of these additional terms tend to increase the bargaining leverage of the merged firm.  If we instead examined a merger between retailer/wholesaler $rw$ and an unaffiliated wholesaler, that would augment the EDM recapture effect with sales of retailer $r$ for two merged wholesalers, rather than sales through two merged retailers.  Similarly, the retail recapture leverage effect would be replaced with an additional wholesale recapture leverage effect for the newly merged wholesaler.  For a merger between two integrated retailer/wholesaler firms, all of these effects would enter.
%%%%%%%%%%%%%%%%%%%%%%%%%%%%%%%%%%%%%%%%%%
\subsection*{Data Generating Process}
This section provides an overview of our methodology, with additional details appearing in the Appendix.  For vertical mergers, we only consider mergers between an unintegrated wholesaler and unintegrated retailer. In addition, we consider upstream and downstream mergers between either two unintegrated firms or between one vertically integrated and one unintegrated firm.  Finally, we assume that an integrated merger occurs between two vertically integrated firms\textbf{consider adding diagonal mergers by allowing for heterogeneity in the bargaining parameter.}.

We simulate markets by randomly sampling shares from a Dirichlet distribution for 2, 3, 4, or 5 retailers or wholesalers, respectively.\footnote{We parameterize the Dirichlet distribution so it is equivalent to a uniform distribution.  }  Because integrated mergers must have two vertically integrated incumbents in the pre-merger state, we increase the maximum allowable number of wholesalers and retailers in our simulated markets to 7. We also assume that in the pre-merger state, there are anywhere from 0 to 4 vertically integrated incumbents (6 for integrated mergers). We assume that vertically integrated firms are not siloed: integrated wholesalers supply inputs to non-integrated retailers and that vertically integrated retailers purchase inputs from non-integrated wholesalers. The price coefficient $\alpha$ is calibrated by assuming that in the pre-merger world, there is a vertically integrated outside option available to all customers.  The other goods are differenced relative to this option, which maintains the outside good normalization.  The market size is set to 1.

We specify values for the bargaining parameter ranging from from 0.1 (wholesalers have the advantage) to 0.9 (retailers have the advantage). To better understand the relative bargaining strength of these parameter values, we report our results in terms of $(1-\lambda)/\lambda$, which range from $9$ (wholesaler power is nine times greater than retailer power) to $1/9$ (retailer power is nine times greater than wholesaler power).  The bargaining parameter is identical for all of the retailers in each simulation, unless noted otherwise.

For each combination of number of retailers, number of wholesalers, number of incumbent integrated firms, and bargaining parameter, we draw 1,000 different sets of market primitives.  This results in 1.96 million merger simulations.  We then eliminate mergers where the merger is unprofitable to the merging firms, as well as markets that do not pass the Hypothetical Monopolist Test, yielding 1.37 million markets.\footnote{The Hypothetical Monopolist Test requires that were a monopolist to jointly own all products in a candidate market, that firm would raise the price of at least one of the merging producers' products by at least a ``small but significant non-transitory increase in price'' (SSNIP), which we take to be 5\%. } All 1.37 million markets treat as primitives the number of retailers, the number of wholesalers, the bargaining parameter, and the wholesaler and retailer marginal costs, which we allow to be either constant or linear.

When simulating a horizontal merger, we assign the products produced by the two largest firms in the market to a single entity post-merger. Similarly, when simulating a vertical merger, we assign the products produced by the largest wholesaler and the largest retailer to a single entity post-merger. This assignment is purposefully skewed towards mergers that are more likely to have competitive effects and to come under agency review.

Table \ref{tab:simsum} provides summary statistics across our various simulations.\footnote{The \href{https://CRAN.R-project.org/package=antitrust}{\texttt{antitrust} R package} contains the computer code needed to calibrate and simulate the effects of mergers in a range of competitive scenarios, including the ones described here.}  The median average wholesale pre-merger price is \$4.6, and the median average retail pre-merger price is \$14.  Because the market size is set to 1, these average prices are equal to total pre-merger expenditures.  Pre-merger HHIs range between 2,814 at the $25^{th}$ percentile to 5,030 at the $75^{th}$, with a median of 3,703. HHIs for vertical mergers increase by 1,277 points at the median, resulting in a median post-merger HHI equal to 4,889.\footnote{We compute the post-merger HHI for vertical mergers by calculating the merged firms' market share as the sum of all the shares of downstream products that either incorporate the upstream partner's input or are sold by the downstream partner.  }  HHIs for upstream mergers increase by 2,004 points at the median, resulting in a median post-merger HHI equal to 5,231. HHIs for horizontal downstream mergers increase by 2,054 points at the median, resulting in a median post-merger HHI equal to 4,889.   HHIs for integrated mergers increase by 2,746 points at the median, resulting in a median post-merger HHI equal to 7,220. Many of these markets fall into the span designated by the DOJ/FTC Horizontal Merger Guidelines as ``Highly Concentrated Markets,'' with post-merger HHIs over 2,500 points and HHI Changes greater than 200 points.\footnote{HHI thresholds are discussed in the 2010 Horizontal Merger Guidelines, Section 5.3.  }

%%%%%%%%%%%%%%%%%%%%%%%%%%%%%%%%%%%%%%%%%%
\subsection*{Results Overview}
Our overall results are depicted in Figure \ref{fig:surplussum}, which  is divided into four panels, each showing how the distribution of surplus changes for a particular set of agents (consumers, retailers, or wholesalers), as well as the net effect on the market as a whole.  Surplus is presented as a percentage change relative to total pre-merger expenditure in the downstream market.

Each panel contains four box and whisker plots, with each plot corresponding to a different type of merger. The whiskers display the $5^{th}$ and $95^{th}$ percentiles of the outcome distribution, the boxes denote the $25^{th}$ and $75^{th}$ percentiles, and the solid horizontal line marks the median. Note that negative outcome values imply agent harm, and positive values imply agent benefits.

We focus first on the results for consumers in the left-most panel of Figure \ref{fig:surplussum}.  The median change is negative, indicating harm, across all four types of mergers for both cost specifications.  However, the distributions and magnitudes differ.  In particular, there is only a partial rank-ordering of consumer harm across different types of mergers: consumer harm from integrated mergers first-order stochastically dominates consumer harm from all other merger  types, while downstream and upstream mergers each first-order stochastically dominate consumer harm from vertical mergers, but do not stochastically dominate one other.   Median consumer harm from integrated mergers is almost 14\% of pre-merger total expenditures, 1.25 times the magnitude of that from downstream mergers, 2.3 times the magnitude of that from upstream mergers, and more than 4 times the magnitude of vertical mergers.


%\textbf{Is it worth exploring which effect dominates: the lessening of EDM or the strengthening of RRC? Maybe do this by allowing assymetric cost structures between merging and non-merging parties?}

%However, the fact that downstream mergers under Bertrand are almost never beneficial and frequently quite harmful is somewhat surprising, given the potential for countervailing bargaining leverage to lower wholesale prices.



Turning to retailers in the second panel of Figure \ref{fig:surplussum}, we find that while downstream mergers, vertical mergers and integrated mergers always benefit retailers, upstream mergers  harm retailers in about 67\% of all simulations. Moreover, there is a partial rank-ordering across mergers that is distinct from the consumer rank-ordering: the retailer surplus distribution from integrated mergers first-order stochastically dominates the retailer surplus distribution from downstream mergers, which dominates the retailer surplus distribution from upstream mergers, but not the retailer surplus distribution from vertical mergers.

%We also find that  for upstream, vertical, and integrated mergers, retailer surplus under constant marginal cost first-order stochastically dominates the retail surplus distribution under linear costs, again suggesting that the incentive to raise rivals' costs dominates the benefits from EDM. By contrast, for downstream mergers, there is no clear rank-ordering between markets with constant marginal costs and markets with linear costs.

% Finally, it is interesting that the harm experienced in the latter instances is quite small in relative terms: median retailer harm from upstream mergers is about one-quarter of the median consumer harm from an upstream merger. This, along with the narrow inter-quartile range of retailer harm (less than 1\% of pre-merger revenues) indicates that retailers are passing on the bulk of the wholesale price increase to consumers.\footnote{This phenomena may also be seen by noting that the distribution of wholesaler gain from upstream mergers is similar in magnitude to the distribution of consumer losses from upstream mergers, though less skewed.  }

As for wholesaler surplus, which appears in the third panel, the effects seen there are largely the reflection of those for retailers: wholesaler surplus increases in about 77\% of all upstream simulated mergers, 23\% of simulated vertical mergers, 20\% of simulated downstream mergers, and about 15\% of simulated integrated mergers. %\textbf{comparing these results to \citeN{ST2020}, the effects are less stark (i.e. less of a transfer.}
However, only wholesaler surplus from upstream mergers first-order stochastically dominates the wholesaler surplus of the other merger types.

%Whereas wholesalers benefit from vertical mergers in about 17\% of simulations, retailers benefit from vertical mergers in about 99\% of simulations. Moreover, the median benefit to retailers is 7.5\% of pre-merger expenditures, comparable to the median loss to wholesalers of -9\%. Together, these observations indicate that vertical mergers are largely a rent transfer from wholesalers to retailers and consumers, with retailers keeping most of the gain.

In terms of total welfare, approximately 18\% of vertical mergers and 15\% of integrated mergers with constant costs that are beneficial, whereas only 3\% of upstream mergers and 2\% of downstream mergers are beneficial. Moreover, there is a partial rank ordering of mergers, with total harm from downstream mergers first-order stochastically dominating total harm from upstream mergers, which dominates totals harm from vertical mergers. Median consumer harm from integrated mergers is about 9.6\% of pre-merger total expenditures, 1.5 times the magnitude of that from downstream mergers, 2.5 times the magnitude of that from upstream mergers, and more than 4.3 times the magnitude of vertical mergers.

%Finally, for each merger type, total harm under linear costs first-order stochastically dominates total harm under constant marginal costs, with the greatest differences occurring for vertical and integrated mergers.

%%%%%%%%%%%%%%%%%%%%%%%%%%%%%%%%%%%%%%%%%%
\subsection*{Vertically Integrated Incumbent Firms}
Here, we examine merger outcomes when a vertically integrated merges with either an unintegrated upstream or unintegrated downstream firm, or another integrated firm. We also examine merger outcomes for  mergers as the number of incumbent integrated non-merging rivals increases.

Figure \ref{fig:CVvertincumbBW_updownBW} depicts box and whisker plots summarizing consumer harm (top) and total harm (bottom) as the number of incumbent integrated firms increases from 0 to 6 firms. The box and whisker  plots when the number of  incumbent integrated firms equals 0 correspond to the results depicted in Figure 1 of \citeN{ST2020} and is included as reference.\footnote{An important difference between the simulations depicted in Figure \ref{fig:CVvertincumbBW} and those in Figure 1 of \citeN{ST2020} is that these results exclude downstream markets where prices are set according to a 2nd score auction}  For vertical mergers, the plots when the number of  incumbent integrated firms equals 1 summarize market outcomes where an unintegrated wholesaler merging with an unintegrated retailer, a single unintegrated 3rd party is integrated, and the remaining 3rd parties (if any) are unintegrated. By contrast,for downstream and upstream  mergers, the plots when the number of  incumbent integrated firms equals 1 depict the outcome of a merger between an integrated firm and an unintegrated rival, with no integrated third parties. Finally, for integrated mergers, the plots when the number of  incumbent integrated firms equals 2 depict a merger between two integrated firms, again with no integrated rivals.

Starting with vertical mergers, absent incumbent integration (0 integrated firms), about 36\% of vertical mergers benefit consumers and about 23\% of vertical mergers are net beneficial (i.e. increase total surplus).  The presence of a single 3rd party integrated firm (1 integrated firm) decreases the percentage of mergers that benefit consumers to 25\% and the percentage of mergers that increase  total surplus to 16\%, but leads median harm largely unchanged. Interestingly, while adding a second integrated incumbent to the market also lowers the percentage of beneficial mergers, adding a third integrated firm increases the percentage of mergers that benefits consumers to about 27\% but does not increase the percentage of net beneficial mergers. Adding a fourth integrated incumbent increases the percentage of mergers that benefit consumers to 32\% but decreases the percentage of net beneficial mergers to about 11\%. Taken together, Figure \ref{fig:CVvertincumbBW_updownBW} indicates that the presence of incumbent integrated firms is harmful, and that the harm does not meaningfully lessen as additional integrated incumbents are included.

By contrast, not only are upstream mergers absent incumbent vertical integration rarely beneficial, but median consumer harm  from upstream mergers is about 6\% of pre-merger revenues  and total harm is about 3\% of pre-merger revenues. A merger between an incumbent integrated firm and an unintegrated retailer when all other market participants are unintegrated (1 integrated firm)  increases the proportion of beneficial mergers but does little to change median harm: consumer welfare increases in almost 16\% of mergers and total welfare increases in 4.3\% of mergers, but median consumer harm increases to 6.3\% and median total harm increases to 4.8\% of pre-merger revenues. Increasing the number of integrated 3rd parties to 1 (2 integrated firms) disproportionately shrinks the inter-quartile range-- relatively more harmful mergers are eliminated than beneficial ones-- but the presence of a single integrated 3rd party has little impact on median harm. Adding additional integrated 3rd parties (3,4 integrated firms) further decreases the range of outcomes while leaving median harm largely unchanged. In summary, additional incumbent integrated firms do not substantially impact median harm, but do substantially reduce the likelihood of observing extremely harmful mergers.

Like upstream mergers, downstream mergers absent incumbent vertical integration are rarely beneficial, with median consumer harm from downstream mergers equal to about 9\% of pre-merger revenues while total harm is about 5\% of pre-merger revenues. A merger between an incumbent integrated firm and an unintegrated retailer when all other market participants are unintegrated (1 integrated firm)  increases the range of outcomes while also increasing median harm: consumer welfare increases in about 18\% of mergers and total welfare increases in 5.7\%, but median consumer harm increases to 14\% and median total harm increases to 7.3\% of pre-merger revenues. Increasing the number of integrated 3rd parties to 1 (2 integrated firms) shrinks the inter-quartile range but has little impact on median harm. Adding additional integrated 3rd parties (3 integrated firms) further decreases the range of outcomes while also decreasing median harm t 5.6\%. With 3 integrated 3rd parties (4 integrated firms), a downstream merger between an integrated and unintegrated firms benefits consumers in about 27\% of markets and shrinks median consumer harm to 6.4\% of pre-merger revenues. By contrast, with 3 integrated 3rd parties, the inter-quartile range of total harm shrinks, and median total harm falls to about 3.8\% of pre-merger revenues. These results indicate that unlike either vertical upstream mergers, integrated downstream mergers are more harmful than unintegrated downstream mergers, and that the presence of additional integrated rival incumbents reduces harm, though slowly.


Finally, absent the presence of rival incumbent integrated firms, mergers between two integrated firms benefits consumers in about 7\% of simulated markets and is net beneficial in 15\% of simulated markets. While adding one additional rival incumbent integrated firm (3 integrated firms) has little effect on either the distribution of either outcome, adding a second integrated rival (4 integrated firms) eliminates the most harmful integrated mergers while also increasing median consumer harm to 14\%  and median total harm to 10.6\%. The presence of additional integrated incumbents has little incremental effect on  median harm.

%%%%%%%%%%%%%%%%%%%%%%%%%%%%%%%%%%%%%%%%%%
\subsection*{Bargaining Power}
\citeN{ST2020} show using numerical simulation that downstream and vertical mergers where wholesalers have relatively more bargaining power are less harmful than mergers where retailers have relatively more bargaining power. Here, we show that this result continues to hold when there are rival incumbent integrated firms. We also show that this relationship holds for integrated mergers as well as downstream and upstream mergers when one of the merging parties is vertically integrated.

%Figures \ref{fig:CVbargupBW}-\ref{fig:CVbargbothBW}  depicts box and whisker plots summarizing the consumer, retailer, wholesaler and total welfare effects for each merger as the number of incumbent integrated firms increases. For upstream, downstream and vertical mergers, the figures depict box and whisker plots for 0,1 and 4 incumbent vertically integrated firms.  For downstream and upstream  mergers (Figures \ref{fig:CVbargupBW} and \ref{fig:CVbargdownBW}), the plots when the number of  incumbent integrated firms equals 1 depict the outcome of a merger between an integrated firm and an unintegrated rival, and all 3rd parties are unintegrated. By contrast, for vertical mergers ((Figures \ref{fig:CVbargvertBW}), the plots when the number of  incumbent integrated firms equals 1 depict the outcome from an unintegrated wholesaler merging with an unintegrated retailer when a single 3rd party is integrated. Finally, Figure \ref{fig:CVbargbothBW} depicts box and whisker plots for integrated mergers with 2,4, or 6 integrated firms.  When the number of  incumbent integrated firms equals 2, Figure \ref{fig:CVbargbothBW} depicts a merger between two integrated firms, with all rivals are unintegrated. Plots with either 4 or 6 integrated firms depict mergers between integrated incumbents when there are either 2 or 4 rival integrated firms in the market.

Figure \ref{fig:CVbargnobothBW}  depicts box and whisker plots summarizing the consumer (top) and total (bottom) welfare effects for downstream, upstream, and vertical mergers as the relative bargaining power parameter increases from 1/9 (retailers have the advantage to 9 (wholesalers have the advantage). also depicted are three sets of box and whisker plots that correspond to the number of incumbent integrated firms included in the simulated markets: 0 (light blue), 1 (darker blue), and 4 (darkest blue)incumbent vertically integrated firms.  As in Figure \ref{fig:CVvertincumbBW_updownBW}, plots when the number of integrated firms is 0 assume that pre-merger, no firms in the market are vertically integrated: these are comparable to those in \citeN{ST2020}. For vertical mergers (left), the plots when the number of  incumbent integrated firms equals 1 depict the outcome from an unintegrated wholesaler merging with an unintegrated retailer when a single 3rd party is integrated. By contrast, for upstream (middle) and downstream (left) mergers, the plots when the number of  incumbent integrated firms equals 1 depict the outcome of a merger between an integrated firm and an unintegrated rival, and all 3rd parties are unintegrated. Finally, plots with 4 integrated firms increases the number of rival incumbent integrated firms in upstream and downstream mergers to 3 and in vertical mergers to 4 .


For vertical mergers, the presence of integrated incumbents has little impact on the strong, positive relationship between bargaining power and harm for vertical mergers. Figure \ref{fig:CVbargnobothBW} shows that consumer harm when there are fewer incumbent integrated firms first-order stochastically dominates consumer harm when there are more incumbent integrated firms, the difference is smallest when wholesalers and retailers have equal bargaining power and greatest as wholesaler power increases relative to retailer power. Across all three incumbent integrated firm scenarios, median consumer harm decreases from about 7\% when retailers have the advantage to -24\% when wholesalers have the advantage. Similarly, median net harm decreases from about 2.9\% of pre-merger expenditures when retailers have relatively more bargaining power to about -14\% of pre-merger expenditures when wholesalers have relatively more bargaining power.

Turning to upstream mergers, when no incumbent firms are in the market (0 integrated firms), there is a positive relationship between bargaining power and both consumer and total harm.  Median consumer harm increases from 2.2\% of pre-merger expenditures when retailers have relatively more bargaining power to about 29\% of pre-merger expenditures when wholesalers have relatively more bargaining power. Likewise, median total harm increases from 0.7\% of pre-merger expenditures when retailers have relatively more bargaining power to about 12\% when wholesalers have relatively more bargaining power. By contrast, for mergers between an integrated and unintegrated upstream supplier (1 integrated firms), there is a positive relationship between bargaining power and both consumer and total surplus. Median consumer harm decreases from about 8.4\% of pre-merger expenditures when retailers have relatively more bargaining power to about -24\% of pre-merger expenditures when wholesalers have relatively more bargaining power. Likewise, median total harm increases from about 3.5\% of pre-merger expenditures when retailers have relatively more bargaining power to -8.4\% when wholesalers have relatively more bargaining power. Adding 2 additional incumbent integrated firms (4 integrated firms) mitigates some of the adverse effects of mergers while still preserving the positive relationship between bargaining power and harm.

For downstream mergers, when no incumbent firms are in the market (0 integrated firms), there is a positive relationship between bargaining power and both consumer and total harm. Median consumer harm decreases from about 18\% of pre-merger expenditures when retailers have relatively more bargaining power to about 1\% of pre-merger expenditures when wholesalers have relatively more bargaining power. Likewise, median total harm decreases from 7.6\% of pre-merger expenditures when retailers have relatively more bargaining power to about 2\% when wholesalers have relatively more bargaining power. By contrast, mergers between an integrated and unintegrated upstream supplier (1 integrated firms), causes the box and whisker plots to rotate clockwise around equal bargaining power, strengthening the positive relationship between bargaining power and both consumer and total surplus.  Increasing wholesaler bargaining power causes median consumer harm to fall from 25\% of pre-merger revenues to -33\% of pre-merger revenues, while median total harm falls from 12.2\% to about 1\% of pre-merger revenues. Adding additional rival incumbents (4 integrated firms) has the reverse effect, rotating the box and whisker plots counter-clockwise and weakening the still positive relationship between bargaining power and surplus. Here, increasing wholesaler bargaining power causes median consumer harm to fall from 14\% of pre-merger revenues to -33\% of pre-merger revenues, while median total harm falls from 5.5\% to 3.99\% of pre-merger revenues.


Finally, Figure \ref{fig:CVbargbothBW} depicts box and whisker plots summarizing consumer, retailer, wholesaler and total surplus  for integrated mergers with 2,4, or 6 integrated firms.  When the number of  incumbent integrated firms equals 2, Figure \ref{fig:CVbargbothBW} depicts a merger between two integrated firms: all rivals are unintegrated. Plots with either 4 or 6 integrated firms depict mergers between integrated incumbents when there are either 2 or 4 rival integrated firms in the market.

In terms of the relationship between bargaining power and harm, integrated mergers are perhaps most similar to vertical mergers. Like vertical mergers, there is a strong monotonic relationship between bargaining power and harm. Median consumer harm increases from about 26\% of pre-merger expenditures when retailers have relatively more bargaining power to about 3.75\% of pre-merger expenditures when wholesalers have relatively more bargaining power. Likewise, median total harm increases from 16\% of pre-merger expenditures when retailers have relatively more bargaining power to -14.5\% when wholesalers have relatively more bargaining power.

A second similarity between vertical and integrated mergers is that the presence of integrated incumbents has little impact on the strong, monotonic relationship between  bargaining power and harm. However, unlike vertical mergers, consumer harm from integrated mergers when there are fewer incumbent integrated firms does not first-order stochastically dominate consumer harm when there are more incumbent integrated firms.

An important difference between vertical and integrated mergers is that while vertical mergers are often beneficial when wholesalers have relatively more bargaining power, integrated mergers are often harmful unless wholesaler relative bargaining power exceeds 7/3. In this way, integrated mergers are more similar to downstream mergers, or upstream mergers in markets with vertically integrated incumbents.

\textbf{Describe the counterintuitive result that increasing retailer bargaining power harms retailers but help wholesalers.}

\subsection*{Costs}

Here, we explore the role that the constant marginal cost assumption plays in driving merger outcomes. In theory, constant marginal costs can increase the magnitude of EDM, as the wholesaler price reduction applies equally to all units produced by the retailer. By contrast, if either the integrated wholesaler or retailer face linear marginal costs, increased output from the EDM is limited by the increased marginal costs, which in turn limits the benefit to the parties. Likewise, in theory, RRC could be more profitable for the merging parties under linear than constant marginal costs, as linear marginal costs cause the integrated wholesaler's prices to increase more rapidly.

Because EDM is plausibly greater under constant marginal costs while RRC is also plausibly less, then one might expect that on net mergers are more beneficial under constant than linear marginal costs. We explore this hypothesis in Figure \ref{fig:surplussumcost}, which displays box and whisker plots that depict the net effect of these different marginal cost specifications on consumer and total welfare under four different scenarios:
constant marginal costs for all firms (top row, blue), linear marginal costs for all firms (top row, orange), constant marginal costs for the merging parties' products but linear costs for the non-merging parties products (bottom row, blue), and linear marginal costs for the merging parties' products but constant costs for the non-merging parties products (bottom row, orange).

Overall, harm is less prevalent for mergers where firms have constant marginal costs than where firms have linear costs. Mergers with constant-cost firms increases consumer surplus  in about 15\% of all mergers and total surplus in 10.5\% of all mergers, while mergers with linear-cost firms increase consumer surplus in only 2.3\% of mergers and total surplus in 0.8\% of all mergers. However, while the consumer and total harm distributions for each merger type under the all-firm linear marginal costs scenario first-order stochastically dominates their counterparts under the all-firm constant marginal costs scenario, the consumer and total harm distributions under the party linear marginal costs scenario only first-order stochastically dominates the their counterparts under the party constant marginal costs scenario for vertical and integrated mergers. The median consumer harm from markets with linear marginal costs is about 9.3\% of pre-merger total expenditures, about 1.3 times the magnitude of markets with constant marginal costs.


Changing the cost structure consistently yields the greatest differences in outcomes for vertical mergers. When all firms have constant marginal costs, vertical mergers increase consumer surplus in about 30\% of all vertical merger simulations and total surplus in about 18\% of all vertical merger simulations, but only increase consumer surplus in about 1\% of vertical merger simulations and total surplus in about 0.9\% of vertical merger simulations when all firms have linear costs. Moreover, qualitatively similar results hold for vertical mergers when only the merging parties' products have constant marginal costs, compared to when only the merging parties' products have linear costs.

The differences between the constant and linear cost structures is more muted for both upstream and downstream mergers. Upstream mergers with all constant marginal cost firms benefit consumers in about 12\% of simulations, but only 0.2\% of simulations when firms have linear costs. Downstream mergers with all constant marginal cost firms benefit consumers in about 11\% of simulations, but only about 6\% of simulations when all firms have linear costs. Again, qualitatively similar results hold for vertical mergers when only the merging parties' products have constant marginal costs, compared to when only the merging parties' products have linear costs.


Finally, while there is a marked difference in outcomes between the constant and linear cost structures for integrated  mergers under the all-firms scenario, the differences in outcomes are more muted under the party scenario, especially for consumer surplus. In particular, median consumer harm when all firms face constant marginal costs is 13.6\%, compared to the median consumer harm of 17.5\% when all firms face linear marginal costs. By contrast, median consumer harm when only the merging parties' have constant costs is 15.1\%, compared to the median consumer harm of 16.5\% when the parties' have linear costs.  %Integrated mergers benefit consumers in about 10\% of simulations when all firms have constant costs, but less than 1\% of simulations when all firms have linear costs.

%%%%%%%%%%%%%%%%%%%%%%%%%%%%%%%%%%%%%%%%%%
\section*{Empirical Setting}

The U.S. waste and recycling industry generates approximately \$80 billion in annual revenues. In recent years the industry has experienced significant merger activity including several large acquisitions between vertically integrated, national competitors. However, solid waste companies tend not to be homogeneous in their degree of vertical integration across geographies. As a result, each of these mergers exhibits a variety of vertical supply chain configurations both pre- and post-merger across their relevant local markets, making it an excellent application for further study of the welfare impacts of mergers that have both horizontal and vertical aspects.

The vertical supply chain in the solid waste industry is primarily comprised of waste collection operations or ``haulers'' and waste disposal facilities. Haulers collect municipal solid waste (MSW) from businesses and residences and must dispose of it at a lawful disposal site, predominantly landfills. Waste disposal (upstream) is a required input into waste collection services (downstream). Some haulers are vertically integrated and operate their own disposal facilities. Vertically-integrated haulers typically prefer to dispose of waste at their own disposal facilities and may also sell a portion of their disposal capacity. Disposal customers include private waste haulers without their own disposal assets (“independent haulers”) as well as local governments that collect their citizens’ waste themselves. Due to strict laws and regulations that govern the disposal of MSW, there are no reasonable substitutes for MSW disposal. Thus, mergers that combine hauling and disposal assets may incentivize the merged entity to raise its hauling rivals’ cost of disposal in order to benefit its own collection operations. Whether or not the merged entity is both able and incentivized to undertake such action depends upon the extent of its market power in the local disposal market, the merging parties’ profit margins in each line of business and their intensity of hauling competition with prospective disposal customers.

In 2020, Waste Management, the largest waste management company in the U.S., acquired Advanced Disposal Services (ADS), previously the fourth largest company, for \$4.6 billion. GFL Environmental also acquired WCA Waste Corporation for \$1.2 billion in 2020. Republic Services, the second largest waste management company in the U.S., obtained Department of Justice (DOJ) approval to acquire Santek Environmental in March 2021. These three merged companies along with Waste Connections, the third largest waste management company, are estimated to control over 60\% of available landfill capacity nationally and also rank among the top haulers nationwide.\footnote{\tiny{Waste Business Journal, \url{https://www.wastedive.com/news/public-companies-increased-control-of-74b-us-waste-industry-in-2018/556079/ }}} Concentration in local markets varies substantially, however, in both the upstream and downstream markets.

Concerns about vertical competitive effects were raised in both the Republic-Santek and the Waste Management-ADS transactions. Both the Solid Waste Agency of Lake County, IL and the Solid Waste Agency of Northern Cook County submitted comments in opposition to the proposed asset divestiture from Waste Management-ADS, stating that a vertically integrated competitor was needed to maintain competition in their local market post-merger.\footnote{\tiny{See  Comments by SWALCO and SWANCC : U.S. and Plaintiff States v. Waste Management, Inc., and Advanced Disposal Services, Inc., \url{https://www.justice.gov/atr/case-document/file/1377646/download}.}} In the DOJ complaint filed in the Republic-Santek case, vertical competitive effects were alleged to arise from the combination of their integrated assets in the Chattanooga area.\footnote{\tiny{See U.S. and State of Alabama v. Republic Services, Inc. and Santek Waste Services, LLC, \url{https://www.justice.gov/atr/case-document/file/1382031/download}}}

Our application in progress further identifies local markets in which these acquisitions result in: 1) only horizontal combinations of assets, 2) only vertical combinations of assets, and 3) combinations of vertically integrated assets - in the presence of differing numbers of existing integrated competitors to analyze the welfare effects from mergers with complex vertical arrangements.

\section{Conclusion}

This paper relaxes two strong assumptions made in \citeN{ST2020}: that mergers occur in vertical supply chains where in the pre-merger state, no firms are vertically integrated, and that all firms employ constant marginal cost technology. Relaxing these constraints is important as competition authorities are called upon to investigate mergers where either the merging parties or a third party are already vertically integrated, or where the presence of a scarce input could reduce the benefit from EDM or increase the harm from RRC. Here, we show using both numerical simulations that relaxing these assumptions can result in merger outcomes that are markedly different from those when these assumptions are maintained. We also show using simulation that relative bargaining power can serve as a useful indicia for predicting consumer and total harm from mergers where these two assumptions are relaxed. Finally, we intend to use two recent trash mergers to investigate the importance of these assumptions in predicting merger harm. Our hope is that future research will focus on some of the other strong assumptions made in \citeN{ST2020}, such as how the presence of either two-part tariffs or large fixed costs affect the Nash bargaining game and therefore merger outcomes.

%%%%%%%%%%%%%%%%%%%%%%%%%%%%%%%%%%%%%%%%%%%%%%%%%%%%%%%%%%%%%%%%%%%%%%%%%%%%%%%%%
% Bibliography
%%%%%%%%%%%%%%%%%%%%%%%%%%%%%%%%%%%%%%%%%%%%%%%%%%%%%%%%%%%%%%%%%%%%%%%%%%%%%%%%%
\newpage
\bibliographystyle{chicago}
\bibliography{master-bib}
\clearpage

%%%%%%%%%%%%%%%%%%%%%%%%%%%%%%%%%%%%%%%%%%%%%%%%%%%%%%%%%%%%%%%%%%%%%%%%%%%%%%%%%
% Tables and Figures
%%%%%%%%%%%%%%%%%%%%%%%%%%%%%%%%%%%%%%%%%%%%%%%%%%%%%%%%%%%%%%%%%%%%%%%%%%%%%%%%%

\begin{table}

\begin{tabular}{l|l|l|r}
\hline
variable & merger & quant & val\\
\hline
down &  & Min & 2\\

up &  & Min & 2\\

vert &  & Min & 0\\

barg &  & Min & 0\\

nestParm &  & Min & 0\\

avgpricepre.up &  & Min & 1\\

avgpricepre.down &  & Min & 6\\

mktElast & \multirow{-8}{*}{\raggedright\arraybackslash all} & Min & -60\\
\cline{1-4}
hhipre &  & Min & 2008\\

hhipost &  & Min & 2915\\

hhidelta & \multirow{-3}{*}{\raggedright\arraybackslash up} & Min & 0\\
\cline{1-4}
hhipre &  & Min & 2011\\

hhipost &  & Min & 2931\\

hhidelta & \multirow{-3}{*}{\raggedright\arraybackslash down} & Min & 0\\
\cline{1-4}
hhipre &  & Min & 2100\\

hhipost &  & Min & 3120\\

hhidelta & \multirow{-3}{*}{\raggedright\arraybackslash vertical} & Min & 32\\
\cline{1-4}
hhipre &  & Min & 2205\\

hhipost &  & Min & 3633\\

hhidelta & \multirow{-3}{*}{\raggedright\arraybackslash both} & Min & 2\\
\cline{1-4}
down &  & p25 & 3\\

up &  & p25 & 3\\

vert &  & p25 & 0\\

barg &  & p25 & 0\\

nestParm &  & p25 & 0\\

avgpricepre.up &  & p25 & 2\\

avgpricepre.down &  & p25 & 10\\

mktElast & \multirow{-8}{*}{\raggedright\arraybackslash all} & p25 & -1\\
\cline{1-4}
hhipre &  & p25 & 2393\\

hhipost &  & p25 & 4011\\

hhidelta & \multirow{-3}{*}{\raggedright\arraybackslash up} & p25 & 1546\\
\cline{1-4}
hhipre &  & p25 & 2572\\

hhipost &  & p25 & 4135\\

hhidelta & \multirow{-3}{*}{\raggedright\arraybackslash down} & p25 & 1431\\
\cline{1-4}
hhipre &  & p25 & 3623\\

hhipost &  & p25 & 5069\\

hhidelta & \multirow{-3}{*}{\raggedright\arraybackslash vertical} & p25 & 1051\\
\cline{1-4}
hhipre &  & p25 & 3591\\

hhipost &  & p25 & 6242\\

hhidelta & \multirow{-3}{*}{\raggedright\arraybackslash both} & p25 & 2463\\
\cline{1-4}
down &  & p50 & 4\\

up &  & p50 & 4\\

vert &  & p50 & 1\\

barg &  & p50 & 1\\

nestParm &  & p50 & 0\\

avgpricepre.up &  & p50 & 5\\

avgpricepre.down &  & p50 & 13\\

mktElast & \multirow{-8}{*}{\raggedright\arraybackslash all} & p50 & -1\\
\cline{1-4}
hhipre &  & p50 & 2876\\

hhipost &  & p50 & 4963\\

hhidelta & \multirow{-3}{*}{\raggedright\arraybackslash up} & p50 & 2027\\
\cline{1-4}
hhipre &  & p50 & 3166\\

hhipost &  & p50 & 5207\\

hhidelta & \multirow{-3}{*}{\raggedright\arraybackslash down} & p50 & 2040\\
\cline{1-4}
hhipre &  & p50 & 4391\\

hhipost &  & p50 & 6016\\

hhidelta & \multirow{-3}{*}{\raggedright\arraybackslash vertical} & p50 & 1292\\
\cline{1-4}
hhipre &  & p50 & 4275\\

hhipost &  & p50 & 7272\\

hhidelta & \multirow{-3}{*}{\raggedright\arraybackslash both} & p50 & 2800\\
\cline{1-4}
down &  & p75 & 5\\

up &  & p75 & 5\\

vert &  & p75 & 2\\

barg &  & p75 & 1\\

nestParm &  & p75 & 0\\

avgpricepre.up &  & p75 & 10\\

avgpricepre.down &  & p75 & 19\\

mktElast & \multirow{-8}{*}{\raggedright\arraybackslash all} & p75 & 0\\
\cline{1-4}
hhipre &  & p75 & 3794\\

hhipost &  & p75 & 6780\\

hhidelta & \multirow{-3}{*}{\raggedright\arraybackslash up} & p75 & 2891\\
\cline{1-4}
hhipre &  & p75 & 4257\\

hhipost &  & p75 & 7391\\

hhidelta & \multirow{-3}{*}{\raggedright\arraybackslash down} & p75 & 2934\\
\cline{1-4}
hhipre &  & p75 & 5679\\

hhipost &  & p75 & 7293\\

hhidelta & \multirow{-3}{*}{\raggedright\arraybackslash vertical} & p75 & 1755\\
\cline{1-4}
hhipre &  & p75 & 5358\\

hhipost &  & p75 & 8648\\

hhidelta & \multirow{-3}{*}{\raggedright\arraybackslash both} & p75 & 3187\\
\cline{1-4}
down &  & Max & 5\\

up &  & Max & 5\\

vert &  & Max & 4\\

barg &  & Max & 1\\

nestParm &  & Max & 0\\

avgpricepre.up &  & Max & 270\\

avgpricepre.down &  & Max & 297\\

mktElast & \multirow{-8}{*}{\raggedright\arraybackslash all} & Max & 0\\
\cline{1-4}
hhipre &  & Max & 10000\\

hhipost &  & Max & 10000\\

hhidelta & \multirow{-3}{*}{\raggedright\arraybackslash up} & Max & 5000\\
\cline{1-4}
hhipre &  & Max & 10000\\

hhipost &  & Max & 10000\\

hhidelta & \multirow{-3}{*}{\raggedright\arraybackslash down} & Max & 5000\\
\cline{1-4}
hhipre &  & Max & 9962\\

hhipost &  & Max & 10000\\

hhidelta & \multirow{-3}{*}{\raggedright\arraybackslash vertical} & Max & 4314\\
\cline{1-4}
hhipre &  & Max & 9998\\

hhipost &  & Max & 10000\\

hhidelta & \multirow{-3}{*}{\raggedright\arraybackslash both} & Max & 5000\\
\cline{1-4}
down &  & Markets & 2202997\\

up &  & Markets & 2202997\\

vert &  & Markets & 2202997\\

barg &  & Markets & 2202997\\

nestParm &  & Markets & 2202997\\

avgpricepre.up &  & Markets & 2202997\\

avgpricepre.down &  & Markets & 2202997\\

mktElast & \multirow{-8}{*}{\raggedright\arraybackslash all} & Markets & 2202997\\
\cline{1-4}
hhipre &  & Markets & 654731\\

hhipost &  & Markets & 654731\\

hhidelta & \multirow{-3}{*}{\raggedright\arraybackslash up} & Markets & 654731\\
\cline{1-4}
hhipre &  & Markets & 701936\\

hhipost &  & Markets & 701936\\

hhidelta & \multirow{-3}{*}{\raggedright\arraybackslash down} & Markets & 701936\\
\cline{1-4}
hhipre &  & Markets & 462005\\

hhipost &  & Markets & 462005\\

hhidelta & \multirow{-3}{*}{\raggedright\arraybackslash vertical} & Markets & 462005\\
\cline{1-4}
hhipre &  & Markets & 384325\\

hhipost &  & Markets & 384325\\

hhidelta & \multirow{-3}{*}{\raggedright\arraybackslash both} & Markets & 384325\\
\hline
\end{tabular}

\caption{Summary Statistics}
\label{tab:simsum}
\end{table}

\begin{sidewaysfigure}
\centering
\includegraphics[scale=0.9]{../output/surplussum.png}
\caption{The figure displays box and whisker plots summarizing the extent to which mergers affect consumer, retailer, wholesaler, and total surplus. Each blue box depicts the effects assuming that firms face constant marginal costs, while each orange box depicts the effects assuming that firms face linear marginal costs. Whiskers depict the $5^{th}$ and $95^{th}$ percentiles of a particular outcome, boxes depict the $25^{th}$ and $75^{th}$ percentiles, and the solid horizontal line depicts the median. }
\label{fig:surplussum}
\end{sidewaysfigure}


\begin{sidewaysfigure}
\centering
\includegraphics[scale=0.9]{../output/CVvertincumbBW.png}
\caption{The figure displays box and whisker plots summarizing the extent to which mergers affect consumer (blue,left) and total (orange,right) surplus as the number of vertically integrated firms present in a market change.  Whiskers depict the $5^{th}$ and $95^{th}$ percentiles of a particular outcome, boxes depict the $25^{th}$ and $75^{th}$ percentiles, and the solid horizontal line depicts the median. }
\label{fig:CVvertincumbBW}
\end{sidewaysfigure}

% \begin{sidewaysfigure}
% \centering
% \includegraphics[scale=0.9]{../output/CVvertincumb_updownBW.png}
% \caption{The figure displays box and whisker plots summarizing the extent to which mergers affect consumer (blue,left) and total (orange,right) surplus as the number of vertically integrated firms present in a market change.  Whiskers depict the $5^{th}$ and $95^{th}$ percentiles of a particular outcome, boxes depict the $25^{th}$ and $75^{th}$ percentiles, and the solid horizontal line depicts the median. }
% \label{fig:CVvertincumbupdownBW}
% \end{sidewaysfigure}
%
% \begin{sidewaysfigure}
% \centering
% \includegraphics[scale=0.9]{../output/CVvertincumb_vertincBW.png}
% \caption{The figure displays box and whisker plots summarizing the extent to which mergers affect consumer (blue,left) and total (orange,right) surplus as the number of vertically integrated firms present in a market change.  Whiskers depict the $5^{th}$ and $95^{th}$ percentiles of a particular outcome, boxes depict the $25^{th}$ and $75^{th}$ percentiles, and the solid horizontal line depicts the median. }
% \label{fig:CVvertincumbvertincBW}
% \end{sidewaysfigure}

% \begin{sidewaysfigure}
% \centering
% \includegraphics[scale=0.9]{../output/CVbargupBW.png}
% \caption{The figure displays box and whisker plots summarizing the extent to which mergers among an integrated and unintegrated wholesaler affect consumer, retailer, wholesaler, and total surplus as the bargaining power of wholesalers relative to retailers changes. The different colored boxes display how outcomes change as the number of vertically  integrated firms increases. Whiskers depict the $5^{th}$ and $95^{th}$ percentiles of a particular outcome, boxes depict the $25^{th}$ and $75^{th}$ percentiles, and the solid horizontal line depicts the median.}
% \label{fig:CVbargupBW}
% \end{sidewaysfigure}
%
% \begin{sidewaysfigure}
% \centering
% \includegraphics[scale=0.9]{../output/CVbargdownBW.png}
% \caption{The figure displays box and whisker plots summarizing the extent to which mergers among an integrated and unintegrated retailer affect consumer, retailer, wholesaler, and total surplus as the bargaining power of wholesalers relative to retailers changes. The different colored boxes display how outcomes change as the number of vertically  integrated firms increases. Whiskers depict the $5^{th}$ and $95^{th}$ percentiles of a particular outcome, boxes depict the $25^{th}$ and $75^{th}$ percentiles, and the solid horizontal line depicts the median.}
% \label{fig:CVbargdownBW}
% \end{sidewaysfigure}
%
% \begin{sidewaysfigure}
% \centering
% \includegraphics[scale=0.9]{../output/CVbargvertBW.png}
% \caption{The figure displays box and whisker plots summarizing the extent to which mergers among an unintegrated wholesaler and unintegrated retailer affect consumer, retailer, wholesaler, and total surplus as the bargaining power of wholesalers relative to retailers changes. The different colored boxes display how outcomes change as the number of vertically  integrated firms increases. Whiskers depict the $5^{th}$ and $95^{th}$ percentiles of a particular outcome, boxes depict the $25^{th}$ and $75^{th}$ percentiles, and the solid horizontal line depicts the median.}
% \label{fig:CVbargvertBW}
% \end{sidewaysfigure}

\begin{sidewaysfigure}
\centering
\includegraphics[scale=0.9]{../output/CVbargnobothBW.png}
\caption{The figure displays box and whisker plots summarizing the extent to which downstream, upstream, and vertical mergers affect consumer, and total surplus as the bargaining power of wholesalers relative to retailers changes. The different colored boxes display how outcomes change as the number of vertically  integrated firms increases. Whiskers depict the $5^{th}$ and $95^{th}$ percentiles of a particular outcome, boxes depict the $25^{th}$ and $75^{th}$ percentiles, and the solid horizontal line depicts the median.}
\label{fig:CVbargnobothBW}
\end{sidewaysfigure}

\begin{sidewaysfigure}
\centering
\includegraphics[scale=0.85]{../output/CVbargbothBW.png}
\caption{The figure displays box and whisker plots summarizing the extent to which mergers among two integrated wholesalers and retailers affect consumer, retailer, wholesaler, and total surplus as the bargaining power of wholesalers relative to retailers changes. The different colored boxes display how outcomes change as the number of vertically  integrated firms increases. Whiskers depict the $5^{th}$ and $95^{th}$ percentiles of a particular outcome, boxes depict the $25^{th}$ and $75^{th}$ percentiles, and the solid horizontal line depicts the median.}
\label{fig:CVbargbothBW}
\end{sidewaysfigure}


\begin{sidewaysfigure}
\centering
\includegraphics[scale=0.9]{../output/surplussum_cost.png}
\caption{The figure displays box and whisker plots summarizing the extent to which merger outcomes change according to 4 different cost scenarios. The boxes in the top row (``All") either assume all firms face either constant marginal costs (blue,left) or linear marginal costs (orange, right). The boxes in the bottom row (``Party") eitehr assume that the merging parties face constant marginal costs while other firms face linear marginal costs (blue, left), or the merging parties face linear marginal costs while other firms face constant marginal costs (orange, right). Whiskers depict the $5^{th}$ and $95^{th}$ percentiles of a particular outcome, boxes depict the $25^{th}$ and $75^{th}$ percentiles, and the solid horizontal line depicts the median. }
\label{fig:surplussumcost}
\end{sidewaysfigure}
\end{document}
