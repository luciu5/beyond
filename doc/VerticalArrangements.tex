\documentclass[12pt]{article}
\usepackage{chicago}    % bibliography package
\usepackage{graphicx}   % insert PostScript figures
\usepackage{setspace}   % controls line spacing
\usepackage{amsmath,amsthm,amssymb,amstext} % controls equation entry and symbols
\usepackage{rotating}   % rotates graphics
\usepackage{soul}       % controls hyphenation
\usepackage{epsfig}     % helps with including graphics
\usepackage{pdflscape}  % helps with displaying rotated graphics in PDFs
\usepackage{lscape}     % helps with rotating pages
\usepackage{caption}    % controls  captions
\usepackage{adjustbox}  % shrink tables
\usepackage[margin=1in]{geometry} % margins
\usepackage{hyperref}   % displays URLs
%packages for kable tables
 \usepackage{booktabs}
 \usepackage{longtable}
 \usepackage{array}
 \usepackage{multirow}
 \usepackage{wrapfig}
 \usepackage{float}
 \usepackage{colortbl}
 \usepackage{pdflscape}
 \usepackage{tabu}
 \usepackage{threeparttable}
 \usepackage{threeparttablex}
 \usepackage[normalem]{ulem}
 \usepackage{makecell}
 \usepackage{xcolor}

\graphicspath{{../output/}}

\interfootnotelinepenalty=10000

\parskip     2.0mm       % space between paragraphs

\alph{footnote}         % make title footnotes alpha-numeric

\captionsetup[figure]{labelsep=space,labelfont=bf} % remove colon from figure name

%%%%%%%%%%%%%%%%%%%%%%%%%%%%%%%%%%%%%%%%%%%%%%%%%%%%%%%%%%%%%%%%%%%%%%%%%%%
% Title Page
%%%%%%%%%%%%%%%%%%%%%%%%%%%%%%%%%%%%%%%%%%%%%%%%%%%%%%%%%%%%%%%%%%%%%%%%%%%
\title{Beyond ``Horizontal'' and ``Vertical'':\\ The Welfare Effects of Complex Integration\footnote{The analysis and conclusions set forth are those of the authors and do not indicate concurrence by other members of the Board research staff or by the Federal Reserve Board of Governors. Furthermore, the views expressed here should not be purported to reflect those of the U.S. Department of Justice. This article has benefited from conversations with Alison Oldale, Ted Rosenbaum, and Nathan Wilson.}}

\newcommand*\samethanks[1][\value{footnote}]{\footnotemark[#1]}
\author{Margaret Loudermilk\footnote{U.S. Department of Justice, margaret.loudermilk@usdoj.gov} \\  U.S Department of Justice \and Gloria Sheu\footnote{Board of Governors of the Federal Reserve System, gloria.sheu@frb.gov. } \\ Federal Reserve Board \and  Charles Taragin\footnote{Board of Governors of the Federal Reserve System, charles.s.taragin@frb.gov}  \\  Federal Reserve Board }

\date{\today}

\begin{document}

\pagenumbering{roman}       % Roman numerals from abstract to text
\maketitle                  % print title information
\thispagestyle{empty}       % no page number on THIS page


\begin{abstract}
\noindent We study the welfare impacts of mergers in markets where some firms are already vertically integrated.  Our model features logit Bertrand competition downstream and Nash Bargaining upstream.  We numerically simulate four merger types: vertical mergers between an unintegrated retailer and an unintegrated wholesaler, downstream ``horizontal'' mergers between an unintegrated retailer and an integrated retailer/wholesaler, upstream ``horizontal'' mergers between an unintegrated wholesaler and an integrated retailer/wholesaler, and integrated mergers between two integrated retailer/wholesaler pairs. We find that mergers that have both horizontal and vertical characteristics typically harm consumers.  We apply the model to the Republic/Santek merger as a real-world example.
\end{abstract}

\bigbreak Keywords: bargaining models; merger simulation; vertical markets; vertical mergers

JEL classification: L13; L40; L41; L42

\newpage                    % start a new page
\pagenumbering{arabic}      % Arabic page numbers from now on
\doublespacing

%%%%%%%%%%%%%%%%%%%%%%%%%%%%%%%%%%%%%%%%%%%%%%%%%%%%%%%%%%%%%%%%%%%%%%%%%%%%%%%%%
% Body of Paper
%%%%%%%%%%%%%%%%%%%%%%%%%%%%%%%%%%%%%%%%%%%%%%%%%%%%%%%%%%%%%%%%%%%%%%%%%%%%%%%%%
\section{Introduction}
Given the many different ways a firm may be organized, it is rare that any particular merger can be neatly categorized as purely ``horizontal'' or ``vertical.''  This is particularly true for the mergers involving large firms that usually attract the most scrutiny.  However, for both historical and practical reasons, antitrust research and practice has generally maintained this dichotomy.  Horizontal mergers are cast in terms of whether they create significant upward pricing pressure (UPP), whereas the focus for vertical mergers is on assessing the net effects of the elimination of double marginalization (EDM) versus raising rivals' costs (RRC).  As a result, the antitrust community has little consensus for what welfare impacts one should expect from a merger that combines all of these effects simultaneously.

The horizontal versus vertical distinction has been enshrined in U.S. Department of Justice (DOJ) and Federal Trade Commission (FTC) Horizontal Merger Guidelines and their companion document, the Vertical Merger Guidelines.\footnote{The FTC withdrew its approval from the Vertical Merger Guidelines in 2021.  The DOJ did not.  }  Given these policy documents, it is not surprising that the agencies typically frame most mergers as either horizontal or vertical.  However, as part of the 2022 effort by the DOJ and the FTC to revise their merger guidelines, these agencies requested comments on whether any ``modern market realities may be lost by focusing on [horizontal and vertical] relationships categorically.''\footnote{The full question is 1.g. in the ``Request for Information on Merger Enforcement,'' dated January 18, 2022.  The text of the question reads: ``Should the guidelines' traditional distinctions between horizontal and vertical mergers be revisited in light of recent economic trends in the modern economy? What aspects of modern market realities may be lost by focusing on these relationships categorically? Should the guidelines address all mergers in a common framework that covers all market relationships relevant to competition? If so, how?''.  }  In this paper, our goal is to provide guidance to practitioners on how these mergers can be simulated in a familiar framework and to also show the range of welfare outcomes that are likely given this modeling environment.

Our model is drawn directly from \citeN{ST2021}, and features a Bertrand logit downstream setup alongside a Nash Bargaining wholesale negotiation upstream.  We use this model to numerically simulate four types of mergers: (1) vertical mergers between an unintegrated retailer and an unintegrated wholesaler, (2) downstream ``horizontal'' mergers between an unintegrated retailer and an integrated retailer/wholesaler pair, (3) upstream ``horizontal'' mergers between an unintegrated wholesaler and an integrated retailer/wholesaler pair, and (4) integrated mergers between two previously integrated retailer/wholesaler pairs.  In each of these simulations, we calculate the impacts on consumer welfare.  Furthermore, we examine how the presence of additional integrated rivals (separate from the merging firms) changes our results.

Our simulations show certain patterns across mergers.  For horizontal upstream and downstream mergers, we find that consumers are net harmed in most instances.  The exceptions occur when upstream wholesalers have more bargaining power than downstream retailers, which is when we expect pre-merger input prices to be high and the gains from EDM to be larger.  Furthermore, we find that integrated mergers are typically net harmful to consumers as well.  The exceptions happen only when upstream firms have significantly more bargaining power compared to downstream firms.  Thus, although these mergers have both horizontal and vertical aspects, they appear to be qualitatively similar to pure horizontal mergers in their overall consumer impacts.  Furthermore, we find that the presence of integrated rival firms lowers the spread of welfare outcomes for consumers and slightly shifts these outcomes towards less consumer harm.  The existence of rivals with their own sources of bargaining leverage stemming from vertical integration appears to reduce the variance of merger outcomes, but does not substantially mitigate consumer harm.

We then employ this model to investigate the 2021 merger of two solid waste management firms, Republic Services and Santek Waste Services, in one geographic market, Chattanooga, Tennessee. The supply chain of this industry involves collection haulers at the downstream level that bring trash to disposal facilities at the upstream level.  Both Republic and Santek operated upstream and downstream in Chattanooga, as did one of their largest rivals, Waste Connections.  In this market, the DOJ negotiated a settlement package meant to address harms in downstream collection and in upstream disposal.  Using a small set of public data inputs, our merger simulation model is able to produce rich results for the changes in horizontal and vertical competition, including RRC and EDM effects.  We estimate that without divestitures, consumers would have faced approximately \$16 million in annual harm in the Chattanooga market.  Furthermore, we show that modeling the merger as a purely horizontal upstream or downstream transaction would miss key vertical impacts.

As a caveat to our analysis, it is based on one specific model of competition and mergers.  Certain alternative structures are likely to give different welfare results.  Indeed, we found in \citeN{ST2021} that using a second score auction to model downstream competition, as opposed to assuming Bertrand pricing, could meaningfully impact the estimated consumer harm from mergers, although results were similar between these models for vertical mergers.  This paper is an attempt to provide some information about possible welfare effects from complex mergers within a framework that is used by practitioners.  Exploring other commonly used frameworks would be a natural extension in future research.

Our paper is related to the literature on methods for measuring the welfare impacts of mergers, particularly papers about vertical mergers.  The closest merger simulation article is our previous work in \citeN{ST2021}, which uses the same model we feature here.  That paper in turn takes the setup from \citeN{WF1994} to characterize downstream competition, and pairs it with a Nash Bargaining model for the upstream market.  \citeN{DVDS2018} and \citeN{DS2019} provide additional vertical merger frameworks that make alternative assumptions about demand and pricing.  \citeN{MS2013} and \citeN{Rogerson2020} develop UPP-style tools for assessing the impact of vertical mergers.  However, none of these papers have results for the wide range of merger types that we study.

Our model draws upon the empirical industrial organization literature on bargaining in supply chains, typified by \citeN{DKVB2010}.  These models have been applied to several sectors that are the frequent target of complex mergers, such as healthcare (e.g. \citeN{HL2017}) and video programming distribution (e.g. \citeN{CLWY2017}).  These structural empirical papers explore the welfare effects of various competitive interactions in certain industries.  Unlike these papers, our framework has been simplified for use with limited data, with the goal of being more accessible to practitioners.  These simplifications allow us to provide numerical results over a wide range of market conditions, which complements the insights that come from the structural empirical literature on specific markets.

The paper proceeds as follows.  In Section \ref{sec:theory}, we introduce the theoretical framework and explain how we model mergers.  Section \ref{sec:sims} explains how we construct our numerical simulations and provides our results, and our application to Republic/Santek appears in Section \ref{sec:application}.  Section \ref{sec:concl} concludes. Additional simulation results and information on the data used in the application appear in the Appendix.

%%%%%%%%%%%%%%%%%%%%%%%%%%%%%%%%%%%%%%%%%%%%%%%%%%%%%%%%%%%%%%%%%%%%%%%%%%%%%%%%%
\section{Theory\label{sec:theory}}
We begin by describing our basic framework, taken from \citeN{ST2021}, which uses a downstream Bertrand logit model embedded in an upstream Nash Bargaining setup.  This framework allows us to study a variety of merger configurations.

%%%%%%%%%%%%%%%%%%%%%%%%%%%%%%%%%%%%%%%%%%
\subsection{Downstream and Upstream Competition}
Assume there is a set of consumers who each choose to buy a single product sold by a retailer.  Retailers are indexed by $r$, and the wholesalers that supply these retailers are indexed by $w$.  Prior to any mergers taking place, each wholesaler offers only one product, although each retailer can purchase from multiple wholesalers.  We denote the set of all retailers by $\mathbb{R}=\{1, \dots, \left\vert{\mathbb{R}}\right\vert\}$ and the set of all wholesalers by $\mathbb{W}=\{1, \dots, \left\vert{\mathbb{W}}\right\vert\}$.  We divide the set $\mathbb{W}$ into overlapping subsets, each labeled $\mathbb{W}^r$, to indicate which wholesalers' products are carried by which retailers.  Similarly, we divide the set of retailers $\mathbb{R}$ into overlapping subsets, each labeled $\mathbb{R}^w$, to indicate the retailers that carry the product sold by each wholesaler.

The share of consumers that choose product $w$ sold by retailer $r$ has the logit form,
\begin{equation}
s_{rw} = \frac{\exp(\delta_{rw} - \alpha p_{rw})}{1 + \sum_{t \in \mathbb{R}} \sum_{x \in \mathbb{W}^t} \exp(\delta_{tx} - \alpha p_{tx})},
\label{eq: nb share}
\end{equation}
where $\delta_{rw}$ is a quality parameter and $\alpha$ captures sensitivity to price $p_{rw}$.  There is an outside good whose quality parameter and price have been normalized to zero.  The retailer's profit is given by $\pi^r = \sum_{w \in \mathbb{W}^r} [p_{rw} - p^W_{rw} - c^R_{rw}] s_{rw} M$, where $p^W_{rw}$ is the unit fee charged by wholesaler $w$ to retailer $r$, $c^R_{rw}$ captures any additional marginal costs borne by the retailer, and $M$ is the market size.  Downstream prices are set in Bertrand equilibrium, according to
\begin{equation}
\sum_{x \in \mathbb{W}^r} [p_{rx} - p^W_{rx} - c^R_{rx}] \frac{\partial s_{rx}}
{\partial p_{rw}} + s_{rw} = 0,
\label{eq: nb downstream foc}
\end{equation}
which is the first order condition for product $w$ sold by retailer $r$.

Wholesale prices are set via Nash Bargaining between retailers and wholesalers.  We assume that the negotiation for a given input price treats other input prices and all downstream prices as given.\footnote{As explained in \citeN{ST2021}, this assumption is equivalent to a situation where all negotiations and choices for downstream prices happen simultaneously.  }  Profits for wholesaler $w$ are given by $\pi^w = \sum_{r \in \mathbb{R}^w} [p^W_{rw} - c^W_{rw}] s_{rw} M$, where $c^W_{rw}$ is the wholesale marginal cost when dealing with retailer $r$.  The first order condition for the negotiation between retailer $r$ and wholesaler $w$ is
\begin{equation}
\begin{split}
&\overbrace{[p^W_{rw} - c^W_{rw}]s_{rw} - \sum_{t \in \mathbb{R}^w \setminus \{r\}} [p^W_{tw} - c^W_{tw}] \Delta s_{tw}(\mathbb{W}^r \setminus \{w\})}^{\text{wholesaler GFT}} = \\
&\frac{1-\lambda}{\lambda} \left(\underbrace{[p_{rw} - p^W_{rw} - c^R_{rw}]s_{rw} - \sum_{x \in \mathbb{W}^r \setminus \{w\}} [p_{rx} - p^W_{rx} - c^R_{rx}] \Delta s_{rx}(\mathbb{W}^r \setminus \{w\})}_{\text{retailer GFT}}\right),
\end{split}
\label{eq: upstream foc}
\end{equation}
where $\lambda \in [0,1]$ captures the bargaining power of the retailer relative to the wholesaler.  The $\Delta s_{tx}(\mathbb{W}^r \setminus \{w\}) \equiv s_{tx}(\mathbb{W}^r \setminus \{w\}) - s_{tx}$ is the difference in the share of good $x$ sold by retailer $t$ when good $w$ is not offered by retailer $r$ versus when good $w$ is offered by retailer $r$.  Thus, the wholesale price $p_{rw}$ is set such that the payoff to wholesaler $w$ when it sells to retailer $r$ less the payoff when it does not (that is, the gains from trade or ``GFT''), divided by the payoff to retailer $r$ when it buys from wholesaler $w$ less the payoff when it does not, equals the ratio of wholesaler to retailer bargaining power.

Together the series of downstream and upstream first order conditions determine market equilibrium.  This model can be solved and calibrated as described in \citeN{ST2021}.  In our baseline configuration, we assume that wholesale and retail marginal costs are constant.  We allow for linear increasing marginal costs as a robustness check that appears in Appendix \ref{app:mc}.

%%%%%%%%%%%%%%%%%%%%%%%%%%%%%%%%%%%%%%%%%%
\subsection{Mergers}
Here we describe the manner in which mergers are modeled in this framework.  We begin by describing vertical mergers, as all cases we examine have some vertical aspects.  Suppose that retailer $r$ and wholesaler $w$ were to merge.  Then the first order condition for profit maximization for the joint firm when setting the downstream price for product $rw$ is given by
\begin{equation}
\begin{split}
&\sum_{x \in \mathbb{W}^r \setminus \{w\}} [p_{rx} - p^W_{rx} - c^R_{rx}]\frac{\partial s_{rx}}{\partial p_{rw}} + s_{rw} + \overbrace{[p_{rw} - c^W_{rw} - c^R_{rw}]\frac{\partial s_{rw}}{\partial p_{rw}}}^{\text{EDM effect}} \\
&+ \underbrace{\sum_{t \in \mathbb{R}^w \setminus \{r\}} [p^W_{tw} - c^W_{tw}] \frac{\partial s_{tw}}{\partial p_{rw}}}_{\text{upstream UPP effect}} = 0.
\end{split}
\label{eq: vmerger own ret downstream foc}
\end{equation}
The pricing problem balances two effects.  On the one hand, the term labeled ``EDM effect'' captures the impact of retailer $r$ being able to access product $w$ at marginal cost.  This force would tend to lower the resulting price.  On the other hand, the term labeled ``upstream UPP effect'' captures the incentive to raise prices for retailer $r$ in order to divert sales to wholesaler $w$.  This force would tend to raise the resulting consumer price.

Turning to input prices, when wholesaler $w$ bargains with unaffiliated retailer $s$, the first order condition becomes
\begin{equation}
\begin{split}
&[p^W_{sw} - c^W_{sw}]s_{sw} - \sum_{t \in \mathbb{R}^w \setminus \{r,s\}} [p^W_{tw} - c^W_{tw}] \Delta s_{tw}(\mathbb{W}^s \setminus \{w\})\\
& - \overbrace{\overbrace{[p_{rw}-c^W_{rw}-c^R_{rw}] \Delta s_{rw}(\mathbb{W}^s \setminus \{w\})}^{\text{indirect EDM effect}} - \sum_{x \in \mathbb{W}^r \setminus \{w\}} [p_{rx} - p^W_{rx} - c^R_{rx}] \Delta s_{rx}(\mathbb{W}^s \setminus \{w\})}^{\text{RRC effect}}=\\
&\frac{1-\lambda}{\lambda} \left([p_{sw} - p^W_{sw} - c^R_{sw}]s_{sw} - \sum_{x \in \mathbb{W}^s \setminus \{w\}} [p_{sx} - p^W_{sx} - c^R_{sx}] \Delta s_{sx}(\mathbb{W}^s \setminus \{w\})\right).
\end{split}
\label{eq: vmerger wh upstream foc}
\end{equation}
which reflects the change in the disagreement payoff coming from the merger with retailer $r$.  Now when the wholesaler considers the possible loss of sales upon ceasing to trade with retailer $s$, these losses are softened due to a potential for diversion to retailer $r$, which we label the ``RRC effect.''  Furthermore, the margin on product $w$ sold by retailer $r$ is potentially higher due to EDM, as shown through the expression labeled ``indirect EDM effect,'' which can further compensate the firm.  These impacts tend to raise the resulting input price.

When the merged firm is bargaining with the unaffiliated wholesaler $v$ over what input price to pay, the bargaining first order condition becomes
\begin{equation}
\begin{split}
&[p^W_{rv} - c^W_{rv}]s_{rv} - \sum_{t \in \mathbb{R}^v \setminus \{r\}} [p^W_{tv} - c^W_{tv}] \Delta s_{tv}(\mathbb{W}^r \setminus \{v\}) = \\
&\frac{1-\lambda}{\lambda} \left([p_{rv} - p^W_{rv} - c^R_{rv}]s_{rv} - \sum_{x \in \mathbb{W}^r \setminus \{w,v\}} [p_{rx} - p^W_{rx} - c^R_{rx}] \Delta s_{rx}(\mathbb{W}^r \setminus \{v\})\right.\\
&\left.- \underbrace{[p_{rw} - c^W_{rw} - c^R_{rw}] \Delta s_{rw}(\mathbb{W}^r \setminus \{v\})}_{\text{EDM recapture effect}}-\underbrace{\sum_{t \in \mathbb{R}^w \setminus \{r\}} [p^W_{tw} - c^W_{tw}] \Delta s_{tw}(\mathbb{W}^r \setminus \{v\})}_{\text{wholesale recapture leverage effect}}\right).
\end{split}
\label{eq: vmerger ret upstream foc}
\end{equation}
In this case, the merged firm has two channels for potential additional profits should it cease to trade with wholesaler $v$.  First, if retail sales are diverted to product $w$ sold by retailer $r$, those sales will earn a higher margin due to lower marginal costs stemming from what we call the ``EDM recapture effect.''  Second, the loss of product $v$ carried by retailer $r$ could increase sales by wholesaler $w$ through other retailers, which we call the ``wholesale recapture leverage effect.''  Both of these effects would tend to lower the resulting input price.

Next consider a merger between an integrated retailer/wholesaler $rw$ and a standalone retailer $s$.  Such a combination has a vertical component and a horizontal component.  The first order condition for setting the downstream price $p_{rw}$ becomes
\begin{equation}
\begin{split}
&\sum_{x \in \mathbb{W}^r \setminus \{w\}} [p_{rx} - p^W_{rx} - c^R_{rx}]\frac{\partial s_{rx}}{\partial p_{rw}} + s_{rw} + \overbrace{[p_{rw} - c^W_{rw} - c^R_{rw}]\frac{\partial s_{rw}}{\partial p_{rw}}}^{\text{direct EDM effect}}+ \underbrace{\sum_{t \in \mathbb{R}^w \setminus \{r\}} [p^W_{tw} - c^W_{tw}] \frac{\partial s_{tw}}{\partial p_{rw}}}_{\text{upstream UPP effect}} \\ &+\underbrace{\underbrace{[p_{sw} - c^W_{sw} - c^R_{sw}]\frac{\partial s_{sw}}{\partial p_{rw}}}_{\text{indirect EDM Effect}} + \sum_{x \in \mathbb{W}^s \setminus \{w\}} [p_{sx} - p^W_{sx} - c^R_{sx}] \frac{\partial s_{sx}} {\partial p_{rw}}}_{\text{downstream UPP effect}}= 0.
\end{split}
\label{eq: intdmerger own ret downstream foc}
\end{equation}
Now there is the possibility for what we call the ``downstream UPP effect'' in the retail market, as the merged firm can recapture sales that are diverted to retailer $s$ when $r$ raises its prices.  EDM between $w$ and $s$ can actually increase this UPP impact, because when sales are diverted to product $w$ sold by retailer $s$, those units may earn a larger margin.  This impact is what we label the ``indirect EDM effect,'' which comes through the interaction of EDM with UPP.

If instead an integrated retailer/wholesaler $rw$ were to merge with a standalone wholesaler $v$, then the resulting first order condition for product $rw$ would look similar to equation \eqref{eq: intdmerger own ret downstream foc}.  However,  the downstream UPP effect would be replaced with an additional upstream UPP effect capturing the value of sales diverted to customers of wholesaler $v$.  The condition would also include an indirect EDM component, reflecting the ability of retailer $r$ to obtain product $v$ at marginal cost, which in turn would raise the value of diverted sales to that product.  Whether these additional incentives to raise prices will dominate the direct EDM impact is an empirical question.  Furthermore, if we instead considered a merger between two integrated retailer/wholesaler pairs, both upstream and downstream UPP effects would enter.

As for wholesale prices, again consider a merger between an integrated retailer/wholesaler $rw$ and a standalone retailer $s$.  When wholesaler $w$ bargains with an unaffiliated retailer $u$ we have the first order condition given by,
\begin{equation}
\begin{split}
&[p^W_{uw} - c^W_{uw}]s_{uw} - \sum_{t \in \mathbb{R}^w \setminus \{r,s,u\}} [p^W_{tw} - c^W_{tw}] \Delta s_{tw}(\mathbb{W}^u \setminus \{w\})\\
& - \overbrace{\sum_{t \in \{r,s\}}\left(\overbrace{[p_{tw}-c^W_{tw}-c^R_{tw}] \Delta s_{tw}(\mathbb{W}^u \setminus \{w\})}^{\text{indirect EDM effect}} - \sum_{x \in \mathbb{W}^t \setminus \{w\}} [p_{tx} - p^W_{tx} - c^R_{tx}] \Delta s_{tx}(\mathbb{W}^u \setminus \{w\})\right)}^{\text{RRC effect}}=\\
&\frac{1-\lambda}{\lambda} \left([p_{uw} - p^W_{uw} - c^R_{uw}]s_{uw} - \sum_{x \in \mathbb{W}^u \setminus \{w\}} [p_{ux} - p^W_{ux} - c^R_{ux}] \Delta s_{ux}(\mathbb{W}^u \setminus \{w\})\right).
\end{split}
\label{eq: intdmerger wh upstream foc}
\end{equation}
The RRC effect is augmented with the profits emanating from the sales of retailer $s$, in addition to the sales of retailer $r$, both of which may potentially recapture sales should retailer $u$ lose access to product $w$.  The merged firm has higher bargaining leverage as a result.  If instead the integrated retailer/wholesaler $rw$ were to merge with a standalone wholesaler $v$, then the profits earned by firm $rw$ should the negotiation with retailer $u$ fail are augmented with the earnings of wholesaler $v$ rather than of retailer $s$.  This adds a term similar to the wholesale recapture leverage effect seen in equation \eqref{eq: vmerger ret upstream foc} to the left-hand side of the bargaining first order condition.\footnote{For simplicity, we assume that when a retailer fails to reach an agreement with wholesaler $w$, that retailer's contract with wholesaler $v$ remains in place.  This assumption can be loosened.  }    In the case of a merger between two integrated retailer/wholesaler firms, all of these effects would appear.

Returning to a merger between an integrated retailer/wholesaler $rw$ and a standalone retailer $s$, when the merged firm bargains with an unaffiliated wholesaler $v$ to supply retailer $r$, the first order condition becomes
\begin{equation}
\begin{split}
&[p^W_{rv} - c^W_{rv}]s_{rv} - \sum_{t \in \mathbb{R}^v \setminus \{r\}} [p^W_{tv} - c^W_{tv}] \Delta s_{tv}(\mathbb{W}^r \setminus \{v\}) = \\
&\frac{1-\lambda}{\lambda} \left([p_{rv} - p^W_{rv} - c^R_{rv}]s_{rv} - \sum_{x \in \mathbb{W}^r \setminus \{w,v\}} [p_{rx} - p^W_{rx} - c^R_{rx}] \Delta s_{rx}(\mathbb{W}^r \setminus \{v\})\right.\\
&- \underbrace{\sum_{t \in \{r,s\}}[p_{tw} - c^W_{tw} - c^R_{tw}] \Delta s_{tw}(\mathbb{W}^r \setminus \{v\})}_{\text{EDM recapture effect}}
- \underbrace{\sum_{x \in \mathbb{W}^s \setminus \{w\}} [p_{sx} - p^W_{sx} - c^R_{sx}] \Delta s_{sx}(\mathbb{W}^r \setminus \{v\})}_{\text{retail recapture leverage effect}}\\
&-\left.\underbrace{\sum_{t \in \mathbb{R}^w \setminus \{r,s\}} [p^W_{tw} - c^W_{tw}] \Delta s_{tw}(\mathbb{W}^r \setminus \{v\})}_{\text{wholesale recapture leverage effect}}\right).
\end{split}
\label{eq: intdmerger ret upstream foc}
\end{equation}
Compared to equation \eqref{eq: vmerger ret upstream foc}, now the EDM recapture effect applies to sales of product $w$ at both affiliated retailers $r$ and $s$.  Furthermore, there is also the possibility that sales will be diverted to retailer $s$ should retailer $r$ lose access to product $v$, which creates an additional ``retail recapture leverage effect'' alongside the wholesale recapture leverage effect.  All of these additional terms tend to increase the bargaining leverage of the merged firm.  If we instead examined a merger between retailer/wholesaler $rw$ and an unaffiliated wholesaler, that would augment the EDM recapture effect with sales of retailer $r$ for two merged wholesalers, rather than sales through two merged retailers.  Similarly, the retail recapture leverage effect would be replaced with an additional wholesale recapture leverage effect for the newly merged wholesaler.  For a merger between two integrated retailer/wholesaler firms, all of these effects would enter.

The discussion thus far has not dealt with the possible existence of integrated rivals to the merging firms.  If such rival firms exist, they would also have pricing and bargaining incentives similar to those captured in the above post-merger first order conditions.  For example, if the integrated firm $rw$ were bargaining over what input price to charge a rival integrated firm $sv$, the first order condition would be
\begin{equation}
\begin{split}
&[p^W_{sw} - c^W_{sw}]s_{sw} - \sum_{t \in \mathbb{R}^w \setminus \{r,s\}} [p^W_{tw} - c^W_{tw}] \Delta s_{tw}(\mathbb{W}^s \setminus \{w\})\\
& - \overbrace{\overbrace{[p_{rw}-c^W_{rw}-c^R_{rw}] \Delta s_{rw}(\mathbb{W}^s \setminus \{w\})}^{\text{indirect EDM effect}} - \sum_{x \in \mathbb{W}^r \setminus \{w\}} [p_{rx} - p^W_{rx} - c^R_{rx}] \Delta s_{rx}(\mathbb{W}^s \setminus \{w\})}^{\text{RRC effect}}=\\
&\frac{1-\lambda}{\lambda} \left([p_{sw} - p^W_{sw} - c^R_{sw}]s_{sw} - \sum_{x \in \mathbb{W}^s \setminus \{w,v\}} [p_{sx} - p^W_{sx} - c^R_{sx}] \Delta s_{sx}(\mathbb{W}^s \setminus \{w\})\right.\\
&\left.- \underbrace{[p_{sv} - c^W_{sv} - c^R_{sv}] \Delta s_{sv}(\mathbb{W}^s \setminus \{w\})}_{\text{EDM recapture effect}}-\underbrace{\sum_{t \in \mathbb{R}^v \setminus \{s\}} [p^W_{tv} - c^W_{tv}] \Delta s_{tv}(\mathbb{W}^s \setminus \{w\})}_{\text{wholesale recapture leverage effect}}\right).
\end{split}
\label{eq: intmerger wh upstream foc}
\end{equation}
The existing integration of firm $sv$ gives it countervailing bargaining leverage that may counteract any change in leverage caused by a merger of retailer $r$ and wholesaler $w$.  The extent of the net impact depends on how these leverage terms balance out in equilibrium.

%%%%%%%%%%%%%%%%%%%%%%%%%%%%%%%%%%%%%%%%%%%%%%%%%%%%%%%%%%%%%%%%%%%%%%%%%%%%%%%%%
\section{Numerical Simulations \label{sec:sims}}
In this section, we use a similar simulation setup to that in \citeN{ST2021}, but allow for additional types of mergers beyond purely horizontal and purely vertical combinations.

%%%%%%%%%%%%%%%%%%%%%%%%%%%%%%%%%%%%%%%%%%
\subsection{Data Generating Process}
Broadly speaking, we consider four categories of mergers: downstream, upstream, vertical, and integrated.  We define a downstream merger as a merger between two unintegrated retailers or between an unintegrated retailer and a vertically integrated wholesaler/retailer combination.  Similarly, upstream mergers are those between two unintegrated wholesalers or between an unintegrated wholesaler and a vertically integrated wholesaler/retailer.  We define vertical mergers as those between an unintegrated wholesaler and unintegrated retailer.  Finally, an integrated merger is between two firms that are both already vertically integrated pre-merger.

For downstream, upstream, and vertical mergers, we simulate markets by randomly sampling shares from a Dirichlet distribution for 2 to 5 retailers or wholesalers, respectively.\footnote{We parametrize the Dirichlet distribution so it is equivalent to a uniform distribution.  }  Because integrated mergers must have two vertically integrated incumbents in the pre-merger state, for those simulations we increase the maximum allowable number of wholesalers and retailers to 7. We also assume that in the pre-merger state, there are anywhere from 0 to 4 vertically integrated incumbents (2 to 6 for integrated mergers). Vertically integrated firms are not siloed: integrated wholesalers supply inputs to retailers other than their integrated partner, and integrated retailers purchase inputs from wholesalers other than their integrated partner.

Our simulations focus on mergers that are more likely to have anti-competitive effects and to therefore come under agency scrutiny. For horizontal merger simulations, we assign the products sold by the two largest firms in the market to a single firm post-merger. Similarly, when simulating a vertical merger, we assign the products sold by the largest wholesaler and the largest retailer to a single firm post-merger.

The bargaining parameter ranges from from 0.1 (where wholesalers have the advantage) to 0.9 (where retailers have the advantage). We report our results in terms of relative bargaining power, $(1-\lambda)/\lambda$.  The bargaining parameter is identical for all of the retailers in each simulation.  We calibrate the price coefficient $\alpha$ assuming that in the pre-merger world, there is a vertically integrated outside option that sets its Bertrand equilibrium price to \$5, has 15\% market share, and has zero marginal costs.  All other goods are differenced relative to this option, resulting in the mean zero outside good normalization.  We assume that pre-merger, upstream marginal costs are 25\% of upstream firm margins and 10\% of downstream firm margins. We set the market size to 1.

For each combination of number of retailers, number of wholesalers, number of incumbent integrated firms, and bargaining parameter, we draw 1,000 different sets of market primitives.  After eliminating situations where the merger is unprofitable to the merging firms, as well as downstream markets that do not pass the Hypothetical Monopolist Test, we have 1.6 million markets remaining.\footnote{The Hypothetical Monopolist Test is a market definition exercise that checks whether a monopolist that owns all products in a candidate market would raise the price of at least one of the merging producers' products by at least a ``small but significant non-transitory increase in price'' (SSNIP), which we take to be 5\%. } Each market treats as primitives the number of retailers, the number of wholesalers, the bargaining parameter, and the wholesaler and retailer marginal costs.

%%%%%%%%%%%%%%%%%%%%%%%%%%%%%%%%%%%%%%%%%%
\subsection{Overview of Simulated Output}

Table \ref{tab:simsum} provides summary statistics for our simulations. The 2010 DOJ/FTC Horizontal Merger Guidelines categorize ``Highly Concentrated Markets'' as those with Herfindahl Hirschman Index (HHI) values over 2,500 points.  The Guidelines state that mergers with HHI changes greater than 200 points that result in Highly Concentrated markets are ``presumed likely to enhance market power.''\footnote{See \S 5.3 of the Guidelines.  }  The vast majority of our mergers meet these conditions.\footnote{We compute the post-merger HHI for vertical mergers by treating the merged firms' market share as the sum of all the shares of downstream products that either use the upstream partner's input or are sold by the downstream partner.  }  The median average wholesale pre-merger price is \$5.80, and the median average retail pre-merger price is \$15.  Recall that the market size is set to 1, meaning that average price is equal to total pre-merger expenditures.

Figure \ref{fig:surplussum} summarizes our results for welfare.  Each of the four panels shows how mergers impact surplus for a particular set of agents (consumers, retailers, wholesalers, or the entire market combined, respectively).  Surplus is reported as a percentage change of total pre-merger expenditures in the downstream market.  Each panel contains four box and whisker plots, with each plot corresponding to a different type of merger. The whiskers show the $5^{th}$ and $95^{th}$ percentiles of the outcome distribution, the boxes denote the $25^{th}$ and $75^{th}$ percentiles, and the solid horizontal line marks the median. Negative outcome values imply harm, and positive values imply benefits.

We focus first on the results for consumers in the left-most panel of Figure \ref{fig:surplussum}.  Less than half of all simulated vertical mergers are harmful, while the majority of upstream, downstream, and integrated mergers show harm.  In particular, there is a partial rank-ordering of consumer harm across types of mergers: consumer harm from downstream and integrated mergers first-order stochastically dominates consumer harm from upstream mergers which in turn dominates consumer harm from vertical mergers. However, while consumer harm is typically greater under integrated mergers than downstream mergers, it is not first-order stochastically dominant. Median consumer harm from integrated mergers is about 14\% of pre-merger expenditures, 1.3 times the magnitude of downstream mergers, and 3 times the magnitude from upstream mergers.

Moving on to retailers in the second panel of Figure \ref{fig:surplussum}, we find that whereas vertical mergers, downstream mergers, and integrated mergers almost always benefit retailers, upstream mergers harm retailers in about 60\% of all simulations. Moreover, there is a partial rank-ordering across mergers that is distinct from the consumer rank-ordering: the retailer surplus distribution from integrated mergers first-order stochastically dominates that from downstream mergers, while the retailer surplus distribution from vertical mergers stochastically dominates the distribution from upstream mergers. The median gain to retailers from integrated mergers is about 13\% of pre-merger expenditures,  1.5 times the magnitude of downstream mergers, and 1.1 times the magnitude of that from vertical mergers.

Turning to wholesaler surplus in the third panel of Figure \ref{fig:surplussum}, the effects seen there are largely reversed from those for retailers: wholesaler surplus increases in about 64\% of all upstream mergers, 11\% of vertical mergers, and about 16\% of downstream  and integrated mergers. Here, wholesaler surplus under integrated, downstream and upstream mergers each stochastically dominate the surplus from vertical mergers, but not one another.

As for total welfare, approximately 29\% of vertical mergers and 17\% of integrated mergers are beneficial, whereas only 9\% of upstream mergers and 2\% of downstream mergers are beneficial. Moreover, there is a partial rank ordering of mergers, with total harm from downstream mergers first-order stochastically dominating total harm from upstream  mergers, which dominates total harm from vertical mergers. Median consumer harm from integrated mergers is about 9\% of pre-merger total expenditures, 1.4 times the magnitude of that from downstream mergers, 2.6 times the magnitude of that from upstream mergers, and more than 4 times the magnitude of vertical mergers.

Throughout these simulations, we maintain the assumption that marginal costs are constant.  However, given that many of the mergers we study have vertical aspects where input cost considerations may drive the realization of EDM and RRC, it is reasonable to wonder how our results would change in an alternative marginal cost environment.  In Appendix \ref{app:mc}, we provide results that allow marginal costs to increase linearly.  We find that this alternative assumption tends to result in mergers that are more harmful for consumers and total welfare.

%%%%%%%%%%%%%%%%%%%%%%%%%%%%%%%%%%%%%%%%%%
\subsection{Mergers and Vertically Integrated Incumbent Firms}

We examine merger outcomes both with and without preexisting vertical integration among the merging firms and as the number of rival incumbent integrated firms varies.  Figure \ref{fig:CVvertincumbBW_consumer} depicts box and whisker plots summarizing consumer welfare as the number of incumbent integrated firms increases. For the results on total welfare, see Appendix \ref{app:total}.  The plots when the number of incumbent integrated firms equals 0 correspond to the results depicted in Figure 1 of \citeN{ST2021} and are included as reference.\footnote{An important difference between the simulations depicted in Figure \ref{fig:CVvertincumbBW_consumer} and those in Figure 1 of \citeN{ST2021} is that here we do not include downstream markets where prices are set according to a second score auction.}  For vertical mergers, the plot when the number of incumbent integrated firms equals 1 summarizes market outcomes where an unintegrated wholesaler merges with an unintegrated retailer, a single third party is integrated, and any remaining rivals are unintegrated. By contrast, for upstream and downstream  mergers, the plots when the number of incumbent integrated firms equals 1 depict the outcome of a merger between an integrated firm and an unintegrated wholesaler or retailer, respectively, with no integrated third parties. For integrated mergers, the plot when the number of incumbent integrated firms equals 0 depicts a merger between two integrated firms, again with no integrated rivals.

We begin with vertical mergers, as they are our simplest case, because there is never any pre-existing integration at either merging firm.  Absent incumbent integration by rivals (denoted by 0 integrated firms in Figure \ref{fig:CVvertincumbBW_consumer}), about half of vertical mergers benefit consumers.  The addition of a single rival integrated firm (denoted by 1 integrated firm in the figure) moves median consumer welfare from about 0\% of pre-merger revenues to a positive 2\% and widens the distribution of outcomes slightly. Adding more integrated rivals does not have much additional effect.

Moving to the second panel in Figure \ref{fig:CVvertincumbBW_consumer}, upstream mergers absent any incumbent vertical integration (0 integrated firms) are never beneficial in the range we study, with median consumer harm at about 6\% of pre-merger revenues. This result is to be expected, as upstream mergers without additional integration do not have the countervailing effects for welfare possible in the context of downstream mergers (where combined retailers may bargain for lower input prices) or vertical mergers (where EDM can create benefits).  Once one of the merging firms is allowed to be integrated (1 integrated firm in the figure) beneficial mergers appear.  Consumer welfare increases in almost 30\% of mergers, though median consumer harm is roughly steady at approximately 4\% of pre-merger revenues.  The additional vertical integration creates some opportunities for EDM to enhance welfare, but this must be balanced against the potential for RRC.  Here we find that harm dominates in most simulations.  Adding in integration by rival firms (2-4 integrated firms in the figure) narrows the inter-quartile range and somewhat trims the whiskers while having little impact on median harm.

Next we examine the welfare impacts of downstream mergers in the third panel of Figure \ref{fig:CVvertincumbBW_consumer}. As with upstream mergers, downstream mergers absent incumbent vertical integration (0 integrated firms in the figure) are never beneficial in the $5^{th}$ to $95^{th}$ percentile range, with median consumer harm from downstream mergers equal to about 9\% of pre-merger revenues. \citeN{ST2021} found the same result for mergers in downstream logit Bertrand markets.  Allowing one of the merging firms to be vertically integrated when all other market participants are unintegrated (1 integrated firm in the figure) increases the range of outcomes while also leading to more median harm: consumer welfare increases in about 20\% of mergers, but median consumer harm grows to about 12\% of pre-merger revenues. Once one of the merging firms is already integrated, the merger now has potential EDM and RRC effects, which raises the possibility of both benefits and harms to welfare.  Similar to what we saw for upstream mergers, we find that the harms dominate in most instances.  Adding integrated third party rivals (2-4 integrated firms in the figure) somewhat moves the distribution towards less harm.  With 3 integrated third parties (denoted by 4 integrated firms in the figure), the merger benefits consumers in about 34\% of markets and shrinks median consumer harm to about 5\% of pre-merger revenues.

We finish by examining integrated mergers in the right-most panel of Figure \ref{fig:CVvertincumbBW_consumer}.  Absent the presence of rival incumbent integrated firms (0 integrated firms in the figure), mergers between two integrated firms benefit consumers in about 6\% of simulated markets. Thus, we find that mergers between two firms that are already integrated are harmful to consumers in the vast majority of cases.  Adding one additional rival incumbent integrated firm (1 integrated firm in the figure) does not have a significant impact on the median or on the range of outcomes. Adding more integrated rivals (2-4 integrated firms in the figure) moves median consumer welfare to be slightly more negative.

To summarize, our analysis of Figure \ref{fig:CVvertincumbBW_consumer} yields some overall conclusions.  First, we find that mergers where one or both of the merging firms are already integrated (and all rivals are unintegrated) are harmful to consumers in most instances.  Although these mergers have the possibility of creating benefits through EDM, these gains appear to be outweighed in the majority of the mergers we study.  Second, the presence of integrated rivals has some impact on the distribution of consumer harm, but the observed patterns are noisy and the resulting box plots are largely similar to those where such rivals are absent.  Therefore, having such rival firms in and of itself does not generate significantly better outcomes for consumers.

%%%%%%%%%%%%%%%%%%%%%%%%%%%%%%%%%%%%%%%%%%
\subsection{Mergers and Bargaining Power}
\citeN{ST2021} show using numerical simulations that downstream and vertical mergers when wholesalers have relatively more bargaining power are less harmful compared to when retailers have relatively more bargaining power. Here, we find that this relationship also holds for downstream and upstream mergers when one of the merging parties is already vertically integrated and for integrated mergers.  We also show that the result persists when there are rival non-merging integrated firms.

Figure \ref{fig:CVbargincumbent_consumer} depicts box and whisker plots summarizing the consumer welfare effects for vertical, upstream, downstream, and integrated mergers as the relative bargaining power parameter goes from 9 (wholesalers have the advantage) to 1/9 (retailers have the advantage). Also depicted for each bargaining power parameter are three sets of box and whisker plots that correspond to the number of incumbent vertically integrated firms included in the simulated markets: 0 (light blue), 1 (darker blue), and 4 (darkest blue) firms.  Results for total welfare are in Appendix \ref{app:total}.  For vertical, upstream, and downstream mergers, plots when the number of integrated firms is 0 assume that pre-merger, no firms in the market are vertically integrated; these are comparable to results in \citeN{ST2021}. For vertical mergers, the plots when the number of incumbent integrated firms equals 1 depict the outcome from an unintegrated wholesaler merging with an unintegrated retailer when one third party rival is integrated. For upstream and downstream mergers, the plots when the number of incumbent integrated firms equals 1 depict the outcome of a merger between an integrated firm and an unintegrated firm when all third parties are unintegrated. In turn, plots with 4 integrated firms increase the number of rival incumbent integrated firms in vertical mergers to 4 and in upstream and downstream mergers to 3.  Finally, in integrated mergers, when the number of incumbent integrated firms equals 0, this depicts a merger between two integrated firms with all rivals are unintegrated. Plots with either 1 or 4 integrated firms depict mergers between integrated incumbents when there are either 1 or 4 rival integrated firms in the market, respectively.

We begin by examining vertical mergers in the leftmost panel of Figure \ref{fig:CVbargincumbent_consumer}.  We focus first on the case when there are no incumbent integrated firms (the light blue plots).  Consistent with the patterns seen in \citeN{ST2021}, mergers tend to benefit consumers when wholesalers have relatively more power, and harm consumers when retailers have relatively more power.  Larger wholesaler bargaining power offers more possibilities for EDM, as pre-merger input prices are likely to be high.  This channel becomes less relevant as retailers gain more power.  The impact on consumers is zero in the neighborhood of retailer to wholesaler bargaining power ratios of 1 to 2/3.  As we add 1 and then 4 integrated rival firms (shown by the darker and darkest blue box plots, respectively), the range of welfare outcomes narrows and shifts upwards, though we still see the negative relationship between consumer welfare and relative retailer bargaining power.  Across all three integrated incumbent scenarios, median consumer harm moves from about -24\% of pre-merger expenditures when wholesalers have the advantage to 7\% when retailers have the advantage.

We turn next to upstream mergers in the second panel of Figure \ref{fig:CVbargincumbent_consumer}.  Without incumbent integrated firms in the market (the light blue plots), the relationship between harm and bargaining power is unlike that for downstream and vertical mergers, as it is not monotonic.  By contrast, for mergers between an integrated firm and an unintegrated upstream supplier (the darker blue plots), the relationship between harm and bargaining power is more in line with that for downstream and vertical mergers.  This is intuitive, as now all three types of mergers involve bargaining interactions that can lower input prices, the savings from which are then passed through to consumers to varying degrees. Median consumer harm goes from roughly -21\% of pre-merger expenditures when wholesalers have relatively more bargaining power to almost 8\% of pre-merger expenditures when retailers have relatively more bargaining power. Adding 3 incumbent integrated rivals (the darkest blue plots) preserves the relationship between bargaining power and harm, though the spread of outcomes narrows and the medians shift up slightly.

For downstream mergers, like with vertical mergers, we find that higher retailer relative bargaining power leads to more negative impacts on welfare.  Absent any incumbent integration (the light blue plots), all the outcomes shown in Figure \ref{fig:CVbargincumbent_consumer} harm consumers, which mirrors our findings in Figure \ref{fig:CVvertincumbBW_consumer}.  Median consumer harm goes from about 1\% of pre-merger expenditures when wholesalers have relatively more bargaining power to 19\% of pre-merger expenditures when retailers have relatively more bargaining power. In situations where retailers already have more bargaining power, retailers are likely to have extracted significant surplus from wholesalers prior to the merger, which limits any potential benefits from increased bargaining leverage.  Thus, there is little to counteract harms from decreased downstream competition.

Once we allow one of the merging firms to be integrated (the darker blue plots), some beneficial mergers appear, particularly when wholesalers have relatively more bargaining power.  The box and whisker plots in Figure \ref{fig:CVbargincumbent_consumer} now show a stronger relationship between bargaining power and extent of consumer harm.  Median consumer harm goes from approximately -28\% of pre-merger revenues when wholesalers have the advantage to 24\% of pre-merger revenues when retailers have the advantage.  The existence of pre-merger integration at one of the merging firms creates an opportunity for EDM, the benefits from which are likely to be largest when wholesalers have high bargaining power and therefore charge high pre-merger input prices.  Compared to vertical mergers, where impacts to consumers were roughly neutral at 2/3 bargaining power, here the neutrality crossing shifts left, when wholesalers have a bit more than 3/2 times the power of retailers.  It appears that wholesalers must be relatively more powerful (and thus the likely gains from EDM larger) in order to generate net consumer benefits, compared to in a vertical merger.  This is intuitive, as here the downstream merger also causes an additional lessening in horizontal competition.  Once we add 3 rival integrated firms (the darkest blue plots) the box and whisker plots shift up. The range of outcomes also shrinks.  This is consistent with what we see for vertical and upstream mergers.  Starting from when wholesalers have relatively more bargaining power and moving right, median consumer harm goes from approximately -35\% of pre-merger revenues to 13\% of pre-merger revenues.

The last panel in Figure \ref{fig:CVbargincumbent_consumer} displays results for integrated mergers.  In terms of bargaining power and resulting harm, integrated mergers are perhaps most similar to vertical mergers. Like vertical mergers, there is a strong monotonic relationship between bargaining power and harm. Median consumer harm moves from about 0\% of pre-merger expenditures when wholesalers have relatively more bargaining power to about 26\% of pre-merger expenditures when retailers have relatively more bargaining power. A second similarity between vertical and integrated mergers is that the presence of rival integrated incumbents (see the darker and darkest blue plots) tends to narrow the range of outcomes and somewhat shift the distribution upwards. However, a key difference between vertical and integrated mergers is that while vertical mergers are often beneficial when wholesalers have relatively more bargaining power, integrated mergers are often harmful unless wholesaler relative bargaining power is roughly 4 times or more. The crossing at zero harm is shifted to the left.

In summary, our review of Figure \ref{fig:CVbargincumbent_consumer} has revealed some systematic patterns.  First, we see that vertical mergers, upstream and downstream mergers where one of the merging firms is integrated, and integrated mergers all exhibit a negative relationship between consumer harm and increasing relative retailer bargaining power.  Consumers seem most likely to benefit from these mergers when wholesalers have significant bargaining power, which are situations when we would expect the gains from EDM to be largest.  Second, the point at which this monotonic relationship between consumer welfare and bargaining power crosses zero varies across merger types, and appears to be farthest to the left in integrated mergers.  That is, wholesalers need to have a substantial bargaining power advantage over retailers in order for consumers to benefit from an integrated merger.  Third, the presence of integrated rivals tends to shrink the range of welfare outcomes and to shift this distribution upwards.  This pattern for integrated rivals is much less noisy than in Figure \ref{fig:CVvertincumbBW_consumer}, where we did not condition on relative bargaining power.

%%%%%%%%%%%%%%%%%%%%%%%%%%%%%%%%%%%%%%%%%%
\section{Empirical Application: Republic/Santek \label{sec:application}}

The U.S. waste and recycling industry generates approximately \$80 billion in annual revenues.\footnote{See Waste Dive, \tiny{\url{https://www.wastedive.com/news/public-companies-increased-control-of-74b-us-waste-industry-in-2018/556079/}.}} In recent years the industry has experienced significant merger activity, including several acquisitions between vertically integrated, nationally active competitors. Furthermore, there are a variety of vertical supply chain configurations across local markets, making the industry an excellent application to study the welfare impacts of mergers that have both horizontal and vertical aspects.

Republic Services, the second largest waste management company in the U.S., acquired Santek Environmental in 2021.\footnote{See Competitive Impact Statement: U.S. and State of Alabama v. Republic Services, Inc.and Santek Waste Services, LLC, \url{https://www.justice.gov/atr/case-document/file/1382626/download}.}  The DOJ filed a complaint and negotiated a settlement in the case, alleging horizontal anti-competitive effects for four small-container commercial waste (SCCW) collection markets and two municipal solid waste (MSW) disposal markets. In addition, vertical anti-competitive effects were alleged to arise from the combination of Republic and Santek integrated assets in the Chattanooga region.\footnote{See U.S. and State of Alabama v. Republic Services, Inc. and Santek Waste Services, LLC, \url{https://www.justice.gov/atr/case-document/file/1382031/download}.}  In this section, we apply our model to simulate the effects of the merger in Chattanooga.\footnote{The \href{https://CRAN.R-project.org/package=antitrust}{\texttt{antitrust} R package} contains the computer code needed to run the merger simulations described here.}

%%%%%%%%%%%%%%%%%%%%%%%%%%%%%%%%%%%%%%%%%%
\subsection{Market Background}

The vertical supply chain in the solid waste industry is primarily comprised of waste collection operations or ``haulers'' and waste disposal facilities. Haulers collect MSW from businesses and residences and must dispose of it at a lawful disposal site, usually a landfill. Waste disposal (upstream) is a required input into waste collection services (downstream). Some haulers are vertically integrated and operate their own disposal facilities. Vertically-integrated haulers typically prefer to dispose of waste at their own disposal facilities and may also sell a portion of their disposal capacity. Disposal customers include private waste haulers without their own disposal assets as well as local governments that collect their citizens' waste themselves. Due to strict laws and regulations that govern the disposal of MSW, there are no reasonable substitutes for MSW disposal.

The Competitive Impact Statement (CIS) filed by the DOJ in association with the Republic/Santek merger describes the alleged lost competition in the ``Chattanooga, Tennessee and North Georgia area,'' subsequently referred to as the Chattanooga Area, due to lost horizontal competition in MSW disposal and SCCW collection as well as due to RRC in the SCCW collection market by raising the MSW disposal costs of independent haulers. The CIS notes that pre-merger, Republic and Santek combined served approximately 73 percent of the SCCW collection market with three other significant competitors. In MSW disposal, the CIS identifies only one other significant competitor pre-merger and Republic and Santek combined as serving approximately 82 percent of the market, disposed of either directly in the merging parties' landfills within the area or passing through their transfer stations before ultimately being disposed of in the parties' landfills elsewhere. Thus, pre-merger both parties were large, vertically integrated competitors in the Chattanooga Area.

In addition, another large, vertically integrated company, Waste Connections, existed in the market at the time of the merger and was the parties' sole competitor in MSW disposal. National firms Waste Management and ADS owned collection assets in the area but were not vertically integrated in this market, as demonstrated by MSW disposal data discussed in Appendix \ref{app:trashdata}.\footnote{Firms that are national competitors and vertically integrated in other markets are known to enter into contracts with each other to dispose of waste on advantageous terms that may make them effectively vertically integrated. Ignoring these contracting relationships may underestimate the number of effectively vertically integrated competitors in the market.} We treat Waste Management and ADS as a single entity (henceforth ``WM-ADS'') because they merged with one another in 2020, and that transaction was completed before the filing of the CIS for Republic/Santek. The final significant participant in the SCCW collection market was a major regional firm that is not identified by name in the CIS.  (We label this firm ``Regional'' in what follows.)

Using data described in Appendix \ref{app:trashdata}, the supply relationships between the upstream disposal market participants (Republic, Santek, and Waste Connections) and downstream collection market participants (Republic, Santek, Waste Connections, WM-ADS, and Regional) can be inferred under the assumption that vertically-integrated haulers first dispose of waste at their own disposal facilities and then sell any residual disposal capacity to rival haulers. Santek and Waste Connections exhibit excess disposal capacity. However, Republic's collection volumes are estimated to exceed their ability to self-supply disposal in the Chattanooga market. As a result, Republic needs to purchase additional disposal from Santek and Waste Connections, presumably at a higher marginal cost. We assume that this disposal is purchased from both Santek and Waste connections according to their shares of available excess MSW disposal capacity of 63.8\% and 36.2\%, respectively. We further assume that WM-ADS and Regional purchase disposal capacity from both Santek and Republic according to these shares. The resulting relationships between the upstream and downstream market participants are captured in Figure \ref{fig:chatt_tree}.

Our final input data set is presented in Table \ref{tab:trashdata}.  As discussed above and in Appendix \ref{app:trashdata}, we need to make several simplifying assumptions in order to transform the available information into the data required for our model.  We expect that the DOJ would have had access to more accurate confidential information during their investigation.  Therefore, we view our results as only illustrative of the types of analysis and results that could have been produced for this case.  Note that, other than imposing that Republic only offers disposal to itself, we do not structurally incorporate capacity constraints into the simulation model. We assume that the changes in volume that would be caused by the merger would not be large enough to shift which constraints would be binding.\footnote{We also assume that the Waste Connections-Republic product remains on the market, which may only be true in the short run.}

%%%%%%%%%%%%%%%%%%%%%%%%%%%%%%%%%%%%%%%%%%
\subsection{Merger Simulation Results}

Overall, the merger between Republic and Santek is estimated to lower upstream prices by 2.8\% and increase downstream prices by 12\%, resulting in \$16 million of annual harm to consumers and total annual producer benefits of approximately \$13.4 million. Thus, on net the model predicts the merger would be harmful despite the presence of significant EDM that reduces prices upstream.\footnote{Our calibrated bargaining parameters for retailers against Santek ($\lambda=1$) and Republic ($\lambda=0.7$) give a substantial edge to downstream firms, which, based on our analysis of the previous section, suggests consumers are likely to be harmed.  High bargaining power parameters are needed to explain the high collection margin compared to the disposal margin.  (See Table \ref{tab:trashdata}.)}  These effects are illustrated in more detail in Figure \ref{fig:TrashSimsFirms}.

Disposal prices for pre-merger integrated pairs are unchanged post-merger. However, MSW disposal prices increase to all rival haulers post-merger (i.e., all unintegrated pairs). The merged firms' disposal prices to downstream rivals are estimated to increase 88\% post-merger, and Waste Connections' prices to its downstream rivals are estimated to increase 22\%. These large price increases are offset by the large decrease in the disposal price for the integrating pair post-merger due to EDM, as the cost to Republic for volume disposed with Santek decreases by 60\%.

Turning to the downstream market, all collection prices increase post-merger. The merging parties' post-merger collection prices increase by approximately 20\%, and their share decreases by about 8 percentage points overall. The Republic/Republic product's price increases 21\%, and the Santek/Santek product's price increases by 23\%. The newly integrated Santek/Republic product's price is estimated to increase by 9\%, demonstrating that EDM is not fully passed-through to consumers. The integrated rival, Waste Connections, increases its collection price by approximately 5\%, and its share increases by 6 percentage points. The unintegrated rivals' collection prices each increase by about 9\%, which again demonstrates incomplete pass-through of the change in upstream costs, and their shares increase slightly.

In Figure \ref{fig:TrashSimsCompareAll} we compare the full vertical model results, which compute both post-merger downstream and upstream price effects, to two other models that hold the prices of one level fixed.\footnote{This approach uses the same calibrated demand and costs parameters across all three models.} The ``downstream-only'' model computes collection price effects holding disposal prices fixed at pre-merger levels, and the ``upstream-only'' model computes disposal price effects holding collection prices fixed at pre-merger levels.

Ignoring these vertical relationships in the downstream-only model results in an estimated 11\% increase in collection prices with a greater than 16\% increase in the merging parties' price, consumer harm of \$10 million annually, and net harm of approximately \$2 million. Estimated consumer harm is 60\% higher in the full vertical model and is offset by markedly larger producer benefits. The downstream-only model misses the substantial RRC effect experienced by the Santek/WM-ADS and Santek/Regional products.  In the vertical model, the downstream prices for these products rise more dramatically due to this RRC, and other prices increase alongside them.  This causes more harm to consumers.

In the upstream-only model, downstream prices and therefore shares are fixed at their pre-merger levels. We still allow for EDM for integrated products.  Overall disposal prices increase by 8.2\% post-merger, and the merger effect is a pure transfer between firms. Thus, both consumer and net harm are estimated to be zero.  The main impact of the merger is in changing the bargaining leverage for Santek, now that it is combined with Republic.  This effect causes Santek to raise its upstream disposal prices to both WM-ADS and Regional.  The price increase is even larger in the upstream-only model compared to the full model, as the lack of downstream price adjustments means that WM-ADS and Regional cannot shift their sales towards their cheaper supplier, Waste Connections, in equilibrium, which makes having access to Santek's disposal facilities relatively more valuable than in the full model, driving up Santek's price.  In this sense, the upstream-only model overstates the increase in leverage that Santek receives from the merger.

The Republic/Santek merger was ultimately settled with the DOJ through a divestiture. According to the Final Judgment, in the Chattanooga Area the parties were required to divest Santek's SCCW collection assets  as well as two landfills and a transfer station.\footnote{See Final Judgment: U.S and Plaintiff States v. Republic Services, Inc. and Santek, LLC, \url{https://www.justice.gov/atr/case-document/file/1408616/download}.} Our results suggest that a divestiture in the collection market alone likely would not have sufficiently remedied the anti-competitive effects of the merger.

%%%%%%%%%%%%%%%%%%%%%%%%%%%%%%%%%%%%%%%%%%%%%%%%%%%%%%%%%%%%%%%%%%%%%%%%%%%%%%%%%
\section{Conclusion \label{sec:concl}}

This paper relaxes an assumption made in much of the merger simulation literature: that mergers occur in vertical supply chains where in the pre-merger state, no firms are vertically integrated. Removing this limitation is important as competition authorities are routinely called upon to investigate mergers where either the merging parties or a third party are vertically integrated. Using a series of numerical simulations and the model of \citeN{ST2021}, we find that most mergers involving one or more firms that are already vertically integrated harm consumers.  The exceptions occur when wholesalers have more bargaining power compared to retailers, which are instances where the benefits from EDM would tend to be larger.  We also show that having integrated non-merging rivals limits the range of welfare outcomes and tends to make mergers somewhat more beneficial to consumers, although these firms' presence is not a panacea for consumer harm. We apply our model to the Republic/Santek merger, where we find that ignoring existing vertical relationships gives an incomplete picture of the merger's welfare impacts.

Our results suggest that mergers that have horizontal aspects remain harmful for consumers in many instances, even when these mergers also have vertical aspects.
This appears to be especially true when retailers have relatively more bargaining power, which in our model occurs when retailer margins are relatively large compared to wholesaler margins.  Antitrust agencies may have reason to be especially cautious when reviewing mergers in these situations.  However, we of course have the caveat that our results are specific to our modeling framework.  An important question for future research would be to explore merger outcomes in other contexts.

%%%%%%%%%%%%%%%%%%%%%%%%%%%%%%%%%%%%%%%%%%%%%%%%%%%%%%%%%%%%%%%%%%%%%%%%%%%%%%%%%
% Bibliography
%%%%%%%%%%%%%%%%%%%%%%%%%%%%%%%%%%%%%%%%%%%%%%%%%%%%%%%%%%%%%%%%%%%%%%%%%%%%%%%%%
\newpage
\bibliographystyle{chicago}
\bibliography{master-bib}
\clearpage

%%%%%%%%%%%%%%%%%%%%%%%%%%%%%%%%%%%%%%%%%%%%%%%%%%%%%%%%%%%%%%%%%%%%%%%%%%%%%%%%%
% Tables and Figures
%%%%%%%%%%%%%%%%%%%%%%%%%%%%%%%%%%%%%%%%%%%%%%%%%%%%%%%%%%%%%%%%%%%%%%%%%%%%%%%%%
\begin{table}
\caption{Summary Statistics for Numerical Simulations}
\label{tab:simsum}
\small{
\begin{tabular}{l|l|l|r}
\hline
variable & merger & quant & val\\
\hline
down &  & Min & 2\\

up &  & Min & 2\\

vert &  & Min & 0\\

barg &  & Min & 0\\

nestParm &  & Min & 0\\

avgpricepre.up &  & Min & 1\\

avgpricepre.down &  & Min & 6\\

mktElast & \multirow{-8}{*}{\raggedright\arraybackslash all} & Min & -60\\
\cline{1-4}
hhipre &  & Min & 2008\\

hhipost &  & Min & 2915\\

hhidelta & \multirow{-3}{*}{\raggedright\arraybackslash up} & Min & 0\\
\cline{1-4}
hhipre &  & Min & 2011\\

hhipost &  & Min & 2931\\

hhidelta & \multirow{-3}{*}{\raggedright\arraybackslash down} & Min & 0\\
\cline{1-4}
hhipre &  & Min & 2100\\

hhipost &  & Min & 3120\\

hhidelta & \multirow{-3}{*}{\raggedright\arraybackslash vertical} & Min & 32\\
\cline{1-4}
hhipre &  & Min & 2205\\

hhipost &  & Min & 3633\\

hhidelta & \multirow{-3}{*}{\raggedright\arraybackslash both} & Min & 2\\
\cline{1-4}
down &  & p25 & 3\\

up &  & p25 & 3\\

vert &  & p25 & 0\\

barg &  & p25 & 0\\

nestParm &  & p25 & 0\\

avgpricepre.up &  & p25 & 2\\

avgpricepre.down &  & p25 & 10\\

mktElast & \multirow{-8}{*}{\raggedright\arraybackslash all} & p25 & -1\\
\cline{1-4}
hhipre &  & p25 & 2393\\

hhipost &  & p25 & 4011\\

hhidelta & \multirow{-3}{*}{\raggedright\arraybackslash up} & p25 & 1546\\
\cline{1-4}
hhipre &  & p25 & 2572\\

hhipost &  & p25 & 4135\\

hhidelta & \multirow{-3}{*}{\raggedright\arraybackslash down} & p25 & 1431\\
\cline{1-4}
hhipre &  & p25 & 3623\\

hhipost &  & p25 & 5069\\

hhidelta & \multirow{-3}{*}{\raggedright\arraybackslash vertical} & p25 & 1051\\
\cline{1-4}
hhipre &  & p25 & 3591\\

hhipost &  & p25 & 6242\\

hhidelta & \multirow{-3}{*}{\raggedright\arraybackslash both} & p25 & 2463\\
\cline{1-4}
down &  & p50 & 4\\

up &  & p50 & 4\\

vert &  & p50 & 1\\

barg &  & p50 & 1\\

nestParm &  & p50 & 0\\

avgpricepre.up &  & p50 & 5\\

avgpricepre.down &  & p50 & 13\\

mktElast & \multirow{-8}{*}{\raggedright\arraybackslash all} & p50 & -1\\
\cline{1-4}
hhipre &  & p50 & 2876\\

hhipost &  & p50 & 4963\\

hhidelta & \multirow{-3}{*}{\raggedright\arraybackslash up} & p50 & 2027\\
\cline{1-4}
hhipre &  & p50 & 3166\\

hhipost &  & p50 & 5207\\

hhidelta & \multirow{-3}{*}{\raggedright\arraybackslash down} & p50 & 2040\\
\cline{1-4}
hhipre &  & p50 & 4391\\

hhipost &  & p50 & 6016\\

hhidelta & \multirow{-3}{*}{\raggedright\arraybackslash vertical} & p50 & 1292\\
\cline{1-4}
hhipre &  & p50 & 4275\\

hhipost &  & p50 & 7272\\

hhidelta & \multirow{-3}{*}{\raggedright\arraybackslash both} & p50 & 2800\\
\cline{1-4}
down &  & p75 & 5\\

up &  & p75 & 5\\

vert &  & p75 & 2\\

barg &  & p75 & 1\\

nestParm &  & p75 & 0\\

avgpricepre.up &  & p75 & 10\\

avgpricepre.down &  & p75 & 19\\

mktElast & \multirow{-8}{*}{\raggedright\arraybackslash all} & p75 & 0\\
\cline{1-4}
hhipre &  & p75 & 3794\\

hhipost &  & p75 & 6780\\

hhidelta & \multirow{-3}{*}{\raggedright\arraybackslash up} & p75 & 2891\\
\cline{1-4}
hhipre &  & p75 & 4257\\

hhipost &  & p75 & 7391\\

hhidelta & \multirow{-3}{*}{\raggedright\arraybackslash down} & p75 & 2934\\
\cline{1-4}
hhipre &  & p75 & 5679\\

hhipost &  & p75 & 7293\\

hhidelta & \multirow{-3}{*}{\raggedright\arraybackslash vertical} & p75 & 1755\\
\cline{1-4}
hhipre &  & p75 & 5358\\

hhipost &  & p75 & 8648\\

hhidelta & \multirow{-3}{*}{\raggedright\arraybackslash both} & p75 & 3187\\
\cline{1-4}
down &  & Max & 5\\

up &  & Max & 5\\

vert &  & Max & 4\\

barg &  & Max & 1\\

nestParm &  & Max & 0\\

avgpricepre.up &  & Max & 270\\

avgpricepre.down &  & Max & 297\\

mktElast & \multirow{-8}{*}{\raggedright\arraybackslash all} & Max & 0\\
\cline{1-4}
hhipre &  & Max & 10000\\

hhipost &  & Max & 10000\\

hhidelta & \multirow{-3}{*}{\raggedright\arraybackslash up} & Max & 5000\\
\cline{1-4}
hhipre &  & Max & 10000\\

hhipost &  & Max & 10000\\

hhidelta & \multirow{-3}{*}{\raggedright\arraybackslash down} & Max & 5000\\
\cline{1-4}
hhipre &  & Max & 9962\\

hhipost &  & Max & 10000\\

hhidelta & \multirow{-3}{*}{\raggedright\arraybackslash vertical} & Max & 4314\\
\cline{1-4}
hhipre &  & Max & 9998\\

hhipost &  & Max & 10000\\

hhidelta & \multirow{-3}{*}{\raggedright\arraybackslash both} & Max & 5000\\
\cline{1-4}
down &  & Markets & 2202997\\

up &  & Markets & 2202997\\

vert &  & Markets & 2202997\\

barg &  & Markets & 2202997\\

nestParm &  & Markets & 2202997\\

avgpricepre.up &  & Markets & 2202997\\

avgpricepre.down &  & Markets & 2202997\\

mktElast & \multirow{-8}{*}{\raggedright\arraybackslash all} & Markets & 2202997\\
\cline{1-4}
hhipre &  & Markets & 654731\\

hhipost &  & Markets & 654731\\

hhidelta & \multirow{-3}{*}{\raggedright\arraybackslash up} & Markets & 654731\\
\cline{1-4}
hhipre &  & Markets & 701936\\

hhipost &  & Markets & 701936\\

hhidelta & \multirow{-3}{*}{\raggedright\arraybackslash down} & Markets & 701936\\
\cline{1-4}
hhipre &  & Markets & 462005\\

hhipost &  & Markets & 462005\\

hhidelta & \multirow{-3}{*}{\raggedright\arraybackslash vertical} & Markets & 462005\\
\cline{1-4}
hhipre &  & Markets & 384325\\

hhipost &  & Markets & 384325\\

hhidelta & \multirow{-3}{*}{\raggedright\arraybackslash both} & Markets & 384325\\
\hline
\end{tabular}
}
\end{table}

\begin{table}
\caption{Republic/Santek Merger Simulation Inputs}
\small{%\caption{\label{tab:trashdata}Republic/Santek Merger Simulation Inputs. Volume is reported in thousands of pounds, while prices and margins are reported in dollars.}
%\centering
\resizebox{\linewidth}{!}{
\begin{tabular}[t]{rrccccc}
\toprule
Disposal Firm & Collection Firm & Volume & Disposal Price & Disposal Margin & Collection Margin & Collection Cost\\
\midrule
Republic & Republic & 165 & 42 & 0 & 44 & 93\\
\cmidrule{1-7}
 & Republic & 34 & 36 & 20 &  & 93\\

 & Santek & 218 & 16 & 0 &  & 74\\

 & WMADS & 30 & 36 & 20 &  & 148\\

\multirow{-4}{*}{\raggedleft\arraybackslash Santek} & Regional & 30 & 36 & 20 &  & 148\\
\cmidrule{1-7}
 & Republic & 19 & 25 & 14 &  & 93\\

 & WasteConn & 48 & 11 & 0 &  & 63\\

 & WMADS & 17 & 25 & 14 &  & 148\\

\multirow{-4}{*}{\raggedleft\arraybackslash WasteConn} & Regional & 17 & 25 & 14 &  & 148\\
\bottomrule
\end{tabular}}
}
\footnotesize{Notes: ``WasteConn'' stands for Waste Connections.  Volume is reported in thousands of tons, whereas prices and margins are reported in dollars per ton.}
\label{tab:trashdata}
\end{table}

\begin{sidewaysfigure}
\centering
\includegraphics[scale=0.9]{output/surplussum.png}
\caption{The figure displays box and whisker plots summarizing the extent to which mergers affect consumer, retailer, wholesaler, and total surplus. Whiskers depict the $5^{th}$ and $95^{th}$ percentiles of a particular outcome, boxes depict the $25^{th}$ and $75^{th}$ percentiles, and the solid horizontal line depicts the median. }
\label{fig:surplussum}
\end{sidewaysfigure}

\begin{sidewaysfigure}
\centering
\includegraphics[scale=0.9]{output/CVvertincumbBW_consumer.png}
\caption{The figure displays box and whisker plots summarizing the extent to which mergers affect consumer surplus as the number of vertically integrated firms present in a market changes.  Whiskers depict the $5^{th}$ and $95^{th}$ percentiles of a particular outcome, boxes depict the $25^{th}$ and $75^{th}$ percentiles, and the solid horizontal line depicts the median. }
\label{fig:CVvertincumbBW_consumer}
\end{sidewaysfigure}

\begin{sidewaysfigure}
\centering
\includegraphics[scale=0.9]{output/CVbargincumbent_consumer.png}
\caption{The figure displays box and whisker plots summarizing the extent to which vertical, upstream, downstream and integrated mergers affect consumer surplus as the bargaining power of wholesalers relative to retailers changes. The different colored boxes display how outcomes change as the number of vertically  integrated firms changes. Whiskers depict the $5^{th}$ and $95^{th}$ percentiles of a particular outcome, boxes depict the $25^{th}$ and $75^{th}$ percentiles, and the solid horizontal line depicts the median.}
\label{fig:CVbargincumbent_consumer}
\end{sidewaysfigure}

\begin{figure}
\centering
\includegraphics[scale=1.0]{output/chattanooga_tree.png}
\caption{Structure of the Chattanooga Area Disposal and Collection Market}
\label{fig:chatt_tree}
\end{figure}


\begin{figure}
\centering
\caption{Republic/Santek Simulation Results}
\includegraphics[scale=1.0]{output/TrashSimsFirm.png}\\
\footnotesize{Notes: ``WasteConn'' stands for Waste Connections.}
\label{fig:TrashSimsFirms}
\end{figure}

\begin{figure}
\centering
\caption{Comparison between Downstream-Only, Upstream-Only, and Full Vertical Simulations for Republic/Santek}
\includegraphics{output/TrashSimsCompareAll.png}\\
\footnotesize{Notes: ``WasteConn'' stands for Waste Connections.}
\label{fig:TrashSimsCompareAll}
\end{figure}


%%%%%%%%%%%%%%%%%%%%%%%%%%%%%%%%%%%%%%%%%%%%%%%%%%%%%%%%%%%%%%%%%%%%%%%%%%%%%%%%%
% Appendix
%%%%%%%%%%%%%%%%%%%%%%%%%%%%%%%%%%%%%%%%%%%%%%%%%%%%%%%%%%%%%%%%%%%%%%%%%%%%%%%%%
\appendix
\newpage
\clearpage

\numberwithin{equation}{section}
\numberwithin{figure}{section}
\numberwithin{table}{section}

%%%%%%%%%%%%%%%%%%%%%%%%%%%%%%%%%%%%%%%%%%%%%%%%%%%%%%%%%%%%%%%%%%%%%%%%%%%%%%%%%
\section{Appendix: Increasing Marginal Costs\label{app:mc}}

Here, we explore the role that the constant marginal cost assumption plays in driving our results by comparing our simulations to an alternative set that allows costs to increase linearly. Figure \ref{fig:surplussumcost} displays box and whisker plots for the net effect on consumer and total welfare under four different scenarios: constant marginal costs for all firms (top row, blue), linear marginal costs for all firms (top row, orange), constant marginal costs for the merging parties' products but linear costs for the non-merging parties' products (bottom row, blue), and linear marginal costs for the merging parties' products but constant costs for the non-merging parties' products (bottom row, orange).

In theory, having linear marginal costs rather than constant costs could result in fewer gains from EDM, as efforts to increase sales due to a wholesale price reduction would be counteracted by rising costs. Likewise, in theory, RRC could be more profitable for the merging parties under linear compared to constant marginal costs, as linear marginal costs could cause input prices to increase more rapidly or downstream rival retailers to reduce output more quickly. Because impacts from EDM are plausibly less under linear marginal costs while those from RRC are plausibly more, one might expect that on net mergers are more harmful under linear rather than constant marginal costs.

Our simulations confirm this hypothesis.  See the top two panels in Figure \ref{fig:surplussumcost}, which compare a scenario where all firms have constant marginal costs to one where they all have linear marginal costs.  Overall, harm is greater for mergers with linear costs. The difference is largest for vertical mergers.  When all firms have constant marginal costs, vertical mergers increase consumer surplus in about 55\% of simulations and total surplus in about 29\% of simulations, but only increase consumer surplus in about 36\% of simulations and total surplus in about 2\% of simulations when all firms have linear costs.

The bottom panel of Figure \ref{fig:surplussumcost} compares a situation where only the merging firms have constant marginal costs to one where only the merging firms have linear marginal costs.  We see that the distributions in these bottom panels are largely the same as those in the top panels.  Thus, it appears that the key driver of our results is whether or not the merging parties have constant marginal costs.  When the merging firms' marginal costs are constant, the distribution of outcomes moves towards less harm.

\begin{sidewaysfigure}
\centering
\includegraphics[scale=0.9]{output/surplussum_cost.png}
\caption{The figure displays box and whisker plots summarizing the extent to which merger outcomes change according to 4 different cost scenarios. The boxes in the top row (``All") either assume all firms face either constant marginal costs (blue, left) or linear marginal costs (orange, right). The boxes in the bottom row (``Party") either assume that the merging parties face constant marginal costs while other firms face linear marginal costs (blue, left), or the merging parties face linear marginal costs while other firms face constant marginal costs (orange, right). Whiskers depict the $5^{th}$ and $95^{th}$ percentiles of a particular outcome, boxes depict the $25^{th}$ and $75^{th}$ percentiles, and the solid horizontal line depicts the median. }
\label{fig:surplussumcost}
\end{sidewaysfigure}

%%%%%%%%%%%%%%%%%%%%%%%%%%%%%%%%%%%%%%%%%%%%%%%%%%%%%%%%%%%%%%%%%%%%%%%%%%%%%%%%%
\newpage
\section{Appendix: Data for Republic/Santek\label{app:trashdata}}

Data on MSW disposal come from information collected by the Tennessee Department of Environmental Quality (TDEQ) consisting of Class 1 landfill ownership, Class 1 Solid Waste Origin Reports, and waste disposal by county for 2019. These data identify the origin county and destination landfill, including owner and operator information, for all MSW produced in Tennessee as well as waste volumes passing through transfer stations in Tennessee. The MSW disposal market definition follows that outlined in the DOJ CIS and attributes share to the company owning the final disposal landfill (i.e., ignoring transfer stations that are an intermediate disposal site only). Thus, the total market quantity is defined as all MSW volumes originating in Hamilton County, Tennessee where Chattanooga is located.

Disposal volumes are measured in tons and have been combined across landfills owned by the same firm. The TDEQ data identify landfills owned or operated by Republic, Santek, Waste Connections, and three other market participants (each with less than 0.5\% share) receiving MSW volumes originating in Hamilton County.\footnote{The City of Chattanooga and Marion County landfills are municipally owned and operated, accounting for less than 1\% share combined. Global Envirotech is a privately-owned transfer station that sends its 0.03\% share to an out-of-state landfill in Georgia.} These three fringe participants have been excluded from the analysis. After re-scaling, the resulting market shares are Republic, 28.4\%, Santek, 54\%, and Waste Connections, 17.6\%.

However, the market shares required for implementation of our model are expressed as upstream-downstream market participant pairs. Following the discussion of supply relationships in Section \ref{sec:application}, vertically-integrated pairs without capacity constraints (i.e., Santek disposal and collection, Waste Connection disposal and collection) are assigned their full collection share. Capacity constrained integrated firms (Republic disposal and collection) are assigned their collection share up to their available capacity with the remainder allocated by residual disposal share to pairs with the respective upstream firms (i.e., Santek disposal and Republic collection, Waste Connections disposal and Republic collection). Unintegrated firms' collection shares are  distributed among the upstream suppliers with available capacity by residual share as well.

MSW disposal prices are collected from the 2019 Waste Business Journal's Directory of Waste Processing \& Disposal Sites. The measure of price is the ``gate rate,'' which is the posted price at the landfill, measured in dollars per ton.\footnote{Disposal prices for large customers may be bilaterally negotiated instead of paying the gate rate.} Republic and Santek both operate multiple landfills in the market with different prices. The price used in the analysis for each is the volume-weighted average for their landfills.

The CIS states that in the Chattanooga Area the post-merger HHI for SCCW collection would be approximately 5,551 post-merger with an increase of 2,660 points and that the combined market share of the merging parties is 73\%. Taking these figures as given we can recover the collection market shares under the assumptions that 1) the merging parties are of equal size, and 2) the non-merging parties are of equal size. After re-scaling, this produces downstream market shares for Republic, 37.6\%, Santek, 37.6\%, Waste Connections, 8.3\%, WM-ADS, 8.3\%, and Regional, 8.3\%.

Collection and disposal margins are calculated for Republic from data on revenue by line of service and from components of the cost of operations in the company's 2019 annual report. The data are reported at the company level and are not specific to the Chattanooga Area, but revenues are reported by collection segment.  Collection costs in dollars per ton are estimated from Republic and Waste Management 2019 10k financial statements. These costs are reported at the company level across all segments. The estimated share of these costs attributable to the Chattanooga market, based on the number of markets in which the companies operate, is divided by the tons disposed for each company. Santek, Waste Connections, and Regional do not publicly produce comparable financial statements. Instead, we assume that the cost structure is the same for the vertically-integrated companies, taking into account their individual tons disposed, and that WM-ADS and Regional share the same cost structure due to their lack of internal disposal capacity.

%%%%%%%%%%%%%%%%%%%%%%%%%%%%%%%%%%%%%%%%%%%%%%%%%%%%%%%%%%%%%%%%%%%%%%%%%%%%%%%%%
\section{Appendix: Additional Figures\label{app:total}}

Figures \ref{fig:CVvertincumbBW_total} and \ref{fig:CVbargincumbent_total} are the analogous results as in Figures \ref{fig:CVvertincumbBW_consumer} and \ref{fig:CVbargincumbent_consumer}, but for total welfare. The patterns for total welfare are largely similar to those for consumer welfare.  We see in Figure \ref{fig:CVvertincumbBW_total} that most upstream, downstream, and integrated mergers are harmful.  Figure \ref{fig:CVbargincumbent_total} shows that net beneficial
mergers typically only occur when wholesalers have relatively more bargaining power.  The presence of integrated rivals can narrow the range of outcomes.

\begin{sidewaysfigure}
\centering
\includegraphics[scale=0.9]{output/CVvertincumbBW_total.png}
\caption{The figure displays box and whisker plots summarizing the extent to which mergers affect total surplus as the number of vertically integrated firms present in a market changes.  Whiskers depict the $5^{th}$ and $95^{th}$ percentiles of a particular outcome, boxes depict the $25^{th}$ and $75^{th}$ percentiles, and the solid horizontal line depicts the median. }
\label{fig:CVvertincumbBW_total}
\end{sidewaysfigure}

\begin{sidewaysfigure}
\centering
\includegraphics[scale=0.9]{output/CVbargincumbent_total.png}
\caption{The figure displays box and whisker plots summarizing the extent to which vertical, upstream, downstream and integrated mergers affect total surplus as the bargaining power of wholesalers relative to retailers changes. The different colored boxes display how outcomes change as the number of vertically  integrated firms changes. Whiskers depict the $5^{th}$ and $95^{th}$ percentiles of a particular outcome, boxes depict the $25^{th}$ and $75^{th}$ percentiles, and the solid horizontal line depicts the median.}
\label{fig:CVbargincumbent_total}
\end{sidewaysfigure}


\end{document}

\begin{figure}
\centering
\includegraphics{output/TrashSimsCompare.png}
\caption{Comparison between Collection-Only and Full Vertical Simulations for Republic/Santek}
\label{fig:TrashSimsCompare}
\end{figure}

%Concerns about vertical competitive effects were raised in both the Republic-Santek and the Waste Management-ADS transactions. Both the Solid Waste Agency of Lake County, IL and the Solid Waste Agency of Northern Cook County submitted comments in opposition to the proposed asset divestiture from Waste Management-ADS, stating that a vertically integrated competitor was needed to maintain competition in their local market post-merger.\footnote{\tiny{See  Comments by SWALCO and SWANCC : U.S. and Plaintiff States v. Waste Management, Inc., and Advanced Disposal Services, Inc., \url{https://www.justice.gov/atr/case-document/file/1377646/download}.}}
%For example, in the DOJ complaint filed in the Republic-Santek case, horizontal anti-competitive effects were alleged for four SCCW collection markets and two MSW disposal markets. In addition, vertical anti-competitive effects were alleged to arise from the combination of their integrated assets in the Chattanooga area.\footnote{See U.S. and State of Alabama v. Republic Services, Inc. and Santek Waste Services, LLC, \url{https://www.justice.gov/atr/case-document/file/1382031/download}}
%Our application in progress further identifies local markets in which these acquisitions result in: 1) only horizontal combinations of assets, 2) only vertical combinations of assets, and 3) combinations of vertically integrated assets in the presence of existing integrated competitors to analyze the welfare effects from mergers with complex vertical arrangements. The next sections present preliminary results on the merger of vertically integrated assets in the presence of other integrated and unintegrated rivals in the context of Republic and Santek's merger in the Chattanooga area.
%The next sections present results of applying the model of Section \ref{sec:theory} to the merger of vertically integrated assets in the presence of other integrated and unintegrated rivals in the context of Republic and Santek's merger in the Chattanooga area.

% \begin{sidewaysfigure}
% \centering
% \includegraphics[scale=0.9]{../CVvertincumb_updownBW.png}
% \caption{The figure displays box and whisker plots summarizing the extent to which mergers affect consumer (blue,left) and total (orange,right) surplus as the number of vertically integrated firms present in a market change.  Whiskers depict the $5^{th}$ and $95^{th}$ percentiles of a particular outcome, boxes depict the $25^{th}$ and $75^{th}$ percentiles, and the solid horizontal line depicts the median. }
% \label{fig:CVvertincumbupdownBW}
% \end{sidewaysfigure}
%
% \begin{sidewaysfigure}
% \centering
% \includegraphics[scale=0.9]{../CVvertincumb_vertincBW.png}
% \caption{The figure displays box and whisker plots summarizing the extent to which mergers affect consumer (blue,left) and total (orange,right) surplus as the number of vertically integrated firms present in a market change.  Whiskers depict the $5^{th}$ and $95^{th}$ percentiles of a particular outcome, boxes depict the $25^{th}$ and $75^{th}$ percentiles, and the solid horizontal line depicts the median. }
% \label{fig:CVvertincumbvertincBW}
% \end{sidewaysfigure}

% \begin{sidewaysfigure}
% \centering
% \includegraphics[scale=0.9]{../CVbargupBW.png}
% \caption{The figure displays box and whisker plots summarizing the extent to which mergers among an integrated and unintegrated wholesaler affect consumer, retailer, wholesaler, and total surplus as the bargaining power of wholesalers relative to retailers changes. The different colored boxes display how outcomes change as the number of vertically  integrated firms increases. Whiskers depict the $5^{th}$ and $95^{th}$ percentiles of a particular outcome, boxes depict the $25^{th}$ and $75^{th}$ percentiles, and the solid horizontal line depicts the median.}
% \label{fig:CVbargupBW}
% \end{sidewaysfigure}
%
% \begin{sidewaysfigure}
% \centering
% \includegraphics[scale=0.9]{../CVbargdownBW.png}
% \caption{The figure displays box and whisker plots summarizing the extent to which mergers among an integrated and unintegrated retailer affect consumer, retailer, wholesaler, and total surplus as the bargaining power of wholesalers relative to retailers changes. The different colored boxes display how outcomes change as the number of vertically  integrated firms increases. Whiskers depict the $5^{th}$ and $95^{th}$ percentiles of a particular outcome, boxes depict the $25^{th}$ and $75^{th}$ percentiles, and the solid horizontal line depicts the median.}
% \label{fig:CVbargdownBW}
% \end{sidewaysfigure}
%
% \begin{sidewaysfigure}
% \centering
% \includegraphics[scale=0.9]{../CVbargvertBW.png}
% \caption{The figure displays box and whisker plots summarizing the extent to which mergers among an unintegrated wholesaler and unintegrated retailer affect consumer, retailer, wholesaler, and total surplus as the bargaining power of wholesalers relative to retailers changes. The different colored boxes display how outcomes change as the number of vertically  integrated firms increases. Whiskers depict the $5^{th}$ and $95^{th}$ percentiles of a particular outcome, boxes depict the $25^{th}$ and $75^{th}$ percentiles, and the solid horizontal line depicts the median.}
% \label{fig:CVbargvertBW}
% \end{sidewaysfigure}

% \begin{sidewaysfigure}
% \centering
% \includegraphics[scale=0.85]{CVbargbothBW.png}
% \caption{The figure displays box and whisker plots summarizing the extent to which mergers among two integrated wholesalers and retailers affect consumer, retailer, wholesaler, and total surplus as the bargaining power of wholesalers relative to retailers changes. The different colored boxes display how outcomes change as the number of vertically  integrated firms increases. Whiskers depict the $5^{th}$ and $95^{th}$ percentiles of a particular outcome, boxes depict the $25^{th}$ and $75^{th}$ percentiles, and the solid horizontal line depicts the median.}
% \label{fig:CVbargbothBW}
% \end{sidewaysfigure}
