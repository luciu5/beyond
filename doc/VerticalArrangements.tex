\documentclass[12pt]{article}
%\documentclass[12pt,review,authoryear]{elsarticle}
%\usepackage{ecrc}       % needed for elsarticle
\usepackage{chicago}    % bibliography package
\usepackage{graphicx}   % insert PostScript figures
\usepackage{setspace}   % controls line spacing
\usepackage{amsmath,amsthm,amssymb,amstext} % controls equation entry and symbols
\usepackage{rotating}   % rotates graphics
\usepackage{soul}       % controls hyphenation
\usepackage{epsfig}     % helps with including graphics
\usepackage{pdflscape}  % helps with displaying rotated graphics in PDFs
\usepackage{lscape}     % helps with rotating pages
\usepackage{caption}    % controls  captions
\usepackage{adjustbox}  % shrink tables
\usepackage[margin=1in]{geometry} % margins
\usepackage{hyperref}   % displays URLs
%packages for kable tables
 \usepackage{booktabs}
 \usepackage{longtable}
 \usepackage{array}
 \usepackage{multirow}
 \usepackage{wrapfig}
 \usepackage{float}
 \usepackage{colortbl}
 \usepackage{pdflscape}
 \usepackage{tabu}
 \usepackage{threeparttable}
 \usepackage{threeparttablex}
 \usepackage[normalem]{ulem}
 \usepackage{makecell}
 \usepackage{xcolor}

\graphicspath{{../output/}}

\interfootnotelinepenalty=10000

\parskip     2.0mm       % space between paragraphs

\alph{footnote}         % make title footnotes alpha-numeric

\captionsetup[figure]{labelsep=space,labelfont=bf} % remove colon from figure name

%%%%%%%%%%%%%%%%%%%%%%%%%%%%%%%%%%%%%%%%%%%%%%%%%%%%%%%%%%%%%%%%%%%%%%%%%%%
% Title Page
%%%%%%%%%%%%%%%%%%%%%%%%%%%%%%%%%%%%%%%%%%%%%%%%%%%%%%%%%%%%%%%%%%%%%%%%%%%

%\newcommand*\samethanks[1][\value{footnote}]{\footnotemark[#1]}
%\author{Margaret Loudermilk\footnote{U.S. Department of Justice, margaret.loudermilk@usdoj.gov} \\  U.S Department of Justice \and Gloria Sheu\footnote{Board of Governors of the Federal Reserve System, gloria.sheu@frb.gov. } \\ Federal Reserve Board \and  Charles Taragin\footnote{Board of Governors of the Federal Reserve System, charles.s.taragin@frb.gov}  \\  Federal Reserve Board }

% \begin{frontmatter}
%
% %% Title, authors and addresses
%
% %% use the tnoteref command within \title for footnotes;
% %% use the tnotetext command for the associated footnote;
% %% use the fnref command within \author or \address for footnotes;
% %% use the fntext command for the associated footnote;
% %% use the corref command within \author for corresponding author footnotes;
% %% use the cortext command for the associated footnote;
% %% use the ead command for the email address,
% %% and the form \ead[url] for the home page:
% %%
% %% \title{Title\tnoteref{label1}}
% %% \tnotetext[label1]{}
% %% \author{Name\corref{cor1}\fnref{label2}}
% %% \ead{email address}
% %% \ead[url]{home page}
% %% \fntext[label2]{}
% %% \cortext[cor1]{}
% %% \address{Address\fnref{label3}}
% %% \fntext[label3]{}
%
% %%\dochead{}
% %% Use \dochead if there is an article header, e.g. \dochead{Short communication}
%
% \title{Beyond ``Horizontal'' and ``Vertical'':\\ The Welfare Effects of Complex Integration\tnoteref{t1} }
% \tnotetext[t1]{The analysis and conclusions set forth are those of the authors and do not indicate concurrence by other members of the Board research staff or by the Federal Reserve Board of Governors. Furthermore, the views expressed here should not be purported to reflect those of the U.S. Department of Justice. This article has benefited from conversations with Alison Oldale, Ted Rosenbaum, and Nathan Wilson.}
%
% %% use optional labels to link authors explicitly to addresses:
% %% \author[label1,label2]{<author name>}
% %% \address[label1]{<address>}
% %% \address[label2]{<address>}
%
% \author[1]{Margaret Loudermilk\fnref{fn1}}
% \ead{margaret.loudermilk@usdoj.gov}
% \affiliation[1]{organization={US Department of Justice}, addressline= {450 5th St. NW}, city={Washington, DC},postcode={20530}}
% \author[2]{Gloria Sheu\fnref{fn1}}
% \ead{gloria.sheu@frb.gov}
% \affiliation[2]{organization={Federal Reserve Board}, addressline= { 2051 Constitution Ave. NW}, city={Washington, DC},postcode={20418} }
% \author[3]{Charles Taragin\corref{cor1}\fnref{fn1}}
%  \cortext[cor1]{Corresponding author.}
% \ead{charles.s.taragin@frb.gov}
% \affiliation[3]{organization={Federal Reserve Board}, addressline= { 2051 Constitution Ave. NW}, city={Washington, DC},postcode={20418} }
% \fntext[fn1]{Declarations of interest: none.}



\title{Beyond ``Horizontal'' and ``Vertical'':\\ The Welfare Effects of Complex Integration\footnote{The analysis and conclusions set forth are those of the authors and do not indicate concurrence by other members of the Board research staff or by the Federal Reserve Board of Governors. Furthermore, the views expressed here should not be purported to reflect those of the U.S. Department of Justice. This article has benefited from conversations with Alison Oldale, Ted Rosenbaum, and Nathan Wilson.}}

\newcommand*\samethanks[1][\value{footnote}]{\footnotemark[#1]}
\author{Margaret Loudermilk\footnote{margaret.loudermilk@usdoj.gov} \\  U.S Department of Justice \and Gloria Sheu\footnote{gloria.sheu@frb.gov. } \\ Federal Reserve Board \and  Charles Taragin\footnote{charles.s.taragin@frb.gov (corresponding author).}  \\  Federal Reserve Board }

\date{\today}

\begin{document}

\pagenumbering{roman}       % Roman numerals from abstract to text
\maketitle                  % print title information
\thispagestyle{empty}       % no page number on THIS page

\begin{abstract}
\noindent We study the welfare impacts of mergers when some firms are already vertically integrated. Our model features logit Bertrand competition downstream and Nash Bargaining upstream.  We numerically simulate four merger types: vertical mergers between an unintegrated retailer and an unintegrated wholesaler, downstream ``horizontal'' mergers between an unintegrated retailer and an integrated retailer/wholesaler, upstream ``horizontal'' mergers between an unintegrated wholesaler and an integrated retailer/wholesaler, and integrated mergers between two integrated retailer/wholesaler pairs. In our model, we find that mergers with both horizontal and vertical characteristics typically harm consumers.  We apply the model to the Republic/Santek merger as a real-world example.

\end{abstract}

% \begin{keyword}
% bargaining models\sep merger simulation\sep vertical markets\sep vertical mergers
% \JEL  L13\sep L40\sep L41\sep L42
% \end{keyword}
%
% \end{frontmatter}

\bigbreak Keywords: bargaining models; merger simulation; vertical markets; vertical mergers

JEL classification: L13; L40; L41; L42



\newpage                    % start a new page
\pagenumbering{arabic}      % Arabic page numbers from now on
\doublespacing

%%%%%%%%%%%%%%%%%%%%%%%%%%%%%%%%%%%%%%%%%%%%%%%%%%%%%%%%%%%%%%%%%%%%%%%%%%%%%%%%%
% Body of Paper
%%%%%%%%%%%%%%%%%%%%%%%%%%%%%%%%%%%%%%%%%%%%%%%%%%%%%%%%%%%%%%%%%%%%%%%%%%%%%%%%%
\section{Introduction}
Given the many different ways a firm may be organized, it is rare that any particular merger can be neatly categorized as purely ``horizontal'' or ``vertical.''  This is particularly true for the mergers involving large firms that usually attract the most scrutiny.  However, for both historical and practical reasons, antitrust research and practice has generally maintained this dichotomy.  Horizontal mergers are cast in terms of whether they create significant upward pricing pressure (UPP), whereas the focus for vertical mergers is on assessing the net effects of the elimination of double marginalization (EDM) versus raising rivals' costs (RRC).  As a result, the antitrust community has little consensus regarding how ``complex'' mergers, which here we define as mergers that possess both horizontal and vertical aspects, will affect welfare.

The horizontal versus vertical distinction has been enshrined in U.S. Department of Justice (DOJ) and Federal Trade Commission (FTC) Horizontal Merger Guidelines and their companion document, the Vertical Merger Guidelines.\footnote{The FTC withdrew its approval from the Vertical Merger Guidelines in 2021.  The DOJ did not.  }  Given these policy documents, it is not surprising that the agencies typically frame most mergers as either horizontal or vertical.  Similarly, most antitrust research maintains this separate categorization, leaving us with little systematic evidence for how complex mergers impact consumers.  However, as part of the current effort by the DOJ and the FTC to revise their merger guidelines, these agencies requested comments on whether any ``modern market realities may be lost by focusing on [horizontal and vertical] relationships categorically.''\footnote{The full question is 1.g. in the ``Request for Information on Merger Enforcement,'' dated January 18, 2022.  The text of the question reads: ``Should the guidelines' traditional distinctions between horizontal and vertical mergers be revisited in light of recent economic trends in the modern economy? What aspects of modern market realities may be lost by focusing on these relationships categorically? Should the guidelines address all mergers in a common framework that covers all market relationships relevant to competition? If so, how?''  }  In this paper, we address this question using a vertical supply chain model to explore the welfare implications of complex mergers.

Our model is drawn from \citeN{ST2021}, and features a Bertrand logit downstream setup alongside a Nash Bargaining wholesale negotiation upstream.  We use this model to explore the welfare implications of four types of mergers: (1) vertical mergers between an unintegrated retailer and an unintegrated wholesaler, (2) downstream ``horizontal'' mergers between an unintegrated retailer and an integrated retailer/wholesaler pair, (3) upstream ``horizontal'' mergers between an unintegrated wholesaler and an integrated retailer/wholesaler pair, and (4) integrated mergers between two previously integrated retailer/wholesaler pairs.  Relative to \citeN{ST2021}, the first category, vertical mergers, appears in the previous paper and here serves as a baseline for comparison.  The other three cases, which feature at least one merging firm that is already integrated, are new.

Our model-based simulations show certain patterns across mergers.  For horizontal upstream and downstream mergers, we find that the consumer harm from UPP and RRC typically exceeds any gains from EDM.  The exceptions occur when upstream wholesalers have more bargaining power than downstream retailers, which is when we expect pre-merger input prices to be high and the gains from EDM to be larger.  Likewise, we find that integrated mergers are typically net harmful to consumers as well.  The exceptions happen only when upstream firms have significantly more bargaining power compared to downstream firms.  Thus, although these complex mergers have both horizontal and vertical aspects, they appear to be qualitatively similar to pure horizontal mergers in their overall consumer impacts.  We also examine how the presence of additional integrated rivals (separate from the merging firms) changes our results. We find that the existence of integrated rival firms lowers the spread of welfare outcomes for consumers and slightly shifts these outcomes towards less harm.

We then employ this model to investigate the 2021 merger of two vertically integrated solid waste management firms, Republic Services and Santek Waste Services, in one geographic market, Chattanooga, Tennessee. The supply chain of this industry involves collection haulers at the downstream level that bring trash to disposal facilities at the upstream level.  Both Republic and Santek operated upstream and downstream in Chattanooga, as did one of their largest rivals, Waste Connections.  In this market, the DOJ negotiated a settlement package meant to address harms in downstream collection and in upstream disposal.  Using a small set of public data inputs, our merger simulation model is able to produce rich results for the changes in horizontal and vertical competition, including RRC and EDM effects.  We estimate that without divestitures, consumers would have faced approximately \$16 million in annual harm in the Chattanooga market.  Furthermore, we show that modeling the merger as a purely horizontal upstream or downstream transaction would miss key vertical impacts.

There are a few caveats to our analysis.  First, products in our merger simulations do not experience any changes in their qualities or costs, other than movements in their wholesale prices due to shifts in bargaining leverage.  Therefore, we rule out merger efficiencies as they are conceptualized in, for example, \citeN{werden96}, and we instead focus on EDM as the main channel through which mergers may benefit consumers.  Second, we model upstream bargaining as being over linear fees, rather than over two-part tariffs or other compensation schedules.  Although this assumption means bargaining is inefficient, linear input prices are widely observed in some industries, perhaps because of difficulties with implementing or monitoring more complicated contracts.\footnote{One example of an industry with linear wholesale pricing that has been featured in several vertical merger proceedings is cable programming distribution, as detailed in \citeN{Rogerson2014}.}  Third, our model assumes that upstream and downstream prices are set treating all other prices as fixed.  Although we make this assumption in part for simplicity, it also may be appropriate for industries where prices take time to adjust.\footnote{Examples include situations where products are sold under long-term contracts.  In contrast, \citeN{BBMM2023} allow downstream firms to treat their own prices as adjustable during upstream negotiations.  The authors argue that this setup is useful for settings like grocery stores, where menu costs are low.  }  Fourth, we rely on the logit for our demand function, which limits the ways in which consumers substitute between products.  In \citeN{ST2012}, we found that assuming the merging firms operate in the same group within a nested logit model tends to increase the magnitude of consumer harm relative to the flat logit.

Our paper is related to the literature on methods for measuring the welfare impacts of mergers, particularly papers about vertical mergers.  The closest merger simulation article is our previous work in \citeN{ST2021}, which uses the same model we feature here but assumes that no firm is vertically integrated pre-merger.  That paper in turn takes the setup from \citeN{WF1994} to characterize downstream competition, and pairs it with a Nash Bargaining model for the upstream market.  \citeN{DVDS2018} and \citeN{DS2019} provide additional vertical merger frameworks that make alternative assumptions about demand and pricing.  \citeN{MS2013} and \citeN{Rogerson2020} develop UPP-style tools for assessing the impact of vertical mergers.  However, none of these papers study complex mergers.

Our model draws upon the empirical industrial organization literature on bargaining in supply chains, typified by \citeN{DKVB2010}.  These models have been applied to several sectors that are the frequent target of complex mergers, such as healthcare (e.g. \citeN{HL2017}) and video programming distribution (e.g. \citeN{CLWY2017}).  These structural empirical papers explore the welfare effects of various competitive interactions in certain industries.  Unlike these papers, our framework has been simplified for use with limited data, with the goal of being more accessible to practitioners.  These simplifications allow us to provide numerical results over a wide range of market conditions, which complements the insights that come from the structural empirical literature on specific case studies.

The paper proceeds as follows.  In Section \ref{sec:theory}, we introduce the theoretical framework and explain how we model mergers.  Section \ref{sec:sims} explains how we construct our numerical simulations and provides our results, and our application to Republic/Santek appears in Section \ref{sec:application}.  Section \ref{sec:concl} concludes. Additional simulation results and information on the data used in the application appear in the Appendix.

%%%%%%%%%%%%%%%%%%%%%%%%%%%%%%%%%%%%%%%%%%%%%%%%%%%%%%%%%%%%%%%%%%%%%%%%%%%%%%%%%
\section{Theory\label{sec:theory}}
We begin by describing our basic framework, taken from \citeN{ST2021}, which uses a downstream Bertrand logit model embedded in an upstream Nash Bargaining setup.  This framework allows us to study a variety of complex merger configurations.

%%%%%%%%%%%%%%%%%%%%%%%%%%%%%%%%%%%%%%%%%%
\subsection{Downstream and Upstream Competition}
Assume there is a set of consumers who each choose to buy a single product sold by a retailer.  Retailers are indexed by $r$, and the wholesalers that supply these retailers are indexed by $w$.  Prior to any mergers taking place, each wholesaler offers only one product, although each retailer can purchase from multiple wholesalers.  We denote the set of all retailers by $\mathbb{R}=\{1, \dots, \left\vert{\mathbb{R}}\right\vert\}$ and the set of all wholesalers by $\mathbb{W}=\{1, \dots, \left\vert{\mathbb{W}}\right\vert\}$.  We divide the set $\mathbb{W}$ into overlapping subsets, each labeled $\mathbb{W}^r$, to indicate which wholesalers' products are carried by which retailers.  Similarly, we divide the set of retailers $\mathbb{R}$ into overlapping subsets, each labeled $\mathbb{R}^w$, to indicate the retailers that carry the product sold by each wholesaler.

The share of consumers that choose product $w$ sold by retailer $r$ has the logit form,
\begin{equation}
s_{rw} = \frac{\exp(\delta_{rw} - \alpha p_{rw})}{1 + \sum_{t \in \mathbb{R}} \sum_{x \in \mathbb{W}^t} \exp(\delta_{tx} - \alpha p_{tx})},
\label{eq: nb share}
\end{equation}
where $\delta_{rw}$ is a quality parameter and $\alpha$ captures sensitivity to price $p_{rw}$.  There is an outside good whose quality parameter and price have been normalized to zero.  The retailer's profit is given by $\pi^r = \sum_{w \in \mathbb{W}^r} [p_{rw} - p^W_{rw} - c^R_{rw}] s_{rw} M$, where $p^W_{rw}$ is the unit fee charged by wholesaler $w$ to retailer $r$, $c^R_{rw}$ captures any additional marginal costs borne by the retailer, and $M$ is the market size.  Downstream prices are set in Bertrand equilibrium, according to
\begin{equation}
\sum_{x \in \mathbb{W}^r} [p_{rx} - p^W_{rx} - c^R_{rx}] \frac{\partial s_{rx}}
{\partial p_{rw}} + s_{rw} = 0,
\label{eq: nb downstream foc}
\end{equation}
which is the first order condition for product $w$ sold by retailer $r$.

Wholesale prices are set via Nash Bargaining between retailers and wholesalers.  We assume that the negotiation for a given input price treats other input prices and all downstream prices as given.\footnote{As explained in \citeN{ST2021}, this assumption is equivalent to a situation where all negotiations and choices for downstream prices happen simultaneously.  }  Profits for wholesaler $w$ are given by $\pi^w = \sum_{r \in \mathbb{R}^w} [p^W_{rw} - c^W_{rw}] s_{rw} M$, where $c^W_{rw}$ is the wholesale marginal cost when dealing with retailer $r$.  The first order condition for the negotiation between retailer $r$ and wholesaler $w$ is
\begin{equation}
\begin{split}
&\overbrace{[p^W_{rw} - c^W_{rw}]s_{rw} - \sum_{t \in \mathbb{R}^w \setminus \{r\}} [p^W_{tw} - c^W_{tw}] \Delta s_{tw}(\mathbb{W}^r \setminus \{w\})}^{\text{wholesaler GFT}} = \\
&\frac{1-\lambda}{\lambda} \left(\underbrace{[p_{rw} - p^W_{rw} - c^R_{rw}]s_{rw} - \sum_{x \in \mathbb{W}^r \setminus \{w\}} [p_{rx} - p^W_{rx} - c^R_{rx}] \Delta s_{rx}(\mathbb{W}^r \setminus \{w\})}_{\text{retailer GFT}}\right),
\end{split}
\label{eq: upstream foc}
\end{equation}
where $\lambda \in [0,1]$ captures the bargaining power of the retailer relative to the wholesaler.  The $\Delta s_{tx}(\mathbb{W}^r \setminus \{w\}) \equiv s_{tx}(\mathbb{W}^r \setminus \{w\}) - s_{tx}$ is the difference in the share of good $x$ sold by retailer $t$ when good $w$ is not offered by retailer $r$ versus when good $w$ is offered by retailer $r$.  Thus, the wholesale price $p_{rw}$ is set such that the payoff to wholesaler $w$ when it sells to retailer $r$ less the payoff when it does not (that is, the gains from trade or ``GFT''), divided by the payoff to retailer $r$ when it buys from wholesaler $w$ less the payoff when it does not, equals the ratio of wholesaler to retailer bargaining power.

Together the series of downstream and upstream first order conditions determine market equilibrium.  This model can be solved and calibrated as described in \citeN{ST2021}.  In our baseline configuration, we assume that wholesale and retail marginal costs are constant.  We allow for linear increasing marginal costs as a robustness check that appears in Appendix \ref{app:mc}.

%%%%%%%%%%%%%%%%%%%%%%%%%%%%%%%%%%%%%%%%%%
\subsection{Mergers}
Here we describe the manner in which complex mergers are modeled in this framework.  Throughout we assume that each product's quality, $\delta_{rw}$, and marginal costs, $c^R_{rw}$ and $c^W_{rw}$, remain unchanged after the merger.  We begin by describing vertical mergers, as all cases we examine have some vertical aspects.  Suppose that retailer $r$ and wholesaler $w$ were to merge.  Then the first order condition for profit maximization for the joint firm when setting the downstream price for product $rw$ is given by
\begin{equation}
\begin{split}
&\sum_{x \in \mathbb{W}^r \setminus \{w\}} [p_{rx} - p^W_{rx} - c^R_{rx}]\frac{\partial s_{rx}}{\partial p_{rw}} + s_{rw} + \overbrace{[p_{rw} - c^W_{rw} - c^R_{rw}]\frac{\partial s_{rw}}{\partial p_{rw}}}^{\text{EDM effect}} \\
&+ \underbrace{\sum_{t \in \mathbb{R}^w \setminus \{r\}} [p^W_{tw} - c^W_{tw}] \frac{\partial s_{tw}}{\partial p_{rw}}}_{\text{upstream UPP effect}} = 0.
\end{split}
\label{eq: vmerger own ret downstream foc}
\end{equation}
The pricing problem balances two effects.  On the one hand, the term labeled ``EDM effect'' captures the impact of retailer $r$ being able to access product $w$ at marginal cost.  This force would tend to lower the resulting price.  On the other hand, the term labeled ``upstream UPP effect'' captures the incentive to raise prices for retailer $r$ in order to divert sales to wholesaler $w$.  This force would tend to raise the resulting consumer price.

Turning to input prices, when wholesaler $w$ bargains with unaffiliated retailer $s$, the first order condition becomes
\begin{equation}
\begin{split}
&[p^W_{sw} - c^W_{sw}]s_{sw} - \sum_{t \in \mathbb{R}^w \setminus \{r,s\}} [p^W_{tw} - c^W_{tw}] \Delta s_{tw}(\mathbb{W}^s \setminus \{w\})\\
& - \overbrace{\overbrace{[p_{rw}-c^W_{rw}-c^R_{rw}] \Delta s_{rw}(\mathbb{W}^s \setminus \{w\})}^{\text{indirect EDM effect}} - \sum_{x \in \mathbb{W}^r \setminus \{w\}} [p_{rx} - p^W_{rx} - c^R_{rx}] \Delta s_{rx}(\mathbb{W}^s \setminus \{w\})}^{\text{RRC effect}}=\\
&\frac{1-\lambda}{\lambda} \left([p_{sw} - p^W_{sw} - c^R_{sw}]s_{sw} - \sum_{x \in \mathbb{W}^s \setminus \{w\}} [p_{sx} - p^W_{sx} - c^R_{sx}] \Delta s_{sx}(\mathbb{W}^s \setminus \{w\})\right).
\end{split}
\label{eq: vmerger wh upstream foc}
\end{equation}
which reflects the change in the disagreement payoff coming from the merger with retailer $r$.  Now when the wholesaler considers the possible loss of sales upon ceasing to trade with retailer $s$, these losses are softened due to a potential for diversion to retailer $r$, which we label the ``RRC effect.''  Furthermore, the margin on product $w$ sold by retailer $r$ is potentially higher due to EDM, as shown through the expression labeled ``indirect EDM effect,'' which can further compensate the firm.  These impacts tend to raise the resulting input price.

When the merged firm is bargaining with the unaffiliated wholesaler $v$ over what input price to pay, the bargaining first order condition becomes
\begin{equation}
\begin{split}
&[p^W_{rv} - c^W_{rv}]s_{rv} - \sum_{t \in \mathbb{R}^v \setminus \{r\}} [p^W_{tv} - c^W_{tv}] \Delta s_{tv}(\mathbb{W}^r \setminus \{v\}) = \\
&\frac{1-\lambda}{\lambda} \left([p_{rv} - p^W_{rv} - c^R_{rv}]s_{rv} - \sum_{x \in \mathbb{W}^r \setminus \{w,v\}} [p_{rx} - p^W_{rx} - c^R_{rx}] \Delta s_{rx}(\mathbb{W}^r \setminus \{v\})\right.\\
&\left.- \underbrace{[p_{rw} - c^W_{rw} - c^R_{rw}] \Delta s_{rw}(\mathbb{W}^r \setminus \{v\})}_{\text{EDM recapture effect}}-\underbrace{\sum_{t \in \mathbb{R}^w \setminus \{r\}} [p^W_{tw} - c^W_{tw}] \Delta s_{tw}(\mathbb{W}^r \setminus \{v\})}_{\text{wholesale recapture leverage effect}}\right).
\end{split}
\label{eq: vmerger ret upstream foc}
\end{equation}
In this case, the merged firm has two channels for potential additional profits should it cease to trade with wholesaler $v$.  First, if retail sales are diverted to product $w$ sold by retailer $r$, those sales will earn a higher margin due to lower marginal costs stemming from what we call the ``EDM recapture effect.''  Second, the loss of product $v$ carried by retailer $r$ could increase sales by wholesaler $w$ through other retailers, which we call the ``wholesale recapture leverage effect.''  Both of these effects would tend to lower the resulting input price.

Next consider a merger between an integrated retailer/wholesaler $rw$ and a standalone retailer $s$.  Such a combination has a vertical component and a horizontal component.  The first order condition for setting the downstream price $p_{rw}$ becomes
\begin{equation}
\begin{split}
&\sum_{x \in \mathbb{W}^r \setminus \{w\}} [p_{rx} - p^W_{rx} - c^R_{rx}]\frac{\partial s_{rx}}{\partial p_{rw}} + s_{rw} + \overbrace{[p_{rw} - c^W_{rw} - c^R_{rw}]\frac{\partial s_{rw}}{\partial p_{rw}}}^{\text{direct EDM effect}}+ \underbrace{\sum_{t \in \mathbb{R}^w \setminus \{r\}} [p^W_{tw} - c^W_{tw}] \frac{\partial s_{tw}}{\partial p_{rw}}}_{\text{upstream UPP effect}} \\ &+\underbrace{\underbrace{[p_{sw} - c^W_{sw} - c^R_{sw}]\frac{\partial s_{sw}}{\partial p_{rw}}}_{\text{indirect EDM Effect}} + \sum_{x \in \mathbb{W}^s \setminus \{w\}} [p_{sx} - p^W_{sx} - c^R_{sx}] \frac{\partial s_{sx}} {\partial p_{rw}}}_{\text{downstream UPP effect}}= 0.
\end{split}
\label{eq: intdmerger own ret downstream foc}
\end{equation}
Now there is the possibility for what we call the ``downstream UPP effect'' in the retail market, as the merged firm can recapture sales that are diverted to retailer $s$ when $r$ raises its prices.  EDM between $w$ and $s$ can actually increase this UPP impact, because when sales are diverted to product $w$ sold by retailer $s$, those units may earn a larger margin.  This impact is what we label the ``indirect EDM effect,'' which comes through the interaction of EDM with UPP.

If instead an integrated retailer/wholesaler $rw$ were to merge with a standalone wholesaler $v$, then the resulting first order condition for product $rw$ would look similar to equation \eqref{eq: intdmerger own ret downstream foc}.  However,  the downstream UPP effect would be replaced with an additional upstream UPP effect capturing the value of sales diverted to customers of wholesaler $v$.  The condition would also include an indirect EDM component, reflecting the ability of retailer $r$ to obtain product $v$ at marginal cost, which in turn would raise the value of diverted sales to that product.  Whether these additional incentives to raise prices will dominate the direct EDM impact is an empirical question.  Furthermore, if we instead considered a merger between two integrated retailer/wholesaler pairs, both upstream and downstream UPP effects would enter.

As for wholesale prices, again consider a merger between an integrated retailer/wholesaler $rw$ and a standalone retailer $s$.  When wholesaler $w$ bargains with an unaffiliated retailer $u$ we have the first order condition given by,
\begin{equation}
\begin{split}
&[p^W_{uw} - c^W_{uw}]s_{uw} - \sum_{t \in \mathbb{R}^w \setminus \{r,s,u\}} [p^W_{tw} - c^W_{tw}] \Delta s_{tw}(\mathbb{W}^u \setminus \{w\})\\
& - \overbrace{\sum_{t \in \{r,s\}}\left(\overbrace{[p_{tw}-c^W_{tw}-c^R_{tw}] \Delta s_{tw}(\mathbb{W}^u \setminus \{w\})}^{\text{indirect EDM effect}} - \sum_{x \in \mathbb{W}^t \setminus \{w\}} [p_{tx} - p^W_{tx} - c^R_{tx}] \Delta s_{tx}(\mathbb{W}^u \setminus \{w\})\right)}^{\text{RRC effect}}=\\
&\frac{1-\lambda}{\lambda} \left([p_{uw} - p^W_{uw} - c^R_{uw}]s_{uw} - \sum_{x \in \mathbb{W}^u \setminus \{w\}} [p_{ux} - p^W_{ux} - c^R_{ux}] \Delta s_{ux}(\mathbb{W}^u \setminus \{w\})\right).
\end{split}
\label{eq: intdmerger wh upstream foc}
\end{equation}
The RRC effect is augmented with the profits emanating from the sales of retailer $s$, in addition to the sales of retailer $r$, both of which may potentially recapture sales should retailer $u$ lose access to product $w$.  The merged firm has higher bargaining leverage as a result.  If instead the integrated retailer/wholesaler $rw$ were to merge with a standalone wholesaler $v$, then the profits earned by firm $rw$ should the negotiation with retailer $u$ fail are augmented with the earnings of wholesaler $v$ rather than of retailer $s$.  This adds a term similar to the wholesale recapture leverage effect seen in equation \eqref{eq: vmerger ret upstream foc} to the left-hand side of the bargaining first order condition.\footnote{For simplicity, we assume that when a retailer fails to reach an agreement with wholesaler $w$, that retailer's contract with wholesaler $v$ remains in place.  This assumption can be loosened.  }    In the case of a merger between two integrated retailer/wholesaler firms, all of these effects would appear.

Returning to a merger between an integrated retailer/wholesaler $rw$ and a standalone retailer $s$, when the merged firm bargains with an unaffiliated wholesaler $v$ to supply retailer $r$, the first order condition becomes
\begin{equation}
\begin{split}
&[p^W_{rv} - c^W_{rv}]s_{rv} - \sum_{t \in \mathbb{R}^v \setminus \{r\}} [p^W_{tv} - c^W_{tv}] \Delta s_{tv}(\mathbb{W}^r \setminus \{v\}) = \\
&\frac{1-\lambda}{\lambda} \left([p_{rv} - p^W_{rv} - c^R_{rv}]s_{rv} - \sum_{x \in \mathbb{W}^r \setminus \{w,v\}} [p_{rx} - p^W_{rx} - c^R_{rx}] \Delta s_{rx}(\mathbb{W}^r \setminus \{v\})\right.\\
&- \underbrace{\sum_{t \in \{r,s\}}[p_{tw} - c^W_{tw} - c^R_{tw}] \Delta s_{tw}(\mathbb{W}^r \setminus \{v\})}_{\text{EDM recapture effect}}
- \underbrace{\sum_{x \in \mathbb{W}^s \setminus \{w\}} [p_{sx} - p^W_{sx} - c^R_{sx}] \Delta s_{sx}(\mathbb{W}^r \setminus \{v\})}_{\text{retail recapture leverage effect}}\\
&-\left.\underbrace{\sum_{t \in \mathbb{R}^w \setminus \{r,s\}} [p^W_{tw} - c^W_{tw}] \Delta s_{tw}(\mathbb{W}^r \setminus \{v\})}_{\text{wholesale recapture leverage effect}}\right).
\end{split}
\label{eq: intdmerger ret upstream foc}
\end{equation}
Compared to equation \eqref{eq: vmerger ret upstream foc}, now the EDM recapture effect applies to sales of product $w$ at both affiliated retailers $r$ and $s$.  Furthermore, there is also the possibility that sales will be diverted to retailer $s$ should retailer $r$ lose access to product $v$, which creates an additional ``retail recapture leverage effect'' alongside the wholesale recapture leverage effect.  All of these additional terms tend to increase the bargaining leverage of the merged firm.  If we instead examined a merger between retailer/wholesaler $rw$ and an unaffiliated wholesaler, that would augment the EDM recapture effect with sales of retailer $r$ for two merged wholesalers, rather than sales through two merged retailers.  Similarly, the retail recapture leverage effect would be replaced with an additional wholesale recapture leverage effect for the newly merged wholesaler.  For a merger between two integrated retailer/wholesaler firms, all of these effects would enter.

The discussion thus far has not dealt with the possible existence of integrated rivals to the merging firms.  If such rival firms exist, they would also have pricing and bargaining incentives similar to those captured in the above post-merger first order conditions.  For example, if the integrated firm $rw$ were bargaining over what input price to charge a rival integrated firm $sv$, the first order condition would be
\begin{equation}
\begin{split}
&[p^W_{sw} - c^W_{sw}]s_{sw} - \sum_{t \in \mathbb{R}^w \setminus \{r,s\}} [p^W_{tw} - c^W_{tw}] \Delta s_{tw}(\mathbb{W}^s \setminus \{w\})\\
& - \overbrace{\overbrace{[p_{rw}-c^W_{rw}-c^R_{rw}] \Delta s_{rw}(\mathbb{W}^s \setminus \{w\})}^{\text{indirect EDM effect}} - \sum_{x \in \mathbb{W}^r \setminus \{w\}} [p_{rx} - p^W_{rx} - c^R_{rx}] \Delta s_{rx}(\mathbb{W}^s \setminus \{w\})}^{\text{RRC effect}}=\\
&\frac{1-\lambda}{\lambda} \left([p_{sw} - p^W_{sw} - c^R_{sw}]s_{sw} - \sum_{x \in \mathbb{W}^s \setminus \{w,v\}} [p_{sx} - p^W_{sx} - c^R_{sx}] \Delta s_{sx}(\mathbb{W}^s \setminus \{w\})\right.\\
&\left.- \underbrace{[p_{sv} - c^W_{sv} - c^R_{sv}] \Delta s_{sv}(\mathbb{W}^s \setminus \{w\})}_{\text{EDM recapture effect}}-\underbrace{\sum_{t \in \mathbb{R}^v \setminus \{s\}} [p^W_{tv} - c^W_{tv}] \Delta s_{tv}(\mathbb{W}^s \setminus \{w\})}_{\text{wholesale recapture leverage effect}}\right).
\end{split}
\label{eq: intmerger wh upstream foc}
\end{equation}
The existing integration of firm $sv$ gives it countervailing bargaining leverage that may counteract any change in leverage caused by a merger of retailer $r$ and wholesaler $w$.  The extent of the net impact depends on how these leverage terms balance out in equilibrium.


%%%%%%%%%%%%%%%%%%%%%%%%%%%%%%%%%%%%%%%%%%
\section{Empirical Application: Republic/Santek \label{sec:application}}

The U.S. waste and recycling industry generates approximately \$80 billion in annual revenues.\footnote{See Waste Dive, \tiny{\url{https://www.wastedive.com/news/public-companies-increased-control-of-74b-us-waste-industry-in-2018/556079/}}.} In recent years the industry has experienced significant merger activity, including several acquisitions between vertically integrated, nationally active competitors. Furthermore, there are a variety of vertical supply chain configurations across local markets, making the industry an excellent application to study the welfare impacts of mergers that have both horizontal and vertical aspects.

Republic Services, the second largest waste management company in the U.S., acquired Santek Environmental in 2021.\footnote{See Competitive Impact Statement: U.S. and State of Alabama v. Republic Services, Inc.and Santek Waste Services, LLC, \url{https://www.justice.gov/atr/case-document/file/1382626/download}.}  The DOJ filed a complaint and negotiated a settlement in the case, alleging horizontal anti-competitive effects for four small-container commercial waste (SCCW) collection markets and two municipal solid waste (MSW) disposal markets. In addition, vertical anti-competitive effects were alleged to arise from the combination of Republic and Santek integrated assets in the Chattanooga region.\footnote{See U.S. and State of Alabama v. Republic Services, Inc. and Santek Waste Services, LLC, \url{https://www.justice.gov/atr/case-document/file/1382031/download}.}  In this section, we apply our model to simulate the effects of the merger in Chattanooga.\footnote{The \href{https://CRAN.R-project.org/package=antitrust}{\texttt{antitrust} R package} contains the computer code needed to run the merger simulations described here.}

%%%%%%%%%%%%%%%%%%%%%%%%%%%%%%%%%%%%%%%%%%
\subsection{Market Background}

The vertical supply chain in the solid waste industry is primarily comprised of waste collection operations or ``haulers'' and waste disposal facilities. Haulers collect MSW from businesses and residences and must dispose of it at a lawful disposal site, usually a landfill. Waste disposal (upstream) is a required input into waste collection services (downstream). Some haulers are vertically integrated and operate their own disposal facilities. Vertically-integrated haulers typically prefer to dispose of waste at their own disposal facilities and may also sell a portion of their disposal capacity. Disposal customers include private waste haulers without their own disposal assets as well as local governments that collect their citizens' waste themselves. Due to strict laws and regulations that govern the disposal of MSW, there are no reasonable substitutes for MSW disposal.

The Competitive Impact Statement (CIS) filed by the DOJ in association with the Republic/Santek merger describes the alleged lost competition in the ``Chattanooga, Tennessee and North Georgia area,'' subsequently referred to as the Chattanooga Area, due to lost horizontal competition in MSW disposal and SCCW collection as well as due to RRC in the SCCW collection market by raising the MSW disposal costs of independent haulers. The CIS notes that pre-merger, Republic and Santek combined served approximately 73 percent of the SCCW collection market with three other significant competitors. In MSW disposal, the CIS identifies only one other significant competitor pre-merger and Republic and Santek combined as serving approximately 82 percent of the market, disposed of either directly in the merging parties' landfills within the area or passing through their transfer stations before ultimately being disposed of in the parties' landfills elsewhere. Thus, pre-merger both parties were large, vertically integrated competitors in the Chattanooga Area.

In addition, another large, vertically integrated company, Waste Connections, existed in the market at the time of the merger and was the parties' sole competitor in MSW disposal. National firms Waste Management and ADS owned collection assets in the area but were not vertically integrated in this market, as demonstrated by MSW disposal data discussed in Appendix \ref{app:trashdata}.\footnote{Firms that are national competitors and vertically integrated in other markets are known to enter into contracts with each other to dispose of waste on advantageous terms that may make them effectively vertically integrated. Ignoring these contracting relationships may underestimate the number of effectively vertically integrated competitors in the market.} We treat Waste Management and ADS as a single entity (henceforth ``WM-ADS'') because they merged with one another in 2020, and that transaction was completed before the filing of the CIS for Republic/Santek. The final significant participant in the SCCW collection market was a major regional firm that is not identified by name in the CIS.  (We label this firm ``Regional'' in what follows.)

Using data described in Appendix \ref{app:trashdata}, the supply relationships between the upstream disposal market participants (Republic, Santek, and Waste Connections) and downstream collection market participants (Republic, Santek, Waste Connections, WM-ADS, and Regional) can be inferred under the assumption that vertically-integrated haulers first dispose of waste at their own disposal facilities and then sell any residual disposal capacity to rival haulers. Santek and Waste Connections exhibit excess disposal capacity. However, Republic's collection volumes are estimated to exceed their ability to self-supply disposal in the Chattanooga market. As a result, Republic needs to purchase additional disposal from Santek and Waste Connections, presumably at a higher marginal cost. We assume that this disposal is purchased from both Santek and Waste connections according to their shares of available excess MSW disposal capacity of 63.8\% and 36.2\%, respectively. We further assume that WM-ADS and Regional purchase disposal capacity from both Santek and Republic according to these shares. The resulting relationships between the upstream and downstream market participants are captured in Figure \ref{fig:chatt_tree}.

Our final input data set is presented in Table \ref{tab:trashdata}.  As discussed above and in Appendix \ref{app:trashdata}, we need to make several simplifying assumptions in order to transform the available information into the data required for our model.  We expect that the DOJ would have had access to more accurate confidential information during their investigation.  Therefore, we view our results as only illustrative of the types of analysis and results that could have been produced for this case.  Note that, other than imposing that Republic only offers disposal to itself, we do not structurally incorporate capacity constraints into the simulation model. We assume that the changes in volume that would be caused by the merger would not be large enough to shift which constraints would be binding.\footnote{We also assume that the Waste Connections-Republic product remains on the market, which may only be true in the short run.}

%%%%%%%%%%%%%%%%%%%%%%%%%%%%%%%%%%%%%%%%%%
\subsection{Merger Simulation Results}

Overall, the merger between Republic and Santek is estimated to lower upstream prices by 2.8\% and increase downstream prices by 12\%, resulting in \$16 million of annual harm to consumers and total annual producer benefits of approximately \$13.4 million. Thus, on net the model predicts the merger would be harmful despite the presence of significant EDM that reduces prices upstream.\footnote{Our calibrated bargaining parameters for retailers against Santek ($\lambda=1$) and Republic ($\lambda=0.7$) give a substantial edge to downstream firms, which, based on our analysis of the previous section, suggests consumers are likely to be harmed.  High bargaining power parameters are needed to explain the high collection margin compared to the disposal margin.  (See Table \ref{tab:trashdata}.)}  These effects are illustrated in more detail in Figure \ref{fig:TrashSimsFirms}.

Disposal prices for pre-merger integrated pairs are unchanged post-merger. However, MSW disposal prices increase to all rival haulers post-merger (i.e., all unintegrated pairs). The merged firms' disposal prices to downstream rivals are estimated to increase 88\% post-merger, and Waste Connections' prices to its downstream rivals are estimated to increase 22\%. These large price increases are offset by the large decrease in the disposal price for the integrating pair post-merger due to EDM, as the cost to Republic for volume disposed with Santek decreases by 60\%.

Turning to the downstream market, all collection prices increase post-merger. The merging parties' post-merger collection prices increase by approximately 20\%, and their share decreases by about 8 percentage points overall. The Republic/Republic product's price increases 21\%, and the Santek/Santek product's price increases by 23\%. The newly integrated Santek/Republic product's price is estimated to increase by 9\%, demonstrating that EDM is not fully passed-through to consumers. The integrated rival, Waste Connections, increases its collection price by approximately 5\%, and its share increases by 6 percentage points. The unintegrated rivals' collection prices each increase by about 9\%, which again demonstrates incomplete pass-through of the change in upstream costs, and their shares increase slightly.

In Figure \ref{fig:TrashSimsCompareAll} we compare the full vertical model results, which compute both post-merger downstream and upstream price effects, to two other models that hold the prices of one level fixed.\footnote{This approach uses the same calibrated demand and costs parameters across all three models.} The ``downstream-only'' model computes collection price effects holding disposal prices fixed at pre-merger levels, and the ``upstream-only'' model computes disposal price effects holding collection prices fixed at pre-merger levels.

Ignoring these vertical relationships in the downstream-only model results in an estimated 11\% increase in collection prices with a greater than 16\% increase in the merging parties' price, consumer harm of \$10 million annually, and net harm of approximately \$2 million. Estimated consumer harm is 60\% higher in the full vertical model and is offset by markedly larger producer benefits. The downstream-only model misses the substantial RRC effect experienced by the Santek/WM-ADS and Santek/Regional products.  In the vertical model, the downstream prices for these products rise more dramatically due to this RRC, and other prices increase alongside them.  This causes more harm to consumers.

In the upstream-only model, downstream prices and therefore shares are fixed at their pre-merger levels. We still have upstream prices set to marginal cost for integrated products.  Overall disposal prices increase by 8.2\% post-merger, and the merger effect is a pure transfer between firms. Thus, both consumer and net harm are estimated to be zero.  The main impact of the merger is in changing the bargaining leverage for Santek, now that it is combined with Republic.  This effect causes Santek to raise its upstream disposal prices to both WM-ADS and Regional.  The price increase is even larger in the upstream-only model compared to the full model. The lack of downstream price adjustments means that WM-ADS and Regional cannot shift their sales towards their cheaper supplier, Waste Connections, in equilibrium, which makes having access to Santek's disposal facilities relatively more valuable than in the full model, driving up Santek's price.  In this sense, the upstream-only model overstates the increase in leverage that Santek receives from the merger.

The Republic/Santek merger was ultimately settled with the DOJ through a divestiture. According to the Final Judgment, in the Chattanooga Area the parties were required to divest Santek's SCCW collection assets  as well as two landfills and a transfer station.\footnote{See Final Judgment: U.S and Plaintiff States v. Republic Services, Inc. and Santek, LLC, \url{https://www.justice.gov/atr/case-document/file/1408616/download}.} Our results suggest that a divestiture in the collection market alone likely would not have sufficiently remedied the anti-competitive effects of the merger.

%%%%%%%%%%%%%%%%%%%%%%%%%%%%%%%%%%%%%%%%%%%%%%%%%%%%%%%%%%%%%%%%%%%%%%%%%%%%%%%%%
\section{Conclusion \label{sec:concl}}

This paper relaxes an assumption made in much of the merger simulation literature: that mergers occur in vertical supply chains where in the pre-merger state, no firms are vertically integrated. Removing this limitation is important as competition authorities are routinely called upon to investigate complex mergers where one or both of the merging parties are vertically integrated. Using a series of numerical simulations and the model of \citeN{ST2021}, we find that most mergers involving one or more firms that are already vertically integrated harm consumers.  The exceptions occur when wholesalers have more bargaining power compared to retailers, which are instances where the benefits from EDM would tend to be larger.  We also show that having integrated non-merging rivals limits the range of welfare outcomes and tends to make mergers somewhat more beneficial to consumers, although these firms' presence is not a panacea for consumer harm. We apply our model to the Republic/Santek merger, where we find that ignoring existing vertical relationships gives an incomplete picture of the merger's welfare impacts.

Our results suggest that mergers that have horizontal aspects remain harmful for consumers in many instances, even when these mergers also have vertical aspects. This appears to be especially true when retailers have relatively more bargaining power, which in our model occurs when retailer margins are relatively large compared to wholesaler margins.  Antitrust agencies may have reason to be especially cautious when reviewing mergers in these situations.

However, we of course have the caveat that our results are specific to our modeling framework.  Certain alternative structures are likely to give different welfare results.  Indeed, we found in \citeN{ST2021} that using a second score auction to model downstream competition, as opposed to assuming Bertrand pricing, could meaningfully impact the estimated consumer harm from mergers, although results were similar between these models for vertical mergers.  This paper is an attempt to provide some information about possible welfare effects from complex mergers within a framework that is used by practitioners.  Exploring other commonly used frameworks would be a natural extension in future research.


%%%%%%%%%%%%%%%%%%%%%%%%%%%%%%%%%%%%%%%%%%%%%%%%%%%%%%%%%%%%%%%%%%%%%%%%%%%%%%%%%
% Bibliography
%%%%%%%%%%%%%%%%%%%%%%%%%%%%%%%%%%%%%%%%%%%%%%%%%%%%%%%%%%%%%%%%%%%%%%%%%%%%%%%%%
\newpage
\bibliographystyle{chicago}
%\bibliographystyle{elsarticle-harv}
\bibliography{master-bib}
\clearpage

%%%%%%%%%%%%%%%%%%%%%%%%%%%%%%%%%%%%%%%%%%%%%%%%%%%%%%%%%%%%%%%%%%%%%%%%%%%%%%%%%
% Tables and Figures
%%%%%%%%%%%%%%%%%%%%%%%%%%%%%%%%%%%%%%%%%%%%%%%%%%%%%%%%%%%%%%%%%%%%%%%%%%%%%%%%%


\begin{table}
\caption{Republic/Santek Merger Simulation Inputs}
\small{%\caption{\label{tab:trashdata}Republic/Santek Merger Simulation Inputs. Volume is reported in thousands of pounds, while prices and margins are reported in dollars.}
%\centering
\resizebox{\linewidth}{!}{
\begin{tabular}[t]{rrccccc}
\toprule
Disposal Firm & Collection Firm & Volume & Disposal Price & Disposal Margin & Collection Margin & Collection Cost\\
\midrule
Republic & Republic & 165 & 42 & 0 & 44 & 93\\
\cmidrule{1-7}
 & Republic & 34 & 36 & 20 &  & 93\\

 & Santek & 218 & 16 & 0 &  & 74\\

 & WMADS & 30 & 36 & 20 &  & 148\\

\multirow{-4}{*}{\raggedleft\arraybackslash Santek} & Regional & 30 & 36 & 20 &  & 148\\
\cmidrule{1-7}
 & Republic & 19 & 25 & 14 &  & 93\\

 & WasteConn & 48 & 11 & 0 &  & 63\\

 & WMADS & 17 & 25 & 14 &  & 148\\

\multirow{-4}{*}{\raggedleft\arraybackslash WasteConn} & Regional & 17 & 25 & 14 &  & 148\\
\bottomrule
\end{tabular}}
}
\footnotesize{Notes: ``WasteConn'' stands for Waste Connections.  Volume is reported in thousands of tons, whereas prices and margins are reported in dollars per ton.}
\label{tab:trashdata}
\end{table}



\begin{figure}
\centering
\includegraphics[scale=.7]{output/chattanooga_tree.png}
\caption{Structure of the Chattanooga Area Disposal and Collection Market}
\label{fig:chatt_tree}
\end{figure}


\begin{figure}
\centering
\caption{Republic/Santek Simulation Results}
\includegraphics[scale=1.0]{output/TrashSimsFirm.png}\\
\footnotesize{Notes: ``WasteConn'' stands for Waste Connections.}
\label{fig:TrashSimsFirms}
\end{figure}


\begin{figure}
\centering
\caption{Republic/Santek Simulations: Full vs. Partial Models}
\includegraphics[scale=1.0]{output/TrashSimsCompareAll.png}\\
\footnotesize{Notes: ``WasteConn'' stands for Waste Connections.}
\label{fig:TrashSimsFirms}
\end{figure}

\begin{figure}
\centering
\caption{Simulated Welfare Effects from Mergers between  Republic/Santek, Santek/WasteConn, Santek/WM-ADS, and WM-ADS/Regional}
\includegraphics{output/trashinterestingmergerbar.png}\\
\footnotesize{Notes: ``WasteConn'' stands for Waste Connections.}
\label{fig:TrashFakeOverall}
\end{figure}


\begin{figure}
\centering
\caption{Simulation Results from mergers between Santek/WasteConn, Santek/WM-ADS, and WM-ADS/Regional }
\includegraphics{output/trashinterestingfirmbar.png}\\
\footnotesize{Notes: ``WasteConn'' stands for Waste Connections.}
\label{fig:TrashFakeFirms}
\end{figure}


\begin{figure}
\centering
\caption{The sensitivity of consumer welfare in Santek/Republic merger simulations to changes in the bargaining power parameter.
%`All' changes the bargaining power paramtere for all non-vertically integrated firms. `Santek:Republic' changes the bargaining power parameter for just the Santek/Republic product. `WasteConn:Republic' changes the bargaining power parameter for just the WasteConn:Republic product. The Dashed line displays baseline results.
}
\includegraphics{output/trashbarg.png}\\
\footnotesize{Notes: ``WasteConn'' stands for Waste Connections.}
\label{fig:TrashBarg}
\end{figure}

\begin{figure}
\centering
\caption{The sensitivity of consumer welfare in the Republic/Santek simulations to changes in the outside share. The absolute value of the market elasticity corresponding to the change in outside share is in parentheses. Baseline results assume 0 outside share.}
\includegraphics{output/trashelast.png}\\
\footnotesize{Notes: ``WasteConn'' stands for Waste Connections.}
\label{fig:TrashElast}
\end{figure}

\begin{figure}
\centering
\caption{The sensitivity of consumer welfare in the Republic/Santek simulations to changes in the nesting parameter.`Integrated' assumes that all vertically integrated products are in a single nest, while `Landfill` assumes that all collectors that use the same landfill are in a single nest. A nesting parameter of 0 displays baseline results. }
\includegraphics{output/trashnest.png}\\
\footnotesize{Notes: ``WasteConn'' stands for Waste Connections.}
\label{fig:TrashNest}
\end{figure}

%%%%%%%%%%%%%%%%%%%%%%%%%%%%%%%%%%%%%%%%%%%%%%%%%%%%%%%%%%%%%%%%%%%%%%%%%%%%%%%%%
% Appendix
%%%%%%%%%%%%%%%%%%%%%%%%%%%%%%%%%%%%%%%%%%%%%%%%%%%%%%%%%%%%%%%%%%%%%%%%%%%%%%%%%
\appendix
\newpage
\clearpage

\numberwithin{equation}{section}
\numberwithin{figure}{section}
\numberwithin{table}{section}


%%%%%%%%%%%%%%%%%%%%%%%%%%%%%%%%%%%%%%%%%%%%%%%%%%%%%%%%%%%%%%%%%%%%%%%%%%%%%%%%%
\newpage
\section{Appendix: Data for Republic/Santek\label{app:trashdata}}

Data on MSW disposal come from information collected by the Tennessee Department of Environmental Quality (TDEQ) consisting of Class 1 landfill ownership, Class 1 Solid Waste Origin Reports, and waste disposal by county for 2019. These data identify the origin county and destination landfill, including owner and operator information, for all MSW produced in Tennessee as well as waste volumes passing through transfer stations in Tennessee. The MSW disposal market definition follows that outlined in the DOJ CIS and attributes share to the company owning the final disposal landfill (i.e., ignoring transfer stations that are an intermediate disposal site only). Thus, the total market quantity is defined as all MSW volumes originating in Hamilton County, Tennessee where Chattanooga is located.

Disposal volumes are measured in tons and have been combined across landfills owned by the same firm. The TDEQ data identify landfills owned or operated by Republic, Santek, Waste Connections, and three other market participants (each with less than 0.5\% share) receiving MSW volumes originating in Hamilton County.\footnote{The City of Chattanooga and Marion County landfills are municipally owned and operated, accounting for less than 1\% share combined. Global Envirotech is a privately-owned transfer station that sends its 0.03\% share to an out-of-state landfill in Georgia.} These three fringe participants have been excluded from the analysis. After re-scaling, the resulting market shares are Republic, 28.4\%, Santek, 54\%, and Waste Connections, 17.6\%.

However, the market shares required for implementation of our model are expressed as upstream-downstream market participant pairs. Following the discussion of supply relationships in Section \ref{sec:application}, vertically-integrated pairs without capacity constraints (i.e., Santek disposal and collection, Waste Connection disposal and collection) are assigned their full collection share. Capacity constrained integrated firms (Republic disposal and collection) are assigned their collection share up to their available capacity with the remainder allocated by residual disposal share to pairs with the respective upstream firms (i.e., Santek disposal and Republic collection, Waste Connections disposal and Republic collection). Unintegrated firms' collection shares are  distributed among the upstream suppliers with available capacity by residual share as well.

MSW disposal prices are collected from the 2019 Waste Business Journal's Directory of Waste Processing \& Disposal Sites. The measure of price is the ``gate rate,'' which is the posted price at the landfill, measured in dollars per ton.\footnote{Disposal prices for large customers may be bilaterally negotiated instead of paying the gate rate.} Republic and Santek both operate multiple landfills in the market with different prices. The price used in the analysis for each is the volume-weighted average for their landfills.

The CIS states that in the Chattanooga Area the post-merger HHI for SCCW collection would be approximately 5,551 post-merger with an increase of 2,660 points and that the combined market share of the merging parties is 73\%. Taking these figures as given we can recover the collection market shares under the assumptions that 1) the merging parties are of equal size, and 2) the non-merging parties are of equal size. After re-scaling, this produces downstream market shares for Republic, 37.6\%, Santek, 37.6\%, Waste Connections, 8.3\%, WM-ADS, 8.3\%, and Regional, 8.3\%.

Collection and disposal margins are calculated for Republic from data on revenue by line of service and from components of the cost of operations in the company's 2019 annual report. The data are reported at the company level and are not specific to the Chattanooga Area, but revenues are reported by collection segment.  Collection costs in dollars per ton are estimated from Republic and Waste Management 2019 10k financial statements. These costs are reported at the company level across all segments. The estimated share of these costs attributable to the Chattanooga market, based on the number of markets in which the companies operate, is divided by the tons disposed for each company. Santek, Waste Connections, and Regional do not publicly produce comparable financial statements. Instead, we assume that the cost structure is the same for the vertically-integrated companies, taking into account their individual tons disposed, and that WM-ADS and Regional share the same cost structure due to their lack of internal disposal capacity.



\end{document}

% \begin{figure}
% \centering
% \includegraphics{output/TrashSimsCompare.png}
% \caption{Comparison between Collection-Only and Full Vertical Simulations for Republic/Santek}
% \label{fig:TrashSimsCompare}
% \end{figure}

%Concerns about vertical competitive effects were raised in both the Republic-Santek and the Waste Management-ADS transactions. Both the Solid Waste Agency of Lake County, IL and the Solid Waste Agency of Northern Cook County submitted comments in opposition to the proposed asset divestiture from Waste Management-ADS, stating that a vertically integrated competitor was needed to maintain competition in their local market post-merger.\footnote{\tiny{See  Comments by SWALCO and SWANCC : U.S. and Plaintiff States v. Waste Management, Inc., and Advanced Disposal Services, Inc., \url{https://www.justice.gov/atr/case-document/file/1377646/download}.}}
%For example, in the DOJ complaint filed in the Republic-Santek case, horizontal anti-competitive effects were alleged for four SCCW collection markets and two MSW disposal markets. In addition, vertical anti-competitive effects were alleged to arise from the combination of their integrated assets in the Chattanooga area.\footnote{See U.S. and State of Alabama v. Republic Services, Inc. and Santek Waste Services, LLC, \url{https://www.justice.gov/atr/case-document/file/1382031/download}}
%Our application in progress further identifies local markets in which these acquisitions result in: 1) only horizontal combinations of assets, 2) only vertical combinations of assets, and 3) combinations of vertically integrated assets in the presence of existing integrated competitors to analyze the welfare effects from mergers with complex vertical arrangements. The next sections present preliminary results on the merger of vertically integrated assets in the presence of other integrated and unintegrated rivals in the context of Republic and Santek's merger in the Chattanooga area.
%The next sections present results of applying the model of Section \ref{sec:theory} to the merger of vertically integrated assets in the presence of other integrated and unintegrated rivals in the context of Republic and Santek's merger in the Chattanooga area.
