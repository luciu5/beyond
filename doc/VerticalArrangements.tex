\documentclass[12pt]{article}
%\documentclass[12pt,review,authoryear]{elsarticle}
%\usepackage{ecrc}       % needed for elsarticle
\usepackage{chicago}    % bibliography package
\usepackage{graphicx}   % insert PostScript figures
\usepackage{setspace}   % controls line spacing
\usepackage{amsmath,amsthm,amssymb,amstext} % controls equation entry and symbols
\usepackage{rotating}   % rotates graphics
\usepackage{soul}       % controls hyphenation
\usepackage{epsfig}     % helps with including graphics
\usepackage{pdflscape}  % helps with displaying rotated graphics in PDFs
\usepackage{lscape}     % helps with rotating pages
\usepackage{caption}    % controls  captions
\usepackage{adjustbox}  % shrink tables
\usepackage[margin=1in]{geometry} % margins
\usepackage{hyperref}   % displays URLs
%packages for kable tables
 \usepackage{booktabs}
 \usepackage{longtable}
 \usepackage{array}
 \usepackage{multirow}
 \usepackage{wrapfig}
 \usepackage{float}
 \usepackage{colortbl}
 \usepackage{pdflscape}
 \usepackage{tabu}
 \usepackage{threeparttable}
 \usepackage{threeparttablex}
 \usepackage[normalem]{ulem}
 \usepackage{makecell}
 \usepackage{xcolor}

\graphicspath{{../output/}}

\interfootnotelinepenalty=10000

\parskip     2.0mm       % space between paragraphs

\alph{footnote}         % make title footnotes alpha-numeric

\captionsetup[figure]{labelsep=space,labelfont=bf} % remove colon from figure name

%%%%%%%%%%%%%%%%%%%%%%%%%%%%%%%%%%%%%%%%%%%%%%%%%%%%%%%%%%%%%%%%%%%%%%%%%%%
% Title Page
%%%%%%%%%%%%%%%%%%%%%%%%%%%%%%%%%%%%%%%%%%%%%%%%%%%%%%%%%%%%%%%%%%%%%%%%%%%

%\newcommand*\samethanks[1][\value{footnote}]{\footnotemark[#1]}
%\author{Margaret Loudermilk\footnote{U.S. Department of Justice, margaret.loudermilk@usdoj.gov} \\  U.S Department of Justice \and Gloria Sheu\footnote{Board of Governors of the Federal Reserve System, gloria.sheu@frb.gov. } \\ Federal Reserve Board \and  Charles Taragin\footnote{Board of Governors of the Federal Reserve System, charles.s.taragin@frb.gov}  \\  Federal Reserve Board }

% \begin{frontmatter}
%
% %% Title, authors and addresses
%
% %% use the tnoteref command within \title for footnotes;
% %% use the tnotetext command for the associated footnote;
% %% use the fnref command within \author or \address for footnotes;
% %% use the fntext command for the associated footnote;
% %% use the corref command within \author for corresponding author footnotes;
% %% use the cortext command for the associated footnote;
% %% use the ead command for the email address,
% %% and the form \ead[url] for the home page:
% %%
% %% \title{Title\tnoteref{label1}}
% %% \tnotetext[label1]{}
% %% \author{Name\corref{cor1}\fnref{label2}}
% %% \ead{email address}
% %% \ead[url]{home page}
% %% \fntext[label2]{}
% %% \cortext[cor1]{}
% %% \address{Address\fnref{label3}}
% %% \fntext[label3]{}
%
% %%\dochead{}
% %% Use \dochead if there is an article header, e.g. \dochead{Short communication}
%
% \title{Beyond ``Horizontal'' and ``Vertical'':\\ The Welfare Effects of Complex Integration\tnoteref{t1} }
% \tnotetext[t1]{The analysis and conclusions set forth are those of the authors and do not indicate concurrence by other members of the Board research staff or by the Federal Reserve Board of Governors. Furthermore, the views expressed here should not be purported to reflect those of the U.S. Department of Justice. This article has benefited from conversations with Alison Oldale, Ted Rosenbaum, and Nathan Wilson.}
%
% %% use optional labels to link authors explicitly to addresses:
% %% \author[label1,label2]{<author name>}
% %% \address[label1]{<address>}
% %% \address[label2]{<address>}
%
% \author[1]{Margaret Loudermilk\fnref{fn1}}
% \ead{margaret.loudermilk@usdoj.gov}
% \affiliation[1]{organization={US Department of Justice}, addressline= {450 5th St. NW}, city={Washington, DC},postcode={20530}}
% \author[2]{Gloria Sheu\fnref{fn1}}
% \ead{gloria.sheu@frb.gov}
% \affiliation[2]{organization={Federal Reserve Board}, addressline= { 2051 Constitution Ave. NW}, city={Washington, DC},postcode={20418} }
% \author[3]{Charles Taragin\corref{cor1}\fnref{fn1}}
%  \cortext[cor1]{Corresponding author.}
% \ead{charles.s.taragin@frb.gov}
% \affiliation[3]{organization={Federal Reserve Board}, addressline= { 2051 Constitution Ave. NW}, city={Washington, DC},postcode={20418} }
% \fntext[fn1]{Declarations of interest: none.}



\title{Beyond Horizontal and Vertical:\\ Measuring the Welfare Effects of Complex Integration\footnote{The analysis and conclusions set forth are those of the authors and do not indicate concurrence by other members of the Board research staff or by the Federal Reserve Board of Governors. Furthermore, the views expressed here should not be purported to reflect those of the U.S. Department of Justice. This article has benefited from conversations with Alison Oldale, Ted Rosenbaum, and Nathan Wilson, along with seminar participants at the U.S. Department of Justice and the Federal Trade Commission.}}

\newcommand*\samethanks[1][\value{footnote}]{\footnotemark[#1]}
\author{Margaret Loudermilk\footnote{margaret.loudermilk@usdoj.gov} \\  U.S Department of Justice \and Gloria Sheu\footnote{gloria.sheu@frb.gov. } \\ Federal Reserve Board \and  Charles Taragin\footnote{charles.s.taragin@frb.gov (corresponding author)}  \\  Federal Reserve Board }

%\date{\today}
\date{October 2024}

\begin{document}

\pagenumbering{roman}       % Roman numerals from abstract to text
\maketitle                  % print title information
\thispagestyle{empty}       % no page number on THIS page

\begin{abstract}
\noindent We study mergers that have both horizontal and vertical aspects using a simulation model with Bertrand competition downstream and Nash Bargaining upstream.  We apply this framework to the 2021 Republic/Santek merger in waste management services, comparing the actual merger to different counterfactual mergers involving various vertical and horizontal combinations.  Within this example, we find almost no instances where the elimination of double marginalization outweighs merger harms to consumers.  Furthermore, we show how analyzing the upstream and downstream markets separately can lead to inaccurate results.

\end{abstract}

% \begin{keyword}
% bargaining models\sep merger simulation\sep vertical markets\sep vertical mergers
% \JEL  L13\sep L40\sep L41\sep L42
% \end{keyword}
%
% \end{frontmatter}

\bigbreak Keywords: bargaining models; merger simulation; vertical markets; vertical mergers

JEL classification: L13; L40; L41; L42



\newpage                    % start a new page
\pagenumbering{arabic}      % Arabic page numbers from now on
\doublespacing

%%%%%%%%%%%%%%%%%%%%%%%%%%%%%%%%%%%%%%%%%%%%%%%%%%%%%%%%%%%%%%%%%%%%%%%%%%%%%%%%%
% Body of Paper
%%%%%%%%%%%%%%%%%%%%%%%%%%%%%%%%%%%%%%%%%%%%%%%%%%%%%%%%%%%%%%%%%%%%%%%%%%%%%%%%%
\section{Introduction}
Given the many different ways a firm may be organized, it is rare that any particular merger can be neatly categorized as purely horizontal or purely vertical.  This is especially true for the mergers involving large firms that attract the most scrutiny.  However, for both historical and practical reasons, antitrust research and policy analysis has generally maintained this dichotomy.  Horizontal mergers are cast in terms of whether they create significant upward pricing pressure (UPP), whereas the focus for vertical mergers is on the net effects of the elimination of double marginalization (EDM) versus raising rivals' costs (RRC).  As a result, the antitrust community has done little research into ``complex'' mergers, which here we define as mergers that possess both horizontal and vertical aspects.  We have scarce guidance for how our merger simulation models and related tools apply to these complex mergers.

The horizontal versus vertical distinction had been enshrined in U.S. Department of Justice (DOJ) and Federal Trade Commission (FTC) Horizontal Merger Guidelines and their companion document, the Vertical Merger Guidelines.\footnote{The most recent Horizontal Merger Guidelines are from 2010, while the most recent Vertical Merger Guidelines are from 2020.  They have both been superseded by a joint document issued in 2023.  }  However, the DOJ and FTC revised their Merger Guidelines in 2023, releasing a single document that discusses both horizontal and vertical competition.  As part of the revision effort, these agencies requested public comments on whether any ``modern market realities may be lost by focusing on [horizontal and vertical] relationships categorically.''\footnote{The full question is 1.g. in the ``Request for Information on Merger Enforcement,'' dated January 18, 2022.  The text of the question reads: ``Should the guidelines' traditional distinctions between horizontal and vertical mergers be revisited in light of recent economic trends in the modern economy? What aspects of modern market realities may be lost by focusing on these relationships categorically? Should the guidelines address all mergers in a common framework that covers all market relationships relevant to competition? If so, how?''  }  In this paper, we address this question by exploring the implications that a standard merger simulation model has for the welfare effects of complex mergers.

Our model is drawn from \citeN{ST2021} and features a Bertrand downstream setup alongside a Nash Bargaining wholesale negotiation upstream.  Our goal is to make our analysis accessible to practitioners, so we use a model based in the merger simulation literature that can be calibrated with limited data.  We then simulate the 2021 merger of two vertically integrated solid waste management firms, Republic Services and Santek Waste Services, in one geographic market, Chattanooga, Tennessee. Both Republic and Santek operated upstream and downstream in Chattanooga, as did one of their largest rivals, Waste Connections.  The DOJ negotiated a settlement package meant to address harms in both the downstream and upstream markets.

Using a small set of public data inputs, our merger simulation model is able to produce rich results for the changes in horizontal and vertical competition, including RRC and EDM effects.  We estimate that without divestitures, consumers would have faced approximately \$16 million in annual harm in the Chattanooga market. We highlight how the results are affected by several key parameters and assumptions, namely the market elasticity, the use of flat logit versus nested demand preferences, and the bargaining parameter.  Furthermore, we show that modeling the merger as a purely horizontal upstream or downstream transaction would miss key vertical impacts and misrepresent the extent of consumer harm.

This empirical example allows us to study the implications of the model for complex mergers.  Starting from the Republic/Santek merger as proposed, we compare our baseline simulation results to those for counterfactual mergers between other firms in the market and to those for counterfactual mergers between different combinations of the upstream and downstream components of Republic and Santek.  This exercise facilitates comparison between how the model behaves for complex mergers versus the more straightforward vertical and horizontal mergers that appear in \citeN{ST2021}.  Based on these simulations, we find that the complex mergers we study all result in net harm to consumers, despite the presence of large EDM effects in certain cases.
 Only the counterfactual pure vertical merger between the Santek upstream division and the Republic downstream division provides net benefits to consumers.

There are a few caveats to our analysis.  First, products in our merger simulations do not experience any changes in their qualities or costs, other than movements in their wholesale prices due to shifts in bargaining leverage.  Therefore, we rule out merger efficiencies as they are conceptualized in, for example, \citeN{werden96}, and we instead focus on EDM as the main channel through which mergers may benefit consumers.  Second, we model upstream bargaining as being over linear fees, rather than over two-part tariffs or other compensation schedules.  Although this assumption means bargaining is inefficient, linear input prices are widely observed in some industries, perhaps because of difficulties with implementing or monitoring more complicated contracts.\footnote{One example of an industry with linear wholesale pricing that has been featured in several vertical merger proceedings is cable programming distribution, as detailed in \citeN{Rogerson2014}.}  Third, our model assumes that upstream and downstream prices are set treating all other prices as fixed.  Although we make this assumption in part for simplicity, it also can be appropriate for industries where prices take time to adjust.\footnote{Examples include situations where products are sold under long-term contracts.  \citeN{CLWY2017} use a similar assumption.  In contrast, \citeN{BBMM2023} allow downstream firms to treat their own prices as adjustable during upstream negotiations.  The authors argue that this setup is useful for settings like grocery stores, where menu costs are low.  }  Fourth, we rely on logit demand for our baseline results, which limits the ways in which consumers substitute between products.  We also provide sensitivity analyses using the nested logit.

Our paper is related to the literature on competition and vertical integration that grew out of a number of theory articles in the 1980s and 1990s.  \citeN{Riordan2008} has an excellent summary. Broadly speaking, this research explores how incentives to disadvantage rivals could arise with vertical integration, and as a corollary shows how vertical mergers could negatively impact consumer welfare.  See, for example, \citeN{SalopScheffman1983}, \citeN{Salinger1988}, \citeN{OSS1990}, and \citeN{Riordan1998}.  In a number of these models, vertically integrated firms raise input prices to competitors or exit the input market entirely.  Unlike in our paper, most of these models are not framed explicitly as bargaining problems, instead assuming posted or non-negotiated prices.  However, many of the incentives towards RRC, which usually involve trading off profits from input sales against those from downstream sales, are similar to those that appear in our model. These theory papers tend to focus on pure vertical mergers rather than complex mergers, in order to maintain mathematical tractability.

Meanwhile, the structural empirical industrial organization literature has studied the competitive effects of vertical integration as part of a greater exploration of bargaining power in vertical supply chains. Many of these papers are based on the model developed in \citeN{HW1988}, modified to allow for empirical estimation, as seen in, for example, \citeN{DKVB2010}.  These models have been applied to several sectors that are frequent targets of vertical integration, such as healthcare (e.g. \citeN{HL2017}) and video programming distribution (e.g. \citeN{CLWY2017}).  These papers explore the welfare effects of various competitive interactions, including complex mergers, in specific industries.\footnote{There is a companion empirical literature that relies more on regression analysis to identify changes in market outcomes due to vertical integration (e.g. \citeN{Chipty2001} on video programming distribution).  }  Our framework is similar to the models appearing in these papers, but it has been simplified for use with limited data, with the goal of being more accessible to practitioners and policymakers.

Our work is also related to the literature on practical methods for measuring the welfare impacts of mergers, particularly papers about vertical mergers.  The closest merger simulation article is our previous research in \citeN{ST2021}, which uses the same model we feature here but assumes that no firm is vertically integrated pre-merger, ruling out complex mergers.  That paper in turn takes the setup from \citeN{WF1994} to characterize downstream competition, and pairs it with a Nash Bargaining model for the upstream market.  \citeN{DVDS2018} and \citeN{DS2019} provide additional vertical merger frameworks that make alternative assumptions about demand and pricing.  \citeN{MS2013}, \citeN{Rogerson2020}, and \citeN{DSS2024} develop UPP-style tools for assessing the impact of vertical mergers.  However, these papers do not focus on complex mergers.

The paper proceeds as follows.  In Section \ref{sec:theory}, we introduce the theoretical framework and explain how we model mergers.  Our application to Republic/Santek appears in Section \ref{sec:application}.  Section \ref{sec:concl} concludes. Additional information on the data used in the empirical application appears in the Appendix.

%%%%%%%%%%%%%%%%%%%%%%%%%%%%%%%%%%%%%%%%%%%%%%%%%%%%%%%%%%%%%%%%%%%%%%%%%%%%%%%%%
\section{Theory\label{sec:theory}}
We begin by describing our basic framework, based on \citeN{ST2021}, which uses a downstream Bertrand model embedded in an upstream Nash Bargaining setup.  This framework allows us to study a variety of complex merger configurations.

%%%%%%%%%%%%%%%%%%%%%%%%%%%%%%%%%%%%%%%%%%
\subsection{Downstream and Upstream Competition}
Assume there is a set of consumers who each choose to buy a single product sold by a retailer.  Retailers are indexed by $r$, and the wholesalers that supply these retailers are indexed by $w$.  Each wholesaler offers only one product, although each retailer may purchase from multiple wholesalers.  Therefore, unique products (meaning a wholesale product offered at a given retail outlet) are denoted by pairs of the retailer and wholesaler indices, $rw$.  We denote the set of all retailers by $\mathcal{R}=\{1, \dots, \left\vert{\mathcal{R}}\right\vert\}$ and the set of all wholesalers by $\mathcal{W}=\{1, \dots, \left\vert{\mathcal{W}}\right\vert\}$.  We divide the set $\mathcal{W}$ into potentially overlapping subsets, each labeled $\mathcal{W}^r$, to indicate which wholesalers' products are carried by which retailers.  Similarly, we divide the set of retailers $\mathcal{R}$ into potentially overlapping subsets, each labeled $\mathcal{R}^w$, to indicate the retailers that carry the product sold by each wholesaler.  Note that throughout we take as given the set of retailer-wholesaler combinations being offered in the market.  We do not model the process of firm entry and exit or the way in which retailers choose which wholesale products to offer.

Let the share of consumers that choose to purchase product $w$ sold by retailer $r$ be denoted by $s_{rw}$.  This share is a function of the vector of retail prices $P$ offered across all available products.\footnote{We omit the retail price argument for $s_{rw}(P)$ in what follows for brevity's sake.}  In our simulations we model the share using both the logit and nested logit functions.  The random coefficients logit is another commonly used option.  The profit for retailer $r$ is given by
\begin{equation}
\pi^r = \sum_{x \in \mathcal{W}^r} [p_{rx} - p^W_{rx} - c^R_{rx}] s_{rx} M,
\label{eq: ret profit}
\end{equation}
where $p^W_{rx}$ is the unit fee charged by wholesaler $x$ to retailer $r$, $c^R_{rx}$ captures any additional marginal costs borne by the retailer, and $M$ is the market size.  Downstream prices are set in Bertrand equilibrium according to
\begin{equation}
\sum_{x \in \mathcal{W}^r} [p_{rx} - p^W_{rx} - c^R_{rx}] \frac{\partial s_{rx}}
{\partial p_{rw}} + s_{rw} = 0,
\label{eq: nb downstream foc}
\end{equation}
which is the first order condition for product $w$ sold by retailer $r$, found by maximizing the profit function $\pi^r$ with respect to $p_{rw}$.

Turning to the upstream market, profits for wholesaler $w$ are given by
\begin{equation}
\pi^w = \sum_{x \in \mathcal{R}^w} [p^W_{xw} - c^W_{xw}] s_{xw} M,
\label{eq: wh profit}
\end{equation}
where $c^W_{xw}$ is the wholesale marginal cost when dealing with retailer $x$.  Wholesale prices are set via bilaterial Nash Bargaining between retailers and wholesalers.  The bargaining problem for wholesale price $p^W_{rw}$  takes the form,
\begin{equation}
\max_{p^W_{rw}} (\pi^r - d^r (\mathcal{W}^r \setminus \{w\}))^{\lambda_w} (\pi^w - d^w (\mathcal{R}^w \setminus \{r\}))^{1 - \lambda_w},
\label{eq: bargain}
\end{equation}
where $\lambda_w \in [0,1]$ captures the bargaining power of retailers relative to wholesaler $w$.  The $d^r (\mathcal{W}^r \setminus \{w\})$ is the disagreement payoff for retailer $r$, and $d^w (\mathcal{R}^w\setminus \{r\})$ is the disagreement payoff for wholesaler $w$.  The disagreement payoff for the retailer is the profits it would earn if it ceases to sell this wholesale product,
\begin{equation}
d^r (\mathcal{W}^r \setminus \{w\}) = \sum_{x \in \mathcal{W}^r \setminus \{w\}} [p_{rx} - p^W_{rx} - c^R_{rx}] s_{rx}(\mathcal{W}^r \setminus \{w\}) M,
\end{equation}
where $s_{rx}(\mathcal{W}^r \setminus \{w\})$ is the realized downstream market share for product $rx$ when retailer $r$ does not carry wholesaler $w$'s product.  Similarly, the disagreement payoff of the wholesaler is
\begin{equation}
d^w (\mathcal{R}^w \setminus \{r\}) = \sum_{x \in \mathcal{R}^w \setminus \{r\}} [p^W_{xw} - c^W_{xw}] s_{xw}(\mathcal{W}^r \setminus \{w\}) M,
\end{equation}
which is earned if wholesaler $w$ does not offer its product through retailer $r$.

We assume that the negotiation for a given input price treats other input prices and all downstream prices as given.\footnote{This assumption is equivalent to a situation where all negotiations and choices for downstream prices happen simultaneously.  }  Then the first order condition for the negotiation between retailer $r$ and wholesaler $w$ is
\begin{equation}
\begin{split}
&\overbrace{[p^W_{rw} - c^W_{rw}]s_{rw} - \sum_{x \in \mathcal{R}^w \setminus \{r\}} [p^W_{xw} - c^W_{xw}] \Delta s_{xw}(\mathcal{W}^r \setminus \{w\})}^{\text{wholesaler GFT}} = \\
&\frac{1-\lambda_w}{\lambda_w} \left(\underbrace{[p_{rw} - p^W_{rw} - c^R_{rw}]s_{rw} - \sum_{x \in \mathcal{W}^r \setminus \{w\}} [p_{rx} - p^W_{rx} - c^R_{rx}] \Delta s_{rx}(\mathcal{W}^r \setminus \{w\})}_{\text{retailer GFT}}\right).
\end{split}
\label{eq: upstream foc}
\end{equation}
The $\Delta s_{xy}(\mathcal{W}^r \setminus \{w\}) = s_{xy}(\mathcal{W}^r \setminus \{w\}) - s_{xy}$ is the difference in the share of good $y$ sold by retailer $x$ when good $w$ is not offered by retailer $r$ versus when good $w$ is offered by retailer $r$.  Thus, the wholesale price $p_{rw}$ is set such that the payoff to wholesaler $w$ when it sells to retailer $r$ less the payoff when it does not (that is, the gains from trade or ``GFT''), divided by the payoff to retailer $r$ when it buys from wholesaler $w$ less the payoff when it does not, equals the ratio of wholesaler to retailer bargaining power.

Together the series of downstream and upstream first order conditions determine market equilibrium.  After assuming a functional form for shares, the model can be solved and calibrated along the lines described in \citeN{ST2021}, using these first order conditions.

%%%%%%%%%%%%%%%%%%%%%%%%%%%%%%%%%%%%%%%%%%
\subsection{Mergers}
Here we describe the manner in which complex mergers are modeled in this framework.  Throughout we assume that all marginal costs, $c^R_{rw}$ and $c^W_{rw}$, and any non-price product characteristics remain unchanged after the merger.  We begin by describing vertical mergers, as all cases we examine have some vertical aspects.  In modeling a merger, we assume that the combined firms maximize the sum of their profits, meaning we add together the profit functions in equations \eqref{eq: ret profit} and \eqref{eq: wh profit} for any merging retailers and wholesalers.  Suppose that retailer $t$ and wholesaler $v$ were to merge.  Then the first order condition for profit maximization for the joint firm when setting the downstream price for product $tv$ is given by
\begin{equation}
\begin{split}
&\sum_{x \in \mathcal{W}^t \setminus \{v\}} [p_{tx} - p^W_{tx} - c^R_{tx}]\frac{\partial s_{tx}}{\partial p_{tv}} + s_{tv} + \overbrace{[p_{tv} - c^W_{tv} - c^R_{tv}]\frac{\partial s_{tv}}{\partial p_{tv}}}^{\text{EDM effect}} \\
&+ \underbrace{\sum_{x \in \mathcal{R}^v \setminus \{t\}} [p^W_{xv} - c^W_{xv}] \frac{\partial s_{xv}}{\partial p_{tv}}}_{\text{upstream UPP effect}} = 0.
\end{split}
\label{eq: vmerger own ret downstream foc}
\end{equation}
The pricing problem balances two effects.  On the one hand, the term labeled ``EDM effect'' captures the impact of retailer $t$ being able to access product $v$ at marginal cost.  This force would tend to lower the resulting price.\footnote{However, this lower marginal cost may also encourage a multi-product retailer to raise prices on its other goods $x \in \mathcal{W}^t \setminus \{v\}$, in order to divert sales to product $tv$.  This incentive is modeled in \citeN{Salinger1991} and studied empirically in \citeN{LM2020}, who call it the ``Edgeworth-Salinger effect.''}  On the other hand, the term labeled ``upstream UPP effect'' captures the incentive to raise prices for retailer $t$ in order to divert sales to wholesaler $v$.\footnote{\citeN{Chen2001} models such an effect.}  This force would tend to raise the resulting consumer price.

Turning to input prices, when wholesaler $v$ bargains with unaffiliated retailer $j$, the first order condition becomes
\begin{equation}
\begin{split}
&[p^W_{jv} - c^W_{jv}]s_{jv} - \sum_{x \in \mathcal{R}^v \setminus \{t,j\}} [p^W_{xv} - c^W_{xv}] \Delta s_{xv}(\mathcal{W}^j \setminus \{v\})\\
& - \overbrace{\overbrace{[p_{tv}-c^W_{tv}-c^R_{tv}] \Delta s_{tv}(\mathcal{W}^j \setminus \{v\})}^{\text{indirect EDM effect}} - \sum_{x \in \mathcal{W}^t \setminus \{v\}} [p_{tx} - p^W_{tx} - c^R_{tx}] \Delta s_{tx}(\mathcal{W}^j \setminus \{v\})}^{\text{RRC effect}}=\\
&\frac{1-\lambda_v}{\lambda_v} \left([p_{jv} - p^W_{jv} - c^R_{jv}]s_{jv} - \sum_{x \in \mathcal{W}^j \setminus \{v\}} [p_{jx} - p^W_{jx} - c^R_{jx}] \Delta s_{jx}(\mathcal{W}^j \setminus \{v\})\right).
\end{split}
\label{eq: vmerger wh upstream foc}
\end{equation}
which reflects the change in the disagreement payoff coming from the merger with retailer $t$.  Now when the wholesaler considers the possible loss of sales upon ceasing to trade with retailer $j$, these losses are softened due to a potential for diversion to retailer $t$, which we label the ``RRC effect.''\footnote{The assumption of simultaneous downstream price setting rules out RRC effects in the style of \citeN{SalopScheffman1983}, in that here the wholesaler does not have an incentive to raise prices in order to directly impact the pricing strategies of rival retailers.  Rather, our model has the RRC effect seen in \citeN{CLWY2017}, where the wholesaler has better bargaining leverage over rival retailers due to the possibility of increased sales volume (conditional on prices) for the merged downstream retail partner should negotiations break down.}  Furthermore, the margin on product $v$ sold by retailer $t$ is potentially higher due to EDM than the wholesale margin $v$ would have earned, as shown through the expression labeled ``indirect EDM effect,'' which can further compensate the firm.  These impacts tend to raise the resulting input price to retailer $j$.

When the merged firm is bargaining with the unaffiliated wholesaler $k$ over what input price to pay, the bargaining first order condition becomes
\begin{equation}
\begin{split}
&[p^W_{tk} - c^W_{tk}]s_{tk} - \sum_{x \in \mathcal{R}^k \setminus \{t\}} [p^W_{xk} - c^W_{xk}] \Delta s_{xk}(\mathcal{W}^t \setminus \{k\}) = \\
&\frac{1-\lambda_k}{\lambda_k} \left([p_{tk} - p^W_{tk} - c^R_{tk}]s_{tk} - \sum_{x \in \mathcal{W}^t \setminus \{v,k\}} [p_{tx} - p^W_{tx} - c^R_{tx}] \Delta s_{tx}(\mathcal{W}^t \setminus \{k\})\right.\\
&\left.- \underbrace{[p_{tv} - c^W_{tv} - c^R_{tv}] \Delta s_{tv}(\mathcal{W}^t \setminus \{k\})}_{\text{EDM recapture effect}}-\underbrace{\sum_{x \in \mathcal{R}^v \setminus \{t\}} [p^W_{xv} - c^W_{xv}] \Delta s_{xv}(\mathcal{W}^t \setminus \{k\})}_{\text{wholesale recapture leverage effect}}\right).
\end{split}
\label{eq: vmerger ret upstream foc}
\end{equation}
In this case, the merged firm has two channels for potential additional profits should it cease to trade with wholesaler $k$.  First, if retail sales are diverted to product $v$ sold by retailer $t$, those sales could earn a higher margin due to lower marginal costs stemming from what we call the ``EDM recapture effect.''  Second, the loss of product $k$ carried by retailer $t$ could increase sales by wholesaler $v$ through other retailers, which we call the ``wholesale recapture leverage effect.''  Both of these effects would tend to lower the resulting input price.

Next consider a merger between an integrated retailer/wholesaler $tv$ and a standalone retailer $s$.  Such a combination has a vertical component and a horizontal component.  The first order condition for setting the downstream price $p_{tv}$ becomes
\begin{equation}
\begin{split}
&\sum_{x \in \mathcal{W}^t \setminus \{v\}} [p_{tx} - p^W_{tx} - c^R_{tx}]\frac{\partial s_{tx}}{\partial p_{tv}} + s_{tv} + \overbrace{[p_{tv} - c^W_{tv} - c^R_{tv}]\frac{\partial s_{tv}}{\partial p_{tv}}}^{\text{direct EDM effect}}+ \underbrace{\sum_{x \in \mathcal{R}^v \setminus \{t, s\}} [p^W_{xv} - c^W_{xv}] \frac{\partial s_{xv}}{\partial p_{tv}}}_{\text{upstream UPP effect}} \\ &+\underbrace{\underbrace{[p_{sv} - c^W_{sv} - c^R_{sv}]\frac{\partial s_{sv}}{\partial p_{tv}}}_{\text{indirect EDM Effect}} + \sum_{x \in \mathcal{W}^s \setminus \{v\}} [p_{sx} - p^W_{sx} - c^R_{sx}] \frac{\partial s_{sx}} {\partial p_{tv}}}_{\text{downstream UPP effect}}= 0.
\end{split}
\label{eq: intdmerger own ret downstream foc}
\end{equation}
Now there is the possibility for what we call the ``downstream UPP effect'' in the retail market, as the merged firm can recapture sales that are diverted to retailer $s$ when $t$ raises its price.  EDM between $v$ and $s$ can actually increase this UPP impact, because when sales are diverted to product $v$ sold by retailer $s$, those units may earn a larger margin.  This impact, which comes through the interaction of EDM with UPP, we label the ``indirect EDM effect.''

If instead an integrated retailer/wholesaler $tv$ were to merge with a standalone wholesaler, then the resulting first order condition for product $tv$ would look similar to equation \eqref{eq: intdmerger own ret downstream foc} with some changes.  The downstream UPP effect would be replaced with an additional upstream UPP effect capturing the value of sales diverted to customers of the new partner wholesaler.  The condition would also include an indirect EDM component, reflecting the ability of retailer $r$ to obtain an additional wholesale product at marginal cost, which in turn would tend to raise the value of diverted sales to that product.  Whether these additional incentives to raise prices dominate the direct EDM impact is an empirical question.  Furthermore, if we instead considered a merger between two integrated retailer/wholesaler pairs, both upstream and downstream UPP effects would enter.

As for wholesale prices, again consider a merger between an integrated retailer/wholesaler $tv$ and a standalone retailer $s$.  When wholesaler $v$ bargains with an unaffiliated retailer $j$ we have the first order condition given by,
\begin{equation}
\begin{split}
&[p^W_{jv} - c^W_{jv}]s_{jv} - \sum_{x \in \mathcal{R}^v \setminus \{t,s,j\}} [p^W_{xv} - c^W_{xv}] \Delta s_{xv}(\mathcal{W}^j \setminus \{v\})\\
& - \overbrace{\sum_{x \in \{t,s\}}\left(\overbrace{[p_{xv}-c^W_{xv}-c^R_{xv}] \Delta s_{xv}(\mathcal{W}^j \setminus \{v\})}^{\text{indirect EDM effect}} - \sum_{y \in \mathcal{W}^x \setminus \{v\}} [p_{xy} - p^W_{xy} - c^R_{xy}] \Delta s_{xy}(\mathcal{W}^j \setminus \{v\})\right)}^{\text{RRC effect}}=\\
&\frac{1-\lambda_v}{\lambda_v} \left([p_{jv} - p^W_{jv} - c^R_{jv}]s_{jv} - \sum_{x \in \mathcal{W}^j \setminus \{v\}} [p_{jx} - p^W_{jx} - c^R_{jx}] \Delta s_{jx}(\mathcal{W}^j \setminus \{v\})\right).
\end{split}
\label{eq: intdmerger wh upstream foc}
\end{equation}
The RRC effect is augmented with the profits emanating from the sales of retailer $s$, in addition to the sales of retailer $t$, both of which may potentially recapture sales should retailer $j$ lose access to product $v$.  The merged firm has additional bargaining leverage as a result.  If instead the integrated retailer/wholesaler $tv$ were to merge with a standalone wholesaler, then the profits earned by firm $tv$ should the negotiation with retailer $j$ fail are augmented with the earnings of the additional wholesaler rather than with the earnings of retailer $s$.  This adds a term similar to the wholesale recapture leverage effect seen in equation \eqref{eq: vmerger ret upstream foc} to the left-hand side of the bargaining first order condition.\footnote{For simplicity, we assume that when a retailer fails to reach an agreement with one wholesaler, that retailer's contracts with all other wholesalers remain in place.  }    In the case of a merger between two integrated retailer/wholesaler firms, all of these effects would appear.

Returning to a merger between an integrated retailer/wholesaler $tv$ and a standalone retailer $s$, when the merged firm bargains with an unaffiliated wholesaler $k$ to supply retailer $t$, the first order condition becomes
\begin{equation}
\begin{split}
&[p^W_{tk} - c^W_{tk}]s_{tk} - \sum_{x \in \mathcal{R}^k \setminus \{t\}} [p^W_{xk} - c^W_{xk}] \Delta s_{xk}(\mathcal{W}^t \setminus \{k\}) = \\
&\frac{1-\lambda_k}{\lambda_k} \left([p_{tk} - p^W_{tk} - c^R_{tk}]s_{tk} - \sum_{x \in \mathcal{W}^t \setminus \{v,k\}} [p_{tx} - p^W_{tx} - c^R_{tx}] \Delta s_{tx}(\mathcal{W}^t \setminus \{k\})\right.\\
&- \underbrace{\sum_{x \in \{t,s\}}[p_{xv} - c^W_{xv} - c^R_{xv}] \Delta s_{xv}(\mathcal{W}^t \setminus \{k\})}_{\text{EDM recapture effect}}
- \underbrace{\sum_{x \in \mathcal{W}^s \setminus \{v\}} [p_{sx} - p^W_{sx} - c^R_{sx}] \Delta s_{sx}(\mathcal{W}^t \setminus \{k\})}_{\text{retail recapture leverage effect}}\\
&-\left.\underbrace{\sum_{x \in \mathcal{R}^v \setminus \{t,s\}} [p^W_{xv} - c^W_{xv}] \Delta s_{xv}(\mathcal{W}^t \setminus \{k\})}_{\text{wholesale recapture leverage effect}}\right).
\end{split}
\label{eq: intdmerger ret upstream foc}
\end{equation}
Compared to equation \eqref{eq: vmerger ret upstream foc}, now the EDM recapture effect applies to sales of product $v$ at both affiliated retailers $t$ and $s$.  Furthermore, there is also the possibility that sales will be diverted to retailer $s$ should retailer $t$ lose access to product $k$, which creates an additional ``retail recapture leverage effect'' alongside the wholesale recapture leverage effect.  All of these additional terms tend to increase the bargaining leverage of the merged firm.  If we instead examine a merger between retailer/wholesaler $tv$ and an unaffiliated wholesaler, that would augment the EDM recapture effect with sales of retailer $t$ for two merged wholesalers, rather than sales for wholesaler $v$ through two merged retailers.  Similarly, the retail recapture leverage effect would be replaced with an additional wholesale recapture leverage effect for the newly merged wholesaler.  For a merger between two integrated retailer/wholesaler firms, all of these effects would enter.


%%%%%%%%%%%%%%%%%%%%%%%%%%%%%%%%%%%%%%%%%%%%%%%%%%%%%%%%%%%%%%%%%%%%%%%%%%%%%%%%%
\section{Empirical Application: the Chattanooga Waste Industry \label{sec:application}}
The U.S. waste and recycling industry generates approximately \$80 billion in annual revenues.\footnote{See Waste Dive, \tiny{\url{https://www.wastedive.com/news/public-companies-increased-control-of-74b-us-waste-industry-in-2018/556079/}}.} In recent years the industry has experienced significant merger activity, including several acquisitions between vertically integrated, nationally active competitors. Furthermore, there are a variety of vertical supply chain configurations across local markets, making the industry an excellent application to study mergers that have both horizontal and vertical aspects.  In this section, we use one such merger, between the firms Republic and Santek, to explore how our model captures and weighs EDM, RRC, and UPP effects in a real world setting.  Throughout this exercise, our focus is on mirroring the types of analysis usually used by antitrust agencies in reviewing mergers.  Given this context and the types of constraints that agencies often face, we assume that a limited amount of data are available and tailor our calibration methods accordingly.

%%%%%%%%%%%%%%%%%%%%%%%%%%%%%%%%%%%%%%%%%%
\subsection{Market Background}
Republic Services, the second largest waste management company in the U.S., acquired Santek Environmental in 2021.\footnote{See the complaint: U.S. and State of Alabama v. Republic Services, Inc. and Santek Waste Services, LLC, \url{https://www.justice.gov/atr/case-document/file/1382031/download}, at paragraph 7.}  The DOJ filed a complaint, alleging horizontal anti-competitive effects for four small-container commercial waste (SCCW) collection markets and two municipal solid waste (MSW) disposal markets spread across multiple states.\footnote{See the complaint: U.S. and State of Alabama v. Republic Services, Inc. and Santek Waste Services, LLC at paragraphs 37-43.}   In addition, vertical anti-competitive effects were alleged to arise from the combination of Republic and Santek integrated assets in the Chattanooga region.\footnote{See the complaint: U.S. and State of Alabama v. Republic Services, Inc. and Santek Waste Services, LLC at paragraphs 44-47.}  The merger was ultimately settled with the DOJ through divestitures. As described in the Final Judgment filed with the settlement package, the parties were required to divest Santek's SCCW collection assets as well as two landfills and a transfer station in the Chattanooga region.\footnote{See the Final Judgment: U.S and Plaintiff States v. Republic Services, Inc. and Santek, LLC, \url{https://www.justice.gov/atr/case-document/file/1408616/download}.} In this section, we apply our model to simulate the effects of this merger in Chattanooga.

The vertical supply chain in the solid waste industry is primarily comprised of waste collection operations or ``haulers'' and waste disposal facilities. Haulers collect MSW from businesses and residences and must dispose of it at a lawful disposal site, usually a landfill. Waste disposal (upstream) is a required input into waste collection services (downstream). Some haulers are vertically integrated and operate their own disposal facilities. Integrated haulers typically prefer to dispose of waste at their own disposal facilities and may also sell a portion of their disposal capacity. Disposal customers include private waste haulers without their own disposal assets as well as local governments that collect their citizens' waste themselves.

The Competitive Impact Statement (CIS) filed by the DOJ in association with the Republic/Santek settlement describes the alleged lost competition in the ``Chattanooga, Tennessee and North Georgia area,'' subsequently referred to as the Chattanooga Area, due to lost horizontal competition in MSW disposal and SCCW collection as well as due to RRC in the SCCW collection market by raising the MSW disposal costs of independent haulers.\footnote{See the CIS: U.S. and State of Alabama v. Republic Services, Inc.and Santek Waste Services, LLC, \url{https://www.justice.gov/atr/case-document/file/1382626/download}.} The CIS notes that pre-merger, Republic and Santek combined served approximately 73 percent of the SCCW collection market with three other significant competitors. In MSW disposal, the CIS identifies only one other significant competitor pre-merger and Republic and Santek combined as serving approximately 82 percent of the market, disposed of either directly in the merging parties' landfills within the area or passing through their transfer stations before ultimately being disposed of in the parties' landfills elsewhere. Thus, pre-merger both parties were large, vertically integrated competitors in the Chattanooga Area. The DOJ complaint says that the merger would have raised the Chattanooga Area collection Herfindahl-Hirschman Index (HHI) by 2,660 points to 5,551 and the disposal HHI by 3,018 points to 6,980.\footnote{See the complaint, U.S. and State of Alabama v. Republic Services, Inc. and Santek Waste Services, LLC at paragraphs 38 and 42.}

In addition, another large, vertically integrated company, Waste Connections, existed in Chattanooga at the time of the merger and was the parties' sole competitor in MSW disposal. National firms Waste Management and ADS also owned collection assets in the area but were not vertically integrated in this market, as demonstrated by MSW disposal data discussed below.\footnote{Firms that are national competitors and vertically integrated in other markets are known to enter into contracts with each other to dispose of waste on advantageous terms that may make them effectively vertically integrated. Ignoring these contracting relationships may underestimate the number of effectively vertically integrated competitors in the market.} We treat Waste Management and ADS as a single entity (henceforth ``WMADS'') because they merged with one another in 2020, and that transaction was completed before the filing of the CIS for Republic/Santek. The final significant participant in the SCCW collection market was a major regional firm that is not identified by name in the CIS.  (We label this firm ``Regional'' in what follows.)

Based on the available data for disposal capacity and volumes, the pre-merger supply relationships between the upstream disposal market participants (Republic, Santek, and Waste Connections) and the downstream collection market participants (Republic, Santek, Waste Connections, WMADS, and Regional) can be inferred under the assumption that vertically-integrated haulers first dispose of waste at their own disposal facilities and then sell any residual disposal capacity to rival haulers. Santek and Waste Connections exhibited excess disposal capacity. However, Republic's collection volumes were estimated to exceed their ability to self-supply disposal in the Chattanooga market. As a result, Republic needed to purchase additional disposal from Santek and Waste Connections, presumably at a higher marginal cost.  Other than imposing that Republic only offers disposal to itself, we do not structurally incorporate capacity constraints into the simulation model. We assume that the changes in volume that would be caused by the merger would not be large enough to shift which constraints would be binding.\footnote{We also assume that the Waste Connections-Republic product remains post-merger, which may only be true in the short run.}

The resulting relationships between the upstream and downstream market participants are captured in Figure \ref{fig:chatt_tree}.  There are a total of 9 wholesaler-retailer products in the market.  The complex nature of the merger between Republic and Santek is immediately apparent.  There are potential benefits from EDM for the purchases of disposal by Republic from Santek, but also potential harm from RRC by Santek when supplying disposal to rivals WMADS and Regional, alongside a loss of horizontal competition between Republic and Santek in downstream collection.

%%%%%%%%%%%%%%%%%%%%%%%%%%%%%%%%%%%%%%%%%%
\subsection{Simulation Inputs and Calibration}
Before simulating the merger, we need to specify a form for consumer demand.  We use the multinomial logit.  Although the logit makes some restrictive assumptions, it is often used in merger analysis because of the heavy reliance on market shares in antitrust proceedings.\footnote{\citeN{MS2021} note that given the highly specific scope of the markets used for antitrust analysis, the logit can be a reasonable approximation, and has appeared in past merger reviews.  }  The logit implies that substitution, and hence competition, between products is proportional to market shares, which dovetails intellectually with the antitrust emphasis on closely scrutinizing mergers between firms with large market shares.\footnote{For example, the 2023 DOJ/FTC Merger Guidelines at \S 2.1 state that mergers raise a presumption of illegality when the change in the HHI is more than 100 points and results in a market concentration of at least 1,800 or in a merged firm with at least 30\% market share. Given that the change in the HHI is equal to two times the product of the merging firms' market shares, these presumptions are likely to screen in mergers when both firms have high shares.}  Nevertheless, we also provide some sensitivity analysis using the nested logit.

The logit demand implies the following form for the share of product $rw$,
\begin{equation}
s_{rw} = \frac{\exp(\delta_{rw} - \alpha p_{rw})}{1 + \sum_{x \in \mathcal{R}} \sum_{y \in \mathcal{W}^x} \exp(\delta_{xy} - \alpha p_{xy})},
\label{eq: logit share}
\end{equation}
where $\delta_{rw}$ is a quality parameter and $\alpha$ captures sensitivity to price.  There is an outside good whose quality parameter and price have been normalized to zero.  We use this share equation to substitute into the first order conditions described in Section \ref{sec:theory}, which then form the set of equations used for calibration.

For our sensitivity analysis using nested logit demand, we divide the set of retailers into a series of nests, with each nest indexed by $g$.  The set of retailers within each nest is $\mathcal{R}_g$, and the set of all nests is  $\mathcal{G}$.  Then the share for product $rw$ is given by
\begin{equation}
s_{rw} = \frac{\exp((\delta_{rw}-\alpha p_{rw})/(1-\sigma))}{D_g^{\sigma} \sum_{f \in \mathcal{G}} D_f^{1-\sigma}},
\label{eq: nl share}
\end{equation}
where
\begin{equation}
D_g = \sum_{r \in \mathcal{R}_g} \sum_{w \in \mathcal{W}^r} \exp((\delta_{rw} - \alpha p_{rw})/(1-\sigma)) \quad \forall g \in \mathcal{G},
\end{equation}
and where $\sigma$ is a nesting parameter such that $0 \leq \sigma < 1$.  When $\sigma=0$, the model reverts to the logit.

Returning to the logit version of the model, in order to calibrate it, we need enough data to identify the model parameters and primitives, which consist of the vector of product qualities $\mathbf{\delta}$, the price coefficient $\alpha$, the bargaining parameters $\lambda$, and any unobserved upstream or downstream marginal costs.  In a scholarly research setting, one would typically proceed by gathering a dataset of prices, market shares, and instrumental variables with enough variation to estimate the consumer demand parameters.  Then given that demand system and the firm pricing first order conditions of the model, one would recover estimates of marginal costs.  However, antitrust practitioners frequently do not have access to the large cross section or panel datasets necessary for demand estimation, instead observing only shares and prices for one or two markets, usually at only a single point in time.  In our context, this means having a snapshot of data for the nine products in the Chattanooga Area.  Nevertheless, antitrust practitioners can often supplement these data with some information on profit margins.  Margins are helpful in that, combined with the firm pricing first order conditions, they provide information on the implied price elasticities and demand parameters that caused firms to choose those margins.

We follow calibration methods similar to those described in \citeN{ST2021}.  The derivatives of logit demand are functions only of market shares and the price coefficient $\alpha$, as $\partial s_{rx}/\partial p_{rw} = \alpha s_{rx} s_{rw}$  for $x \neq w$, $\forall r$ and $\partial s_{rw}/\partial p_{rw} = -\alpha s_{rw} (1-s_{rw})$ for $x=w$.  Furthermore, the logit also means that $\Delta s_{xy} (\mathbb{W}^r \setminus \{w\}) = s_{rw} \left(s_{xy}/(1-s_{rw})\right)$.  Therefore, the downstream pricing first order conditions and the upstream bargaining first order conditions described in Section \ref{sec:theory} form a system of equations that are functions only of market shares, margins, the downstream price coefficient, and the upstream bargaining parameters.  Given this system of equations, our model is identified if we observe the upstream margins for all products and a downstream margin for a single product, along with the full vector of market shares, upstream prices, and downstream marginal costs.  Calibration proceeds by numerically solving for the unknown objects in the system of first order condition equations described in Section \ref{sec:theory}, given the input data.  Once the price coefficient $\alpha$ is calibrated, downstream pre-merger equilibrium prices can be recovered from the appropriate downstream pricing first order condition, and  the quality parameters can be recovered from the market shares in equation \eqref{eq: logit share}.  After recovering all parameters and pre-merger prices, we use code from the \texttt{antitrust} R package to run the merger simulations.\footnote{When downstream prices rather than costs are observed, the \texttt{vertical.barg} function may be used to calibrate bargaining and demand parameters and simulate merger effects for all complex mergers discussed here. See  \url{https://CRAN.R-project.org/package=antitrust}.}  We again use the first order conditions from Section 2 to perform these simulations, but now assume that any merged firms jointly maximize their profits.

The intuition behind the identification of the model is as follows.  The process for identifying the downstream parameters is similar to that seen in \citeN{WF1994}.  Conditional on the other data, knowledge of a downstream margin determines how sensitive retail customers are to price, as a higher margin implies less elastic demand.  Given this implied relationship, optimal downstream prices can be recovered using the firm first order conditions with data on shares and costs.  Then knowledge of downstream market shares identifies the quality of each product, as higher shares conditional on prices imply higher quality.  Turning to the upstream model, the remaining unknown objects are the bargaining parameters.  Conditional on the data and the downstream model, bargaining power is identified by the relative sizes of the upstream and downstream profit margins.  When downstream firms have higher margins than upstream firms, that is an indicator that retailers have more bargaining power.

Our data comes from several public sources.  We obtain 2019 information on disposal volumes and landfill usage from the Tennessee Department of Environmental Quality. We assume that Republic purchases disposal from both Santek and Waste connections according to their shares of available excess MSW disposal capacity of 63.8\% and 36.2\%, respectively. We further assume that WMADS and Regional purchase disposal capacity from both Santek and Republic according to these shares. The DOJ complaint states that due to strict laws and regulations that govern the disposal of MSW, there are no reasonable substitutes for MSW disposal.\footnote{See the complaint: U.S. and State of Alabama v. Republic Services, Inc. and Santek Waste Services, LLC at paragraph 25.}  We interpret this statement as implying that the share of the outside good in our simulations is zero.  We gather information on disposal prices from the 2019 Waste Business Journal’s Directory of Waste Processing \& Disposal Sites.  Margin and downstream cost information comes from 2019 annual reports and public financial statements, under the assumption that the national averages reported there are reasonable approximations for the Chattanooga market.  Our data sources and construction are described in more detail in Appendix \ref{app:trashdata}.

Our final input data set is presented in Table \ref{tab:trashdata}.  As discussed above and in Appendix \ref{app:trashdata}, we need to make several simplifying assumptions in order to transform the available information into the data required for our model.  We expect that the DOJ would have had access to confidential information during their investigation, which could have allowed them to generate more precise results.  Nevertheless, we believe our results are qualitatively accurate, as they are consistent with several of the anti-competitive concerns raised in the Republic/Santek complaint and CIS.


%%%%%%%%%%%%%%%%%%%%%%%%%%%%%%%%%%%%%%%%%%
\subsection{Republic/Santek Merger Simulation Results}
The results of our merger simulation for prices and market shares appear in Figure \ref{fig:TrashSimsFirms}.  The merger between Republic and Santek is estimated to lower upstream prices by almost 3\% and increase downstream prices by 12.5\%, resulting in \$16.6 million of annual harm to consumers and total annual producer benefits of approximately \$14 million. Thus, on net the model predicts the merger would be harmful despite the presence of significant EDM that reduces prices upstream.\footnote{Our calibrated bargaining parameters for retailers against Santek ($\lambda_{Santek}=1$) and Waste Connections ($\lambda_{WasteConn}=0.73$) give a substantial edge to downstream firms, which suggests consumers are likely to be harmed because EDM benefits are probably small.  High bargaining power parameters are needed to explain the high collection margin compared to the disposal margin.  (See Table \ref{tab:trashdata}.)}

Disposal prices for pre-merger integrated pairs are unchanged post-merger. However, MSW disposal prices increase to all rival haulers post-merger (i.e., all unintegrated pairs). Santek's disposal prices to downstream rivals are estimated to increase over 88\% post-merger, reflecting RRC.  These higher Santek prices allow Waste Connections to also increase its prices to downstream rivals by 22\%. These large price increases are somewhat offset by the large decrease in the disposal price for the integrating pair post-merger due to EDM, as the cost to Republic for volume disposed with Santek decreases by approximately 63\%.

Turning to the downstream market, all collection prices increase post-merger. The merging parties' post-merger collection prices increase more than 19\%, and their share decreases by about 16 percentage points overall. The Republic-Republic product's price increases 21\%, and the Santek-Santek product's price increases by 23\%. The newly integrated Santek-Republic product's price is estimated to increase by almost 9\%, demonstrating that EDM is not fully passed-through to consumers. The integrated rival product, Waste Connections-Waste Connections, increases its collection price by 5\%, and its market share increases by almost 10 percentage points. The unintegrated rivals' collection prices each increase by about 8\%, which again demonstrates incomplete pass-through of the change in upstream costs.  Given that all downstream prices increase, consumers are harmed by the merger.  Therefore, although the merger results in a large EDM effect for Santek disposal sold to Republic, it is outweighed by RRC effects in the disposal market and the loss of horizontal competition in the collection market.

This simulation admittedly relies on a simplified model, particularly with regards to the use of logit demand.  The logit makes two strong assumptions: (1) that overall consumer welfare relies heavily on the extent of substitution to the outside good and (2) that substitution within the market is proportional to market shares.  We can examine the sensitivity of our results to the first issue by varying our chosen value for the share of the outside good.  The form of the logit implies that this share is proportional to the market elasticity of demand.\footnote{The market elasticity with respect to a price change for all inside products is $ -\alpha \overline{p} s_{00}$, where $s_{00}$ is the share of the outside good and $\overline{p}$ is the share-weighted average price among the inside goods. }  Our baseline results assume that the outside share is zero, following the assertion made by the DOJ that there are essentially no reasonable substitutes for SCCW collection or MSW disposal.\footnote{See the CIS: U.S and Plaintiff States v. Republic Services, Inc. and Santek, LLC.}  There are strict laws that dictate where and how commercial customers can dispose of waste.  Such an assumption naturally tends to increase the harm from the merger, because consumers do not exit the market in response to a price increase. Results for alternative assumptions appear in  Figure \ref{fig:TrashElast}. The baseline results are those for the perfectly inelastic market appearing as the leftmost point.  As we increase the share of the outside good (which, equivalently, increases the absolute value of the market elasticity of demand), the estimated harm to consumers decreases monotonically.  However, the harm is still about a third of the baseline results when the market is unit elastic.  Consumer harm remains over 5\% of pre-merger expenditures when the market elasticity is as high as 1.3.

In order to examine how sensitive our results are to the substitution proportional to share assumption, we can model demand instead as a nested logit.  We present two alternatives.  In the first option (which we call ``integrated'' nesting), we organize the nests into two groups, one for the integrated products, meaning Republic-Republic, Santek-Santek, and Waste Connections-Waste Connections, and the second for all other products where the disposal and collection are performed by different firms.  In the second option (which we call ``landfill'' nesting), we have three nests, one each for Republic, Santek, and Waste Connections, where each nest collects together all products that use the same landfill.  We run these simulations for different possible values of the nesting parameter $\sigma$ between 0 and 0.9, where 0 is equivalent to the baseline logit.  As $\sigma$ increases towards 1, the nests become more differentiated.

Our results appear in Figure \ref{fig:TrashNest}.  The baseline simulation is represented by the point at a nesting parameter of 0 on the left in the figure.  As $\sigma$ increases, consumer harm under integrated nesting increases and consumer harm under landfill nesting decreases.  These findings are intuitive, as with integrated nesting, the two highest selling products, Republic-Republic and Santek-Santek, are in a nest with only one other competitor, the much less popular Waste Connections-Waste Connections.  As the nesting parameter tends towards one, the merger of Republic and Santek approaches a merger to monopoly in this nest, which causes a lot of harm at high nesting parameters, as evidenced by the jump in harm at a nesting parameter of 0.9.  On the other hand, with landfill nesting, these two best selling products are separated into different nests, resulting in less horizontal competition between the merging firms and hence less consumer harm, especially when the nesting parameter is large.  However, consumer harm remains noticeable under landfill nesting, at around 10\% of pre-merger expenditures, when the nesting parameter is around 0.5.

Another key part of our model is the bargaining power parameters. As discussed in \citeN{ST2021}, vertical mergers tend to benefit consumers when wholesalers have relatively more power versus when retailers have relatively more power. Larger wholesaler bargaining power offers more possibilities for EDM, because pre-merger input prices are likely to be high. This channel becomes less relevant as retailers gain more power. Our baseline merger simulation uses relatively large bargaining power parameters of 1 for retailers negotiating with Santek and 0.73 for negotiating with Waste Connections, which reflect the higher relative margin we observe for collection versus disposal.  We test the sensitivity of our results to the bargaining power parameter by varying it in alternative simulations.\footnote{In these simulations, we set all parameters for bargaining power between all non-integrated wholesaler-retailer pairs to be identical.  We found that introducing some variation across wholesaler-retailer combinations did not meaningfully change our results.  }  Figure \ref{fig:TrashBarg} provides our results, with the value for consumer harm in the baseline simulation denoted by the horizontal dashed line towards the top of the graph.  As expected, consumer harm is larger when the bargaining power parameter is higher, similar to what is seen in \citeN{ST2021} for vertical mergers.  Interestingly, consumer harm is close to the baseline value for bargaining parameters as low as 0.4.  Below that, EDM effects become more pronounced, and they even result in net consumer benefits when retailer bargaining power falls down to 0.1.  This result suggests that only in a counterfactual world where retail margins were observed to be much lower than wholesale margins should the DOJ have expected to find EDM effects to outweigh other harms.  The data we have in Table \ref{tab:trashdata} suggests the opposite was true.

In summary, our simulations estimate that the complex merger between Republic and Santek would have caused significant harm to consumers if it had been allowed to proceed without divestitures.  Although the data inputs into the model are admittedly crude (though, in our experience, using data of comparable quality for merger analysis is not uncommon given the constraints that antitrust agencies face), we find that nontrivial consumer harm remains even after significant departures from our baseline parameter values.  Furthermore, even when varying these parameters causes the results to meaningfully change, they shift in ways that are predictable and intuitive, as demonstrated by the monotonic relationships seen in Figures \ref{fig:TrashElast}, \ref{fig:TrashNest}, and \ref{fig:TrashBarg}.  These results suggest that the model is useful for analyzing complex mergers in a policy context.

%%%%%%%%%%%%%%%%%%%%%%%%%%%%%%%%%%%%%%%%%%
\subsection{Comparison to Alternative Models}
In the interests of expediency or due to data and time constraints, it is common for antitrust practitioners to use merger simulations that do not account for the vertical aspects of a given market. In order to examine what impact such an assumption has on the simulations, we can compare the full vertical model results, which compute both post-merger downstream and upstream price effects, to two other models that hold the prices of one level fixed.\footnote{This approach uses the same calibrated demand and costs parameters across all three models.} The ``downstream-only'' model computes collection price effects holding disposal prices fixed at pre-merger levels, and the ``upstream-only'' model computes disposal price effects holding collection prices fixed at pre-merger levels.

The results of this comparison are summarized in  Figure \ref{fig:TrashSimsCompareAll}.  Ignoring vertical relationships in the downstream-only model gives in an estimated 9.2\% increase in collection prices with an 16.4\% increase in the merging parties' price, consumer harm of \$10 million annually, and net overall harm of approximately \$2 million. Estimated consumer harm is 60\% higher in the full vertical model and is offset by markedly larger producer benefits. The downstream-only model misses the substantial RRC effect experienced by the Santek-WMADS and Santek-Regional products.  In the vertical model, the downstream prices for these products rise more dramatically due to this RRC, and other prices increase alongside them.  This causes more harm to consumers.

In the upstream-only model, downstream prices and therefore shares are fixed at their pre-merger levels. Upstream prices are still set to marginal cost for integrated products.  Overall disposal prices increase by 11.2\% post-merger, and the merger effect is a pure transfer between firms. Thus, both consumer and net harm are estimated to be zero.  The main impact of the merger is in changing the bargaining leverage for Santek, now that it is combined with Republic.  This effect causes Santek to raise its upstream disposal prices to both WMADS and Regional.  The price increase is even larger in the upstream-only model compared to the full model. The lack of downstream price adjustments means that WMADS and Regional cannot change prices in order to shift their sales towards their cheaper supplier, Waste Connections, in equilibrium, which makes having access to Santek's disposal facilities relatively more valuable than in the full model, driving up Santek's price.  In this sense, the upstream-only model overstates the increase in leverage that Santek receives from the merger.

Our results suggest that a divestiture in the collection market or in the disposal market alone likely would not have sufficiently remedied the anti-competitive effects of the merger.  The DOJ ultimately required divestitures at both levels of the vertical chain.

%%%%%%%%%%%%%%%%%%%%%%%%%%%%%%%%%%%%%%%%%%
\subsection{Comparison to Alternative Mergers}
The analysis above shows some useful aspects of the model for analyzing the complex merger between Republic and Santek.  However, there are many other forms that a complex merger can take.  In order to explore some of these other scenarios, we simulate counterfactual mergers between firms in the market other than Republic with Santek, specifically Santek/Waste Connections, Santek/WMADS, and WMADS/Regional.  Santek/Waste Connections is an example of two fully integrated firms merging, and, unlike Republic, both firms sell their excess capacity to other downstream haulers.  Santek/WMADS is a merger involving an integrated firm buying a pure downstream competitor.  Filling out the set, WMADS/Regional is a horizontal merger between two firms that only operate downstream.  The estimated aggregate welfare effects of these counterfactual mergers are illustrated alongside those of the Republic/Santek merger in Figure \ref{fig:TrashFakeOverall}.  The results for prices and market shares appear in Figure \ref{fig:TrashFakeFirms}.

We begin with the merger of Santek and Waste Connections, displayed in the second row of Figure \ref{fig:TrashFakeOverall} and the top panel of Figure \ref{fig:TrashFakeFirms}. The post-merger change in the HHI for the disposal market is 1,896 and for the collection market is 620, respectively.  We see in Figure \ref{fig:TrashFakeOverall} that compared to Republic/Santek, the Santek/Waste Connections merger would have generated lower consumer harm and lower producer benefits.  The harm result is interesting, because both Santek and Waste Connections exclusively self-supply their disposal input pre-merger. This eliminates the possibility for positive EDM effects. Furthermore, we see from Figure \ref{fig:TrashFakeFirms} that Santek and Waste Connections significantly raise disposal prices to their downstream rivals, Republic, WMADS, and Regional.  However, the Republic-Republic product has a large downstream market share and does not face an input price increase, which limits the extent of price increases in the collection market. Thus, net consumer harm to the Chattanooga market is \$6.6 million, lower than that estimated for Republic/Santek.

Turning to the Santek/WMADS simulation, this would have been a merger of a small downstream competitor combining with a vertically integrated firm.  The merger results in a post-merger change in the collection market HHI of 620.  Since Santek supplies WMADS with disposal, there is the possibility for EDM.  As can be seen in Figure \ref{fig:TrashFakeOverall}, the welfare impacts are more muted relative to both the Republic/Santek and Santek/Waste Connections mergers.  This is driven by the small size of the WMADS products relative to more popular options like Republic-Republic and Waste Connections-Waste Connections.  Turning to Figure \ref{fig:TrashFakeFirms}, we see that EDM greatly lowers the disposal price between Santek and WMADS, while RRC slightly raises the disposal prices that Santek charges Republic and Regional.  This nets out into the small shift in profits from upstream to downstream firms seen in Figure \ref{fig:TrashFakeOverall}.  Prices in downstream collection rise slightly, which gives a relatively small loss in consumer welfare.

As for the WMADS/Regional simulation, this would have been the combination of two downstream horizontal competitors, neither of which was vertically integrated.  This merger would have resulted in a post-merger change in the HHI for the collection market of 136, which is much lower than any of the other combinations we have considered.  Unsurprisingly, the resulting estimated welfare effects in Figure \ref{fig:TrashFakeOverall} are small, with consumer harm and producer benefits approximately canceling each other out. The more detailed results reported in Figure \ref{fig:TrashFakeFirms} show only minimal changes in prices or market shares post-merger.  These results are consistent with the findings for downstream horizontal mergers seen in \citeN{ST2021}.

Another way to explore the implications of the model for complex mergers is to construct certain counterfactual firms as merger candidates.  Specifically, we take the upstream and downstream divisions of Republic, Santek, and Waste Connections, and unless otherwise indicated, assume they are independent firms.\footnote{We set their bargaining power parameters equal to $0.73$, which is what we calibrated for Waste Connections in the baseline model.  }  This setup allows us to study additional complex mergers involving the most popular products, those sold by Republic and Santek.  Specifically, we examine the following mergers: (1) a pure vertical merger between Santek's upstream disposal and Republic's downstream collection, labeled ``Vertical'' in Figure \ref{fig:TrashFakeRepSan}; (2) a pure upstream merger between Santek's disposal and Republic's disposal, labeled ``Up''; (3) a pure downstream merger between Santek's collection and Republic's collection, labeled ``Down''; (4) a partial upstream merger between a vertically integrated Santek and Republic's disposal, labeled ``Partial Up''; and (5) a partial downstream merger between a vertically integrated Santek and Republic's collection, labeled ``Partial Down.''  The last two are complex mergers, while the first three are similar to the non-complex cases studied in \citeN{ST2021}.  Furthermore, treating Waste Connections as separate upstream and downstream firms makes the overall environment more comparable to \citeN{ST2021}.  In any of the instances where only one part of Republic or Santek are merging, we assume that the other part is operating independently.

 The welfare results for these simulations are in Figure \ref{fig:TrashFakeRepSan}.  The baseline simulation for the Republic/Santek merger as actually proposed is provided at the top for comparison.  The last three rows echo findings in \citeN{ST2021} for non-complex mergers, which showed that our model tends to find consumer harm for most horizontal mergers, whereas vertical mergers have more scope for net consumer benefits.  Starting with the pure vertical case at the bottom of the figure, we see that consumers actually benefit from this merger.  This result occurs because the EDM for Santek selling disposal to Republic is so large, with the disposal price falling by approximately 63\%.  Recall from Figure \ref{fig:TrashSimsFirms}, this was also seen in the baseline simulation.  However, here we find that the RRC effects are much smaller, with Santek raising its disposal prices to WMADS and Regional by about 16\% versus 88\% in the baseline.  This difference occurs because in the pure vertical case, the highly popular Santek-Santek and Republic-Republic products are not fully owned by the merging firms, which means that some of the benefits from customers induced to divert from withholding disposal would accrue to the independent Santek collection and Republic disposal firms.  As a result, the EDM effect dominates and most downstream prices fall, which results in lower collection profits and net consumer benefits.

 Turning to the pure upstream merger in the fifth row of Figure \ref{fig:TrashFakeRepSan}, the combination of the disposal assets of Santek and Republic harms consumers due to a loss in upstream competition.  Both firms raise all of their disposal prices by over 30\%, which harms collection firms who in turn partially pass on these higher costs to final customers.  It is interesting to compare this case to a partial upstream merger between an integrated Santek and Republic's disposal division, which appears in the second row of Figure \ref{fig:TrashFakeRepSan}.  Compared to the pure upstream scenario, the partial upstream merger adds an RRC incentive.\footnote{There is no new EDM incentive, because Republic does not provide disposal to Santek.  }  Now when Republic disposal negotiates with Republic collection, some of the customers that could switch away from the Republic-Republic product could choose a Santek collection option, including the Santek-Santek integrated product that benefits from EDM.  As a result, the price of Republic disposal increases by about 70\%.  Furthermore, the equilibrium prices for Santek disposal sold to rivals are significantly higher pre-merger than in the pure upstream merger case, and they still rise by nearly 30\% after the merger.  This results in high post-merger downstream prices, shifting some of the increased profits from upstream to downstream firms and increasing harm to consumers.

The results for the pure downstream merger between the collection divisions of Republic and Santek appear in the fourth row of Figure \ref{fig:TrashFakeRepSan}.  In this case, consumers are harmed due to the loss of downstream competition, which increases prices in the collection market. Interestingly, we find that upstream disposal prices charged to the merged Republic and Santek fall slightly, due to increased bargaining leverage against input suppliers, which manifests itself in the decrease in upstream profits seen in the third column of Figure \ref{fig:TrashFakeRepSan}.  This countervailing effect is also documented in \citeN{ST2021}.  As is also seen in that previous paper, here we find that the lower input prices do not translate into net benefits for consumers.

When we compare the pure downstream case to the partial downstream merger between an integrated Santek and Republic's collection assets (third row of Figure \ref{fig:TrashFakeRepSan}), we see that consumer harm increases.  This occurs even though the move to the partial downstream merger introduces a large EDM effect for Santek's sales of disposal to Republic.  That input price falls by over 70\%.  The merged firm continues to negotiate a lower input price from Republic's disposal division, due to increased bargaining leverage.  However, Santek also raises prices on disposal to WMADS and Regional due to RRC, which, combined with the loss in downstream competition, increases the harm to consumers.

The preceding analysis demonstrates how various horizontal and vertical effects of mergers interact in the model.  We find that in our empirical application, EDM only results in net benefits for consumers in the pure vertical case, whereas all complex mergers result in consumer harm.  Indeed, we see with the downstream and partial downstream Republic/Santek examples that moving from a pure horizontal merger to a complex merger that adds EDM effects can actually result in more harm to consumers.  In the complex mergers we study, we find that EDM benefits are outweighed by RRC effects and the loss of horizontal competition.

%%%%%%%%%%%%%%%%%%%%%%%%%%%%%%%%%%%%%%%%%%%%%%%%%%%%%%%%%%%%%%%%%%%%%%%%%%%%%%%%%
\section{Conclusion \label{sec:concl}}

This paper relaxes an assumption made in much of the merger simulation literature: that mergers occur in vertical supply chains where in the pre-merger state, no firms are vertically integrated. Removing this limitation is important as competition authorities are routinely called upon to investigate complex mergers where one or both of the merging parties are vertically integrated.

Our results suggest that mergers that have horizontal aspects can remain harmful for consumers in a number of instances, even when these mergers also have vertical aspects. In our examples, the loss in horizontal competition between Republic and Santek and RRC behavior by Santek outweighs even large decreases in input prices from EDM.  This suggests that antitrust agencies have reason to be cautious when reviewing mergers between large firms that tout the possibility of countervailing EDM benefits.

However, we of course have the caveat that our results are specific to our modeling framework and calibrated parameters.  Certain alternative structures are likely to give different welfare results.  Indeed, \citeN{ST2021} find that using a second score auction to model downstream competition, as opposed to assuming Bertrand pricing, could meaningfully impact the estimated consumer harm from mergers, although results were similar between these models for vertical mergers.  This paper is an attempt to provide some information about possible welfare effects from complex mergers within a framework that is used by practitioners.  Exploring other commonly used frameworks would be a natural extension in future research.


%%%%%%%%%%%%%%%%%%%%%%%%%%%%%%%%%%%%%%%%%%%%%%%%%%%%%%%%%%%%%%%%%%%%%%%%%%%%%%%%%
% Bibliography
%%%%%%%%%%%%%%%%%%%%%%%%%%%%%%%%%%%%%%%%%%%%%%%%%%%%%%%%%%%%%%%%%%%%%%%%%%%%%%%%%
\newpage
\bibliographystyle{chicago}
%\bibliographystyle{elsarticle-harv}
\bibliography{master-bib}
\clearpage

%%%%%%%%%%%%%%%%%%%%%%%%%%%%%%%%%%%%%%%%%%%%%%%%%%%%%%%%%%%%%%%%%%%%%%%%%%%%%%%%%
% Tables and Figures
%%%%%%%%%%%%%%%%%%%%%%%%%%%%%%%%%%%%%%%%%%%%%%%%%%%%%%%%%%%%%%%%%%%%%%%%%%%%%%%%%
\begin{table}
\caption{Republic/Santek Merger Simulation Inputs}
\small{%\caption{\label{tab:trashdata}Republic/Santek Merger Simulation Inputs. Volume is reported in thousands of pounds, while prices and margins are reported in dollars.}
%\centering
\resizebox{\linewidth}{!}{
\begin{tabular}[t]{rrccccc}
\toprule
Disposal Firm & Collection Firm & Volume & Disposal Price & Disposal Margin & Collection Margin & Collection Cost\\
\midrule
Republic & Republic & 165 & 42 & 0 & 44 & 93\\
\cmidrule{1-7}
 & Republic & 34 & 36 & 20 &  & 93\\

 & Santek & 218 & 16 & 0 &  & 74\\

 & WMADS & 30 & 36 & 20 &  & 148\\

\multirow{-4}{*}{\raggedleft\arraybackslash Santek} & Regional & 30 & 36 & 20 &  & 148\\
\cmidrule{1-7}
 & Republic & 19 & 25 & 14 &  & 93\\

 & WasteConn & 48 & 11 & 0 &  & 63\\

 & WMADS & 17 & 25 & 14 &  & 148\\

\multirow{-4}{*}{\raggedleft\arraybackslash WasteConn} & Regional & 17 & 25 & 14 &  & 148\\
\bottomrule
\end{tabular}}
}
\footnotesize{Notes: ``WasteConn'' stands for Waste Connections.  Volume is reported in thousands of tons, whereas prices, costs, and margins are reported in dollars per ton.}
\label{tab:trashdata}
\end{table}

\begin{figure}
\centering
\caption{Structure of the Chattanooga Area Disposal and Collection Markets}
\includegraphics[scale=1]{output/chattanooga_tree.png}
\label{fig:chatt_tree}
\end{figure}

\begin{figure}
\centering
\caption{Republic/Santek Merger Simulation Results}
\includegraphics[scale=1.0]{output/TrashSimsFirm.png}\\
\footnotesize{Notes: ``WasteConn'' stands for Waste Connections.}
\label{fig:TrashSimsFirms}
\end{figure}

\begin{figure}
\centering
\caption{Sensitivity of Consumer Harm from Republic/Santek to the Outside Share}
\includegraphics[scale=.9]{output/trashelast.png}\\
\footnotesize{Notes: The absolute value of the market elasticity corresponding to the outside good share is in parentheses. Baseline results assume 0 outside share.  }
\label{fig:TrashElast}
\end{figure}

\begin{figure}
\centering
\caption{Sensitivity of Consumer Harm from Republic/Santek to the Nesting Parameter}
\includegraphics[scale=.9]{output/trashnest.png}\\
\footnotesize{Notes: Simulations use nested logit demand.  ``Integrated'' assumes that all vertically integrated products are in the same nest, while ``Landfill'' assumes that all collection firms that use the same landfill are in the same nest.  Baseline results assume a nesting parameter of 0. }
\label{fig:TrashNest}
\end{figure}

\begin{figure}
\centering
\caption{Sensitivity of Consumer Harm from Republic/Santek to Bargaining Power}
\includegraphics[scale=.9]{output/trashbarg.png}\\
\footnotesize{Notes: The bargaining power parameter is set to be the same across all non-integrated retailer-wholesaler pairs.  The dashed horizontal line denotes the baseline value.}
\label{fig:TrashBarg}
\end{figure}

\begin{figure}
\centering
\caption{Republic/Santek Merger Simulations: Full vs. Partial Models}
\includegraphics[scale=1.0]{output/TrashSimsCompareAll.png}\\
\footnotesize{Notes: ``WasteConn'' stands for Waste Connections.}
\label{fig:TrashSimsCompareAll}
\end{figure}

\begin{figure}
\centering
\caption{Simulated Welfare Effects for Other Mergers}
\includegraphics[scale=1]{output/trashinterestingmergerbar.png}\\
\footnotesize{Notes: ``WasteConn'' stands for Waste Connections. }
\label{fig:TrashFakeOverall}
\end{figure}

\begin{figure}
\centering
\caption{Price and Market Share Simulation Results for Other Mergers }
\includegraphics[scale=.9]{output/trashinterestingfirmbar.png}\\
\footnotesize{Notes: ``WasteConn'' stands for Waste Connections.  The three mergers studied are Santek/Waste Connections (top panel), Santek/WMADS (middle panel), and WMADS/Regional (bottom panel).  }
\label{fig:TrashFakeFirms}
\end{figure}

\begin{figure}
\centering
\caption{Simulation Results for Republic/Santek under Different Firm Configurations  }
\includegraphics[scale=.9]{output/trashinterestingsantrep.png}\\
\footnotesize{Notes: ``Base'' is the baseline merger of Republic/Santek as actually proposed.  ``Partial Up'' is the merger between an integrated Santek and the upstream Republic division.  ``Partial Down'' is the merger between an integrated Santek and the downstream Republic division.  ``Down'' is the merger between the downstream divisions of both Santek and Republic.  ``Up'' is the merger between the upstream divisions of both Santek and Republic.  ``Vertical'' is the merger between the upstream division of Santek and the downstream division of Republic.}
\label{fig:TrashFakeRepSan}
\end{figure}


%%%%%%%%%%%%%%%%%%%%%%%%%%%%%%%%%%%%%%%%%%%%%%%%%%%%%%%%%%%%%%%%%%%%%%%%%%%%%%%%%
% Appendix
%%%%%%%%%%%%%%%%%%%%%%%%%%%%%%%%%%%%%%%%%%%%%%%%%%%%%%%%%%%%%%%%%%%%%%%%%%%%%%%%%
\appendix
\newpage
\clearpage

\numberwithin{equation}{section}
\numberwithin{figure}{section}
\numberwithin{table}{section}


%%%%%%%%%%%%%%%%%%%%%%%%%%%%%%%%%%%%%%%%%%%%%%%%%%%%%%%%%%%%%%%%%%%%%%%%%%%%%%%%%
\newpage
\section{Appendix: Data for Republic/Santek\label{app:trashdata}}

Data on MSW disposal come from information collected by the Tennessee Department of Environmental Quality (TDEQ) consisting of Class 1 landfill ownership, Class 1 Solid Waste Origin Reports, and waste disposal by county for 2019. These data identify the origin county and destination landfill, including owner and operator information, for all MSW produced in Tennessee as well as waste volumes passing through transfer stations in Tennessee. The MSW disposal market definition follows that outlined in the DOJ CIS and attributes share to the company owning the final disposal landfill (i.e., ignoring transfer stations that are an intermediate disposal site only). Thus, the total market quantity is defined as all MSW volumes originating in Hamilton County, Tennessee where Chattanooga is located.

Disposal volumes are measured in tons and have been combined across landfills owned by the same firm. The TDEQ data identify landfills owned or operated by Republic, Santek, Waste Connections, and three other market participants (each with less than 0.5\% share) receiving MSW volumes originating in Hamilton County.\footnote{The City of Chattanooga and Marion County landfills are municipally owned and operated, accounting for less than 1\% share combined. Global Envirotech is a privately-owned transfer station that sends its 0.03\% share to an out-of-state landfill in Georgia.} These three fringe participants have been excluded from the analysis. After re-scaling, the resulting market shares are Republic, 28.4\%, Santek, 54\%, and Waste Connections, 17.6\%.

However, the market shares required for implementation of our model are expressed as upstream-downstream market participant pairs. Following the discussion of supply relationships in Section \ref{sec:application}, vertically-integrated pairs without capacity constraints (i.e., Santek disposal and collection, Waste Connection disposal and collection) are assigned their full collection share. Capacity constrained integrated firms (Republic disposal and collection) are assigned their collection share up to their available capacity, with the remainder allocated by residual disposal share to pairs with the respective upstream firms (i.e., Santek disposal and Republic collection, Waste Connections disposal and Republic collection). Unintegrated firms' collection shares are  distributed among the upstream suppliers with available capacity by residual share as well.

MSW disposal prices are collected from the 2019 Waste Business Journal's Directory of Waste Processing \& Disposal Sites. The measure of price is the ``gate rate,'' which is the posted price at the landfill, measured in dollars per ton.\footnote{Disposal prices for large customers may be bilaterally negotiated instead of paying the gate rate.  We assume this behavior is not significant in Chattanooga.} Republic and Santek both operate multiple landfills in the market with different prices. The price used in the analysis for each is the volume-weighted average for their landfills.

The CIS states that in the Chattanooga Area the post-merger HHI for SCCW collection would be approximately 5,551 post-merger with an increase of 2,660 points and that the combined market share of the merging parties is 73\%. Taking these figures as given, we can recover the collection market shares under the assumptions that 1) the merging parties are of equal size and 2) the non-merging parties are of equal size. After re-scaling, this produces downstream market shares for Republic, 37.6\%, Santek, 37.6\%, Waste Connections, 8.3\%, WMADS, 8.3\%, and Regional, 8.3\%.

Collection and disposal margins are calculated for Republic from data on revenue by line of service and from components of the cost of operations in the company's 2019 annual report. These data are reported at the company level and are not specific to the Chattanooga Area, but revenues are reported by collection segment.  Collection costs in dollars per ton are estimated from Republic and Waste Management 2019 10k financial statements. These costs are reported at the company level across all segments. The estimated share of these costs attributable to the Chattanooga market, based on the number of markets in which the companies operate, is divided by the tons disposed for each company. Santek, Waste Connections, and Regional do not publicly produce comparable financial statements. Instead, we assume that the cost structure is the same for the integrated companies, taking into account their individual tons disposed, and that WMADS and Regional share the same cost structure due to their lack of internal disposal capacity.


\end{document}


%We use this model to explore the welfare implications of four types of mergers: (1) vertical mergers between an unintegrated retailer and an unintegrated wholesaler, (2) downstream ``horizontal'' mergers between an unintegrated retailer and an integrated retailer/wholesaler pair, (3) upstream ``horizontal'' mergers between an unintegrated wholesaler and an integrated retailer/wholesaler pair, and (4) integrated mergers between two previously integrated retailer/wholesaler pairs.  Relative to \citeN{ST2021}, the first category, vertical mergers, appears in the previous paper and here serves as a baseline for comparison.  The other three cases, which feature at least one merging firm that is already integrated, are new.

%Our model-based simulations show certain patterns across mergers.  For horizontal upstream and downstream mergers, we find that the consumer harm from UPP and RRC typically exceeds any gains from EDM.  The exceptions occur when upstream wholesalers have more bargaining power than downstream retailers, which is when we expect pre-merger input prices to be high and the gains from EDM to be larger.  Likewise, we find that integrated mergers are typically net harmful to consumers as well.  The exceptions happen only when upstream firms have significantly more bargaining power compared to downstream firms.  Thus, although these complex mergers have both horizontal and vertical aspects, they appear to be qualitatively similar to pure horizontal mergers in their overall consumer impacts.  We also examine how the presence of additional integrated rivals (separate from the merging firms) changes our results. We find that the existence of integrated rival firms lowers the spread of welfare outcomes for consumers and slightly shifts these outcomes towards less harm.


%Concerns about vertical competitive effects were raised in both the Republic-Santek and the Waste Management-ADS transactions. Both the Solid Waste Agency of Lake County, IL and the Solid Waste Agency of Northern Cook County submitted comments in opposition to the proposed asset divestiture from Waste Management-ADS, stating that a vertically integrated competitor was needed to maintain competition in their local market post-merger.\footnote{\tiny{See  Comments by SWALCO and SWANCC : U.S. and Plaintiff States v. Waste Management, Inc., and Advanced Disposal Services, Inc., \url{https://www.justice.gov/atr/case-document/file/1377646/download}.}}
%For example, in the DOJ complaint filed in the Republic-Santek case, horizontal anti-competitive effects were alleged for four SCCW collection markets and two MSW disposal markets. In addition, vertical anti-competitive effects were alleged to arise from the combination of their integrated assets in the Chattanooga area.\footnote{See U.S. and State of Alabama v. Republic Services, Inc. and Santek Waste Services, LLC, \url{https://www.justice.gov/atr/case-document/file/1382031/download}}
%Our application in progress further identifies local markets in which these acquisitions result in: 1) only horizontal combinations of assets, 2) only vertical combinations of assets, and 3) combinations of vertically integrated assets in the presence of existing integrated competitors to analyze the welfare effects from mergers with complex vertical arrangements. The next sections present preliminary results on the merger of vertically integrated assets in the presence of other integrated and unintegrated rivals in the context of Republic and Santek's merger in the Chattanooga area.
%The next sections present results of applying the model of Section \ref{sec:theory} to the merger of vertically integrated assets in the presence of other integrated and unintegrated rivals in the context of Republic and Santek's merger in the Chattanooga area.

%Using a series of numerical simulations and the model of \citeN{ST2021}, we find that most mergers involving one or more firms that are already vertically integrated harm consumers.  The exceptions occur when wholesalers have more bargaining power compared to retailers, which are instances where the benefits from EDM would tend to be larger.  We also show that having integrated non-merging rivals limits the range of welfare outcomes and tends to make mergers somewhat more beneficial to consumers, although these firms' presence is not a panacea for consumer harm. We apply our model to the Republic/Santek merger, where we find that ignoring existing vertical relationships gives an incomplete picture of the merger's welfare impacts.
