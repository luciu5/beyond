\documentclass[12pt]{article}
\usepackage{chicago}    % bibliography package
\usepackage{graphicx}   % insert PostScript figures
\usepackage{setspace}   % controls line spacing
\usepackage{amsmath,amsthm,amssymb,amstext} % controls equation entry and symbols
\usepackage{rotating}   % rotates graphics
\usepackage{soul}       % controls hyphenation
\usepackage{epsfig}     % helps with including graphics
\usepackage{pdflscape}  % helps with displaying rotated graphics in PDFs
\usepackage{lscape}     % helps with rotating pages
\usepackage{setspace}   % controls line spacing
\usepackage{caption}    % controls  captions
\usepackage{adjustbox}  % shrink tables
\usepackage[margin=1in]{geometry} % margins
\usepackage{hyperref}   % displays URLs
%packages for kable tables
 \usepackage{booktabs}
 \usepackage{longtable}
 \usepackage{array}
 \usepackage{multirow}
 \usepackage{wrapfig}
 \usepackage{float}
 \usepackage{colortbl}
 \usepackage{pdflscape}
 \usepackage{tabu}
 \usepackage{threeparttable}
 \usepackage{threeparttablex}
 \usepackage[normalem]{ulem}
 \usepackage{makecell}
 \usepackage{xcolor}

\interfootnotelinepenalty=10000

\parskip     2.0mm       % space between paragraphs

\alph{footnote}         % make title footnotes alpha-numeric

\captionsetup[figure]{labelsep=space,labelfont=bf} % remove colon from figure name

%%%%%%%%%%%%%%%%%%%%%%%%%%%%%%%%%%%%%%%%%%%%%%%%%%%%%%%%%%%%%%%%%%%%%%%%%%%
% Title Page
%%%%%%%%%%%%%%%%%%%%%%%%%%%%%%%%%%%%%%%%%%%%%%%%%%%%%%%%%%%%%%%%%%%%%%%%%%%
\title{Beyond ``Horizontal'' and ``Vertical'':\\ The Welfare Effects of Complex Integration\footnote{The analysis and conclusions set forth are those of the authors and do not indicate concurrence by other members of the Board research staff or by the Federal Reserve Board of Governors. Furthermore, the views expressed here should not be purported to reflect those of the Federal Trade Commission or the U.S. Department of Justice. This article has benefited from conversations with Alison Oldale, Ted Rosenbaum, and Nathan Wilson.}}

\newcommand*\samethanks[1][\value{footnote}]{\footnotemark[#1]}
\author{Margaret Loudermilk\footnote{U.S. Department of Justice, margaret.loudermilk@usdoj.gov} \\  U.S Department of Justice \and Gloria Sheu\footnote{Board of Governors of the Federal Reserve System, gloria.sheu@frb.gov. } \\ Federal Reserve Board \and  Charles Taragin\footnote{Federal Trade Commission, ctaragin@ftc.gov}  \\  Federal Trade Commission }

\date{\today}

\begin{document}

\pagenumbering{roman}       % Roman numerals from abstract to text
\maketitle                  % print title information
\thispagestyle{empty}       % no page number on THIS page

\begin{center}
Preliminary and Incomplete
\end{center}

\begin{abstract}
\noindent We use a standard vertical supply-chain model to study the welfare impacts of mergers that have both horizontal and vertical aspects.  The model features logit Bertrand competition downstream and Nash Bargaining upstream.  We numerically simulate four types of mergers: (1) vertical mergers between an unintegrated retailer and an unintegrated wholesaler, (2) mixed downstream horizontal/vertical mergers between an unintegrated retailer and an integrated retailer/wholesaler pair, (3) mixed upstream horizontal/vertical mergers between an unintegrated wholesaler and an integrated retailer/wholesaler pair, and (4) integrated mergers between two previously integrated retailer/wholesaler pairs. The breadth of options we include better captures the variety of merger structures observed in practice, compared to the typical ``horizontal vs. vertical'' dichotomy.   We further extend our analysis to accommodate preexisting vertical integration by third-party firms and linear marginal costs.
\end{abstract}

\bigbreak Keywords: bargaining models; merger simulation; vertical markets; vertical mergers

JEL classification: L13; L40; L41; L42

\newpage                    % start a new page
\pagenumbering{arabic}      % Arabic page numbers from now on
\doublespacing

%%%%%%%%%%%%%%%%%%%%%%%%%%%%%%%%%%%%%%%%%%%%%%%%%%%%%%%%%%%%%%%%%%%%%%%%%%%%%%%%%
% Body of Paper
%%%%%%%%%%%%%%%%%%%%%%%%%%%%%%%%%%%%%%%%%%%%%%%%%%%%%%%%%%%%%%%%%%%%%%%%%%%%%%%%%
\section{Introduction}
Given the many different ways a firm may be organized, it is rare that any particular merger can be neatly categorized as purely ``horizontal'' or ``vertical.''  Indeed, as the 2020 U.S. Department of Justice (DOJ) and Federal Trade Commission (FTC) Vertical Merger Guidelines (henceforth, the ``VMG'') state, mergers often have both horizontal and vertical aspects.  However, the theories and models used to analyze mergers often fail to address these complexities.  Horizontal mergers are cast in terms of whether they create significant upward pricing pressure (UPP), while the focus for vertical mergers is on assessing the net effects of the elimination of double marginalization (EDM) versus raising rivals' costs (RRC).  Yet, a given merger may combine all of these effects simultaneously, which raises the necessity of balancing their impacts in a unified framework.

In this paper, we use a merger simulation model to assess the welfare implications of these complex mergers.  Our model is drawn directly from \citeN{ST2021}, and features a Bertrand logit downstream setup alongside a Nash Bargaining wholesale negotiation upstream.  We use this model to numerically simulate four types of mergers: (1) vertical mergers between an unintegrated retailer and an unintegrated wholesaler, (2) mixed downstream horizontal/vertical mergers between an unintegrated retailer and an integrated retailer/wholesaler pair, (3) mixed upstream horizontal/vertical mergers between an unintegrated wholesaler and an integrated retailer/wholesaler pair, and (4) integrated mergers between two previously integrated retailer/wholesaler pairs.  In each of these simulations, we calculate the impacts on consumer welfare and firm profits.  Furthermore, we examine how the presence of additional integrated rivals in the market changes our results. Our baseline simulations assume that units are produced with constant marginal costs, but we later extend our analysis to linear upward-sloping marginal costs.

%We find.

We then employ this model to investigate the 2021 merger of two solid waste management firms, Republic and Santek, in one geographic market: Chattanooga, Tennessee. The solid waste management industry has seen a large number of mergers with both horizontal and vertical elements.  Broadly speaking, the supply chain of this sector involves haulers at the downstream level that bring trash to disposal facilities at the upstream level.  A variety of firms participate in these services, with some owning their own hauling and disposal assets, while others are active in only a segment of the market. We find that ignoring vertical integration substantially understates harm to consumers in Chattanooga.

%Our paper is related to the literature on mergers and vertical integration.

%The paper proceeds as follows.

%%%%%%%%%%%%%%%%%%%%%%%%%%%%%%%%%%%%%%%%%%%%%%%%%%%%%%%%%%%%%%%%%%%%%%%%%%%%%%%%%
\section{Theory\label{sec:theory}}
We begin by describing our basic framework, which uses a downstream Bertrand logit model embedded in an upstream Nash Bargaining setup, taken from \citeN{ST2021}.  This framework allows us to study a variety of merger configurations.

%%%%%%%%%%%%%%%%%%%%%%%%%%%%%%%%%%%%%%%%%%
\subsection{Downstream and Upstream Competition}
Assume there is a set of consumers who each choose to buy a product sold by a single retailer.  Retailers are indexed by $r$, while the wholesalers that supply these retailers are indexed by $w$.  Prior to any mergers taking place, each wholesaler offers only one product, although each retailer can purchase from multiple wholesalers.  We denote the set of all retailers by $\mathbb{R}=\{1, \dots, \left\vert{\mathbb{R}}\right\vert\}$, and the set of all wholesalers by $\mathbb{W}=\{1, \dots, \left\vert{\mathbb{W}}\right\vert\}$.  We divide the set $\mathbb{W}$ into overlapping subsets, each labeled $\mathbb{W}^r$, to indicate which wholesalers' products are carried by which retailers.  Similarly, we divide the set of retailers $\mathbb{R}$ into overlapping subsets, each labeled $\mathbb{R}^w$, to indicate the retailers that carry the product sold by each wholesaler.

The share of consumers that choose product $w$ sold by retailer $r$ has the logit form,
\begin{equation}
s_{rw} = \frac{\exp(\delta_{rw} - \alpha p_{rw})}{1 + \sum_{t \in \mathbb{R}} \sum_{x \in \mathbb{W}^t} \exp(\delta_{tx} - \alpha p_{tx})},
\label{eq: nb share}
\end{equation}
where $\delta_{rw}$ is a quality parameter and $\alpha$ captures sensitivity to price $p_{rw}$.  There is an outside good whose quality parameter and price have been normalized to zero.  The retailer's profit is given by $\pi^r = \sum_{w \in \mathbb{W}^r} [p_{rw} - p^W_{rw} - c^R_{rw}] s_{rw} M$, where $p^W_{rw}$ is the unit fee charged by wholesaler $w$ to retailer $r$, $c^R_{rw}$ captures any additional marginal costs borne by the retailer, and $M$ is the market size.  Downstream prices are set in Bertrand equilibrium, according to
\begin{equation}
\sum_{x \in \mathbb{W}^r} [p_{rx} - p^W_{rx} - c^R_{rx}] \frac{\partial s_{rx}}
{\partial p_{rw}} + s_{rw} = 0,
\label{eq: nb downstream foc}
\end{equation}
which is the first order condition for product $w$ sold by retailer $r$.

Wholesale prices are set via Nash Bargaining between retailers and wholesalers.  We assume that the negotiation for a given input price treats other input prices and all downstream prices as given.\footnote{As explained in \citeN{ST2021}, this assumption is equivalent to a situation where all negotiations and choices for downstream prices happen simultaneously.  }  Profits for wholesaler $w$ are given by $\pi^w = \sum_{r \in \mathbb{R}^w} [p^W_{rw} - c^W_{rw}] s_{rw} M$, where $c^W_{rw}$ is the wholesale marginal cost when dealing with retailer $r$.  The first order condition for the negotiation between retailer $r$ and wholesaler $w$ is
\begin{equation}
\begin{split}
&\overbrace{[p^W_{rw} - c^W_{rw}]s_{rw} - \sum_{t \in \mathbb{R}^w \setminus \{r\}} [p^W_{tw} - c^W_{tw}] \Delta s_{tw}(\mathbb{W}^r \setminus \{w\})}^{\text{wholesaler GFT}} = \\
&\frac{1-\lambda}{\lambda} \left(\underbrace{[p_{rw} - p^W_{rw} - c^R_{rw}]s_{rw} - \sum_{x \in \mathbb{W}^r \setminus \{w\}} [p_{rx} - p^W_{rx} - c^R_{rx}] \Delta s_{rx}(\mathbb{W}^r \setminus \{w\})}_{\text{retailer GFT}}\right),
\end{split}
\label{eq: upstream foc}
\end{equation}
where $\lambda \in [0,1]$ captures the bargaining power of the retailer relative to the wholesaler.  The $\Delta s_{tx}(\mathbb{W}^r \setminus \{w\}) \equiv s_{tx}(\mathbb{W}^r \setminus \{w\}) - s_{tx}$ is the difference in the share of good $x$ sold by retailer $t$ when good $w$ is not offered by retailer $r$ versus when good $w$ is offered by retailer $r$.  Thus, the wholesale price $p_{rw}$ is set such that the payoff to wholesaler $w$ when it sells to retailer $r$ less the payoff when it does not (that is, the gains from trade or ``GFT''), divided by the payoff to retailer $r$ when it buys from wholesaler $w$ less the payoff when it does not, equals the ratio of wholesaler to retailer bargaining power.

Together the series of downstream and upstream first order conditions determine market equilibrium.  This model can be solved and calibrated as described in \citeN{ST2021}.  In our baseline configuration, we assume that wholesale and retail marginal costs are constant.  We loosen this restriction to allow for linear increasing marginal cost as a robustness check.

%%%%%%%%%%%%%%%%%%%%%%%%%%%%%%%%%%%%%%%%%%
\subsection{Mergers}
Here we describe the manner in which mergers are modeled in this framework.  We begin by describing vertical mergers, as all cases we examine have some vertical aspects.  Suppose that retailer $r$ and wholesaler $w$ were to merge.  Then the first order condition for profit maximization for the joint firm when setting the downstream price for product $w$ is given by
\begin{equation}
\begin{split}
&\sum_{x \in \mathbb{W}^r \setminus \{w\}} [p_{rx} - p^W_{rx} - c^R_{rx}]\frac{\partial s_{rx}}{\partial p_{rw}} + s_{rw} + \overbrace{[p_{rw} - c^W_{rw} - c^R_{rw}]\frac{\partial s_{rw}}{\partial p_{rw}}}^{\text{EDM effect}} \\
&+ \underbrace{\sum_{t \in \mathbb{R}^w \setminus \{r\}} [p^W_{tw} - c^W_{tw}] \frac{\partial s_{tw}}{\partial p_{rw}}}_{\text{upstream UPP effect}} = 0.
\end{split}
\label{eq: vmerger own ret downstream foc}
\end{equation}
The pricing problem balances two effects.  On the one hand, the term labeled ``EDM effect'' captures the impact of retailer $r$ being able to access product $w$ at marginal cost.  This force would tend to lower the resulting price.  On the other hand, the term labeled ``upstream UPP effect'' captures the incentive to raise prices for retailer $r$ in order to divert sales to wholesaler $w$.  This force would tend to raise the resulting consumer price.

Turning to input prices, when wholesaler $w$ bargains with unaffiliated retailer $s$, the first order condition becomes
\begin{equation}
\begin{split}
&[p^W_{sw} - c^W_{sw}]s_{sw} - \sum_{t \in \mathbb{R}^w \setminus \{r,s\}} [p^W_{tw} - c^W_{tw}] \Delta s_{tw}(\mathbb{W}^s \setminus \{w\})\\
& - \overbrace{\overbrace{[p_{rw}-c^W_{rw}-c^R_{rw}] \Delta s_{rw}(\mathbb{W}^s \setminus \{w\})}^{\text{indirect EDM effect}} - \sum_{x \in \mathbb{W}^r \setminus \{w\}} [p_{rx} - p^W_{rx} - c^R_{rx}] \Delta s_{rx}(\mathbb{W}^s \setminus \{w\})}^{\text{RRC effect}}=\\
&\frac{1-\lambda}{\lambda} \left([p_{sw} - p^W_{sw} - c^R_{sw}]s_{sw} - \sum_{x \in \mathbb{W}^s \setminus \{w\}} [p_{sx} - p^W_{sx} - c^R_{sx}] \Delta s_{sx}(\mathbb{W}^s \setminus \{w\})\right).
\end{split}
\label{eq: vmerger wh upstream foc}
\end{equation}
which reflects the change in the disagreement payoff coming from the merger with retailer $r$.  Now when the wholesaler considers the possible loss of sales upon ceasing to trade with retailer $s$, these losses are softened due to a potential for diversion to retailer $r$, which we label the ``RRC effect.''  Furthermore, the margin on product $w$ sold by retailer $r$ is potentially higher due to EDM, as shown through the expression labeled ``indirect EDM effect,'' which can further compensate the firm.  These impacts tend to raise the resulting input price.

When the merged firm is bargaining with the unaffiliated wholesaler $v$ over what input price to pay as a retailer, the bargaining first order condition becomes
\begin{equation}
\begin{split}
&[p^W_{rv} - c^W_{rv}]s_{rv} - \sum_{t \in \mathbb{R}^v \setminus \{r\}} [p^W_{tv} - c^W_{tv}] \Delta s_{tv}(\mathbb{W}^r \setminus \{v\}) = \\
&\frac{1-\lambda}{\lambda} \left([p_{rv} - p^W_{rv} - c^R_{rv}]s_{rv} - \sum_{x \in \mathbb{W}^r \setminus \{w,v\}} [p_{rx} - p^W_{rx} - c^R_{rx}] \Delta s_{rx}(\mathbb{W}^r \setminus \{v\})\right.\\
&\left.- \underbrace{[p_{rw} - c^W_{rw} - c^R_{rw}] \Delta s_{rw}(\mathbb{W}^r \setminus \{v\})}_{\text{EDM recapture effect}}-\underbrace{\sum_{t \in \mathbb{R}^w \setminus \{r\}} [p^W_{tw} - c^W_{tw}] \Delta s_{tw}(\mathbb{W}^r \setminus \{v\})}_{\text{wholesale recapture leverage effect}}\right).
\end{split}
\label{eq: vmerger ret upstream foc}
\end{equation}
In this case, the merged firm has two channels for potential additional profits should it cease to trade with wholesaler $v$.  First, if retail sales are diverted to product $w$ sold by retailer $r$, those sales will earn a higher margin due to lower marginal costs stemming from what we call the ``EDM recapture effect.''  Second, the loss of product $v$ carried by retailer $r$ could increase sales by wholesaler $w$ through other retailers, which we call the ``wholesale recapture leverage effect.''  Both of these effects would tend to lower the resulting input price.

Next consider a merger between an integrated retailer/wholesaler $rw$ and a standalone retailer $s$.  Such a combination has a vertical component and a horizontal component.  The first order condition for setting the downstream price $p_{rw}$ becomes
\begin{equation}
\begin{split}
&\sum_{x \in \mathbb{W}^r \setminus \{w\}} [p_{rx} - p^W_{rx} - c^R_{rx}]\frac{\partial s_{rx}}{\partial p_{rw}} + s_{rw} + \overbrace{[p_{rw} - c^W_{rw} - c^R_{rw}]\frac{\partial s_{rw}}{\partial p_{rw}}}^{\text{direct EDM effect}}+ \underbrace{\sum_{t \in \mathbb{R}^w \setminus \{r\}} [p^W_{tw} - c^W_{tw}] \frac{\partial s_{tw}}{\partial p_{rw}}}_{\text{upstream UPP effect}} \\ &+\underbrace{\underbrace{[p_{sw} - c^W_{sw} - c^R_{sw}]\frac{\partial s_{sw}}{\partial p_{rw}}}_{\text{indirect EDM Effect}} + \sum_{x \in \mathbb{W}^s \setminus \{w\}} [p_{sx} - p^W_{sx} - c^R_{sx}] \frac{\partial s_{sx}} {\partial p_{rw}}}_{\text{downstream UPP effect}}= 0.
\end{split}
\label{eq: intdmerger own ret downstream foc}
\end{equation}
Now there is the possibility for what we call the ``downstream UPP effect'' in the retail market, as the merged firm can recapture sales that are diverted to retailer $s$ when $r$ raises its prices.  EDM between $w$ and $s$ can actually increase this UPP impact, because when sales are diverted to product $w$ sold by retailer $s$, those units earn a larger margin.  This impact is what we label the ``indirect EDM effect,'' which comes through the interaction of EDM with UPP.

If instead an integrated retailer/wholesaler $rw$ were to merge with a standalone wholesaler $v$, then the resulting first order condition for product $rw$ would look similar to equation \eqref{eq: intdmerger own ret downstream foc}.  However,  the downstream UPP effect would be replaced with an additional upstream UPP effect capturing the value of sales diverted to customers of wholesaler $v$.  The condition would also include an indirect EDM component, reflecting the ability of retailer $r$ to obtain product $v$ at marginal cost, which in turn would raise the value of diverted sales to that product.  Whether these additional incentives to raise prices will dominate the direct EDM impact is an empirical question.  When considering a merger between two integrated retailer/wholesaler pairs, both upstream and downstream UPP effects would enter.

Turning to wholesale prices, again consider a merger between a preexisting integrated retailer/wholesaler $rw$ and a standalone retailer $s$.  When wholesaler $w$ bargains with an unaffiliated retailer $u$ we have the first order condition given by,
\begin{equation}
\begin{split}
&[p^W_{uw} - c^W_{uw}]s_{uw} - \sum_{t \in \mathbb{R}^w \setminus \{r,s,u\}} [p^W_{tw} - c^W_{tw}] \Delta s_{tw}(\mathbb{W}^u \setminus \{w\})\\
& - \overbrace{\sum_{t \in \{r,s\}}\left(\overbrace{[p_{tw}-c^W_{tw}-c^R_{tw}] \Delta s_{tw}(\mathbb{W}^u \setminus \{w\})}^{\text{indirect EDM effect}} - \sum_{x \in \mathbb{W}^t \setminus \{w\}} [p_{tx} - p^W_{tx} - c^R_{tx}] \Delta s_{tx}(\mathbb{W}^u \setminus \{w\})\right)}^{\text{RRC effect}}=\\
&\frac{1-\lambda}{\lambda} \left([p_{uw} - p^W_{uw} - c^R_{uw}]s_{uw} - \sum_{x \in \mathbb{W}^u \setminus \{w\}} [p_{ux} - p^W_{ux} - c^R_{ux}] \Delta s_{ux}(\mathbb{W}^u \setminus \{w\})\right).
\end{split}
\label{eq: intdmerger wh upstream foc}
\end{equation}
The RRC effect is augmented with the profits emanating from the sales of retailer $s$, in addition to the sales of retailer $r$, both of which may potentially recapture sales should retailer $u$ lose access to product $w$.  The merged firm has higher bargaining leverage as a result.  If instead the integrated retailer/wholesaler $rw$ were to merge with a standalone wholesaler $v$, then the profits earned by firm $rw$ should the negotiation with retailer $u$ fail are augmented with the earnings of wholesaler $v$ rather than of retailer $s$.  This adds a term similar to the wholesale recapture leverage effect seen in equation \eqref{eq: vmerger ret upstream foc} to the left-hand side of the bargaining first order condition.\footnote{For simplicity, we assume that when a retailer fails to reach an agreement with wholesaler $w$, that retailer's contract with wholesaler $v$ remains in place.  This assumption can be loosened.  }    In the case of a merger between two integrated retailer/wholesaler firms, all of these effects would appear.

Returning to a merger between an integrated retailer/wholesaler $rw$ and a standalone retailer $s$, when the merged firm bargains with an unaffiliated wholesaler $v$ to supply retailer $r$, the first order condition becomes
\begin{equation}
\begin{split}
&[p^W_{rv} - c^W_{rv}]s_{rv} - \sum_{t \in \mathbb{R}^v \setminus \{r\}} [p^W_{tv} - c^W_{tv}] \Delta s_{tv}(\mathbb{W}^r \setminus \{v\}) = \\
&\frac{1-\lambda}{\lambda} \left([p_{rv} - p^W_{rv} - c^R_{rv}]s_{rv} - \sum_{x \in \mathbb{W}^r \setminus \{w,v\}} [p_{rx} - p^W_{rx} - c^R_{rx}] \Delta s_{rx}(\mathbb{W}^r \setminus \{v\})\right.\\
&- \underbrace{\sum_{t \in \{r,s\}}[p_{tw} - c^W_{tw} - c^R_{tw}] \Delta s_{tw}(\mathbb{W}^r \setminus \{v\})}_{\text{EDM recapture effect}}
- \underbrace{\sum_{x \in \mathbb{W}^s \setminus \{w\}} [p_{sx} - p^W_{sx} - c^R_{sx}] \Delta s_{sx}(\mathbb{W}^r \setminus \{v\})}_{\text{retail recapture leverage effect}}\\
&-\underbrace{\left.\sum_{t \in \mathbb{R}^w \setminus \{r,s\}} [p^W_{tw} - c^W_{tw}] \Delta s_{tw}(\mathbb{W}^r \setminus \{v\})\right)}_{\text{wholesale recapture leverage effect}}.
\end{split}
\label{eq: intdmerger ret upstream foc}
\end{equation}
Compared to equation \eqref{eq: vmerger ret upstream foc}, now the EDM recapture effect applies to sales of product $w$ at both affiliated retailers $r$ and $s$.  Furthermore, there is also the possibility that sales will be diverted to retailer $s$ should retailer $r$ lose access to product $v$, which creates an additional ``retail recapture leverage effect'' alongside the wholesale recapture leverage effect.  All of these additional terms tend to increase the bargaining leverage of the merged firm.  If we instead examined a merger between retailer/wholesaler $rw$ and an unaffiliated wholesaler, that would augment the EDM recapture effect with sales of retailer $r$ for two merged wholesalers, rather than sales through two merged retailers.  Similarly, the retail recapture leverage effect would be replaced with an additional wholesale recapture leverage effect for the newly merged wholesaler.  For a merger between two integrated retailer/wholesaler firms, all of these effects would enter.

%Integrated rivals.

%%%%%%%%%%%%%%%%%%%%%%%%%%%%%%%%%%%%%%%%%%%%%%%%%%%%%%%%%%%%%%%%%%%%%%%%%%%%%%%%%
\section{Numerical Simulations}
In this section, we use a similar setup to that in \citeN{ST2021}, but allow for additional types of mergers beyond purely horizontal and purely vertical combinations and allow for the presence of integrated rival firms.

%%%%%%%%%%%%%%%%%%%%%%%%%%%%%%%%%%%%%%%%%%
\subsection{Data Generating Process}
Broadly speaking, we consider four categories of mergers: downstream, upstream, vertical, and integrated.  We define a downstream merger as a merger between two unintegrated retailers or between an unintegrated retailer and a vertically integrated wholesaler/retailer combination.  Similarly, upstream mergers are those between two unintegrated wholesalers or between an unintegrated wholesaler and a vertically integrated wholesaler/retailer.  We define vertical mergers as those between an unintegrated wholesaler and unintegrated retailer.  Finally, we assume that an integrated merger is between two firms that are both already vertically integrated pre-merger.

For downstream, upstream, and vertical mergers, we simulate markets by randomly sampling shares from a Dirichlet distribution for 2 to 5 retailers or wholesalers, respectively.\footnote{We parameterize the Dirichlet distribution so it is equivalent to a uniform distribution.  }  Because integrated mergers must have two vertically integrated incumbents in the pre-merger state, for those simulations we increase the maximum allowable number of wholesalers and retailers in our simulated markets to 7. We also assume that in the pre-merger state, there are anywhere from 0 to 4 vertically integrated incumbents (6 for integrated mergers). Vertically integrated firms are not siloed: integrated wholesalers supply inputs to retailers other than their integrated partner and vertically integrated retailers purchase inputs from wholesalers other than their integrated partner.

Our simulations focus on mergers that are more likely to have competitive effects and to therefore come under agency scrutiny. For horizontal merger simulations, we assign the products sold by the two largest firms in the market to a single firm post-merger. Similarly, when simulating a vertical merger, we assign the products sold by the largest wholesaler and the largest retailer to a single firm post-merger.

The bargaining parameter ranges from from 0.1 (where wholesalers have the advantage) to 0.9 (where retailers have the advantage). We report our results in terms of relative bargaining power, $(1-\lambda)/\lambda$, which ranges from $9$ (wholesaler power is nine times greater than retailer power) to $1/9$ (retailer power is nine times greater than wholesaler power).  The bargaining parameter is identical for all of the retailers in each simulation.  We calibrate the price coefficient $\alpha$ assuming that in the pre-merger world, there is a vertically integrated outside option.  All other goods are differenced relative to this option, resulting in the mean zero outside good normalization.  We set the market size to 1.

For each combination of number of retailers, number of wholesalers, number of incumbent integrated firms, and bargaining parameter, we draw 1,000 different sets of market primitives.  After eliminating situations where the merger is unprofitable to the merging firms, as well as markets that do not pass the Hypothetical Monopolist Test, we have 1.6 million markets remaining.\footnote{The Hypothetical Monopolist Test is a market definition exercise that checks whether a monopolist that owns all products in a candidate market would raise the price of at least one of the merging producers' products by at least a ``small but significant non-transitory increase in price'' (SSNIP), which we take to be 5\%. } Each market treats as primitives the number of retailers, the number of wholesalers, the bargaining parameter, and the wholesaler and retailer marginal costs.


%%%%%%%%%%%%%%%%%%%%%%%%%%%%%%%%%%%%%%%%%%
\subsection{Overview of Simulated Output}

Table \ref{tab:simsum} provides summary statistics for our simulations.\footnote{The \href{https://CRAN.R-project.org/package=antitrust}{\texttt{antitrust} R package} contains the computer code needed to run the merger simulations described here.}  For reference, the 2010 DOJ/FTC Horizontal Merger Guidelines categorize ``Highly Concentrated Markets'' as those with HHIs over 2,500 points.  The Guidelines state that mergers with HHI changes greater than 200 points that result in Highly Concentrated markets are ``presumed likely to enhance market power.''\footnote{See \S 5.3 of the Guidelines.  }

The median average wholesale pre-merger price is \$5.80, and the median average retail pre-merger price is \$15.  Recall that the market size is set to 1, meaning that average price is equal to total pre-merger expenditures.  Pre-merger HHIs range from 2,911 at the $25^{th}$ percentile to 4,981 at the $75^{th}$, with a median of 3,747. HHIs for vertical mergers increase by 1,328 points at the median, resulting in a median post-merger HHI equal to 6,048.\footnote{We compute the post-merger HHI for vertical mergers by treating the merged firms' market share as the sum of all the shares of downstream products that either use the upstream partner's input or are sold by the downstream partner.  }  HHIs for downstream mergers increase by 2,128 points at the median, resulting in a median post-merger HHI equal to 5,153.  HHIs for upstream mergers increase by 2,070 points at the median, resulting in a median post-merger HHI equal to 4,968. HHIs for integrated mergers increase by 2,803 points at the median, resulting in a median post-merger HHI equal to 6,998.

Figure \ref{fig:surplussum} summarizes our results for welfare.  Each of the four panels show how mergers impact surplus for a particular set of agents (consumers, retailers, wholesalers, or the entire market combined, respectively).  Surplus is reported as a percentage change of total pre-merger expenditures in the downstream market.  Each panel contains four box and whisker plots, with each plot corresponding to a different type of merger. The whiskers show the $5^{th}$ and $95^{th}$ percentiles of the outcome distribution, the boxes denote the $25^{th}$ and $75^{th}$ percentiles, and the solid horizontal line marks the median. Negative outcome values imply harm, and positive values imply benefits.

We focus first on the results for consumers in the left-most panel of Figure \ref{fig:surplussum}.  Less than half of all simulated vertical mergers are harmful, while the majority of upstream, downstream, and integrated mergers show harm.  In particular, there is a partial rank-ordering of consumer harm across types of mergers: consumer harm from downstream and integrated mergers first-order stochastically dominates consumer harm from upstream mergers which in turn dominates consumer harm from vertical mergers. However, while consumer harm is typically greater under integrated mergers than downstream mergers, it is not first-order stochastically dominant. Median consumer harm from integrated mergers is about 14\% of pre-merger expenditures, 1.3 times the magnitude of downstream mergers, and 3 times the magnitude from upstream mergers.

Moving on to retailers in the second panel of Figure \ref{fig:surplussum}, we find that whereas vertical mergers, downstream mergers, and integrated mergers almost always benefit retailers, upstream mergers harm retailers in about 60\% of all simulations. Moreover, there is a partial rank-ordering across mergers that is distinct from the consumer rank-ordering: the retailer surplus distribution from integrated mergers first-order stochastically dominates that from downstream mergers, while the retailer surplus distribution from vertical mergers stochastically dominates the distribution from upstream mergers. The median gain to retailers from integrated mergers is about 13\% of pre-merger expenditures,  1.5 times the magnitude of downstream mergers, and 1.1 times the magnitude of that from vertical mergers.

Turning to wholesaler surplus in the third panel of Figure \ref{fig:surplussum}, the effects seen there are largely reversed from those for retailers: wholesaler surplus increases in about 64\% of all upstream mergers, 11\% of vertical mergers, and about 16\% of downstream  and integrated mergers. Here, wholesaler surplus under integrated, downstream and upstream mergers each stochastically dominate the surplus from vertical mergers, but not one another.

As for total welfare, approximately 29\% of vertical mergers and 17\% of integrated mergers are beneficial, whereas only 9\% of upstream mergers and 2\% of downstream mergers are beneficial. Moreover, there is a partial rank ordering of mergers, with total harm from downstream mergers first-order stochastically dominating total harm from upstream  mergers, which dominates total harm from vertical mergers. Median consumer harm from integrated mergers is about 9\% of pre-merger total expenditures, 1.4 times the magnitude of that from downstream mergers, 2.6 times the magnitude of that from upstream mergers, and more than 4 times the magnitude of vertical mergers.


%%%%%%%%%%%%%%%%%%%%%%%%%%%%%%%%%%%%%%%%%%
\subsection{Mergers and Vertically Integrated Incumbent Firms}

Here, we examine merger outcomes both with and without preexisting vertical integration among the merging firms and as the number of rival incumbent integrated firms varies.  Figures \ref{fig:CVvertincumbBW_consumer} and \ref{fig:CVvertincumbBW_total} depict box and whisker plots summarizing consumer harm and total harm as the number of incumbent integrated firms increases. The plots when the number of incumbent integrated firms equals 0 correspond to the results depicted in Figure 1 of \citeN{ST2021} and are included as reference.\footnote{An important difference between the simulations depicted in Figures \ref{fig:CVvertincumbBW_consumer},  \ref{fig:CVvertincumbBW_total} and those in Figure 1 of \citeN{ST2021} is that here we do not include downstream markets where prices are set according to a second score auction.}  For vertical mergers, the plots when the number of  incumbent integrated firms equals 1 summarize market outcomes where an unintegrated wholesaler merges with an unintegrated retailer, a single third party is integrated, and any remaining rivals are unintegrated. By contrast, for downstream and upstream  mergers, the plots when the number of incumbent integrated firms equals 1 depict the outcome of a merger between an integrated firm and an unintegrated retailer or wholesaler, respectively, with no integrated third parties. For integrated mergers, the plots when the number of incumbent integrated firms equals 2 (the lowest number allowed in this instance) depict a merger between two integrated firms, again with no integrated rivals.

We begin with vertical mergers, as they are our simplest case, because there is never any pre-existing integration at either merging firm.  Absent incumbent integration by rivals (denoted by 0 integrated firms in Figures \ref{fig:CVvertincumbBW_consumer} and \ref{fig:CVvertincumbBW_total}), about half of vertical mergers benefit consumers and about 31\% of vertical mergers increase total surplus.  The addition of a single rival integrated firm (1 integrated firm in the figure) decreases median consumer harm from about 0\% of pre-merger revenues to to 2\% while widening the distribution of outcomes slightly. By contrast, additional integrating rivals has little effect on median total harm. Adding more integrated rivals tends to increase the proportion of mergers that benefit consumers, while also increasing the proportion of mergers that reduce total surplus. Thus, the presence of rival integrated firms can limit harm, particularly consumer harm, from vertical mergers.  Integrated rivals have sources of bargaining leverage that mirror the additional leverage a vertical merger brings to the combining firms.  It appears that these rivals can therefore somewhat limit the extent of negative impacts that would otherwise occur through RRC.

Moving to the second panel from the left in Figures \ref{fig:CVvertincumbBW_consumer} and \ref{fig:CVvertincumbBW_total}, upstream mergers absent incumbent vertical integration (0 integrated firms in the figure) are never beneficial in the range we study, with median consumer harm at about 6\% of pre-merger revenues and median total harm at about 3\% of pre-merger revenues. This result is to be expected, as upstream mergers without additional integration do not have the potential countervailing effects for welfare that have been discussed in the context of downstream or vertical mergers.  Once one of the merging firms is allowed to be integrated (1 integrated firm in the figure) beneficial mergers appear, and median harm decreases modestly.  Consumer welfare increases in almost 30\% of mergers and total welfare increases in about 14\% of mergers, but median consumer harm and total harm are roughly steady at approximately 4\% of pre-merger revenues.  The additional vertical integration creates some opportunities for EDM to enhance welfare, but this must be balanced against the potential for RRC.  Just as we saw with downstream mergers, here we find that harm dominates in most simulations.  Adding in integration by rival firms (2-4 integrated firms in the figure) narrows the inter-quartile range and trims the lower whisker-- meaning that relatively more harmful mergers are eliminated than beneficial ones-- while only slightly changing median harm.  Therefore, the presence of third-party incumbent integrated firms, as in the vertical and downstream cases, can limit the likelihood of observing extremely harmful mergers.

Next we examine the welfare impacts of downstream mergers. As with upstream mergers,  downstream mergers absent incumbent vertical integration (0 integrated firms in the figure) are never beneficial in the $5^{th}$ to $95^{th}$ percentile range, with median consumer harm from downstream mergers equal to about 9\% of pre-merger revenues and total harm equal to about 6\% of pre-merger revenues. \citeN{ST2021} found the same result for mergers in downstream logit Bertrand markets.  Allowing one of the merging firms to be vertically integrated when all other market participants are unintegrated (1 integrated firm in the figure) increases the range of outcomes while also leading to more median harm: consumer welfare increases in about 20\% of mergers, and total welfare increases in about 2\%, but median consumer harm grows to about 12\% of pre-merger revenues, and median total harm grows to about 8\%. Once one of the merging firms is already integrated, the merger now has potential EDM and RRC effects, which raises the possibility of both benefits and harms to welfare.  On net, we find that the harms dominate in most instances.  Adding integrated third party rivals (2-4 integrated firms in the figure) somewhat decreases the range of outcomes and moves the distribution towards less harm.  With 3 integrated third parties (4 integrated firms in the figure), the merger benefits consumers in about 34\% of markets and shrinks median consumer harm to about 5\% of pre-merger revenues. The inter-quartile range of total harm narrows, and median total harm falls to about 4\% of pre-merger revenues. As we saw with vertical mergers, it appears that the presence of third party integrated rivals can limit harm, though here the effect is a bit stronger.

We finish by examining integrated mergers in the right-most panel of Figures \ref{fig:CVvertincumbBW_consumer} and\ref{fig:CVvertincumbBW_total}.  Absent the presence of rival incumbent integrated firms (0 integrated firms in the figures), mergers between two integrated firms benefits consumers in about 6\% of simulated markets and are net beneficial in about 17\% of simulated markets. Thus, we find that mergers between two firms that are already integrated are harmful in the vast majority of cases.  Adding one additional rival incumbent integrated firm (1 integrated firms in the figures) lengthens the upper whisker, increasing the percentage of beneficial mergers, but has little effect on either median consumer or total welfare. Likewise, adding a second integrated rival (2 integrated firms in the figures) lengthens the upper whisker while also eliminating the most harmful integrated mergers and slightly increasing median consumer and total harms. Adding further integrated incumbents has modest incremental effects on the distribution of harm.

To summarize, our analysis of Figures \ref{fig:CVvertincumbBW_consumer} and \ref{fig:CVvertincumbBW_total} yield some overall conclusions.  First, we find that mergers where one or both of the merging firms are already integrated (and no rivals are integrated) are only beneficial in a small proportion of instances.  Second, the presence of integrated rivals can decrease the likelihood of observing the most harmful mergers, though the strength of that effect can vary widely across types of mergers. Third, with the exception of downstream mergers,additional integrated firms often tend to increase the incidence of with the exception of integrated mergers, additional integrated firms do not substantively affects median consumer or total surplus, but can greatly effect outcome distribution skewness.


%%%%%%%%%%%%%%%%%%%%%%%%%%%%%%%%%%%%%%%%%%
\subsection{Mergers and Bargaining Power}
\citeN{ST2021} show using numerical simulations that downstream and vertical mergers when wholesalers have relatively more bargaining power are less harmful compared to when retailers have relatively more bargaining power. Here, we find that this relationship also holds for downstream and upstream mergers when one of the merging parties is already vertically integrated and for integrated mergers.  We also show that the result persists for all types of mergers when there are rival non-merging integrated firms.

Figures \ref{fig:CVbargincumbent_consumer}  and \ref{fig:CVbargincumbent_total}  depict box and whisker plots summarizing the consumer and total welfare effects for downstream, upstream, and vertical mergers as the relative bargaining power parameter goes from 9 (wholesalers have the advantage) to 1/9 (retailers have the advantage). Also depicted for each bargaining power parameter are three sets of box and whisker plots that correspond to the number of incumbent vertically integrated firms included in the simulated markets: 0 (light blue), 1 (darker blue), and 4 (darkest blue) firms.  As in Figures \ref{fig:CVvertincumbBW_consumer} and \ref{fig:CVvertincumbBW_total}, plots when the number of integrated firms is 0 assume that pre-merger, no firms in the market are vertically integrated; these are comparable to those in \citeN{ST2021}. For vertical mergers, the plots when the number of incumbent integrated firms equals 1 depict the outcome from an unintegrated wholesaler merging with an unintegrated retailer when one third party rival is integrated. By contrast, for upstream and downstream mergers, the plots when the number of incumbent integrated firms equals 1 depict the outcome of a merger between an integrated firm and an unintegrated firm when all third parties are unintegrated. In turn, plots with 4 integrated firms increase the number of rival incumbent integrated firms in vertical mergers to 4 and in upstream and downstream mergers to 3.  Finally, in integrated mergers, when the number of  incumbent integrated firms equals 0, this depicts a merger between two integrated firms with all rivals are unintegrated. Plots with either 1 or 4 integrated firms depict mergers between integrated incumbents when there are either 1 or 4 rival integrated firms in the market, respectively.

We begin by examining vertical mergers.  We find that the presence of integrated incumbents has little impact on the relationship between relative bargaining power and welfare. Consistent with the patterns seen in \citeN{ST2021}, mergers tend to benefit welfare when wholesalers have relatively more power, and harm welfare when retailers have relatively more power.  Larger wholesaler bargaining power offers more possibilities for EDM, as pre-merger input prices are likely to be high.  This channel becomes less relevant as retailers gain more power.  The impact on consumers is roughly zero when wholesalers and retailers have equal power.  Furthermore, we find that consumer harm when there are 0 incumbent integrated firms first-order stochastically dominates harm when there are 4  incumbent integrated firms, with the difference in magnitudes decreasing as retailer bargaining power increases. Across all three integrated incumbent scenarios, median consumer harm moves from about -24\% of pre-merger expenditures when wholesalers have the advantage to 7\% when retailers have the advantage.  Similarly, median total harm goes from from about -14\% of pre-merger expenditures when wholesales have the advantage to 2.9\% when retailers have the advantage.


We turn next to upstream mergers in Figures \ref{fig:CVbargincumbent_consumer} and   \ref{fig:CVbargincumbent_total}.  Without incumbent integrated firms in the market (0 integrated firms), the relationship between harm and bargaining power is unlike that of downstream and vertical mergers: median harm first increases and then decreases as wholesalers have relatively more bargaining power, with the largest depicted median harm-- 12\% for consumers, 18\% on net-- occurring when relative bargaining power 7/3.  Increasing bargaining power beyond 7/3 evidently diminishes harm because pre-merger, wholesalers have already expropriated much of the retailers' surplus. By contrast, for mergers between an integrated and an unintegrated upstream supplier (denoted by 1 integrated firm in the figures), the relationship between harm and bargaining power is more in line with that for downstream and vertical mergers.  This is intuitive, as now all three types of mergers create some form of EDM that is to varying degrees passed through to consumers. Median consumer harm goes from roughly -21\% of pre-merger expenditures when wholesalers have relatively more bargaining power to almost 8\% of pre-merger expenditures when retailers have relatively more bargaining power. Median total harm increases from about -24\% to more than 3\% of pre-merger expenditures. Adding incumbent integrated rivals (see 4 integrated firms in the figure) preserves the relationship between bargaining power and harm, though the spread of outcomes narrows.

For downstream mergers, we continue to find that higher retailer relative bargaining power leads to more negative impacts on welfare.  Absent any incumbent integration (0 integrated firms), all the outcomes shown in Figures \ref{fig:CVbargincumbent_consumer} and \ref{fig:CVbargincumbent_total} feature a decrease in welfare, which mirrors our findings in Figures \ref{fig:CVvertincumbBW_consumer} and \ref{fig:CVvertincumbBW_total}.  Median consumer harm goes from about 1\% of pre-merger expenditures when wholesalers have relatively more bargaining power to 19\% of pre-merger expenditures when retailers have relatively more bargaining power. Likewise, median total harm moves from about 3\% of pre-merger expenditures to about 8\%.  In situations where retailers already have more bargaining power, retailers are likely to have extracted significant surplus from wholesalers prior to the merger, which limits any potential benefits from increased bargaining leverage.  Thus, there is little to counteract harms from decreased downstream competition.

Once we allow one of the merging firms to be integrated (denoted by 1 integrated firm in the figure), some beneficial mergers appear, particularly when wholesalers have relatively more bargaining power.  The box and whisker plots in Figure \ref{fig:CVbargincumbent_consumer} fold towards one another around a pivot at equal bargaining power, strengthening the relationship between bargaining power and extent of consumer harm.  Median consumer harm goes from approximately -28\% of pre-merger revenues when wholesalers have the advantage to 24\% of pre-merger revenues when retailers have the advantage.  Median total harm goes from almost 5\% to about 11\% of pre-merger revenues.  The existence of pre-merger integration at one of the merging firms creates an opportunity for EDM, the benefits from which are likely to be largest when wholesalers have high bargaining power and therefore charge high pre-merger input prices.  Compared to vertical mergers, where impacts to consumers were roughly neutral at 2/3 bargaining power, here neutrality happens between  3/2 and 7/3, when wholesalers have a bit more than 1.5 times the power of retailers.  It appears that wholesalers must be relatively more powerful (and thus the likely gains from EDM larger) in order to generate net consumer benefits, compared to in a vertical merger.  This is intuitive, as here the downstream merger also causes an additional lessening in horizontal competition.  Once we add rival integrated firms (see 4 integrated firms in the figure) the box and whisker plots rotate counter-clockwise, weakening but not erasing the relationship between bargaining power and harm. The range of outcomes also shrinks.  Starting from when wholesalers have relatively more bargaining power and moving right, median consumer harm goes from approximately -35\% of pre-merger revenues to 13\% of pre-merger revenues, while median total harm goes from almost 3\% to 5\% of pre-merger revenues.

The last column in Figures \ref{fig:CVbargincumbent_consumer} and   \ref{fig:CVbargincumbent_total} display results for integrated mergers.  In terms of bargaining power and resulting harm, integrated mergers are perhaps most similar to vertical mergers. Like vertical mergers, there is a strong monotonic relationship between bargaining power and harm. Median consumer harm moves from about 0\% of pre-merger expenditures when wholesalers have relatively more bargaining power to about 26\% of pre-merger expenditures when retailers have relatively more bargaining power. Likewise, median total harm moves from about -19\% when wholesalers have relatively more bargaining power to 15\% of pre-merger expenditures when retailers have relatively more bargaining power.  A second similarity between vertical and integrated mergers is that the presence of rival integrated incumbents (see 1 or 4 integrated firms in the figure) has modest impact on the relationship between bargaining power and harm. However, a key difference between vertical and integrated mergers is that while vertical mergers are often beneficial when wholesalers have relatively more bargaining power, integrated mergers are often harmful unless wholesaler relative bargaining power is roughly 4 or more. The crossing at zero harm is shifted to the left.

%Finally, Figure \ref{fig:CVbargincumbent_consumer} depicts box and whisker plots summarizing consumer, retailer, wholesaler and total surplus  for integrated mergers with 2,4, or 6 integrated firms.  When the number of  incumbent integrated firms equals 2, Figure \ref{fig:CVbargincumbent_consumer} depicts a merger between two integrated firms: all rivals are unintegrated. Plots with either 4 or 6 integrated firms depict mergers between integrated incumbents when there are either 2 or 4 rival integrated firms in the market.

%\textbf{Describe the counterintuitive result that increasing retailer bargaining power harms retailers but help wholesalers.}

%%%%%%%%%%%%%%%%%%%%%%%%%%%%%%%%%%%%%%%%%%
\subsection{Robustness to Alternative Cost Specifications}

Here, we explore the role that the constant marginal cost assumption plays in driving our results by comparing our simulations to an alternative set that allow costs to increase linearly. Figure \ref{fig:surplussumcost} displays box and whisker plots for the net effect on consumer and total welfare under four different scenarios: constant marginal costs for all firms (top row, blue), linear marginal costs for all firms (top row, orange), constant marginal costs for the merging parties' products but linear costs for the non-merging parties products (bottom row, blue), and linear marginal costs for the merging parties' products but constant costs for the non-merging parties products (bottom row, orange).

In theory, having linear marginal costs rather than constant costs could result in fewer gains from EDM, as efforts to increase sales due to a wholesale price reduction would be counteracted by rising costs. Likewise, in theory, RRC could be more profitable for the merging parties under linear compared to constant marginal costs, as linear marginal costs could cause the integrated wholesaler's prices to increase more rapidly or downstream rival retailers to reduce output more quickly. Because impacts from EDM are plausibly less under linear marginal costs while those from RRC are plausibly more, one might expect that on net mergers are more harmful under linear rather than constant marginal costs.

Our simulations confirm this hypothesis.  See the top two panels in Figure \ref{fig:surplussumcost}, which compare a scenario where all firms have constant marginal costs to one where they all have linear marginal costs.  Overall, harm is greater for mergers with linear costs. The difference is largest for vertical mergers.  When all firms have constant marginal costs, vertical mergers increase consumer surplus in about 55\% of simulations and total surplus in about 29\% of simulations, but only increase consumer surplus in about 36\% of simulations and total surplus in about 2\% of simulations when all firms have linear costs.

The bottom panel of Figure \ref{fig:surplussumcost} compares a situation where only the merging firms have constant marginal costs to one where only the merging firms have linear marginal costs.  We see that the distributions in these bottom panels are largely the same as those in the top panels.  Thus, it appears that the key driver of our results is whether or not the merging parties have constant marginal costs.  When the merging firms' marginal costs are constant, the distribution of outcomes moves towards less harm.

%Mergers with constant-cost firms increases consumer surplus in about 15\% of all mergers and total surplus in 10.5\% of all mergers, while mergers with linear-cost firms increase consumer surplus in only 2.3\% of mergers and total surplus in 0.8\% of all mergers. However, while the consumer and total harm distributions for each merger type under the all-firm linear marginal costs scenario first-order stochastically dominates their counterparts under the all-firm constant marginal costs scenario, the consumer and total harm distributions under the party linear marginal costs scenario only first-order stochastically dominates the their counterparts under the party constant marginal costs scenario for vertical and integrated mergers. The median consumer harm from markets with linear marginal costs is about 9.3\% of pre-merger total expenditures, about 1.3 times the magnitude of markets with constant marginal costs.

%Changing the cost structure consistently yields the greatest differences in outcomes for vertical mergers. When all firms have constant marginal costs, vertical mergers increase consumer surplus in about 30\% of all vertical merger simulations and total surplus in about 18\% of all vertical merger simulations, but only increase consumer surplus in about 1\% of vertical merger simulations and total surplus in about 0.9\% of vertical merger simulations when all firms have linear costs. Moreover, qualitatively similar results hold for vertical mergers when only the merging parties' products have constant marginal costs, compared to when only the merging parties' products have linear costs.

%The differences between the constant and linear cost structures is more muted for both upstream and downstream mergers. Upstream mergers with all constant marginal cost firms benefit consumers in about 12\% of simulations, but only 0.2\% of simulations when firms have linear costs. Downstream mergers with all constant marginal cost firms benefit consumers in about 11\% of simulations, but only about 6\% of simulations when all firms have linear costs. Again, qualitatively similar results hold for vertical mergers when only the merging parties' products have constant marginal costs, compared to when only the merging parties' products have linear costs.

%Finally, while there is a marked difference in outcomes between the constant and linear cost structures for integrated  mergers under the all-firms scenario, the differences in outcomes are more muted under the party scenario, especially for consumer surplus. In particular, median consumer harm when all firms face constant marginal costs is 13.6\%, compared to the median consumer harm of 17.5\% when all firms face linear marginal costs. By contrast, median consumer harm when only the merging parties' have constant costs is 15.1\%, compared to the median consumer harm of 16.5\% when the parties' have linear costs.  %Integrated mergers benefit consumers in about 10\% of simulations when all firms have constant costs, but less than 1\% of simulations when all firms have linear costs.

%%%%%%%%%%%%%%%%%%%%%%%%%%%%%%%%%%%%%%%%%%
\section*{Empirical Application}

The U.S. waste and recycling industry generates approximately \$80 billion in annual revenues.\footnote{\tiny{Waste Dive, \url{https://www.wastedive.com/news/public-companies-increased-control-of-74b-us-waste-industry-in-2018/556079/}}} In recent years the industry has experienced significant merger activity including several large acquisitions between vertically integrated, national competitors. However, solid waste companies tend not to be homogeneous in their degree of vertical integration across geographies. As a result, each of these mergers exhibits a variety of vertical supply chain configurations both pre- and post-merger across their relevant local markets, making it an excellent application for further study of the welfare impacts of mergers that have both horizontal and vertical aspects.

The vertical supply chain in the solid waste industry is primarily comprised of waste collection operations or ``haulers'' and waste disposal facilities. Haulers collect municipal solid waste (MSW) from businesses and residences and must dispose of it at a lawful disposal site, predominantly landfills. Waste disposal (upstream) is a required input into waste collection services (downstream). Some haulers are vertically integrated and operate their own disposal facilities. Vertically-integrated haulers typically prefer to dispose of waste at their own disposal facilities and may also sell a portion of their disposal capacity. Disposal customers include private waste haulers without their own disposal assets (“independent haulers”) as well as local governments that collect their citizens' waste themselves. Due to strict laws and regulations that govern the disposal of MSW, there are no reasonable substitutes for MSW disposal. Thus, mergers that combine hauling and disposal assets may incentivize the merged entity to raise its hauling rivals' cost of disposal in order to benefit its own collection operations. Whether or not the merged entity is both able and incentivized to undertake such action depends upon the extent of its market power in the local disposal market, the merging parties’ profit margins in each line of business, and their intensity of hauling competition with prospective disposal customers.

In 2020, Waste Management, the largest waste management company in the U.S., acquired Advanced Disposal Services (ADS), previously the fourth largest company, for \$4.6 billion.\footnote{Competitive Impact Statement: U.S. and Plaintiff States v. Waste Management, Inc. and Advanced Disposal Services, Inc., \url{https://www.justice.gov/atr/case-document/file/1330596/download}} GFL Environmental also acquired WCA Waste Corporation for \$1.2 billion in 2020.\footnote{Waste 360,\tiny{ \url{https://www.waste360.com/business/breaking-gfl-acquires-wca-waste-corp-121-billion}}} Republic Services, the second largest waste management company in the U.S., acquired Santek Environmental in 2021.\footnote{Competitive Impact Statement: U.S. and State of Alabama v. Republic Services, Inc.and Santek Waste Services, LLC, \url{https://www.justice.gov/atr/case-document/file/1382626/download}}  These three merged companies along with Waste Connections, the third largest waste management company in the U.S., are estimated to control over 60\% of available landfill capacity nationally and also rank among the top haulers nationwide.\footnote{Waste Business Journal,\tiny{ \url{https://www.wastedive.com/news/public-companies-increased-control-of-74b-us-waste-industry-in-2018/556079/ }}} Concentration in local markets varies substantially, however, in both the upstream and downstream markets.

%Concerns about vertical competitive effects were raised in both the Republic-Santek and the Waste Management-ADS transactions. Both the Solid Waste Agency of Lake County, IL and the Solid Waste Agency of Northern Cook County submitted comments in opposition to the proposed asset divestiture from Waste Management-ADS, stating that a vertically integrated competitor was needed to maintain competition in their local market post-merger.\footnote{\tiny{See  Comments by SWALCO and SWANCC : U.S. and Plaintiff States v. Waste Management, Inc., and Advanced Disposal Services, Inc., \url{https://www.justice.gov/atr/case-document/file/1377646/download}.}}
For example, in the DOJ complaint filed in the Republic-Santek case, horizontal anti-competitive effects were alleged for four SCCW collection markets and two MSW disposal markets. In addition, vertical anti-competitive effects were alleged to arise from the combination of their integrated assets in the Chattanooga area.\footnote{See U.S. and State of Alabama v. Republic Services, Inc. and Santek Waste Services, LLC, \url{https://www.justice.gov/atr/case-document/file/1382031/download}}
%Our application in progress further identifies local markets in which these acquisitions result in: 1) only horizontal combinations of assets, 2) only vertical combinations of assets, and 3) combinations of vertically integrated assets in the presence of existing integrated competitors to analyze the welfare effects from mergers with complex vertical arrangements. The next sections present preliminary results on the merger of vertically integrated assets in the presence of other integrated and unintegrated rivals in the context of Republic and Santek's merger in the Chattanooga area.
The next sections present results of applying the model of Section \ref{sec:theory} to the merger of vertically integrated assets in the presence of other integrated and unintegrated rivals in the context of Republic and Santek's merger in the Chattanooga area.

\subsection{Combination of vertically integrated assets: Chattanooga}

The Competitive Impact Statement (CIS) filed by the DOJ in association with the Republic-Santek merger describes the alleged lost competition in the ``Chattanooga, Tennessee and North Georgia area'', subsequently referred to as the Chattanooga Area, due to lost horizontal competition in MSW disposal and small-container commercial waste (SCCW) collection as well as raising rivals costs in the SCCW collection market by raising the MSW disposal costs of independent haulers. The CIS notes that pre-merger Republic and Santek combined served approximately 73 percent of the SCCW collection market with three other significant competitors. In MSW disposal, the CIS identifies only one other significant competitor pre-merger and Republic and Santek combined as serving approximately 82 percent of the market, disposed of either directly in the merging parties’ landfills within the area or passing through their transfer stations in Chattanooga before ultimately being disposed of in the parties’ landfills elsewhere. Thus, pre-merger both parties were large, vertically integrated competitors in the Chattanooga Area.

In addition, another large, vertically integrated waste company existed in the market at the time of the merger, Waste Connections, and was the parties' sole competitor in the MSW disposal market. Waste Management and ADS owned collection assets in the area but were not vertically integrated in this market, as demonstrated by MSW disposal data discussed in Section{\ref{chatt_disposal_vol}}.\footnote{Firms that are national competitors and vertically integrated in other markets are known to enter contracts with each other to dispose of waste on advantageous terms that may make them effectively vertically integrated. Ignoring these contracting relationships may underestimate the number of effectively vertically integrated competitors in the market.} We treat Waste Management and ADS as a single, merged entity regardless of whether the data source predates the merger consummation since the merger was completed before the filing of the CIS for the Republic-Santek merger. The final significant participant in the SCCW collection market is a major regional firm that is not identified by name in the CIS.

\subsubsection{MSW Disposal Volume Data}
\label{chatt_disposal_vol}

Data on MSW disposal comes from data collected by TDEQ (Tennessee Department of Environmental Quality) consisting of Class 1 landfill ownership, Class 1 Solid Waste Origin Reports, and waste disposal by county for 2019. This data identifies the origin county and destination landfill, including owner and operator information, for all MSW produced in Tennessee as well as waste volumes passing through transfer stations in Tennessee. The MSW disposal market definition follows that outlined in the DOJ CIS and attributes share to the company owning the final disposal landfill (i.e., ignoring transfer stations which are an intermediate disposal site only). Thus, the total market quantity is defined as all MSW volumes originating in Hamilton County, Tennessee where Chattanooga is located.

Disposal volumes are measured in tons and have been combined across landfills owned by the same firm. The TDEQ data identifies landfills owned or operated by Republic, Santek, Waste Connections, and three other market participants (each with less than 0.5\% share) receiving MSW volumes originating in Hamilton County.\footnote{The City of Chattanooga and Marion County landfills are municipally owned and operated, accounting for less than 1\% combined. Global Envirotech is a privately-owned transfer station that sends its 0.03\% share to an out of state landfill in GA.} These three fringe participants have been excluded from the analysis. After re-scaling, the resulting market shares are Republic, 28.4\%, Santek, 54\%, and Waste Connections, 17.6\%.

Using this data in conjunction with information on collection market volumes, discussed in Section \ref{chatt_collection_share}, the supply relationships between the upstream disposal market participants (\textit{Republic, Santek, and Waste Connections}) and downstream collection market participants (\textit{Republic, Santek, Waste Connections, WM-ADS, and ``Regional''}) can be inferred under the assumption that vertically-integrated haulers first dispose of waste at their own disposal facilities and then sell any residual disposal capacity to rival haulers. Santek and Waste Connections exhibit excess disposal capacity. However, Republic's collection volumes are estimated to exceed their ability to self-supply disposal in the Chattanooga market. As a result, Republic will need to purchase additional disposal from Santek and Waste Connections, presumably at a higher marginal cost. We assume that this disposal is purchased from both Santek and Waste connections according to their share of available excess MSW disposal capacity of 63.8\% and 36.2\%, respectively. We further assume that WM-ADS and Regional purchased disposal capacity from both Santek and Republic according to these shares. The resulting relationships between the upstream and downstream market participants are captured in Figure \ref{fig:chatt_tree}.

\subsubsection{MSW Disposal Price Data}
MSW disposal prices are collected from the 2019 Waste Business Journal's Directory of Waste Processing \& Disposal Sites. The measure of price reflected is the ``gate rate'', which is the posted price at the landfill, measured in \$/Ton.\footnote{Disposal prices for large customer's may be bilaterally negotiated instead of paying the gate rate.} Republic and Santek both operate multiple landfills in the market with different prices. The price used in the analysis for each is the volume weighted average for their landfills.

\subsubsection{Collection Market Share Data}
\label{chatt_collection_share}

The CIS states that in the Chattanooga Area the post-merger HHI for SCCW collection would be approximately 5,551 post-merger with an increase of 2,660 points and that the combined market share of the merging parties is 73\%. Taking these figures as given we can recover the collection market shares under the assumptions that 1) the merging parties are of equal size, and 2) the non-merging parties are of equal size. After re-scaling, this produces downstream market shares for Republic, 37.6\%, Santek, 37.6\%, Waste Connections, 8.3\%, WM-ADS, 8.3\%, and Regional, 8.3\%.

However, the market shares required for implementation of the Sheu-Taragin (\textit{\textbf{add ref}}) framework are expressed as upstream-downstream market participant pairs. Following the discussion of supply relationships in Section \ref{chatt_disposal_vol}, vertically-integrated pairs without capacity constraints (i.e., Santek disposal-Santek collection, Waste Connection disposal-Waste Connections collection) are assigned their full collection share. Capacity constrained, integrated firms (Republic disposal-Republic collection) are assigned their collection share up to their available capacity with the remainder allocated by residual disposal share to pairs with the respective upstream firms (i.e., Santek disposal-Republic collection, Waste Connections disposal-Republic collection). Unintegrated firms' collection shares are  distributed among the upstream suppliers with available capacity by residual share as well.

\subsubsection{Cost, Margin and  Elasticity Data}

Collection and disposal margins are calculated for Republic from data on revenue by line of service and components of cost of operations in the company's 2019 annual report. The data is reported at the company level and is not specific to the Chattanooga Area, but revenues are reported by collection segment.

Collection costs in \$/Ton are estimated from Republic and Waste Management 2019 10k financial statements. These costs are reported at the company level across all segments. The estimated share of these costs attributable to the Chattanooga market, based on the number of markets in which the companies operate, is divided by the tons disposed for each company. Santek, Waste Connections, and Regional do not publicly produce comparable financial statements. Instead, we assume that the cost structure is the same for the vertically-integrated companies, taking into account their individual tons disposed, and that WM-ADS and Regional share the same cost structure in this market due to their lack of internal disposal capacity.

A survey of demand elasticity estimates for the collection market are present in Bel and Gradus (2016) with an average of -0.34. These estimates are not specific to the SCCW market, but available evidence suggests that the commercial segment should be more inelastic than other segments such that analysis using these estimates would tend to be conservative for estimating the size of merger induced price effects.\footnote{Bel, G. and R. Gradus, 2016, Effects of unit-based pricing on household waste collection demand: A meta-regression analysis, Resource and Energy Economics, 44, 169-182.}

\subsubsection{Chattanooga Area Merger Simulation Results}

Overall, the merger between Republic and Santek is estimated to lower upstream prices by 2.8\% and increase downstream prices by 12\%, resulting in \$16 million of harm to consumers and total producer benefits of approximately \$13.4 million. Thus, on net the model predicts the merger would be harmful despite the presence of significant efficiencies from EDM that reduce prices upstream. These aggregate effects obscure the complicated competitive effects between these interdependent firms though. These effects are illustrated in more detail in Figure \ref{fig:TrashSimsFirms}.

Disposal prices for pre-merger integrated pairs are unchanged post-merger. However, MSW disposal prices increase to all rival haulers post-merger (i.e., all unintegrated pairs). The merged firms' disposal prices to downstream rivals are estimated to increase 88\% post-merger, and Waste Connections prices to its downstream rivals are estimated to increase 22\%. For the merged firm the price increase is driven by both the effects of increased bargaining leverage and upward pricing pressure, whereas, Waste Connections' price increase is driven only by the change in bargaining leverage. These large price increases are offset in the aggregate, volume-weighted estimate by the large decrease in the disposal price for the integrating pair post-merger due to the EDM effect, and the cost to Republic for volume disposed with Santek decreases by 60\%.

Turning to the downstream market, all collection prices increase post-merger. The merging parties' post-merger collection prices increase by approximately 20\% and share decreases by about 8 percentage points overall. The pre-merger integrated Republic pair's price increases 21\% from the effect of upward pricing pressure alone and Santek's increases by 23\% from a combination of UPP and increased bargaining leverage. The newly integrated Santek-Republic pair's price is estimated to increase by 9\%, demonstrating that the upstream efficiencies are not fully passed-through to consumers. Pre-merger integrated rival, Waste Connections, collection price increases by approximately 5\% and share increases by 6 percentage points. Unintegrated rivals' collection prices each increase by about 9\%, which again demonstrates incomplete pass-through of the change in upstream costs, and share increases by 2 percentage points each.

\subsubsection{Merger Simulation of Collection Market Only}

We compare these results to those from a merger simulation considering only a merger in the collection market under logit demand, treating each upstream-downstream pair as if it were an independent firm, as shown in Figure \ref{fig:TrashSimsCompare}. Ignoring these vertical relationships results in an estimated 9.4\% increase in collection prices with a 15\% increase in the merging parties' price, consumer harm of \$9.6 million and a producer benefit of \$7.1 million. While net harm is nearly the same under both models at about \$2.5 million, estimated consumer harm is 67\% higher in the model including vertical relationships and is offset by markedly larger producer benefits. Further, the collection-only model overestimates diversion to the unintegrated rivals and as a result underestimates the market price effect due to UPP from the merger. (See also Table \ref{tab:trash_merged}.)

The Republic-Santek merger was ultimately settled with the DOJ through a divestiture. According to the final judgement, in the Chattanooga Area the parties were required to divest Santek's SCCW collection assets  as well as two landfills and a transfer station in the Chattanooga area.\footnote{Final Judgment: U.S and Plaintiff States v. Republic Services, Inc. and Santek, LLC, \url{https://www.justice.gov/atr/case-document/file/1408616/download}} Our results suggest that a divestiture in the collection market alone likely would not have sufficiently remedied the anti-competitive effects of the merger.

%%%%%%%%%%%%%%%%%%%%%%%%%%%%%%%%%%%%%%%%%%%%%%%%%%%%%%%%%%%%%%%%%%%%%%%%%%%%%%%%%
\section{Conclusion}

This paper relaxes an assumption made in \citeN{ST2021}: that mergers occur in vertical supply chains where in the pre-merger state, no firms are vertically integrated. Relaxing this constraint is important as competition authorities are routinely called upon to investigate mergers where either the merging parties or a third party are already vertically integrated. Here, we show using numerical simulations that relaxing these assumptions can result in merger outcomes that are markedly different from those when these assumptions are maintained.  This is further illustrated by the results from our simulations in the Republic/Santek merger, where we find that ignoring existing vertical relationships drastically understates harm to Chattanooga consumers. Our hope is that future research will focus on some of the other assumptions made in \citeN{ST2021}, such as how the presence of siloed firms, two-part tariffs, or large fixed costs affect the Nash bargaining game and therefore merger outcomes.


%%%%%%%%%%%%%%%%%%%%%%%%%%%%%%%%%%%%%%%%%%%%%%%%%%%%%%%%%%%%%%%%%%%%%%%%%%%%%%%%%
% Bibliography
%%%%%%%%%%%%%%%%%%%%%%%%%%%%%%%%%%%%%%%%%%%%%%%%%%%%%%%%%%%%%%%%%%%%%%%%%%%%%%%%%
\newpage
\bibliographystyle{chicago}
\bibliography{master-bib}
\clearpage

%%%%%%%%%%%%%%%%%%%%%%%%%%%%%%%%%%%%%%%%%%%%%%%%%%%%%%%%%%%%%%%%%%%%%%%%%%%%%%%%%
% Tables and Figures
%%%%%%%%%%%%%%%%%%%%%%%%%%%%%%%%%%%%%%%%%%%%%%%%%%%%%%%%%%%%%%%%%%%%%%%%%%%%%%%%%

\begin{table}
\small{
\begin{tabular}{l|l|l|r}
\hline
variable & merger & quant & val\\
\hline
down &  & Min & 2\\

up &  & Min & 2\\

vert &  & Min & 0\\

barg &  & Min & 0\\

nestParm &  & Min & 0\\

avgpricepre.up &  & Min & 1\\

avgpricepre.down &  & Min & 6\\

mktElast & \multirow{-8}{*}{\raggedright\arraybackslash all} & Min & -60\\
\cline{1-4}
hhipre &  & Min & 2008\\

hhipost &  & Min & 2915\\

hhidelta & \multirow{-3}{*}{\raggedright\arraybackslash up} & Min & 0\\
\cline{1-4}
hhipre &  & Min & 2011\\

hhipost &  & Min & 2931\\

hhidelta & \multirow{-3}{*}{\raggedright\arraybackslash down} & Min & 0\\
\cline{1-4}
hhipre &  & Min & 2100\\

hhipost &  & Min & 3120\\

hhidelta & \multirow{-3}{*}{\raggedright\arraybackslash vertical} & Min & 32\\
\cline{1-4}
hhipre &  & Min & 2205\\

hhipost &  & Min & 3633\\

hhidelta & \multirow{-3}{*}{\raggedright\arraybackslash both} & Min & 2\\
\cline{1-4}
down &  & p25 & 3\\

up &  & p25 & 3\\

vert &  & p25 & 0\\

barg &  & p25 & 0\\

nestParm &  & p25 & 0\\

avgpricepre.up &  & p25 & 2\\

avgpricepre.down &  & p25 & 10\\

mktElast & \multirow{-8}{*}{\raggedright\arraybackslash all} & p25 & -1\\
\cline{1-4}
hhipre &  & p25 & 2393\\

hhipost &  & p25 & 4011\\

hhidelta & \multirow{-3}{*}{\raggedright\arraybackslash up} & p25 & 1546\\
\cline{1-4}
hhipre &  & p25 & 2572\\

hhipost &  & p25 & 4135\\

hhidelta & \multirow{-3}{*}{\raggedright\arraybackslash down} & p25 & 1431\\
\cline{1-4}
hhipre &  & p25 & 3623\\

hhipost &  & p25 & 5069\\

hhidelta & \multirow{-3}{*}{\raggedright\arraybackslash vertical} & p25 & 1051\\
\cline{1-4}
hhipre &  & p25 & 3591\\

hhipost &  & p25 & 6242\\

hhidelta & \multirow{-3}{*}{\raggedright\arraybackslash both} & p25 & 2463\\
\cline{1-4}
down &  & p50 & 4\\

up &  & p50 & 4\\

vert &  & p50 & 1\\

barg &  & p50 & 1\\

nestParm &  & p50 & 0\\

avgpricepre.up &  & p50 & 5\\

avgpricepre.down &  & p50 & 13\\

mktElast & \multirow{-8}{*}{\raggedright\arraybackslash all} & p50 & -1\\
\cline{1-4}
hhipre &  & p50 & 2876\\

hhipost &  & p50 & 4963\\

hhidelta & \multirow{-3}{*}{\raggedright\arraybackslash up} & p50 & 2027\\
\cline{1-4}
hhipre &  & p50 & 3166\\

hhipost &  & p50 & 5207\\

hhidelta & \multirow{-3}{*}{\raggedright\arraybackslash down} & p50 & 2040\\
\cline{1-4}
hhipre &  & p50 & 4391\\

hhipost &  & p50 & 6016\\

hhidelta & \multirow{-3}{*}{\raggedright\arraybackslash vertical} & p50 & 1292\\
\cline{1-4}
hhipre &  & p50 & 4275\\

hhipost &  & p50 & 7272\\

hhidelta & \multirow{-3}{*}{\raggedright\arraybackslash both} & p50 & 2800\\
\cline{1-4}
down &  & p75 & 5\\

up &  & p75 & 5\\

vert &  & p75 & 2\\

barg &  & p75 & 1\\

nestParm &  & p75 & 0\\

avgpricepre.up &  & p75 & 10\\

avgpricepre.down &  & p75 & 19\\

mktElast & \multirow{-8}{*}{\raggedright\arraybackslash all} & p75 & 0\\
\cline{1-4}
hhipre &  & p75 & 3794\\

hhipost &  & p75 & 6780\\

hhidelta & \multirow{-3}{*}{\raggedright\arraybackslash up} & p75 & 2891\\
\cline{1-4}
hhipre &  & p75 & 4257\\

hhipost &  & p75 & 7391\\

hhidelta & \multirow{-3}{*}{\raggedright\arraybackslash down} & p75 & 2934\\
\cline{1-4}
hhipre &  & p75 & 5679\\

hhipost &  & p75 & 7293\\

hhidelta & \multirow{-3}{*}{\raggedright\arraybackslash vertical} & p75 & 1755\\
\cline{1-4}
hhipre &  & p75 & 5358\\

hhipost &  & p75 & 8648\\

hhidelta & \multirow{-3}{*}{\raggedright\arraybackslash both} & p75 & 3187\\
\cline{1-4}
down &  & Max & 5\\

up &  & Max & 5\\

vert &  & Max & 4\\

barg &  & Max & 1\\

nestParm &  & Max & 0\\

avgpricepre.up &  & Max & 270\\

avgpricepre.down &  & Max & 297\\

mktElast & \multirow{-8}{*}{\raggedright\arraybackslash all} & Max & 0\\
\cline{1-4}
hhipre &  & Max & 10000\\

hhipost &  & Max & 10000\\

hhidelta & \multirow{-3}{*}{\raggedright\arraybackslash up} & Max & 5000\\
\cline{1-4}
hhipre &  & Max & 10000\\

hhipost &  & Max & 10000\\

hhidelta & \multirow{-3}{*}{\raggedright\arraybackslash down} & Max & 5000\\
\cline{1-4}
hhipre &  & Max & 9962\\

hhipost &  & Max & 10000\\

hhidelta & \multirow{-3}{*}{\raggedright\arraybackslash vertical} & Max & 4314\\
\cline{1-4}
hhipre &  & Max & 9998\\

hhipost &  & Max & 10000\\

hhidelta & \multirow{-3}{*}{\raggedright\arraybackslash both} & Max & 5000\\
\cline{1-4}
down &  & Markets & 2202997\\

up &  & Markets & 2202997\\

vert &  & Markets & 2202997\\

barg &  & Markets & 2202997\\

nestParm &  & Markets & 2202997\\

avgpricepre.up &  & Markets & 2202997\\

avgpricepre.down &  & Markets & 2202997\\

mktElast & \multirow{-8}{*}{\raggedright\arraybackslash all} & Markets & 2202997\\
\cline{1-4}
hhipre &  & Markets & 654731\\

hhipost &  & Markets & 654731\\

hhidelta & \multirow{-3}{*}{\raggedright\arraybackslash up} & Markets & 654731\\
\cline{1-4}
hhipre &  & Markets & 701936\\

hhipost &  & Markets & 701936\\

hhidelta & \multirow{-3}{*}{\raggedright\arraybackslash down} & Markets & 701936\\
\cline{1-4}
hhipre &  & Markets & 462005\\

hhipost &  & Markets & 462005\\

hhidelta & \multirow{-3}{*}{\raggedright\arraybackslash vertical} & Markets & 462005\\
\cline{1-4}
hhipre &  & Markets & 384325\\

hhipost &  & Markets & 384325\\

hhidelta & \multirow{-3}{*}{\raggedright\arraybackslash both} & Markets & 384325\\
\hline
\end{tabular}
}
\caption{Summary Statistics}
\label{tab:simsum}
\end{table}


\small{%\caption{\label{tab:trashdata}Republic/Santek Merger Simulation Inputs. Volume is reported in thousands of pounds, while prices and margins are reported in dollars.}
%\centering
\resizebox{\linewidth}{!}{
\begin{tabular}[t]{rrccccc}
\toprule
Disposal Firm & Collection Firm & Volume & Disposal Price & Disposal Margin & Collection Margin & Collection Cost\\
\midrule
Republic & Republic & 165 & 42 & 0 & 44 & 93\\
\cmidrule{1-7}
 & Republic & 34 & 36 & 20 &  & 93\\

 & Santek & 218 & 16 & 0 &  & 74\\

 & WMADS & 30 & 36 & 20 &  & 148\\

\multirow{-4}{*}{\raggedleft\arraybackslash Santek} & Regional & 30 & 36 & 20 &  & 148\\
\cmidrule{1-7}
 & Republic & 19 & 25 & 14 &  & 93\\

 & WasteConn & 48 & 11 & 0 &  & 63\\

 & WMADS & 17 & 25 & 14 &  & 148\\

\multirow{-4}{*}{\raggedleft\arraybackslash WasteConn} & Regional & 17 & 25 & 14 &  & 148\\
\bottomrule
\end{tabular}}
}
\small{
> kable(vert_sum,format = "latex",
+       booktabs = TRUE,
+       caption = "Republic/Santex Simulation Effects") #%>%
\begin{table}

\caption{Republic/Santex Simulation Effects}
\centering
\begin{tabular}[t]{llllrrr}
\toprule
Level & Effect & Disposal & Collector & Pre-merger & Post-merger & Change (\%)\\
\midrule
Disposal & Prices & Republic & Republic & 42 & 42 & 0\\
Disposal & Prices & Santek & WM-ADS & 36 & 69 & 88\\
Disposal & Prices & Santek & Regional & 36 & 69 & 88\\
Disposal & Prices & Santek & Santek & 16 & 16 & 0\\
Disposal & Prices & Santek & Republic & 36 & 14 & -63\\
\addlinespace
Disposal & Prices & WasteConn & WM-ADS & 25 & 31 & 22\\
Disposal & Prices & WasteConn & Republic & 25 & 31 & 22\\
Disposal & Prices & WasteConn & Regional & 25 & 31 & 22\\
Disposal & Prices & WasteConn & WasteConn & 11 & 11 & 0\\
Collection & Prices & Republic & Republic & 179 & 217 & 21\\
\addlinespace
Collection & Prices & Santek & Santek & 140 & 172 & 23\\
Collection & Prices & Santek & WM-ADS & 214 & 247 & 15\\
Collection & Prices & Santek & Regional & 214 & 247 & 15\\
Collection & Prices & Santek & Republic & 173 & 189 & 9\\
Collection & Prices & WasteConn & Republic & 162 & 206 & 27\\
\addlinespace
Collection & Prices & WasteConn & WasteConn & 105 & 111 & 5\\
Collection & Prices & WasteConn & WM-ADS & 203 & 209 & 3\\
Collection & Prices & WasteConn & Regional & 203 & 209 & 3\\
Collection & Shares & Republic & Republic & 31 & 23 & -28\\
Collection & Shares & Santek & Republic & 6 & 9 & 48\\
\addlinespace
Collection & Shares & Santek & Santek & 32 & 29 & -9\\
Collection & Shares & Santek & WM-ADS & 7 & 6 & -17\\
Collection & Shares & Santek & Regional & 7 & 6 & -17\\
Collection & Shares & WasteConn & WasteConn & 5 & 11 & 104\\
Collection & Shares & WasteConn & WM-ADS & 4 & 7 & 94\\
\addlinespace
Collection & Shares & WasteConn & Regional & 4 & 7 & 94\\
Collection & Shares & WasteConn & Republic & 3 & 2 & -39\\
\bottomrule
\end{tabular}
\end{table}

>   #collapse_rows(1:2,row_group_label_position="stack")
> sink()
}

\begin{table}
\centering
\small{\begin{tabular}{l|r|r|r}
\hline
Firm   & Pre-Merger & Vertical &  Collection Only \\
\hline
Republic-Santek   & 73.2\%  & 63.1\%  & 59.4\%  \\
Waste Connections & 5.5\%   & 10.9\%  & 8.4\%  \\
Waste Management  & 10.6\%  & 13.0\%  & 16.2\%   \\
Regional          & 10.6\%  & 13.0\%  & 16.2\%   \\
\hline
\end{tabular}
}
\caption{Predicted Market Shares for Integrated Model vs. Collection Only Model}
\label{tab:trash_merged}
\end{table}


\begin{sidewaysfigure}
\centering
\includegraphics[scale=0.9]{output/surplussum.png}
\caption{The figure displays box and whisker plots summarizing the extent to which mergers affect consumer, retailer, wholesaler, and total surplus. Whiskers depict the $5^{th}$ and $95^{th}$ percentiles of a particular outcome, boxes depict the $25^{th}$ and $75^{th}$ percentiles, and the solid horizontal line depicts the median. }
\label{fig:surplussum}
\end{sidewaysfigure}


\begin{sidewaysfigure}
\centering
\includegraphics[scale=0.9]{output/CVvertincumbBW_consumer.png}
\caption{The figure displays box and whisker plots summarizing the extent to which mergers affect consumer surplus as the number of vertically integrated firms present in a market change.  Whiskers depict the $5^{th}$ and $95^{th}$ percentiles of a particular outcome, boxes depict the $25^{th}$ and $75^{th}$ percentiles, and the solid horizontal line depicts the median. }
\label{fig:CVvertincumbBW_consumer}
\end{sidewaysfigure}

\begin{sidewaysfigure}
\centering
\includegraphics[scale=0.9]{output/CVvertincumbBW_total.png}
\caption{The figure displays box and whisker plots summarizing the extent to which mergers affect total surplus as the number of vertically integrated firms present in a market change.  Whiskers depict the $5^{th}$ and $95^{th}$ percentiles of a particular outcome, boxes depict the $25^{th}$ and $75^{th}$ percentiles, and the solid horizontal line depicts the median. }
\label{fig:CVvertincumbBW_total}
\end{sidewaysfigure}

% \begin{sidewaysfigure}
% \centering
% \includegraphics[scale=0.9]{../output/CVvertincumb_updownBW.png}
% \caption{The figure displays box and whisker plots summarizing the extent to which mergers affect consumer (blue,left) and total (orange,right) surplus as the number of vertically integrated firms present in a market change.  Whiskers depict the $5^{th}$ and $95^{th}$ percentiles of a particular outcome, boxes depict the $25^{th}$ and $75^{th}$ percentiles, and the solid horizontal line depicts the median. }
% \label{fig:CVvertincumbupdownBW}
% \end{sidewaysfigure}
%
% \begin{sidewaysfigure}
% \centering
% \includegraphics[scale=0.9]{../output/CVvertincumb_vertincBW.png}
% \caption{The figure displays box and whisker plots summarizing the extent to which mergers affect consumer (blue,left) and total (orange,right) surplus as the number of vertically integrated firms present in a market change.  Whiskers depict the $5^{th}$ and $95^{th}$ percentiles of a particular outcome, boxes depict the $25^{th}$ and $75^{th}$ percentiles, and the solid horizontal line depicts the median. }
% \label{fig:CVvertincumbvertincBW}
% \end{sidewaysfigure}

% \begin{sidewaysfigure}
% \centering
% \includegraphics[scale=0.9]{../output/CVbargupBW.png}
% \caption{The figure displays box and whisker plots summarizing the extent to which mergers among an integrated and unintegrated wholesaler affect consumer, retailer, wholesaler, and total surplus as the bargaining power of wholesalers relative to retailers changes. The different colored boxes display how outcomes change as the number of vertically  integrated firms increases. Whiskers depict the $5^{th}$ and $95^{th}$ percentiles of a particular outcome, boxes depict the $25^{th}$ and $75^{th}$ percentiles, and the solid horizontal line depicts the median.}
% \label{fig:CVbargupBW}
% \end{sidewaysfigure}
%
% \begin{sidewaysfigure}
% \centering
% \includegraphics[scale=0.9]{../output/CVbargdownBW.png}
% \caption{The figure displays box and whisker plots summarizing the extent to which mergers among an integrated and unintegrated retailer affect consumer, retailer, wholesaler, and total surplus as the bargaining power of wholesalers relative to retailers changes. The different colored boxes display how outcomes change as the number of vertically  integrated firms increases. Whiskers depict the $5^{th}$ and $95^{th}$ percentiles of a particular outcome, boxes depict the $25^{th}$ and $75^{th}$ percentiles, and the solid horizontal line depicts the median.}
% \label{fig:CVbargdownBW}
% \end{sidewaysfigure}
%
% \begin{sidewaysfigure}
% \centering
% \includegraphics[scale=0.9]{../output/CVbargvertBW.png}
% \caption{The figure displays box and whisker plots summarizing the extent to which mergers among an unintegrated wholesaler and unintegrated retailer affect consumer, retailer, wholesaler, and total surplus as the bargaining power of wholesalers relative to retailers changes. The different colored boxes display how outcomes change as the number of vertically  integrated firms increases. Whiskers depict the $5^{th}$ and $95^{th}$ percentiles of a particular outcome, boxes depict the $25^{th}$ and $75^{th}$ percentiles, and the solid horizontal line depicts the median.}
% \label{fig:CVbargvertBW}
% \end{sidewaysfigure}

\begin{sidewaysfigure}
\centering
\includegraphics[scale=0.9]{output/CVbargincumbent_consumer.png}
\caption{The figure displays box and whisker plots summarizing the extent to which vertical, upstream, downstream and integrated mergers affect consumer surplus as the bargaining power of wholesalers relative to retailers changes. The different colored boxes display how outcomes change as the number of vertically  integrated firms increases. Whiskers depict the $5^{th}$ and $95^{th}$ percentiles of a particular outcome, boxes depict the $25^{th}$ and $75^{th}$ percentiles, and the solid horizontal line depicts the median.}
\label{fig:CVbargincumbent_consumer}
\end{sidewaysfigure}

\begin{sidewaysfigure}
\centering
\includegraphics[scale=0.9]{output/CVbargincumbent_total.png}
\caption{The figure displays box and whisker plots summarizing the extent to which vertical, upstream, downstream and integrated mergers affect total surplus as the bargaining power of wholesalers relative to retailers changes. The different colored boxes display how outcomes change as the number of vertically  integrated firms increases. Whiskers depict the $5^{th}$ and $95^{th}$ percentiles of a particular outcome, boxes depict the $25^{th}$ and $75^{th}$ percentiles, and the solid horizontal line depicts the median.}
\label{fig:CVbargincumbent_total}
\end{sidewaysfigure}

% \begin{sidewaysfigure}
% \centering
% \includegraphics[scale=0.85]{output/CVbargbothBW.png}
% \caption{The figure displays box and whisker plots summarizing the extent to which mergers among two integrated wholesalers and retailers affect consumer, retailer, wholesaler, and total surplus as the bargaining power of wholesalers relative to retailers changes. The different colored boxes display how outcomes change as the number of vertically  integrated firms increases. Whiskers depict the $5^{th}$ and $95^{th}$ percentiles of a particular outcome, boxes depict the $25^{th}$ and $75^{th}$ percentiles, and the solid horizontal line depicts the median.}
% \label{fig:CVbargbothBW}
% \end{sidewaysfigure}


\begin{sidewaysfigure}
\centering
\includegraphics[scale=0.9]{output/surplussum_cost.png}
\caption{The figure displays box and whisker plots summarizing the extent to which merger outcomes change according to 4 different cost scenarios. The boxes in the top row (``All") either assume all firms face either constant marginal costs (blue,left) or linear marginal costs (orange, right). The boxes in the bottom row (``Party") either assume that the merging parties face constant marginal costs while other firms face linear marginal costs (blue, left), or the merging parties face linear marginal costs while other firms face constant marginal costs (orange, right). Whiskers depict the $5^{th}$ and $95^{th}$ percentiles of a particular outcome, boxes depict the $25^{th}$ and $75^{th}$ percentiles, and the solid horizontal line depicts the median. }
\label{fig:surplussumcost}
\end{sidewaysfigure}

\begin{figure}
\centering
\includegraphics[scale=1.0]{output/chattanooga_tree.png}
\caption{Chattanooga Area MSW Disposal and Collection Market}
\label{fig:chatt_tree}
\end{figure}


\begin{figure}
\centering
\includegraphics[scale=1.0]{output/TrashSimsFirm.png}
\caption{Republic/Santek Simulation Results}
\label{fig:TrashSimsFirms}
\end{figure}


\begin{figure}
\centering
\includegraphics{output/TrashSimsCompare.png}
\caption{Merger Simulation Results Comparison}
\label{fig:TrashSimsCompare}
\end{figure}

\end{document}
