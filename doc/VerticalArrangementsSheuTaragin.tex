\documentclass[12pt]{article}
\usepackage{chicago}    % bibliography package
\usepackage{graphicx}   % insert PostScript figures
\usepackage{setspace}   % controls line spacing
\usepackage{amsmath,amsthm,amssymb,amstext} % controls equation entry and symbols
\usepackage{rotating}   % rotates graphics
\usepackage{soul}       % controls hyphenation
\usepackage{epsfig}     % helps with including graphics
\usepackage{pdflscape}  % helps with displaying rotated graphics in PDFs
\usepackage{lscape}     % helps with rotating pages
\usepackage{setspace}   % controls line spacing
\usepackage{caption}    % controls  captions
\usepackage{adjustbox}  % shrink tables
\usepackage[margin=1in]{geometry} % margins
\usepackage{hyperref}   % displays URLs
%packages for kable tables
 \usepackage{booktabs}
 \usepackage{longtable}
 \usepackage{array}
 \usepackage{multirow}
 \usepackage{wrapfig}
 \usepackage{float}
 \usepackage{colortbl}
 \usepackage{pdflscape}
 \usepackage{tabu}
 \usepackage{threeparttable}
 \usepackage{threeparttablex}
 \usepackage[normalem]{ulem}
 \usepackage{makecell}
 \usepackage{xcolor}

\parskip     2.0mm       % space between paragraphs

\alph{footnote}         % make title footnotes alpha-numeric

\captionsetup[figure]{labelsep=space,labelfont=bf} % remove colon from figure name

%%%%%%%%%%%%%%%%%%%%%%%%%%%%%%%%%%%%%%%%%%%%%%%%%%%%%%%%%%%%%%%%%%%%%%%%%%%
% Title Page
%%%%%%%%%%%%%%%%%%%%%%%%%%%%%%%%%%%%%%%%%%%%%%%%%%%%%%%%%%%%%%%%%%%%%%%%%%%
\title{The Effects of Vertical Arrangements\\in a Vertical Supply Chain\footnote{The analysis and conclusions set forth are those of the authors and do not indicate concurrence by other members of the Board research staff, by the Federal Reserve Board of Governors, by the Federal Trade Commission, or by its Commissioners.  }}

\newcommand*\samethanks[1][\value{footnote}]{\footnotemark[#1]}
\author{Gloria Sheu\footnote{Board of Governors of the Federal Reserve System, gloria.sheu@frb.gov. } \\ Federal Reserve Board \and  Charles Taragin\footnote{Federal Trade Commission, ctaragin@ftc.gov}  \\  Federal Trade Commission}

\date{\today}

\begin{document}

\pagenumbering{roman}       % Roman numerals from abstract to text
\maketitle                  % print title information
\thispagestyle{empty}       % no page number on THIS page


\begin{abstract}
\noindent
\end{abstract}

\bigbreak Keywords: bargaining models; merger simulation; vertical markets

JEL classification: L13; L40; L41; L42

\newpage                    % start a new page
\pagenumbering{arabic}      % Arabic page numbers from now on
\doublespacing

%%%%%%%%%%%%%%%%%%%%%%%%%%%%%%%%%%%%%%%%%%%%%%%%%%%%%%%%%%%%%%%%%%%%%%%%%%%%%%%%%
% Body of Paper
%%%%%%%%%%%%%%%%%%%%%%%%%%%%%%%%%%%%%%%%%%%%%%%%%%%%%%%%%%%%%%%%%%%%%%%%%%%%%%%%%
\section{Introduction}


%%%%%%%%%%%%%%%%%%%%%%%%%%%%%%%%%%%%%%%%%%%%%%%%%%%%%%%%%%%%%%%%%%%%%%%%%%%%%%%%%


%%%%%%%%%%%%%%%%%%%%%%%%%%%%%%%%%%%%%%%%%%


%%%%%%%%%%%%%%%%%%%%%%%%%%%%%%%%%%%%%%%%%%
\subsection*{Data Generating Process}
This section provides an overview of our methodology, with additional details appearing in the Appendix.  We simulate markets by randomly sampling shares from a Dirichlet distribution for 2, 3, 4, or 5 retailers or wholesalers, respectively.\footnote{We parametrize the Dirichlet distribution so it is equivalent to a uniform distribution.  }  We also assume that in the pre-merger state, anywhere from 0 to 4 retailers are vertically integrated with a single wholesaler. We assume that vertically integrated wholesalers supply inputs to non-integrated retailers and that vertically integrated retailers purchase inputs from non-integrated wholesalers. The price coefficient $\alpha$ is calibrated by assuming that in the pre-merger world, there is a vertically integrated outside option available to all customers.  The other goods are differenced relative to this option, which maintains the outside good normalization.  The market size is set to 1.

We specify values for the bargaining parameter ranging from from 0.1 (wholesalers have the advantage) to 0.9 (retailers have the advantage). To better understand the relative bargaining strength of these parameter values, we report our results in terms of $(1-\lambda)/\lambda$, which range from $9$ (wholesaler power is nine times greater than retailer power) to $1/9$ (retailer power is nine times greater than wholesaler power).  The bargaining parameter is identical for all of the retailers in each simulation, unless noted otherwise.

For each combination of number of retailers, number of wholesalers, and bargaining parameter, we draw 1,000 different sets of market primitives.  This results in 3.85 million merger simulations.  We then eliminate mergers where the merger is unprofitable to the merging firms, as well as markets that do not pass the Hypothetical Monopolist Test, yielding 2.69 million markets.\textbf{discrepancy between res.nests dimension and ALL Counts in Table 1}\footnote{The Hypothetical Monopolist Test requires that were a monopolist to jointly own all products in a candidate market, that firm would raise the price of at least one of the merging producers' products by at least a ``small but significant non-transitory increase in price'' (SSNIP), which we take to be 5\%. } All 2.69 million markets treat as primitives the number of retailers, the number of wholesalers, the bargaining parameter, and the wholesaler and retailer marginal costs, which we allow to be either constant or linear.

When simulating a horizontal merger, we assign the products produced by the two largest firms in the market to a single entity post-merger. Similarly, when simulating a vertical merger, we assign the products produced by the largest wholesaler and the largest retailer to a single entity post-merger. This assignment is purposefully skewed towards mergers that are more likely to have competitive effects and to come under agency review.

Table \ref{tab:simsum} provides summary statistics across our various simulations.\footnote{The \href{https://CRAN.R-project.org/package=antitrust}{\texttt{antitrust} R package} contains the computer code needed to calibrate and simulate the effects of mergers in a range of competitive scenarios, including the ones described here.}  The median average wholesale pre-merger price is \$4.5, and the median average retail pre-merger price is \$13.  Because the market size is set to 1, these average prices are equal to total pre-merger expenditures.  Pre-merger HHIs range between 2,820 at the $25^{th}$ percentile to 5,013 at the $75^{th}$, with a median of 3,689. HHIs for horizontal downstream mergers increase by 2,015 points at the median, resulting in a median post-merger HHI equal to 4,914. HHIs for upstream mergers increase by 2,063 points at the median, resulting in a median post-merger HHI equal to 5,238. HHIs for vertical mergers increase by 1,055 points at the median, resulting in a median post-merger HHI equal to 5,995.\footnote{We compute the post-merger HHI for vertical mergers by calculating the merged firms' market share as the sum of all the shares of downstream products that either incorporate the upstream partner's input or are sold by the downstream partner.  }  HHIs for integrated mergers increase by 2,770 points at the median, resulting in a median post-merger HHI equal to 7,280. Many of these markets fall into the span designated by the DOJ/FTC Horizontal Merger Guidelines as ``Highly Concentrated Markets,'' with post-merger HHIs over 2,500 points and HHI Changes greater than 200 points.\footnote{HHI thresholds are discussed in the 2010 Horizontal Merger Guidelines, Section 5.3.  }

%%%%%%%%%%%%%%%%%%%%%%%%%%%%%%%%%%%%%%%%%%
\subsection*{Results Overview}
Our overall results are depicted in Figure \ref{fig:surplussum}, which  is divided into four panels, each showing how the distribution of surplus changes for a particular set of agents (consumers, retailers, or wholesalers), as well as the net effect on the market as a whole.  Surplus is presented as a percentage change relative to total pre-merger expenditure in the downstream market.

Each panel contains four pairs of box and whisker plots, with each pair corresponding to a different type of merger. The blue box and whisker plots (on the left in each pair) depict outcomes assuming that marginal costs are constant, and the orange box and whisker plots (on the right in each pair) show outcomes assuming that marginal costs are linear.  The whiskers display the $5^{th}$ and $95^{th}$ percentiles of the outcome distribution, the boxes denote the $25^{th}$ and $75^{th}$ percentiles, and the solid horizontal line marks the median. Note that negative outcome values imply agent harm, and positive values imply agent benefits.

We focus first on the results for consumers in the left-most panel of Figure \ref{fig:surplussum}.  The median change is negative, indicating harm, across all four types of mergers for both cost specifications.  However, the distributions and magnitudes differ.  In particular, there is only a partial rank-ordering of consumer harm across different types of mergers: consumer harm from integrated mergers first-order stochastically dominates consumer harm from all other merger  types, while downstream and upstream mergers each first-order stochastically dominate consumer harm from vertical mergers, but do not stochastically dominate one other.   Median consumer harm from integrated mergers is about 16\% of pre-merger total expenditures, 1.37 times the magnitude of that from downstream mergers, 2.4 times the magnitude of that from upstream mergers, and more than 3.5 times the magnitude of vertical mergers.




%Consumer harm is less prevalent for mergers where firms have constant marginal costs then when firms have linear costs. Mergers with constant-cost firms benefit consumers in about 15\% of all mergers, while mergers with linear-cost firms benefit consumers in only 2.3\% of mergers. Indeed, consumer harm under linear marginal costs first-order stochastically dominates consumer harm under constant marginal costs. The median consumer harm from markets with linear marginal costs is about 9.3\% of pre-merger total expenditures, about 1.3 times the magnitude of markets with constant marginal costs.


Consumer harm is less prevalent for mergers where firms have constant marginal costs then when firms have linear costs. Vertical mergers when firms have constant marginal costs benefit consumers in about 30\% of all simulations , but only in about 2.3\% of simulations when firms have linear costs. Upstream mergers with constant marginal cost firms benefit consumers in about 12\% of simulations, but only 0.2\% of simulations when firms have linear costs. Downstream mergers with constant marginal cost firms benefit consumers in about 11\% of simulations, but only about 6\% of simulations when firms have linear costs. Finally, integrated  mergers benefit consumers in about 10\% of all simulations when firms have constant costs, but less than 1\% of simulations when firms have linear costs.


%mergers are unlikely to benefit consumers: upstream under both constant and linear marginal costs, vertical mergers with linear marginal costs, and integrated mergers with linear marginal costs.  That upstream horizontal mergers are largely net harmful is somewhat surprising, as it indicates that any EDM from the unintegrated upstream firm merging with its integrated rival is outweighed by the recapture effect. Similar logic applies to integrated mergers. Likewise, the fact that vertical mergers with linear marginal costs are almost never beneficial when marginal costs are linear is surprising, as it suggests that the incentive to raise rivals' cost dominates the benefit from EDM. %especially given that about 40\% of vertical mergers with constant marginal costs are beneficial.
\textbf{Is it worth exploring which effect dominates: the lessening of EDM or the strengthening of RRC? Maybe do this by allowing assymetric cost structures between merging and non-merging parties?}

%However, the fact that downstream mergers under Bertrand are almost never beneficial and frequently quite harmful is somewhat surprising, given the potential for countervailing bargaining leverage to lower wholesale prices.



Turning to retailers in the second panel of Figure \ref{fig:surplussum}, we find that while downstream mergers, vertical mergers and integrated mergers always benefit retailers, upstream mergers  harm retailers in about 70\% of all simulations. Moreover, there is a partial rank-ordering across mergers: the retailer surplus distribution from integrated mergers first-order stochastically dominates the retailer surplus distribution from downstream mergers, which dominates the retailer surplus distribution from upstream mergers, but not the retailer surplus distribution from vertical mergers.  We also find that  for upstream, vertical, and integrated mergers, retailer surplus under constant marginal cost first-order stochastically dominates the retail surplus distribution under linear costs, again suggesting that the incentive to raise rivals' costs dominates the benefits from EDM. By contrast, for downstream mergers, there is no clear rank-ordering between markets with constant marginal costs and markets with linear costs.

% Finally, it is interesting that the harm experienced in the latter instances is quite small in relative terms: median retailer harm from upstream mergers is about one-quarter of the median consumer harm from an upstream merger. This, along with the narrow inter-quartile range of retailer harm (less than 1\% of pre-merger revenues) indicates that retailers are passing on the bulk of the wholesale price increase to consumers.\footnote{This phenomena may also be seen by noting that the distribution of wholesaler gain from upstream mergers is similar in magnitude to the distribution of consumer losses from upstream mergers, though less skewed.  }

As for wholesaler surplus, which appears in the third panel, the effects seen there are largely the reflection of those for retailers: wholesaler surplus increases in about 82\% of all upstream simulated mergers, 28\% of simulated vertical mergers, 20\% of simulated downstream mergers, and 17\% of simulated integrated mergers. %\textbf{comparing these results to \citeN{ST2017}, the effects are less stark (i.e. less of a transfer.}
However, only wholesaler surplus from upstream mergers first-order stochastically dominates the wholesaler surplus of the other merger types.

%Whereas wholesalers benefit from vertical mergers in about 17\% of simulations, retailers benefit from vertical mergers in about 99\% of simulations. Moreover, the median benefit to retailers is 7.5\% of pre-merger expenditures, comparable to the median loss to wholesalers of -9\%. Together, these observations indicate that vertical mergers are largely a rent transfer from wholesalers to retailers and consumers, with retailers keeping most of the gain.

In terms of total welfare, with the exception of the approximately 10\% of vertical mergers and 8.5\% of integrated mergers with constant costs that are beneficial, our simulated mergers are almost always net harmful. Moreover there is a complete rank ordering of mergers, with total harm from integrated mergers first-order stochastically dominating total harm from downstream mergers, which dominates total harm from upstream mergers, which dominates totals harm from vertical mergers. Finally, for each merger type, total harm under linear costs first-order stochastically dominates total harm under constant marginal costs, with the greatest differences occurring for vertical and integrated mergers.

%%%%%%%%%%%%%%%%%%%%%%%%%%%%%%%%%%%%%%%%%%
\subsection*{Vertically Integrated Incumbent Firms}
Here, we investigate how the presence of one or more incumbent vertically integrated firms affects merger outcomes when firms either have constant or linear marginal costs. In particular, we examine merger outcomes when a vertically integrated merges with either an unintegrated upstream or unintegrated downstream firm, or another integrated firm. We also examine merger outcomes for these integrated mergers as the number of integrated non-merging firms increases.

Figure \ref{fig:CVvertincumbBW} depicts box and whisker plots summarizing consumer harm (top row) and total harm (bottom row) as the number of incumbent integrated firms increases from 0 to 6 firms. The box and whisker  plots when the number of  incumbent integrated firms equals 0 correspond to the results depicted in Figure 1 of \citeN{ST2017}. For downstream and upstream  mergers, the plots when the number of  incumbent integrated firms equals 1 depict the outcome of a merger between an integrated firm and an unintegrated rival, and all 3rd parties are unintegrated. By contrast, for vertical mergers, the plots when the number of  incumbent integrated firms equals 1 depict the outcome from an unintegrated wholesaler merging with an unintegrated retailer when a single 3rd party is integrated. Likewise, for integrated mergers, the plots when the number of  incumbent integrated firms equals 2 depict a merger between two integrated firms, and all rivals are unintegrated. The blue box and whisker plots (left) assume that firms face constant marginal costs, while the orange box and whisker plots (right) assume that firms face linear marginal costs.

%%%%%%%%%%%%%%%%%%%%%%%%%%%%%%%%%%%%%%%%%%
\subsection*{Bargaining Power}

%%%%%%%%%%%%%%%%%%%%%%%%%%%%%%%%%%%%%%%%%%%%%%%%%%%%%%%%%%%%%%%%%%%%%%%%%%%%%%%%%
% Bibliography
%%%%%%%%%%%%%%%%%%%%%%%%%%%%%%%%%%%%%%%%%%%%%%%%%%%%%%%%%%%%%%%%%%%%%%%%%%%%%%%%%
\newpage
\bibliographystyle{chicago}
\bibliography{master-bib}
\clearpage

%%%%%%%%%%%%%%%%%%%%%%%%%%%%%%%%%%%%%%%%%%%%%%%%%%%%%%%%%%%%%%%%%%%%%%%%%%%%%%%%%
% Tables and Figures
%%%%%%%%%%%%%%%%%%%%%%%%%%%%%%%%%%%%%%%%%%%%%%%%%%%%%%%%%%%%%%%%%%%%%%%%%%%%%%%%%

\begin{table}

\begin{tabular}{l|l|l|r}
\hline
variable & merger & quant & val\\
\hline
down &  & Min & 2\\

up &  & Min & 2\\

vert &  & Min & 0\\

barg &  & Min & 0\\

nestParm &  & Min & 0\\

avgpricepre.up &  & Min & 1\\

avgpricepre.down &  & Min & 6\\

mktElast & \multirow{-8}{*}{\raggedright\arraybackslash all} & Min & -60\\
\cline{1-4}
hhipre &  & Min & 2008\\

hhipost &  & Min & 2915\\

hhidelta & \multirow{-3}{*}{\raggedright\arraybackslash up} & Min & 0\\
\cline{1-4}
hhipre &  & Min & 2011\\

hhipost &  & Min & 2931\\

hhidelta & \multirow{-3}{*}{\raggedright\arraybackslash down} & Min & 0\\
\cline{1-4}
hhipre &  & Min & 2100\\

hhipost &  & Min & 3120\\

hhidelta & \multirow{-3}{*}{\raggedright\arraybackslash vertical} & Min & 32\\
\cline{1-4}
hhipre &  & Min & 2205\\

hhipost &  & Min & 3633\\

hhidelta & \multirow{-3}{*}{\raggedright\arraybackslash both} & Min & 2\\
\cline{1-4}
down &  & p25 & 3\\

up &  & p25 & 3\\

vert &  & p25 & 0\\

barg &  & p25 & 0\\

nestParm &  & p25 & 0\\

avgpricepre.up &  & p25 & 2\\

avgpricepre.down &  & p25 & 10\\

mktElast & \multirow{-8}{*}{\raggedright\arraybackslash all} & p25 & -1\\
\cline{1-4}
hhipre &  & p25 & 2393\\

hhipost &  & p25 & 4011\\

hhidelta & \multirow{-3}{*}{\raggedright\arraybackslash up} & p25 & 1546\\
\cline{1-4}
hhipre &  & p25 & 2572\\

hhipost &  & p25 & 4135\\

hhidelta & \multirow{-3}{*}{\raggedright\arraybackslash down} & p25 & 1431\\
\cline{1-4}
hhipre &  & p25 & 3623\\

hhipost &  & p25 & 5069\\

hhidelta & \multirow{-3}{*}{\raggedright\arraybackslash vertical} & p25 & 1051\\
\cline{1-4}
hhipre &  & p25 & 3591\\

hhipost &  & p25 & 6242\\

hhidelta & \multirow{-3}{*}{\raggedright\arraybackslash both} & p25 & 2463\\
\cline{1-4}
down &  & p50 & 4\\

up &  & p50 & 4\\

vert &  & p50 & 1\\

barg &  & p50 & 1\\

nestParm &  & p50 & 0\\

avgpricepre.up &  & p50 & 5\\

avgpricepre.down &  & p50 & 13\\

mktElast & \multirow{-8}{*}{\raggedright\arraybackslash all} & p50 & -1\\
\cline{1-4}
hhipre &  & p50 & 2876\\

hhipost &  & p50 & 4963\\

hhidelta & \multirow{-3}{*}{\raggedright\arraybackslash up} & p50 & 2027\\
\cline{1-4}
hhipre &  & p50 & 3166\\

hhipost &  & p50 & 5207\\

hhidelta & \multirow{-3}{*}{\raggedright\arraybackslash down} & p50 & 2040\\
\cline{1-4}
hhipre &  & p50 & 4391\\

hhipost &  & p50 & 6016\\

hhidelta & \multirow{-3}{*}{\raggedright\arraybackslash vertical} & p50 & 1292\\
\cline{1-4}
hhipre &  & p50 & 4275\\

hhipost &  & p50 & 7272\\

hhidelta & \multirow{-3}{*}{\raggedright\arraybackslash both} & p50 & 2800\\
\cline{1-4}
down &  & p75 & 5\\

up &  & p75 & 5\\

vert &  & p75 & 2\\

barg &  & p75 & 1\\

nestParm &  & p75 & 0\\

avgpricepre.up &  & p75 & 10\\

avgpricepre.down &  & p75 & 19\\

mktElast & \multirow{-8}{*}{\raggedright\arraybackslash all} & p75 & 0\\
\cline{1-4}
hhipre &  & p75 & 3794\\

hhipost &  & p75 & 6780\\

hhidelta & \multirow{-3}{*}{\raggedright\arraybackslash up} & p75 & 2891\\
\cline{1-4}
hhipre &  & p75 & 4257\\

hhipost &  & p75 & 7391\\

hhidelta & \multirow{-3}{*}{\raggedright\arraybackslash down} & p75 & 2934\\
\cline{1-4}
hhipre &  & p75 & 5679\\

hhipost &  & p75 & 7293\\

hhidelta & \multirow{-3}{*}{\raggedright\arraybackslash vertical} & p75 & 1755\\
\cline{1-4}
hhipre &  & p75 & 5358\\

hhipost &  & p75 & 8648\\

hhidelta & \multirow{-3}{*}{\raggedright\arraybackslash both} & p75 & 3187\\
\cline{1-4}
down &  & Max & 5\\

up &  & Max & 5\\

vert &  & Max & 4\\

barg &  & Max & 1\\

nestParm &  & Max & 0\\

avgpricepre.up &  & Max & 270\\

avgpricepre.down &  & Max & 297\\

mktElast & \multirow{-8}{*}{\raggedright\arraybackslash all} & Max & 0\\
\cline{1-4}
hhipre &  & Max & 10000\\

hhipost &  & Max & 10000\\

hhidelta & \multirow{-3}{*}{\raggedright\arraybackslash up} & Max & 5000\\
\cline{1-4}
hhipre &  & Max & 10000\\

hhipost &  & Max & 10000\\

hhidelta & \multirow{-3}{*}{\raggedright\arraybackslash down} & Max & 5000\\
\cline{1-4}
hhipre &  & Max & 9962\\

hhipost &  & Max & 10000\\

hhidelta & \multirow{-3}{*}{\raggedright\arraybackslash vertical} & Max & 4314\\
\cline{1-4}
hhipre &  & Max & 9998\\

hhipost &  & Max & 10000\\

hhidelta & \multirow{-3}{*}{\raggedright\arraybackslash both} & Max & 5000\\
\cline{1-4}
down &  & Markets & 2202997\\

up &  & Markets & 2202997\\

vert &  & Markets & 2202997\\

barg &  & Markets & 2202997\\

nestParm &  & Markets & 2202997\\

avgpricepre.up &  & Markets & 2202997\\

avgpricepre.down &  & Markets & 2202997\\

mktElast & \multirow{-8}{*}{\raggedright\arraybackslash all} & Markets & 2202997\\
\cline{1-4}
hhipre &  & Markets & 654731\\

hhipost &  & Markets & 654731\\

hhidelta & \multirow{-3}{*}{\raggedright\arraybackslash up} & Markets & 654731\\
\cline{1-4}
hhipre &  & Markets & 701936\\

hhipost &  & Markets & 701936\\

hhidelta & \multirow{-3}{*}{\raggedright\arraybackslash down} & Markets & 701936\\
\cline{1-4}
hhipre &  & Markets & 462005\\

hhipost &  & Markets & 462005\\

hhidelta & \multirow{-3}{*}{\raggedright\arraybackslash vertical} & Markets & 462005\\
\cline{1-4}
hhipre &  & Markets & 384325\\

hhipost &  & Markets & 384325\\

hhidelta & \multirow{-3}{*}{\raggedright\arraybackslash both} & Markets & 384325\\
\hline
\end{tabular}

\label{tab:simsum}
\caption{Summary Statistics}
\end{table}

\begin{sidewaysfigure}
\centering
\includegraphics[scale=0.9]{../output/surplussum.png}
\caption{The figure displays box and whisker plots summarizing the extent to which mergers affect consumer, retailer, wholesaler, and total surplus. Each blue box depicts the effects assuming that firms face constant marginal costs, while each orange box depicts the effects assuming that firms face linear marginal costs. Whiskers depict the $5^{th}$ and $95^{th}$ percentiles of a particular outcome, boxes depict the $25^{th}$ and $75^{th}$ percentiles, and the solid horizontal line depicts the median. }
\label{fig:surplussum}
\end{sidewaysfigure}


\begin{sidewaysfigure}
\centering
\includegraphics[scale=0.9]{../output/CVvertincumbBW.png}
\caption{The figure displays box and whisker plots summarizing the extent to which mergers affect consumer and total surplus as the number of vertically integrated firms present in a market change. Each blue box depicts the effects assuming that firms face constant marginal costs, while each orange box depicts the effects assuming that firms face linear marginal costs. Whiskers depict the $5^{th}$ and $95^{th}$ percentiles of a particular outcome, boxes depict the $25^{th}$ and $75^{th}$ percentiles, and the solid horizontal line depicts the median. }
\label{fig:CVvertincumbBW}
\end{sidewaysfigure}


\begin{sidewaysfigure}
\centering
\includegraphics[scale=0.9]{../output/CVbargupBW.png}
\caption{The figure displays box and whisker plots summarizing the extent to which mergers among an integrated and unintegrated wholesaler affect consumer, retailer, wholesaler, and total surplus as the bargaining power of wholesalers relative to retailers changes. The different colored boxes display how outcomes change as the number of vertically  integrated firms increases. Whiskers depict the $5^{th}$ and $95^{th}$ percentiles of a particular outcome, boxes depict the $25^{th}$ and $75^{th}$ percentiles, and the solid horizontal line depicts the median.}
\label{fig:CVbargupBW}
\end{sidewaysfigure}

\begin{sidewaysfigure}
\centering
\includegraphics[scale=0.9]{../output/CVbargdownBW.png}
\caption{The figure displays box and whisker plots summarizing the extent to which mergers among an integrated and unintegrated retailer affect consumer, retailer, wholesaler, and total surplus as the bargaining power of wholesalers relative to retailers changes. The different colored boxes display how outcomes change as the number of vertically  integrated firms increases. Whiskers depict the $5^{th}$ and $95^{th}$ percentiles of a particular outcome, boxes depict the $25^{th}$ and $75^{th}$ percentiles, and the solid horizontal line depicts the median.}
\label{fig:CVbargdownBW}
\end{sidewaysfigure}

\begin{sidewaysfigure}
\centering
\includegraphics[scale=0.9]{../output/CVbargvertBW.png}
\caption{The figure displays box and whisker plots summarizing the extent to which mergers among an unintegrated wholesaler and unintegrated retailer affect consumer, retailer, wholesaler, and total surplus as the bargaining power of wholesalers relative to retailers changes. The different colored boxes display how outcomes change as the number of vertically  integrated firms increases. Whiskers depict the $5^{th}$ and $95^{th}$ percentiles of a particular outcome, boxes depict the $25^{th}$ and $75^{th}$ percentiles, and the solid horizontal line depicts the median.}
\label{fig:CVbargvertBW}
\end{sidewaysfigure}

\begin{sidewaysfigure}
\centering
\includegraphics[scale=0.9]{../output/CVbargbothBW.png}
\caption{The figure displays box and whisker plots summarizing the extent to which mergers among two integrated wholesalers and retailers affect consumer, retailer, wholesaler, and total surplus as the bargaining power of wholesalers relative to retailers changes. The different colored boxes display how outcomes change as the number of vertically  integrated firms increases. Whiskers depict the $5^{th}$ and $95^{th}$ percentiles of a particular outcome, boxes depict the $25^{th}$ and $75^{th}$ percentiles, and the solid horizontal line depicts the median.}
\label{fig:CVbargbothBW}
\end{sidewaysfigure}

\end{document}
