\documentclass[12pt]{article}
\usepackage{chicago}    % bibliography package
\usepackage{graphicx}   % insert PostScript figures
\usepackage{setspace}   % controls line spacing
\usepackage{amsmath,amsthm,amssymb,amstext} % controls equation entry and symbols
\usepackage{rotating}   % rotates graphics
\usepackage{soul}       % controls hyphenation
\usepackage{epsfig}     % helps with including graphics
\usepackage{pdflscape}  % helps with displaying rotated graphics in PDFs
\usepackage{lscape}     % helps with rotating pages
\usepackage{setspace}   % controls line spacing
\usepackage{caption}    % controls  captions
\usepackage{adjustbox}  % shrink tables
\usepackage[margin=1in]{geometry} % margins
\usepackage{hyperref}   % displays URLs

\parskip     2.0mm       % space between paragraphs

\alph{footnote}         % make title footnotes alpha-numeric

\captionsetup[figure]{labelsep=space,labelfont=bf} % remove colon from figure name

%%%%%%%%%%%%%%%%%%%%%%%%%%%%%%%%%%%%%%%%%%%%%%%%%%%%%%%%%%%%%%%%%%%%%%%%%%%
% Title Page
%%%%%%%%%%%%%%%%%%%%%%%%%%%%%%%%%%%%%%%%%%%%%%%%%%%%%%%%%%%%%%%%%%%%%%%%%%%
\title{The Effects of Vertical Arrangements\\in a Vertical Supply Chain\footnote{The analysis and conclusions set forth are those of the authors and do not indicate concurrence by other members of the Board research staff, by the Federal Reserve Board of Governors, by the Federal Trade Commission, or by its Commissioners.  }}

\newcommand*\samethanks[1][\value{footnote}]{\footnotemark[#1]}
\author{Gloria Sheu\footnote{Board of Governors of the Federal Reserve System, gloria.sheu@frb.gov. } \\ Federal Reserve Board \and  Charles Taragin\footnote{Federal Trade Commission, ctaragin@ftc.gov}  \\  Federal Trade Commission}

\date{\today}

\begin{document}

\pagenumbering{roman}       % Roman numerals from abstract to text
\maketitle                  % print title information
\thispagestyle{empty}       % no page number on THIS page


\begin{abstract}
\noindent
\end{abstract}

\bigbreak Keywords: bargaining models; merger simulation; vertical markets

JEL classification: L13; L40; L41; L42

\newpage                    % start a new page
\pagenumbering{arabic}      % Arabic page numbers from now on
\doublespacing

%%%%%%%%%%%%%%%%%%%%%%%%%%%%%%%%%%%%%%%%%%%%%%%%%%%%%%%%%%%%%%%%%%%%%%%%%%%%%%%%%
% Body of Paper
%%%%%%%%%%%%%%%%%%%%%%%%%%%%%%%%%%%%%%%%%%%%%%%%%%%%%%%%%%%%%%%%%%%%%%%%%%%%%%%%%
\section{Introduction}


%%%%%%%%%%%%%%%%%%%%%%%%%%%%%%%%%%%%%%%%%%%%%%%%%%%%%%%%%%%%%%%%%%%%%%%%%%%%%%%%%


%%%%%%%%%%%%%%%%%%%%%%%%%%%%%%%%%%%%%%%%%%


%%%%%%%%%%%%%%%%%%%%%%%%%%%%%%%%%%%%%%%%%%
\subsection*{Data Generating Process}
This section provides an overview of our methodology, with additional details appearing in the Appendix.  We simulate markets by randomly sampling shares from a Dirichlet distribution for 2, 3, 4, or 5 retailers or wholesalers, respectively.\footnote{We parametrize the Dirichlet distribution so it is equivalent to a uniform distribution.  }  We also assume that in the pre-merger state, anywhere from 0 to 4 retailers are vertically integrated with a single wholesaler. We assume that vertically integrated wholesalers supply inputs to non-integrated retailers and that vertically integrated retailers purchase inputs from non-integrated wholesalers. The price coefficient $\alpha$ is calibrated by assuming that in the pre-merger world, there is a vertically integrated outside option available to all customers.  The other goods are differenced relative to this option, which maintains the outside good normalization.  The market size is set to 1.

We specify values for the bargaining parameter ranging from from 0.1 (wholesalers have the advantage) to 0.9 (retailers have the advantage). To better understand the relative bargaining strength of these parameter values, we report our results in terms of $(1-\lambda)/\lambda$, which range from $9$ (wholesaler power is nine times greater than retailer power) to $1/9$ (retailer power is nine times greater than wholesaler power).  The bargaining parameter is identical for all of the retailers in each simulation, unless noted otherwise.

For each combination of number of retailers, number of wholesalers, and bargaining parameter, we draw 1,000 different sets of market primitives.  This results in XX million merger simulations.  We then eliminate mergers where the merger is unprofitable to the merging firms, as well as markets that do not pass the Hypothetical Monopolist Test, yielding XX million markets.\footnote{The Hypothetical Monopolist Test requires that were a monopolist to jointly own all products in a candidate market, that firm would raise the price of at least one of the merging producers' products by at least a ``small but significant non-transitory increase in price'' (SSNIP), which we take to be 5\%. } All XX million markets treat as primitives the number of retailers, the number of wholesalers, the bargaining parameter, and the wholesaler and retailer marginal costs, which we allow to be either constant or linear.

When simulating a horizontal merger, we assign the products produced by the two largest firms in the market to a single entity post-merger. Similarly, when simulating a vertical merger, we assign the products produced by the largest wholesaler and the largest retailer to a single entity post-merger. This assignment is purposefully skewed towards mergers that are more likely to have competitive effects and to come under agency review.

Table \ref{tab:simsum} provides summary statistics across our various simulations.\footnote{The \href{https://CRAN.R-project.org/package=antitrust}{\texttt{antitrust} R package} contains the computer code needed to calibrate and simulate the effects of mergers in a range of competitive scenarios, including the ones described here.}  The median average wholesale pre-merger price is almost \$5, and the median average retail pre-merger price is \$12.  Because the market size is set to 1, these average prices are equal to total pre-merger expenditures.  Pre-merger HHIs range between 2,029 at the $25^{th}$ percentile to 5,143 at the $75^{th}$, with a median of 2,828. HHIs for horizontal downstream mergers increase by 1,658 points at the median, resulting in a median post-merger HHI equal to 4,270. HHIs for upstream mergers increase by 1,179 points at the median, resulting in a median post-merger HHI equal to 4,243. HHIs for vertical mergers increase by 953 points at the median, resulting in a median post-merger HHI equal to 5,174.\footnote{We compute the post-merger HHI for vertical mergers by calculating the merged firms' market share as the sum of all the shares of downstream products that either incorporate the upstream partner's input or are sold by the downstream partner.  }  Many of these markets fall into the span designated by the DOJ/FTC Horizontal Merger Guidelines as ``Highly Concentrated Markets,'' with HHIs over 2,500.\footnote{HHI thresholds are discussed in the 2010 Horizontal Merger Guidelines, Section 5.3.  }

%%%%%%%%%%%%%%%%%%%%%%%%%%%%%%%%%%%%%%%%%%
\subsection*{Results Overview}
Our overall results are depicted in Figure \ref{fig:surplussum}, which  is divided into four panels, each showing how the distribution of surplus changes for a particular set of agents (consumers, retailers, or wholesalers), as well as the net effect on the market as a whole.  Surplus is presented as a percentage change relative to total pre-merger expenditure in the downstream market.

Each panel contains four pairs of box and whisker plots, with each pair corresponding to a different type of merger. The blue box and whisker plots (on the left in each pair) depict outcomes assuming that marginal costs are constant, and the orange box and whisker plots (on the right in each pair) show outcomes assuming that marginal costs are linear.  The whiskers display the $5^{th}$ and $95^{th}$ percentiles of the outcome distribution, the boxes denote the $25^{th}$ and $75^{th}$ percentiles, and the solid horizontal line marks the median. Note that negative outcome values imply agent harm, and positive values imply agent benefits.

We focus first on the results for consumers in the left-most panel of Figure \ref{fig:surplussum}.  The median change is negative, indicating harm, across all four types of mergers for both cost specifications.  However, the distributions and magnitudes differ.  In Bertrand markets, there is only a partial rank-ordering of consumer harm across different types of mergers: consumer harm from integrated mergers first-order stochastically dominates consumer harm from all other merger  types, while downstream and upstream mergers each first-order stochastically dominate consumer harm from vertical mergers, but do not stochastically dominate one other.   Median harm from downstream  mergers is 4.5\% of pre-merger total expenditures, 1.3 times the magnitude of that from upstream mergers, and 3.1 times the magnitude of that from vertical mergers. By contrast, under second score, consumer harm in upstream mergers stochastically dominates consumer harm in downstream mergers, which in turn dominates harm in vertical mergers. The median harm from upstream mergers is 3.5\% of pre-merger total expenditures, 4.1 times that of downstream mergers, and 15 times that of vertical mergers.

Four types of mergers are almost always harmful: upstream under both constant and linear marginal costs, vertical mergers with linear marginal costs, and integrated mergers with linear marginal costs.  That upstream horizontal mergers are net harmful is somewhat surprising, as it indicates that any EDM from the unintegrated upstream firm merging with its integrated rival is outweighed by the recapture effect.  Likewise, the fact that vertical mergers with linear marginal costs are almost never beneficial when marginal costs are linear is surprising, as it suggests that the incentive to raise rivals' cost dominates the benefit from EDM. %especially given that about 40\% of vertical mergers with constant marginal costs are beneficial.
\textbf{Is it worth exploring which effect dominates: the lessening of EDM or the strengthening of RRC? Maybe do this by allowing assymetric cost structures between merging and non-merging parties?}

%However, the fact that downstream mergers under Bertrand are almost never beneficial and frequently quite harmful is somewhat surprising, given the potential for countervailing bargaining leverage to lower wholesale prices.

Harm is less prevalent for vertical mergers and downstream mergers with constant marginal costs.  Vertical mergers in Bertrand markets benefit consumers in about 40\% of all simulations , and downstream horizontal mergers benefit consumers in XX\% of all simulations.  The median consumer benefit among positive outcomes from vertical mergers equals 4.2\% of pre-merger total expenditures (comparable to the median consumer harm from downstream mergers under Bertrand), and the median consumer benefit from downstream mergers in auction markets equals 1.7\% of pre-merger revenues (less than two-fifths of the median consumer harm from downstream mergers under Bertrand).

Turning to retailers in the second panel of Figure \ref{fig:surplussum}, we find that while downstream mergers, vertical mergers and integrated mergers benefit retailers, upstream mergers frequently -- though not universally-- harm retailers. We also find that for all merger types, the median retailer surplus under constant marginal cost first-order stochastically dominate the retail surplus distribution under linear costs, indicating that the benefits from EDM are typically greater under constant marginal costs than linear marginal costs.

% Finally, it is interesting that the harm experienced in the latter instances is quite small in relative terms: median retailer harm from upstream mergers is about one-quarter of the median consumer harm from an upstream merger. This, along with the narrow inter-quartile range of retailer harm (less than 1\% of pre-merger revenues) indicates that retailers are passing on the bulk of the wholesale price increase to consumers.\footnote{This phenomena may also be seen by noting that the distribution of wholesaler gain from upstream mergers is similar in magnitude to the distribution of consumer losses from upstream mergers, though less skewed.  }

As for wholesaler surplus, which appears in the third panel, the effects seen there are the mirror image of those for retailers, with net harm in almost all downstream mergers and the majority of vertical mergers, and net benefits in all upstream mergers.  Although one would expect wholesalers to be harmed by downstream mergers, the amount of harm in Bertrand markets is about one-third  of that in second score markets. This difference partially explains the difference in the magnitude of consumer harm between downstream mergers with Bertrand versus auctions: namely, retailers' greater ability to extract wholesaler surplus in second score auction markets translates into more savings passed thru to consumers, which offsets the harm from less retail competition.

%Whereas wholesalers benefit from vertical mergers in about 17\% of simulations, retailers benefit from vertical mergers in about 99\% of simulations. Moreover, the median benefit to retailers is 7.5\% of pre-merger expenditures, comparable to the median loss to wholesalers of -9\%. Together, these observations indicate that vertical mergers are largely a rent transfer from wholesalers to retailers and consumers, with retailers keeping most of the gain.

In terms of total welfare, with the exception of second score downstream mergers, our simulated mergers are typically net harmful, with second score vertical mergers yielding the highest median total harm (-2.5\% of pre-merger expenditures), followed by Bertrand downstream mergers (-2.2\%), Bertrand vertical mergers (-1.5\%), and then upstream mergers.  Downstream mergers under auctions are slightly beneficial at the median, with a gain of 0.5\% of pre-merger expenditures. About 16\% of Bertrand vertical mergers and 7\%  of second score vertical mergers are net beneficial.

\begin{sidewaysfigure}
\centering
\includegraphics[scale=0.85]{../output/surplussum.png}
\caption{The figure displays box and whisker plots summarizing the extent to which mergers affect consumer, retailer, wholesaler, and total surplus. Each blue box depicts the effects assuming that retailers are playing a Bertrand pricing game. Whiskers depict the $5^{th}$ and $95^{th}$ percentiles of a particular outcome, boxes depict the $25^{th}$ and $75^{th}$ percentiles, and the solid horizontal line depicts the median. }
\label{fig:surplussum}
\end{sidewaysfigure}


\begin{sidewaysfigure}
\centering
\includegraphics[scale=0.85]{../output/CVvertincumbBW.png}
\caption{The figure displays box and whisker plots summarizing the extent to which mergers affect consumer, retailer, wholesaler, and total surplus as the number of vertically integrrated firms present in a market change. Each blue box depicts the effects assuming that retailers are playing a Bertrand pricing game. Whiskers depict the $5^{th}$ and $95^{th}$ percentiles of a particular outcome, boxes depict the $25^{th}$ and $75^{th}$ percentiles, and the solid horizontal line depicts the median. }
\label{fig:CVvertincumbBW}
\end{sidewaysfigure}


\begin{sidewaysfigure}
\centering
\includegraphics[scale=0.9]{../output/CVbargupBW.png}
\caption{The figure displays box and whisker plots summarizing the extent to which mergers among an integrated and unintegrated wholesaler affect consumer, retailer, wholesaler, and total surplus as the bargaining power of wholesalers relative to retailers changes. The different colored boxes display how outcomes change as the number of vertically  integrated firms increases. Whiskers depict the $5^{th}$ and $95^{th}$ percentiles of a particular outcome, boxes depict the $25^{th}$ and $75^{th}$ percentiles, and the solid horizontal line depicts the median.}
\label{fig:CVbargupBW}
\end{sidewaysfigure}

\begin{sidewaysfigure}
\centering
\includegraphics[scale=0.9]{../output/CVbargdownBW.png}
\caption{The figure displays box and whisker plots summarizing the extent to which mergers among an integrated and unintegrated retailer affect consumer, retailer, wholesaler, and total surplus as the bargaining power of wholesalers relative to retailers changes. The different colored boxes display how outcomes change as the number of vertically  integrated firms increases. Whiskers depict the $5^{th}$ and $95^{th}$ percentiles of a particular outcome, boxes depict the $25^{th}$ and $75^{th}$ percentiles, and the solid horizontal line depicts the median.}
\label{fig:CVbargdownBW}
\end{sidewaysfigure}

\begin{sidewaysfigure}
\centering
\includegraphics[scale=0.9]{../output/CVbargvertBW.png}
\caption{The figure displays box and whisker plots summarizing the extent to which mergers among an unintegrated wholesaler and unintegrated retailer affect consumer, retailer, wholesaler, and total surplus as the bargaining power of wholesalers relative to retailers changes. The different colored boxes display how outcomes change as the number of vertically  integrated firms increases. Whiskers depict the $5^{th}$ and $95^{th}$ percentiles of a particular outcome, boxes depict the $25^{th}$ and $75^{th}$ percentiles, and the solid horizontal line depicts the median.}
\label{fig:CVbargvertBW}
\end{sidewaysfigure}

\begin{sidewaysfigure}
\centering
\includegraphics[scale=0.9]{../output/CVbargbothBW.png}
\caption{The figure displays box and whisker plots summarizing the extent to which mergers among two integrated wholesalers and retailers affect consumer, retailer, wholesaler, and total surplus as the bargaining power of wholesalers relative to retailers changes. The different colored boxes display how outcomes change as the number of vertically  integrated firms increases. Whiskers depict the $5^{th}$ and $95^{th}$ percentiles of a particular outcome, boxes depict the $25^{th}$ and $75^{th}$ percentiles, and the solid horizontal line depicts the median.}
\label{fig:CVbargbothBW}
\end{sidewaysfigure}

\end{document}
