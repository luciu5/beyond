\documentclass[12pt]{article}
\usepackage{chicago}    % bibliography package
\usepackage{graphicx}   % insert PostScript figures
\usepackage{setspace}   % controls line spacing
\usepackage{amsmath,amsthm,amssymb,amstext} % controls equation entry and symbols
\usepackage{rotating}   % rotates graphics
\usepackage{soul}       % controls hyphenation
\usepackage{epsfig}     % helps with including graphics
\usepackage{pdflscape}  % helps with displaying rotated graphics in PDFs
\usepackage{lscape}     % helps with rotating pages
\usepackage{caption}    % controls  captions
\usepackage{adjustbox}  % shrink tables
\usepackage[margin=1in]{geometry} % margins
\usepackage{hyperref}   % displays URLs
\usepackage{parskip}

\newcounter{para}
\newcommand\mypara{\par\textbf{\stepcounter{para}\thepara)\space}}

\begin{document}

\section{Response to Editor}

\subsection{Comments}


\mypara Incidentally, this section imposes a logit assumption in (1), but
is this really needed before turning to the simulation exercises in Sections 3 and 4?


\mypara In the end, I think R2 set a good bar for the simulations: the
point of doing the simulation [is] to try and provide some qualitative results about the
direction/nature of the net effect of the various countervailing forces and what this depends on?
If you can convincingly deliver on this, then the paper can inform practitioners on the key
elements to be sure to precisely capture in a merger simulation exercise.


\mypara When presenting your results, please consider whether the simulations should be presented as the
result of a statistical sampling (as is done now) or more in the vein of counterfactual analysis
where you vary one parameter at a time. The argument for the statistical sampling approach is
that parameters may interact---although you haven?t clearly established that in most of the
current results. The argument for the one parameter at a time approach is that it can be more
closely tied to the institution presented in the theory section. If you are tempted to do the
sampling approach because there are many parameters to vary, consider whether you should
instead focus the reader?s attention on the most important/illustrative results.

\subsection{TO DOs}
\setcounter{para}{0}
\mypara Be very clear on how the central point is parameterized/calibrated. Be clear about what
data is used to pin down what parameters. Also note what parameters you will vary
either because they are not well identified with available data or because you want to
consider sensitivity.
\mypara Be sure to compare simulations that allow for the complex nature of Republic/Santek
versus the more vanilla simulations considered in ST (2021). This is important to capture
this papers contribution.
\mypara Be sure to consider how the nature of bargaining and downstream diversion will
impact the results. For the later, you would need to go beyond a multinational logit
model, but a nested logit should be able to capture the intuition. I'm not asking you to
estimate substitution in this context since it doesn?t seem like you have the data to do so
(although I'd love to be shown wrong on this point!). I do want to know whether varying
substitution patterns which fit the data you do have are likely to alter the qualitative
outcomes of the merger simulation.


\section{Response to Referee 1}
\setcounter{para}{0}
\mypara I hope that the authors provide the code so that researchers can (a) run their own simulations
and (b) use the code for their own research. The authors should follow the AER/ReStud
guidelines that are becoming standard in the profession
( \url{https://aeadataeditor.github.io/aea-de-guidance/preparing-for-data-deposit.html} ). Basically,
make the code easily readable and post it to a code/data archive website like Zenodo (and link
to it from the author's webpage and the paper).
Note: Footnote 16 states that the antitrust R package contains the code to do the
simulations. I could not easily find the relevant parts of the package. It would be great if this
machinery was also added to antitrust , but I would like to see a stand-alone replication
package.
\mypara I think the authors could ditch the Republic/Santek example. The authors admit that the
example is for ``illustrative" purposes. Instead of a single example, the authors could choose a
few representative mergers and look at specific fake data that mimics those cases. Eg. many
upstream and few downstream or vise versa.
\mypara There is one thing I really like about Section 4: the comparison between a naive merger
analysis (eg. downstream/upstream horizontal effects only) and the more comprehensive
analysis that includes all of the vertical effects. I would like to see this type of analysis in Section

\mypara I think that the authors overclaim a bit. In the abstract (and this sentiment is repeated in the
paper, for example on page 2) the authors: ``In our model, we find that mergers with both
horizontal and vertical characteristics typically harm consumers."
First, I think the more important contribution is to point out that a naive analysis that ignores the
vertical aspect of a merger can mis-quantify the effects of the merger (see point 3 above) or that
the EDM benefits are typically small (a point already made).
Second, mergers arise in nature (and are prosecuted) endogenously. The median pre-merger
HHI in the simulations is between 2800 and 4300 and the median Delta HHI is over 1000. Most
of the mergers analyzed would be presumed harmful based on the classic Cournot intuition or
the more recent Nocke/Whinston intuition. I think the authors should avoid statements like in
their abstract that make it seem they are drawing generic conclusions about merger
enforcement.
Smaller Comments
\mypara  Table 1: It seems a bit suspicious that the price quantiles are all integers. Maybe I am missing
something, but the domain should be some subset of the real number line, making integers very
unlikely.
\mypara  Simulation details. I don't understand why shares are drawn from a Dirichlet distribution (or
any distribution), Shares are endogenous functions of model primitives. Do you mean that you
sampled $\delta$ from a distribution?
\mypara Simulation details. It might make more sense to calibrate to a market elasticity (for example
-1) instead of the way it is done in the paper.
\mypara  It might make sense if a retailer and manufacturer had meaning in the model. One suggestion
is to have $\delta$ be comprised of two components, and manufacturer effect, and a retailer effect,
where the retailer effect is perfectly correlated across manufacturer products within a retailer,
and the manufacturer effect is perfectly correlated across retailer for the same product. So
something like  $\delta_{rw} = \delta_{r} + \delta_{w}$
\mypara  In Section 2.2 (first paragraph), referencing a particular retailer with index r is confusing
because r is the generic index for a retailer.
\mypara In Equation 4, I think there is an additional effect in the first term that Luco and Marshall
(2020) would call the Edgeworth-Salinger effect.
\mypara  The authors mention on multiple occasions that a retailer could cease to trade with a
wholesaler (eg. page 8). I don't see how this is a possibility in the model.
\mypara  I don't think ``Vertical" and ``Horizontal" need to be quoted in the title.

\section{Response to Referee 2}
I recommend that the paper be shortened in two ways. First I think that it should report
the results for the simulations of complex mergers but should omit the results for the simulations
of mergers where the number of pre-existing rivals (that do not participate in the merger)
changes. That is, I think the authors should only include simulation results for the case when
there are no pre-existing vertically integrated rivals that do not participate in the merger and
report that the nature of the results does not change substantially when pre-existing vertically
integrated rivals that do not participate in the merger are allowed to exist. In addition to
shortening the paper, this will allow the authors to provide a simpler and much less confusing
description of the set of examples they consider in their simulations. Second, I think that the
material on generalizing the results to the case of increasing marginal costs should be omitted. I
think this a somewhat separate issue that deserves separate treatment if substantial enough results
can be derived.
\section{Response to Referee 3}
\setcounter{para}{0}
\mypara I could not find a contribution that would merit publication at the Journal of Industrial Economics. The paper is very similar to Sheu and Taragin (2021). The model and simulation are essentially the same. At times, the paper reads as an appendix of the previous article.


\mypara  The paper claims to simulate the welfare effects in a new set of mergers called ``complex mergers." Complex mergers are standard vertical and horizontal mergers where a firm is vertically integrated. They have been previously studied by Ho and Lee (2017) and Crawford et al. (2018), among others. (The related literature also needs to be improved.) So, the paper uses a stylized model to simulate standard mergers.


\mypara  Some of the modeling assumptions are very restrictive. For instance, the logit demand system, simultaneous bargaining-Bertrand game, and the linear fees in the bargaining model. While these assumptions seem reasonable for a policy paper for practitioners, they are too strong for the JINDEC, as they hinder the credibility of the main exercises in the paper. The logit demand, in particular, is unnecessarily restrictive in terms of substitution patterns and could be driving the welfare results. For example, vertically and non-vertically integrated firms are clearly different from the consumers' perspective. One would expect consumers self-selecting into the firms' types, which would affect the compensating variation computation. To the very least, random coefficients on price and on whether the firm is vertically integrated are necessary to obtain more credible results.


\mypara  The RRC discussion was perplexing. What the paper calls ``RRC" differs from the classic RRC in the literature in Salop and Scheffman (1983). In the model, it arises due to the greater bargaining power of the integrated firms. But the paper's assumption that ``negotiation for a given input price treats other input prices and all downstream prices as given" (p. 6) rules out the possibility of a standard RRC effect. So the paper contradicts Crawford et al. (2018) footnote 62. Is this an error? It is paramount that this issue is clarified in the future, even at a different journal. Please include the profit functions and derivations in the Appendix (e.g. equation 5 and subsequent).


\mypara  The simulation exercise in Section 3 does not implement the simulation of the model in Section 2. What is the mapping between the two? How do you ensure market shares are consistent with equation 1 and firm pricing decisions? The paper states that: ``we calibrate the price coefficient $\alpha$ assuming that in the pre-merger world, there is a vertically integrated outside option that sets its Bertrand equilibrium price to \$5, has 15\% market share, and has zero marginal costs. All other goods are differenced." But how are prices consistent with the demand and profit maximization computed? Specifically, I could not understand how such a procedure fits the bargaining game upstream. They seem to be inconsistent. For instance, how do you ensure downstream (upstream) prices are consistent with equation 2 (3)? For these reasons, I could not follow the details of the simulation exercise.


\mypara Finally, I found the calibration exercise in Section 4 hard to follow. At times it reads as if the paper is collecting data. But it is closer to a calibration/simulation exercise motivated by the merger. But there is no estimation and the data is minimal. The theoretical model does not fit the setting in the case. First, the bargaining model is not present nor discussed in the DOJ Lawsuit. Second, the RRC discussed in the DOJ lawsuit refers to foreclosure/exclusionary conduct concerns. This feature could typically has different welfare implications. See, e.g., Ordover et al. (1990) and the references therein. I found the analysis in this section not credible.

\end{document}
